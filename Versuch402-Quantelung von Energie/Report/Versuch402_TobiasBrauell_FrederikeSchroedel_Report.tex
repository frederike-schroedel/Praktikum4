%%%%%%%%%%%%%%%%%%%%
%%% Document
%%%%%%%%%%%%%%%%%%%%
\documentclass[pdftex, a4paper,11pt, twoside, ngerman]{report}
% \documentclass[11pt,xcolor=dvipsnames]{beamer}

% \usepackage[utf8x]{inputenc}                 %für deutsche zeichen äüö ohne kile auto-ersetzen
% \usepackage[ansinew]{inputenc}        %kile auto-ersetzen: einstellungen->latex:general-> hacken bei special characters
% \usepackage[UKenglish]{babel}          %Englisch
\usepackage[ngerman]{babel}          %Deutsch
\usepackage{siunitx}


%%%%%%%%%%
%%% Geometry
%%%%%%%%%%
% \usepackage{showframe}
\usepackage[scale=0.8, hmarginratio=4:2]{geometry}
  \geometry{textheight=1.05\textheight, textwidth=.95\textwidth, marginparwidth=25 pt}



%%%%%%%%%%
%%% Packages (aus header datei)
%%%%%%%%%%
\IfFileExists{header_TobiasBrauell-DOCUMENT.tex}{% Copyright © 2014 Tobias Brauell <tobiasbrauell@gmail.com>

% This is my general purpose LaTeX header file for writing German documents.
% Ideally, you include this using a simple ``\input{header.tex}`` in your main
% document and start with ``\title`` and ``\begin{document}`` afterwards.

% If you need to add additional packages, I recommend not doing this in this
% file, but in your main document. That way, you can just drop in a new
% ``header.tex`` and get all the new commands without having to merge manually.

%%%%%%%%%%%%%%%%%%%%%%%%%%%%%
%%% Locale, date
%%%%%%%%%%%%%%%%%%%%%%%%%%%%%
\usepackage[german]{isodate}



%%%%%%%%%%%%%%%%%%%%%%%%%%%%%
%%% Margins and other spacing
%%%%%%%%%%%%%%%%%%%%%%%%%%%%%
\usepackage[activate]{pdfcprot}
% \usepackage[parfill]{parskip}
\usepackage{setspace}
  \setlength{\columnsep}{2 cm}
  \setlength{\parindent}{0 pt}


%%%%%%%%%%%%%%%%%%%%%%%%%%%%%
%%% Input encoding
%%%%%%%%%%%%%%%%%%%%%%%%%%%%%
\usepackage[T1]{fontenc}
\usepackage[utf8x]{inputenc}



%%%%%%%%%%%%%%%%%%%%%%%%%%%%%
%%% Indexing
%%%%%%%%%%%%%%%%%%%%%%%%%%%%%
\usepackage{makeidx}
  \makeindex



%%%%%%%%%%%%%%%%%%%%%%%%%%%%%
%%% Blindtext
%%%%%%%%%%%%%%%%%%%%%%%%%%%%%
\usepackage{blindtext}


%%%%%%%%%%%%%%%%%%%%%%%%%%%%%
%%% Global Counter
%%%%%%%%%%%%%%%%%%%%%%%%%%%%%



%%%%%%%%%%%%%%%%%%%%%%%%%%%%%
%%% Geometry
%%%%%%%%%%%%%%%%%%%%%%%%%%%%%
\usepackage{layout}
% \usepackage[scale=0.8]{geometry}
%   \geometry{textheight=1.05\textheight, marginparwidth=50 pt}

% \usepackage{multirow}
% \usepackage{dcolumn}



%%%%%%%%%%%%%%%%%%%%%%%%%%%%%
%%% Pagestyle
%%%%%%%%%%%%%%%%%%%%%%%%%%%%%
% \usepackage{fancyhdr}
% \usepackage{microtype} 

% \pagestyle{fancy}



%%%%%%%%%%%%%%%%%%%%%%%%%%%%%
%%% Fonts/Colors
%%%%%%%%%%%%%%%%%%%%%%%%%%%%%
\usepackage{lmodern}
\usepackage{xcolor}
% This replaces all fonts with Bitstream Charter, Bitstream Vera Sans and
% Bitstream Vera Mono. Math will be rendered in Charter.
% \usepackage[charter, greekuppercase=italicized]{mathdesign}
% \usepackage{beramono}
% \usepackage{berasans}

% Bold, sans-serif tensors. This fragment is taken from “egreg” from
% http://tex.stackexchange.com/a/82747/8945 and licensed under `CC-BY-SA
% <https://creativecommons.org/licenses/by-sa/3.0/>`_.
% \usepackage{bm}
%   \DeclareMathAlphabet{\mathsfit}{\encodingdefault}{\sfdefault}{m}{sl}
%   \SetMathAlphabet{\mathsfit}{bold}{\encodingdefault}{\sfdefault}{bx}{sl}
%   \newcommand{\tens}[1]{\bm{\mathsfit{#1}}}

% Bold vectors.
% \renewcommand{\vec}[1]{\boldsymbol{#1}}



%%%%%%%%%%%%%%%%%%%%%%%%%%%%%
%%% Code/Listings
%%%%%%%%%%%%%%%%%%%%%%%%%%%%%
\usepackage{listings}



%%%%%%%%%%%%%%%%%%%%%%%%%%%%%
%%% Enumerations
%%%%%%%%%%%%%%%%%%%%%%%%%%%%%
\usepackage{enumitem}
% \usepackage{paralist}


%%%%%%%%%%%%%%%%%%%%%%%%%%%%%
%%% Figures
%%%%%%%%%%%%%%%%%%%%%%%%%%%%%
% \usepackage[pdftex]{graphicx}
\usepackage{graphicx}
\usepackage{epsfig}
\usepackage{epstopdf}
\usepackage{subfigure}
\usepackage{wrapfig}
\makeatletter \newcommand\hyper@makecurrent[1]{} \makeatother
\usepackage{caption}
% \usepackage{subcaption}

\addto\captionsUKenglish{\renewcommand{\figurename}{Fig.}}
\addto\captionsngerman{\renewcommand{\figurename}{Abb.}}



%%%%%%%%%%%%%%%%%%%%%%%%%%%%%
%%% PDF Pages
%%%%%%%%%%%%%%%%%%%%%%%%%%%%%
\usepackage{pdfpages}



%%%%%%%%%%%%%%%%%%%%%%%%%%%%%
%%% Personal Graphics
%%%%%%%%%%%%%%%%%%%%%%%%%%%%%
\usepackage{tikz}
% \usepackage{tikz-3dplot}
  \usetikzlibrary{calc}
  \usetikzlibrary{decorations.markings}



%%%%%%%%%%%%%%%%%%%%%%%%%%%%%
%%% Math
%%%%%%%%%%%%%%%%%%%%%%%%%%%%%
\usepackage{amsmath}
\usepackage{amssymb}
\usepackage{mathtools}
\usepackage{dcolumn}
\usepackage[
    separate-uncertainty  = true,
    uncertainty-separator =  {\,}, 
%   mode = text, 
    output-decimal-marker ={,}, 
    multi-part-units      = brackets, 
    range-units           = brackets, 
    range-phrase          = {\,--\,}
  ]{siunitx}
% \usepackage{feynmf}



%%%%%%%%%%%%%%%%%%%%%%%%%%%%%
%%% Referenzen
%%%%%%%%%%%%%%%%%%%%%%%%%%%%%
\usepackage{hyperref}
\usepackage{url}
% \usepackage{cleveref}%\label{abc}--\cref{abc} \Cref{abc[,def]}-und \crefrange{abc}{def}-bis
\usepackage[ngerman]{cleveref}%\label{abc}--\cref{abc} \Cref{abc[,def]}-und \crefrange{abc}{def}-bis



%%%%%%%%%%%%%%%%%%%%%%%%%%%%%
%%% Table's
%%%%%%%%%%%%%%%%%%%%%%%%%%%%%
\usepackage{rotating}
\usepackage{longtable}
\usepackage{multirow}
\usepackage{tabularx}
  \newcolumntype{L}[1]{>{\raggedright\arraybackslash}p{#1}} % linksbündig mit Breitenangabe
  \newcolumntype{C}[1]{>{\centering\arraybackslash}p{#1}} % zentriert mit Breitenangabe
  \newcolumntype{R}[1]{>{\raggedleft\arraybackslash}p{#1}} % rechtsbündig mit Breitenangabe



%%%%%%%%%%%%%%%%%%%%%%%%%%%%%
%%% Todo's
%%%%%%%%%%%%%%%%%%%%%%%%%%%%%
% \usepackage{xkeyval}
\usepackage{todonotes} %\todo{text} oder \todo[inline]{text}
%   \presetkeys{todonotes}{inline}{}
%   \let\todox\todo
%   \renewcommand\todo{1}{\todox[inline]{#1}}


%%%%%%%%%%%%%%%%%%%%%%%%%%%%%%%%%%%%%%%%%%%%%%%%%%%%%%%%%%
%%% Settings
%%%%%%%%%%%%%%%%%%%%%%%%%%%%%%%%%%%%%%%%%%%%%%%%%%%%%%%%%%
\usepackage{cancel}

\newcommand{\HRule}{\rule{\linewidth}{0.5mm}}



%%%%%%%%%%%%%%%%%%%%%%%%%%%%%
%%% Theme
%%%%%%%%%%%%%%%%%%%%%%%%%%%%%



%%%%%%%%%%%%%%%%%%%%%%%%%%%%%
%%% header
%%%%%%%%%%%%%%%%%%%%%%%%%%%%%
% \lhead{text}
% \chead{text}
% \rhead{text}



%%%%%%%%%%%%%%%%%%%%%%%%%%%%%
%%% footer
%%%%%%%%%%%%%%%%%%%%%%%%%%%%%
%%% Tobias Brauell       	Versuch....		Ruth Jacobs
% \renewcommand\footrulewidth{.4pt}
% \lfoot{\scriptsize Ruth Jacobs - Tobias Brauell \\ {\ \ \ \ \ \ \ \ \ \ } Gruppe $\alpha 9$} 
% \cfoot{\thepage\ / \ \pageref{LastPage}}
% \rfoot{\scriptsize Versuch 518: Höhenstrahlung \\ Tutor: Christoph Krieger {\ \ \ } } 



%%%%%%%%%%%%%%%%%%%%%%%%%%%%%
%%% Title Page
%%%%%%%%%%%%%%%%%%%%%%%%%%%%%
% \title[ITER { } International Thermonuclear Experimental Reactor]{\huge{\bf{ITER}} \\ \large{\bf{International Thermonuclear Experimental Reactor}}}
% \author[T. Brauell]{Tobias Brauell}
% \institute{Universität Bonn}
% 
% \date{09.~Dez.~2013}
% \logo{\includegraphics[width=.15\textwidth]{Figures/toplogo.png}}


}{% Copyright © 2014 Tobias Brauell <tobiasbrauell@gmail.com>

% This is my general purpose LaTeX header file for writing German documents.
% Ideally, you include this using a simple ``\input{header.tex}`` in your main
% document and start with ``\title`` and ``\begin{document}`` afterwards.

% If you need to add additional packages, I recommend not doing this in this
% file, but in your main document. That way, you can just drop in a new
% ``header.tex`` and get all the new commands without having to merge manually.

%%%%%%%%%%%%%%%%%%%%%%%%%%%%%
%%% Locale, date
%%%%%%%%%%%%%%%%%%%%%%%%%%%%%
\usepackage[german]{isodate}



%%%%%%%%%%%%%%%%%%%%%%%%%%%%%
%%% Margins and other spacing
%%%%%%%%%%%%%%%%%%%%%%%%%%%%%
\usepackage[activate]{pdfcprot}
% \usepackage[parfill]{parskip}
\usepackage{setspace}
  \setlength{\columnsep}{2 cm}
  \setlength{\parindent}{0 pt}


%%%%%%%%%%%%%%%%%%%%%%%%%%%%%
%%% Input encoding
%%%%%%%%%%%%%%%%%%%%%%%%%%%%%
\usepackage[T1]{fontenc}
\usepackage[utf8x]{inputenc}



%%%%%%%%%%%%%%%%%%%%%%%%%%%%%
%%% Indexing
%%%%%%%%%%%%%%%%%%%%%%%%%%%%%
\usepackage{makeidx}
  \makeindex



%%%%%%%%%%%%%%%%%%%%%%%%%%%%%
%%% Blindtext
%%%%%%%%%%%%%%%%%%%%%%%%%%%%%
\usepackage{blindtext}


%%%%%%%%%%%%%%%%%%%%%%%%%%%%%
%%% Global Counter
%%%%%%%%%%%%%%%%%%%%%%%%%%%%%



%%%%%%%%%%%%%%%%%%%%%%%%%%%%%
%%% Geometry
%%%%%%%%%%%%%%%%%%%%%%%%%%%%%
\usepackage{layout}
% \usepackage[scale=0.8]{geometry}
%   \geometry{textheight=1.05\textheight, marginparwidth=50 pt}

% \usepackage{multirow}
% \usepackage{dcolumn}



%%%%%%%%%%%%%%%%%%%%%%%%%%%%%
%%% Pagestyle
%%%%%%%%%%%%%%%%%%%%%%%%%%%%%
% \usepackage{fancyhdr}
% \usepackage{microtype} 

% \pagestyle{fancy}



%%%%%%%%%%%%%%%%%%%%%%%%%%%%%
%%% Fonts/Colors
%%%%%%%%%%%%%%%%%%%%%%%%%%%%%
\usepackage{lmodern}
\usepackage{xcolor}
% This replaces all fonts with Bitstream Charter, Bitstream Vera Sans and
% Bitstream Vera Mono. Math will be rendered in Charter.
% \usepackage[charter, greekuppercase=italicized]{mathdesign}
% \usepackage{beramono}
% \usepackage{berasans}

% Bold, sans-serif tensors. This fragment is taken from “egreg” from
% http://tex.stackexchange.com/a/82747/8945 and licensed under `CC-BY-SA
% <https://creativecommons.org/licenses/by-sa/3.0/>`_.
% \usepackage{bm}
%   \DeclareMathAlphabet{\mathsfit}{\encodingdefault}{\sfdefault}{m}{sl}
%   \SetMathAlphabet{\mathsfit}{bold}{\encodingdefault}{\sfdefault}{bx}{sl}
%   \newcommand{\tens}[1]{\bm{\mathsfit{#1}}}

% Bold vectors.
% \renewcommand{\vec}[1]{\boldsymbol{#1}}



%%%%%%%%%%%%%%%%%%%%%%%%%%%%%
%%% Code/Listings
%%%%%%%%%%%%%%%%%%%%%%%%%%%%%
\usepackage{listings}



%%%%%%%%%%%%%%%%%%%%%%%%%%%%%
%%% Enumerations
%%%%%%%%%%%%%%%%%%%%%%%%%%%%%
\usepackage{enumitem}
% \usepackage{paralist}


%%%%%%%%%%%%%%%%%%%%%%%%%%%%%
%%% Figures
%%%%%%%%%%%%%%%%%%%%%%%%%%%%%
% \usepackage[pdftex]{graphicx}
\usepackage{graphicx}
\usepackage{epsfig}
\usepackage{epstopdf}
\usepackage{subfigure}
\usepackage{wrapfig}
\makeatletter \newcommand\hyper@makecurrent[1]{} \makeatother
\usepackage{caption}
% \usepackage{subcaption}

\addto\captionsUKenglish{\renewcommand{\figurename}{Fig.}}
\addto\captionsngerman{\renewcommand{\figurename}{Abb.}}



%%%%%%%%%%%%%%%%%%%%%%%%%%%%%
%%% PDF Pages
%%%%%%%%%%%%%%%%%%%%%%%%%%%%%
\usepackage{pdfpages}



%%%%%%%%%%%%%%%%%%%%%%%%%%%%%
%%% Personal Graphics
%%%%%%%%%%%%%%%%%%%%%%%%%%%%%
\usepackage{tikz}
% \usepackage{tikz-3dplot}
  \usetikzlibrary{calc}
  \usetikzlibrary{decorations.markings}



%%%%%%%%%%%%%%%%%%%%%%%%%%%%%
%%% Math
%%%%%%%%%%%%%%%%%%%%%%%%%%%%%
\usepackage{amsmath}
\usepackage{amssymb}
\usepackage{mathtools}
\usepackage{dcolumn}
\usepackage[
    separate-uncertainty  = true,
    uncertainty-separator =  {\,}, 
%   mode = text, 
    output-decimal-marker ={,}, 
    multi-part-units      = brackets, 
    range-units           = brackets, 
    range-phrase          = {\,--\,}
  ]{siunitx}
% \usepackage{feynmf}



%%%%%%%%%%%%%%%%%%%%%%%%%%%%%
%%% Referenzen
%%%%%%%%%%%%%%%%%%%%%%%%%%%%%
\usepackage{hyperref}
\usepackage{url}
% \usepackage{cleveref}%\label{abc}--\cref{abc} \Cref{abc[,def]}-und \crefrange{abc}{def}-bis
\usepackage[ngerman]{cleveref}%\label{abc}--\cref{abc} \Cref{abc[,def]}-und \crefrange{abc}{def}-bis



%%%%%%%%%%%%%%%%%%%%%%%%%%%%%
%%% Table's
%%%%%%%%%%%%%%%%%%%%%%%%%%%%%
\usepackage{rotating}
\usepackage{longtable}
\usepackage{multirow}
\usepackage{tabularx}
  \newcolumntype{L}[1]{>{\raggedright\arraybackslash}p{#1}} % linksbündig mit Breitenangabe
  \newcolumntype{C}[1]{>{\centering\arraybackslash}p{#1}} % zentriert mit Breitenangabe
  \newcolumntype{R}[1]{>{\raggedleft\arraybackslash}p{#1}} % rechtsbündig mit Breitenangabe



%%%%%%%%%%%%%%%%%%%%%%%%%%%%%
%%% Todo's
%%%%%%%%%%%%%%%%%%%%%%%%%%%%%
% \usepackage{xkeyval}
\usepackage{todonotes} %\todo{text} oder \todo[inline]{text}
%   \presetkeys{todonotes}{inline}{}
%   \let\todox\todo
%   \renewcommand\todo{1}{\todox[inline]{#1}}


%%%%%%%%%%%%%%%%%%%%%%%%%%%%%%%%%%%%%%%%%%%%%%%%%%%%%%%%%%
%%% Settings
%%%%%%%%%%%%%%%%%%%%%%%%%%%%%%%%%%%%%%%%%%%%%%%%%%%%%%%%%%
\usepackage{cancel}

\newcommand{\HRule}{\rule{\linewidth}{0.5mm}}



%%%%%%%%%%%%%%%%%%%%%%%%%%%%%
%%% Theme
%%%%%%%%%%%%%%%%%%%%%%%%%%%%%



%%%%%%%%%%%%%%%%%%%%%%%%%%%%%
%%% header
%%%%%%%%%%%%%%%%%%%%%%%%%%%%%
% \lhead{text}
% \chead{text}
% \rhead{text}



%%%%%%%%%%%%%%%%%%%%%%%%%%%%%
%%% footer
%%%%%%%%%%%%%%%%%%%%%%%%%%%%%
%%% Tobias Brauell       	Versuch....		Ruth Jacobs
% \renewcommand\footrulewidth{.4pt}
% \lfoot{\scriptsize Ruth Jacobs - Tobias Brauell \\ {\ \ \ \ \ \ \ \ \ \ } Gruppe $\alpha 9$} 
% \cfoot{\thepage\ / \ \pageref{LastPage}}
% \rfoot{\scriptsize Versuch 518: Höhenstrahlung \\ Tutor: Christoph Krieger {\ \ \ } } 



%%%%%%%%%%%%%%%%%%%%%%%%%%%%%
%%% Title Page
%%%%%%%%%%%%%%%%%%%%%%%%%%%%%
% \title[ITER { } International Thermonuclear Experimental Reactor]{\huge{\bf{ITER}} \\ \large{\bf{International Thermonuclear Experimental Reactor}}}
% \author[T. Brauell]{Tobias Brauell}
% \institute{Universität Bonn}
% 
% \date{09.~Dez.~2013}
% \logo{\includegraphics[width=.15\textwidth]{Figures/toplogo.png}}


}



%%%%%%%%%%
%%%%%%%%%%
%%%%%%%%%%
\begin{document}
%   \layout
  \input{./Titlepage-Versuch402.tex}
%   \setcounter{page}{2}
  
  \begin{chapter}*{Abstract}
    Ziel des Versuchs ist es den Zusammenhang zwischen Energie und Frequenz des Lichts zu bestimmen. Hierzu wird die Quantelung der Energie sowohl durch den Photoeffekt, als auch durch die Messung der Balmerserie von Wasserstoff und Deuterium bestimmt. Somit erhalten wir Werte für das Plancksche Wirkungsquantum die im Anschluss verglichen werden. 
  \end{chapter}
  
  \tableofcontents
  
  
  
  %%%%%%%%%%%%%%%%%%%%
  %%%%%%%%%%%%%%%%%%%%
  %%%%%%%%%%%%%%%%%%%%
  \begin{chapter}{Theorie des Versuchs}
    \label{chp:Theorie}
    Für die Durchführung eines jeden Labor-Versuches ist es wichtig bereits vor dem eigentlichen Beginn des Versuches die benötigte Theorie zu kennen und zu verstehen. Daher werden hier zu aller erst die beteiligten theoretischen und physikalischen Grundlagen erklärt.
    
    
    
    %%%%%%%%%%%%%%%%%%%%%%%%%%%%%%
    %%%%%%%%%%%%%%%%%%%%%%%%%%%%%%
    %%%%%%%%%%%%%%%%%%%%%%%%%%%%%%
    \begin{section}{Photoelektrische Bestimmung des Planckschen Wirkungsquantum}
      \label{chp:TheoriePhotoelektrischesWirkungsquantum}
      
      
      
      %%%%%%%%%%%%%%%%%%%%%%%%%%%%%%%%%%%%%%%%
      %%%%%%%%%%%%%%%%%%%%%%%%%%%%%%%%%%%%%%%%
      %%%%%%%%%%%%%%%%%%%%%%%%%%%%%%%%%%%%%%%%
      \begin{subsection}{Photoeffekt}
        \label{chp:TheoriePhotoelektrischesWirkungsquantumPhotoeffekt}
        %Unter dem Photoeffekt versteht man den klassisch betrachtet verblüffenden Effekt, dass die Energie eines Elektrons, welches durch ein Photon aus einer Photokathode ausgelöst wird, nur von der Frequenz des Lichts, aber nicht von der Intensität abhängt. 
        Der Photoeffekt, oder auch Photoelektrischer Effekt genannt, bezeichnet eine physikalische Eigenschaft der Atomhülle mithilfe derer die Energie eines Hüllenelektrons beeinflusst werden kann. Dabei wechselwirkt ein energiereiches Photon mit einem Hüllenelektron und übergibt diesem seinem gesamten Impuls und ändert somit die Energie des Elektrons. Die Energie des Photons ist nach Einstein mit der \cref{eq:EnergiePhoton} gegeben.
        \begin{equation}
          \label{eq:EnergiePhoton}
          E=h\nu=\frac{hc}{\lambda}
        \end{equation}
        Daraus geht also hervor, dass die Energie antiproportional zur Wellenlänge des Photons ist. Abhängig von der Energie des absorbierten Photons reagiert das Elektron mit einer Änderung seiner Bahn um den Atomkern. Das nun angeregte Elektron wechselt in eine höhere Schale. Erhält das Elektron genügend Energie, so kann es komplett aus dem Atom entfernt werden. Diese Energie wird als \textit{Ionisierungsenergie} bezeichnet.
        
        \todo[inline]{Reicht das hier schon? vielleicht noch etwas zu Absorptionslinien und Emissionslinien}
        
        \begin{figure}[htbp]
          \centering
          \begin{minipage}{0.48\textwidth}
            \centering
            \includegraphics[width=.9\textwidth]{Figures/photoeffekt.png}
            \caption{Schematische Darstellung der Wirkungsweise des Photoeffekts.\cite{bib:Photoeffekt}}\label{fig:Photoeffekt}
          \end{minipage}\quad
          \begin{minipage}{0.48\textwidth}
            \centering
            \includegraphics[width=.9\textwidth]{Figures/BohrschesAtommodellSerien.png}
            \caption{Bohrsches Atommodell zusammen mit den Übergangsserien.\cite{bib:BohrschesAtommodellSerien}}\label{fig:BohrschesAtommodellSerien}
          \end{minipage}
        \end{figure}
        
      \end{subsection}
      %%%%%%%%%%%%%%%%%%%%%%%%%%%%%%%%%%%%%%%%
      
      
      
      %%%%%%%%%%%%%%%%%%%%%%%%%%%%%%%%%%%%%%%%
      %%%%%%%%%%%%%%%%%%%%%%%%%%%%%%%%%%%%%%%%
      %%%%%%%%%%%%%%%%%%%%%%%%%%%%%%%%%%%%%%%%
      \begin{subsection}{Photozelle}
        \label{chp:TheoriePhotoelektrischesWirkungsquantumPhotozelle}
        In diesem Versuch wird der Photoeffekt u.A. dafür genutzt um Elektronen aus einer Metallplatte heraus zu lösen. Dabei trifft Das Licht einer Quecksilberdampflampe mit einer ionisierenden Wellenlänge auf eine negativ geladene Metallplatte. Um ein Elektron aus der Oberfläche zu lösen muss die spezifische Austrittsarbeit des verwendeten Metalls erreicht werden. Aus der Austrittsarbeit ergibt sich mithilfe von \cref{eq:Grenzfrequenz} die sog. \textit{Grenzfrequenz}.
        \begin{equation}
          \label{eq:Grenzfrequenz}
          \nu_{0}=\frac{W_{A}}{h}
        \end{equation}
        Die ausgelösten Elektronen können anschließend in der Photozelle detektiert werden. Aufgebaut ist die Photozelle aus einem evakuierten Gehäuse in dem eine Metallplatte als Kathode angebracht ist. Auf der gegenüber liegenden Seite innerhalb des Gehäuses ist ein dünner Ringdraht als Anode angeschlossen. An Anode und Kathode kann nun eine Spannung angelegt werden um die ausgelösten Elektronen einem Potential auszusetzen. Wird nun Spannung erhöht, werden die Elektronen zur Anode hin beschleunigt und können als Strom mit einem Amperemeter gemessen werden. Wird die Spannung umgepolt, so werden die Elektronen zurück auf die Metallplatte beschleunigt und der Anodenstrom verringert sich. So kann man die kinetische Energie der ausgelösten Elektronen finden indem man die Grenzspannung bestimmt, bei der kein Anodenstrom mehr gemessen werden kann. Die maximale kinetische Energie ergibt sich dann aus der Differenz der Energie des Photons und der Austrittsarbeit mit \cref{eq:Ekin}.
        \begin{equation}
          \label{eq:Ekin}
          E_{kin}=eU_{0}=h\nu-W_{A}
        \end{equation}
        
        \begin{figure}[b]
          \begin{center}
            \includegraphics[width=.8\textwidth]{Figures/photozelle_mit_aufbau.png}
            \caption{Schematischer Aufbau einer Photozelle.\cite{bib:PhotozelleAufbau}}\label{fig:PhotozelleAufbau}
          \end{center}
        \end{figure}
        
        Um möglichst genaue Messwerte zu erhalten werden Anode und Kathode aus Materialien mit unterschiedlichen Ionisierungsenergien verwendet. Dadurch kann es vermieden werden, dass der Anodenstrom durch Elektronen verfälscht wird, die aus der Anode selbst gelöst wurden. Die Verbindung unterschiedlicher Materialien bringt allerdings den Nachteil mit sich, dass zwischen ihnen ein sog. \textit{Kontaktpotential} entsteht. Dieses Kontaktpotential entsteht, da sich die Potentiale (oder Fermi-Niveaus) beider Kontakte im Gleichgewicht befinden müssen. Daher kommt zu dieser Gleichung noch eine Korrektur für das Kontaktpotential zwischen den Materialien hinzu die in \cref{eq:KontaktpotentialKorrektur} gegeben ist.
        \begin{equation}
          \label{eq:KontaktpotentialKorrektur}
          W_{K}^{12}=\left|W_{A}^{1}-W_{A}^{2}\right|
        \end{equation}
        Daraus ergibt sich somit die Kontaktpotential korrigierte Formel für die kinetische Energie \cref{eq:EkinKontaktpotential}.
        \begin{equation}
          \label{eq:EkinKontaktpotential}
          E_{kin}=eU_{0}-W_{K}=h\nu-W_{A}-W_{K}
        \end{equation}
        
        \todo[inline]{Sind alle Formel richtig?}
        
      \end{subsection}
      %%%%%%%%%%%%%%%%%%%%%%%%%%%%%%%%%%%%%%%%
      
      
    \end{section}
    %%%%%%%%%%%%%%%%%%%%%%%%%%%%%%
    
    
    
    %%%%%%%%%%%%%%%%%%%%%%%%%%%%%%
    %%%%%%%%%%%%%%%%%%%%%%%%%%%%%%
    %%%%%%%%%%%%%%%%%%%%%%%%%%%%%%
    \begin{section}{Bohrsches Atommodell und Balmer-Serie}
      \label{chp:TheorieBohrBalmerSerie}
      
      
      
      %%%%%%%%%%%%%%%%%%%%%%%%%%%%%%%%%%%%%%%%
      %%%%%%%%%%%%%%%%%%%%%%%%%%%%%%%%%%%%%%%%
      %%%%%%%%%%%%%%%%%%%%%%%%%%%%%%%%%%%%%%%%
      \begin{subsection}{Aufbau des Borschen Atoms}
        \label{chp:TheorieBohrBalmerSerieAufbauAtomhuelle}
        Das Bohrsche Atommodell ist das wohl bekannteste Atommodell. Für sein Modell postulierte Bohr $1913$ die folgenden drei Postulate.
        \begin{enumerate}
          \item Elektronen umkreisen den Atomkern auf festen Bahnen.
          \item Das Elektron kann den Kern nur stabil und ohne Energieabstrahlung auf bestimmten, diskreten Bahnen mit festen Energien umkreisen.
          \item Elektronen können ihre Energie nur durch Emission oder Absorption elektromagnetischer Strahlung ändern.
        \end{enumerate}
        Die Radien der diskreten Bahnen dieser Postulate sind mit \cref{eq:Bohrradius} und deren Energien mit \cref{eq:BohrradiusEnergien}.
        \newline
        \begin{minipage}{.48\textwidth}
          \begin{equation}
            \label{eq:Bohrradius}
            r_{n}=\frac{n^{2}}{Z}a_{0}=\frac{n^{2}h^{2}\epsilon_{0}}{\pi\nu Ze^{2}}
          \end{equation}
        \end{minipage}
        \begin{minipage}{.48\textwidth}
          \begin{equation}
            \label{eq:BohrradiusEnergien}
            E_{n}=-R_{y}\frac{Z^{2}}{n^{2}}
          \end{equation}
        \end{minipage}
        
        \todo[inline]{Hier muss noch was weiter gemacht werden. vielleicht etwas zu bahnwechseln ueber Absorption und Emission.}
        
      \end{subsection}
      %%%%%%%%%%%%%%%%%%%%%%%%%%%%%%%%%%%%%%%%
      
      
      
      %%%%%%%%%%%%%%%%%%%%%%%%%%%%%%%%%%%%%%%%
      %%%%%%%%%%%%%%%%%%%%%%%%%%%%%%%%%%%%%%%%
      %%%%%%%%%%%%%%%%%%%%%%%%%%%%%%%%%%%%%%%%
      \begin{subsection}{Spektroskopie}
        \label{chp:TheorieBohrBalmerSerieSpektroskopie}
        Unter den Begriff Spektroskopie fasst man verschiedene Verfahren zusammen, Energie in ihr Spektrum zerlegen.
        Wir haben es hier mit der Spektroskopie an einem Refelktionsgitter zu tun.
        \begin{figure}[b]
          \begin{center}
            \includegraphics[width=.95\textwidth]{Figures/BalmerserieEmissionAbsorption.png}
            \caption{Emissions- und Absorptionslinien der Balmer Übergänge im Wasserstoff Atom.\cite{bib:BalmerserieEmissionAbsorption}}\label{fig:BalmerserieEmissionAbsorption}
          \end{center}
        \end{figure}
        Zur späteren Bestimmung der Gitterkonstante $g$ nutzen wir die Gittergleichung:
        \begin{equation}
          \label{eq:Gittergleichung}
          g(\sin(\alpha)+\sin(\beta))=m\lambda
        \end{equation}
        
        \begin{figure}[htbp]
          \begin{center}
            \includegraphics[width=.8\textwidth]{Figures/Gitteraufbau.png}
            \caption{Schematische Darstellung des Reflektionsgitteraufbaus.\cite{bib:LDDidactic}}\label{fig:Gitteraufbau}
          \end{center}
        \end{figure}
        
        Es werden die Emissionspektren der Quecksilberdampflampe und der Balmer Lampe hiermit sichtbar gemacht.
      
      \end{subsection}
      %%%%%%%%%%%%%%%%%%%%%%%%%%%%%%%%%%%%%%%%
      
      
      
      %%%%%%%%%%%%%%%%%%%%%%%%%%%%%%%%%%%%%%%%
      %%%%%%%%%%%%%%%%%%%%%%%%%%%%%%%%%%%%%%%%
      %%%%%%%%%%%%%%%%%%%%%%%%%%%%%%%%%%%%%%%%
      \begin{subsection}{Linienbreite}
        \label{chp:TheorieBohrBalmerSerieLinienbreite}
        Ein wichtiger Punkt bei der Spektroskopie ist die Linienbreite. Es gibt verschiedene Effekte, welche die Linienbreite beeinflussen. Die natürliche Linienbreite, minimale Breite der Spektrallinie, tritt sogar bei einem völlig isolierten Atom. Sie entsteht auf Grund der endlichen Strahlungsdauer des Atoms. Hierdurch kommt es zu einer Energie-Zeitunschärfe, in der Form einer Resonazkurve. Zusätzlich zu der natürlichen Linienbreite gibt es die Dopplerverbreiterung, welche durch die Summe  der Bewegung der einzelnen Teilchen in unterschiedliche Raumrichtungen entsteht. Bei höheren Temperaturen verstärkt sich dieser Effekt, durch die steigende mittlere Geschwindigkeit der Teilchen. Die Stoßverbreiterung bewirkt bei benachbarten Atomen eine Verschiebung der Energieniveaus. Durch Stöße wird die Lebensdauer eines angeregten Zustandes verkürzt, sodass die Energie unschärfer wird.
        
        \todo[inline]{formel}
        
      \end{subsection}
      %%%%%%%%%%%%%%%%%%%%%%%%%%%%%%%%%%%%%%%%
      
    \end{section}
    %%%%%%%%%%%%%%%%%%%%%%%%%%%%%%

  \end{chapter}
  %%%%%%%%%%%%%%%%%%%%
          
          
          
  %%%%%%%%%%%%%%%%%%%%
  %%%%%%%%%%%%%%%%%%%%
  %%%%%%%%%%%%%%%%%%%%
  \begin{chapter}{Erster Versuchsteil - Photoeffekt}
    \label{chp:Photoeffekt}
    
    
    
    %%%%%%%%%%%%%%%%%%%%%%%%%%%%%%
    %%%%%%%%%%%%%%%%%%%%%%%%%%%%%%
    %%%%%%%%%%%%%%%%%%%%%%%%%%%%%%
    \begin{section}{Aufbau und Justage}
      \label{chp:photoeffekt:sec:AufbauJustage}
      Wie in der Skizze zu sehen besteht der Aufbau aus einer Hg Lampe, einer Blende, einer Linse, dem Filterrad und der Photozelle. 
      Man nutzt eine Hg Lampe, da diese bereits ein diskretes Spektrum hat, so dass Filter, die einige Nanometer um dem Bereichen der Emissinoslinien liegen, ausreichend sind um einen genügend scharfen Wellenlängen bereich zu nutzen. 
        Bei dem Aufbau ist wichtig das alle Bauteile so positioniert sind, dass die in der Photozelle befindliche Kathode mit Licht verschiedener Frequenzen beleuchtet wird.
      Zunächst ist zu beachten das die Bauteile in der oben genannten Reinfolge auf der gleichen höhe angeordnet sind.
      Hierzu wir die Hg Lampe angestellt, die Schutzkappe der Photozelle entfernt und das Filterrad auf $\lambda = \SI{578}{\nano\meter}$ gestellt.
      Nun schließt man die Irisblend soweit, dass zwar ein Punkt auf der Kathode erscheint, aber die Ringanode nicht beleuchtet wird und stellt diesen Punkt mit Hilfe der Linse scharf.
      Zuletzt wird die Schutzkappe wieder über die Photozelle gestülpt und so ausgerichtet, dass das Licht durch das Streulicht begrenzende Rohr Weiterhin auf die Kathode fällt.
      \begin{figure}[htbp]
        \begin{center}
          \includegraphics[width=.9\textwidth]{Figures/Planckaufbau.png}
          \caption{Versuchsaufbau auf der Optischen Bank. \textbf{a}: Hg-Hochdrucklampe, \textbf{b}: Irisblende,  \textbf{c}: Linse f=100mm, \textbf{d}: Frequenz-Filterrad, \textbf{e}: Photozelle. Positionsangaben in \textit{cm}. \cite{bib:LDDidactic}}\label{fig:Planckaufbau}
        \end{center}
      \end{figure}
      Nach der Jusiterung des Aufbaus ist es wichtig die Fotozelle richtig an zu schließen.
      Wir wollen zwischen Anode und Kathode ein variierbares Gegenfeld haben, um die ausgelösten Elektronen auszubremsen und darüber deren Energie zu bestimmen.
      Hierfür schließt man die Anode an den negativen Ausgang eines Spannungsteilers angeschlossen und die Kathode an den Postiven, welcher auf Masse gezogen wird.
      Ein Netzgerät mit bis zu $\SI{12}{\volt}$ liegt an den Spanungsteiler an, zwischen dessen Ausgängen ein Spannungsmessgerät geschaltet ist.
      
    \end{section}
    %%%%%%%%%%%%%%%%%%%%%%%%%%%%%
    
    %%%%%%%%%%%%%%%%%%%%%%%%%%%%%
    %%%%%%%%%%%%%%%%%%%%%%%%%%%%%
    %%%%%%%%%%%%%%%%%%%%%%%%%%%%%
      \begin{section}{Durchführung}
        \label{chp:Aufbau:sec:ERSTERTEIL:subsec:UNTERTEIL}
        Wir beginnen die Durchführung, indem wir das Filterrad auf den kurzwelligsten Filter stellen, an dem Netzgeräte die Gegenspannung hoch drehen, bis kein Photostrom mehr fließt und uns aus der größe der Spannung überlegen wie der Spannungsteile zusammen gestellt werden soll.
       Wir haben aus den vorhanenden Wiederständen (zwei mal $\SI{100}{\ohm}$, und je einmal $\SI{220}{\ohm}$ und $\SI{330}{\ohm}$) ein eine Spannungsteilerschaltung mit einem Verhältnis von $\frac{100}{330}$ gewählt.

       Nun sollen für die verschiedenen Filter jeweils zwei mal die Kennlinien aufgenommen werden.
       %Hierzu ermittelt man zuerst den kleinst möglichen Strom, unser $I_0$ 
       Um die Kennlinie zu bestimmen beginnt man bei $U=\SI{0}{\volt}$ und erhöht langsam die Spannung, bis $I=\SI{0}{\ampere}$, also bis sich die Werte für den Strom nicht mehr ändern.

       An dieser Stelle sind wir zunächst jeden Filter einmal durchgegangen.
       Vor dem zweiten Durchgang haben wir festgestellt, dass die Schutzkappe verrutscht ist, wodurch die Intensität, welche die Kathode erreicht hat mit der Zeit langsam abgeschwächt wurde.
       Um das im nächsten Durchgang zu vermeiden haben wir die Schutzkappe erneut ausgerichtet und mit Klebeband fixiert.
       Hierdurch ist es zu einer großen Intesitätsdifferenz zwischen den beide aufgenommenen Kennlinien gekommen.

       Im Anschluss wird für die Wellenlänge $\lambda = \SI{365}{\nano\meter}$ die Intesität deutlich vergrößert und erneut eine Kennlinie aufgenommen.  

    \end{section}
    %%%%%%%%%%%%%%%%%%%%%%%%%%%%%
    
    
    
    %%%%%%%%%%%%%%%%%%%%%%%%%%%%%%
    %%%%%%%%%%%%%%%%%%%%%%%%%%%%%%
    %%%%%%%%%%%%%%%%%%%%%%%%%%%%%%
    \begin{section}{Auswertung des ersten Versuchstages}
      \label{chp:Photoeffekt:sec:Auswertung}
      \begin{figure}[b!]
        \begin{center}
          \includegraphics[width=\textwidth]{Figures/Versuch402_1_365.png}
          \caption{Messdaten für drei unterschiedliche Intensitäten. Wellenlänge: \SI{365}{\nano\meter}}\label{fig:Versuch402_1_365}
        \end{center}
      \end{figure}
      Um das Plancksche Wirkungsquantum zu bestimmen müssen bei jeder Messung die jeweilige Grenzspannung bestimmt werden. Diese können sehr einfach aus den Graphen abgelesen werden. Die Grenzspannung findet sich für jede Wellenlänge als X-Achsen Schnittpunkt, da die Grenzspannung jene ist, bei der der Anodenstrom verschwindet. Die Messung der niedrigsten Wellenlänge haben wir für drei unterschiedliche Intensitäten vorgenommen um zu zeigen, dass die Granzspannung unabhängig der Intensität ist und ausschließich durch die Wellenlänge bestimmt wird. In \cref{fig:Versuch402_1_365} kann man gut erkennen, dass diese Abhängigkeit tatsächlich zutrifft. In \crefrange{fig:Versuch402_1_405}{fig:Versuch402_1_578} sind alle Messdaten der anderen Wellenlängen ebenfalls in unterschiedlichen Intensitäten dargestellt. Um aus diesen Messwerten tatsächlich das Plancksche Wirkungsquantum zu bestimmen müssen die verschiedenen Grenzspannungen gegen die zugehörige Frequenz der Photonen aufgetragen und eine Gerade an die Messwerte angepasst werden. Aus der Resultierenden Steigung lässt sich mit \cref{eq:Ekin} und der Elektronen Ladung das Plancksche Wirkungsquantum berechnen. Aus unseren Messungen erhalten wir eine Steigung der Grenzspannungen von $\SI{3.7993e-15}{\volt\per\hertz}$. Daraus berechnet sich für unsere Messungen ein Wirkungsquantum $h=\SI{6.0865e-34}{\joule\second}$. 
      
      \begin{figure}[htbp]
        \centering
        \begin{minipage}{0.48\textwidth}
          \centering
          \includegraphics[width=\textwidth]{Figures/Versuch402_1_405.png}
          \caption{Messdaten für zwei unterschiedliche Intensitäten. Wellenlänge: \SI{405}{\nano\meter}}\label{fig:Versuch402_1_405}
        \end{minipage}\quad
        \begin{minipage}{0.48\textwidth}
          \centering
          \includegraphics[width=\textwidth]{Figures/Versuch402_1_436.png}
          \caption{Messdaten für zwei unterschiedliche Intensitäten. Wellenlänge: \SI{436}{\nano\meter}}\label{fig:Versuch402_1_436}
        \end{minipage}
        \begin{minipage}{0.48\textwidth}
          \centering
          \includegraphics[width=\textwidth]{Figures/Versuch402_1_546.png}
          \caption{Messdaten für zwei unterschiedliche Intensitäten. Wellenlänge: \SI{546}{\nano\meter}}\label{fig:Versuch402_1_546}
        \end{minipage}\quad
        \begin{minipage}{0.48\textwidth}
          \centering
          \includegraphics[width=\textwidth]{Figures/Versuch402_1_578.png}
          \caption{Messdaten für zwei unterschiedliche Intensitäten. Wellenlänge: \SI{578}{\nano\meter}}\label{fig:Versuch402_1_578}
        \end{minipage}
      \end{figure}
      
      \begin{figure}[htbp]
        \begin{center}
          \includegraphics[width=.8\textwidth]{Figures/Versuch402_1_PlanckschesWirkungsquant.png}
          \caption{Graphische Darstellung aller Grenzspannungen zur Bestimmung des Planckschen Wirkungsquantum.}\label{fig:Versuch402_1_PlanckschesWirkungsquant}
        \end{center}
      \end{figure}
      
      
    \end{section}
    %%%%%%%%%%%%%%%%%%%%%%%%%%%%%%
    
    
    
    %%%%%%%%%%%%%%%%%%%%%%%%%%%%%%
    %%%%%%%%%%%%%%%%%%%%%%%%%%%%%%
    %%%%%%%%%%%%%%%%%%%%%%%%%%%%%%
    \begin{section}{Fazit - Photoeffekt}
      \label{chp:Photoeffekt:sec:Fazit}
      Zusammenfassend können wir nicht sagen, dass unsere Messung eine tatsächlich genaue Bestimmung des Planckschen Wirkungsquantum lieferte da die Abweichung zum Literaturwert von $h_{Lit}=\SI{6.62607e-34}{\joule\second}$ leider nicht zu leugnen sind.
      
      \todo[inline]{Am besten noch etwas mehr schreiben}
      
    \end{section}
    %%%%%%%%%%%%%%%%%%%%%%%%%%%%%%
    
  \end{chapter}
  %%%%%%%%%%%%%%%%%%%%
  
  
  
  %%%%%%%%%%%%%%%%%%%%
  %%%%%%%%%%%%%%%%%%%%
  %%%%%%%%%%%%%%%%%%%%
  \begin{chapter}{Zweiter Versichsteil - Balmer-Serie}
    \label{chp:Balmer}
    
    
    %%%%%%%%%%%%%%%%%%%%%%%%%%%%%%
    %%%%%%%%%%%%%%%%%%%%%%%%%%%%%%
    %%%%%%%%%%%%%%%%%%%%%%%%%%%%%%
    \begin{section}{Aufbau und Justage}
      \label{chp:Balmer:sec:Aufbau}
      
      \todo[inline]{skizze}
      
    \end{section}
    %%%%%%%%%%%%%%%%%%%%%%%%%%%%%%
    
  \end{chapter}
  %%%%%%%%%%%%%%%%%%%%
  
  \begin{chapter}{Versuchsaufbau und Durchführung - Balmer-Serie}

      \begin{section}{Aufbau}

          \todo[inline]{skizze}
          Ziel dieses Versuchsteils ist es, dass Spektrum einer Wasserstoff-Deuterium-Lampe, die so genannte Balmerserie, zu bestimmen.
          Hierzu nutzen wir ein Rfelexionsgitter, welches zwischen zwei gewinkleten Armen einer optischen Bank steht.
          Der Winkel zwischen beiden Armen kann abgelesen werden.
          Zu Beginn stehen an dem einen Ende der optischen Bank eine Hg Lampe und am anderen ein Okular mit Strichskala, um die Gitterkonstante zu bestimmen. 
          Im Verlauf des Versuchs wir zuerst die Lampe gegen die Wasserstoff-Deuterium-Lampe und später das Okular gegen einen CCD-Kamera ausgetauscht.
          Die gesammt Anordnung ist, wie auch auf dem Bilder zu sehen, Balmer-Lampe, Linse mit einer Brennweite von $f=\SI{50}{\milli\meter}$, ein verstellbarer Spalt, ein Projektionsojektiv mit $f=\SI{150}{\milli\meter}$, das Refelxionsgitter,eine weitere Linse mit $f=\SI{300}{\milli\meter}$ und zu guter letztdas Okular, beziehungsweise die CCD-Kamera.

      \end{section}

      \begin{section}{Justierung und Durchführung}

          \begin{subsection}{Justierung}

          Zunächst ist der Anfangsaufbau mit der Hg Lampe zu Justieren.
        
          Nach einsetzen der Lampe wird die erste Linse ($f=\SI{50}{\milli\meter}$) so eingesetzt, das sie die Lichtquelle auf die schrägen Flanken des senkrecht stehenden Splates abbildet.
          Als nächstes wird das Objektiv eingesetzt und mit Hilfe der Autokollimation justiert.
          Das lässt sich realisieren, indem man das Gitter so dreht, dass die Reflektion erneut durch das Objektiv fällt und somit den Spalt neben den Spalt abbildet. 
          Wenn man diese Abbildung des Splates nun scharf stellt, ist diese nach dem Objektiv auf Unendlich fokussiert.
          Nun legen wir die Abbildung des Spaltes über dem Spalt und stellen den Winkel zwischen den Armen der optischen Bänke auf $\alpha-\beta = \SI{30}{\degree}$.
          Dieser Winkel wird von uns den ganzen Versuch über nicht geändert.
          Am Ende des zweiten Armses wird das Okular so montiert, dass die Skala gut abzulesen ist. 
          Um die verschiedenen Spektrallinien auf die Skala scharf ab zu bilden stellen man die letzte Linse in den Strahlen gang und verschiebt diese je nach bedarf.
          So entsteht ein Beobachtungsteleskop.

      \end{subsection}

      \begin{subsection}{Bestimmung der Gitterkonstante}

          Um Gitterkonstante zu bestimmen nehmen wir die Hg Lampe.
          Diese benutzen wir, da sie eine hohe Intesität hat, wodurch die Spektrallinien gut sichtbar sind.
          Außerdem haben kennen wir deren Wellenlängen und können so diese den beobachteten Wellenlängen identifizieren um die Gitterkonstante zu bestimmen.
          Wir drehen an der Gittersäule so, um die ersten Linien im Okular sichtabr zu machen.
          Nun stellen wir mit Hilfe der $f=\SI{300}{\milli\meter}$ Linse die Linien scharf und notieren uns die Farbe und den Gitterwinkel $\omega_G$. 
          So gehen wir vor bis wir alle sichtbaren Linien notiert haben.
          Durch den vergleich mit den gegebenen Wellenlängen für Spektrallinien lassen sich so den Winkel Wellenlängen zu ordnen.
          \todo[inline]{Tabelle}

      \end{subsection}

      \begin{subsection}{Untersuchung der Balmer-Linie mit einem Okular}

          Als nächstes tauschen wir die Hg Lampe durch eine Balmer-Lampe und überprüfen die Justierung.
          Dann suchen wir durch drehen der Gittersäule erneut die erste Linie und notieren uns die Farbe, den Winkel und die Breite der Linien $d$, welche den Abstand der Aufspaltung der Linienbreite representiert.
          Hierfür ist besonders wichtig die Linen auf die Skala scharf zu stellen.
          Für die anderen Linien wiederholen wir dieses Vorgehen.
          \todo[inline]{tabelle}

      \end{subsection}

      \begin{subsection}{Untersuchung der Balmerlinie mit der CCD-Kamera}

          Für den nächsten Versuchteil wird das Okular durch die CCD-Kamera ersätzt. 
          Dabei muss dadrauf geachtet werden, dass CCD-Zeile beleuchtet wird.
          Die Kamera wird an den PC angeschlossen und mit Strom versorgt.
          Auf dem PC öffnet man Das Programm VideoCom und wählt die Einstellungen aus der Versuchsanleitung.
          Wir wollen die Isotropieaufspaltung messen, also suchen wir nach Ausschlägen, indem wir mit der Gittersäule die Winkel ansteuern, die wir zuvor ermittelt haben.
          Sobald auf dem Monitor einer dieser Ausschläge erscheint, versuchen wir durch Vergrößerung des Abschnittes, Mittelwertbildung der Intesität und verschieben der letzten Linse zwei scharfe getrennte Spitzen oder Überlagerungen zu erhalten.
          Aus den so aufgenommenen Messdaten lässt sich die Linienverbreiterung bestimmen.

          \todo[inline]{messwerte?}

      \end{subsection}

      \end{section}

  \end{chapter}
  %%%%%%%%%%%%%%%%%%%%
  
  
  %%%%%%%%%%%%%%%%%%%%
  %%%%%%%%%%%%%%%%%%%%
  %%%%%%%%%%%%%%%%%%%%
  \begin{appendix}
    \label{Anhang}
    
    
    
    %%%%%%%%%%%%%%%%%%%%%%%%%%%%%%
    %%%%%%%%%%%%%%%%%%%%%%%%%%%%%%
    %%%%%%%%%%%%%%%%%%%%%%%%%%%%%%
    \begin{chapter}{ERSTER TEIL}
      \label{Anhang:chp:ERSTERTEIL}
      
      
      
      %%%%%%%%%%%%%%%%%%%%%%%%%%%%%%%%%%%%%%%%
      %%%%%%%%%%%%%%%%%%%%%%%%%%%%%%%%%%%%%%%%
      %%%%%%%%%%%%%%%%%%%%%%%%%%%%%%%%%%%%%%%%
      \begin{section}{ERSTER UNTERTEIL}
        \label{Anhang:chp:ERSTERTEIL:sec:UNTERTEIL}
       
       
       
      \end{section}
      %%%%%%%%%%%%%%%%%%%%%%%%%%%%%%%%%%%%%%%%
      
      
    \end{chapter}
    %%%%%%%%%%%%%%%%%%%%%%%%%%%%%%
    
  \end{appendix}
  %%%%%%%%%%%%%%%%%%%%
  
  
  
  %%%%%%%%%%%%%%%%%%%%
  %%%%%%%%%%%%%%%%%%%%
  %%%%%%%%%%%%%%%%%%%%
  \begin{thebibliography}{99}
    \scriptsize
    \bibitem{bib:PhotozelleAufbau}\url{http://www.leifiphysik.de/sites/default/files/medien/Kennlinie_einer_Fotozelle_Bild_2.gif}
    \bibitem{bib:Photoeffekt}\url{http://web-docs.gsi.de/~wolle/TELEKOLLEG/ATOM/IMAGES/photoeffekt1.gif}
    \bibitem{bib:BalmerserieEmissionAbsorption}\url{http://scienceblogs.de/primaklima/wp-content/blogs.dir/29/files/2012/06/i-86621a49589f7ddc22387aa065cef996-Balmerserie.jpg}
    \bibitem{bib:BohrschesAtommodellSerien}\url{http://www.systemdesign.ch/images/4/4e/BohrschesAtommodellSerien.jpg}
    \bibitem{bib:LDDidactic} \url{TODO}
    
%     \cite{ALEPH:2005ab}
%     \bibitem{ALEPH:2005ab}
%     S.~Schael {\it et al.}  [ALEPH and DELPHI and L3 and OPAL and SLD and LEP Electroweak Working Group and SLD Electroweak Group and SLD Heavy Flavour Group Collaborations],
%     %``Precision electroweak measurements on the $Z$ resonance,''
%     Phys.\ Rept.\  {\bf 427} (2006) 257
%     [hep-ex/0509008];\\
%     %%CITATION = HEP-EX/0509008;%%
%     %868 citations counted in INSPIRE as of 16 Oct 2013
%     \texttt{http://lepewwg.web.cern.ch/LEPEWWG/plots/winter2012/}.
  \end{thebibliography}
  %%%%%%%%%%%%%%%%%%%%
  
\end{document}
%%%%%%%%%%
