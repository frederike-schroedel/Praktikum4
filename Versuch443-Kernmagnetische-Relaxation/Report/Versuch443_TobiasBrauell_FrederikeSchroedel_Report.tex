%%%%%%%%%%%%%%%%%%%%
%%% Document
%%%%%%%%%%%%%%%%%%%%
\documentclass[pdftex, a4paper,11pt, twoside, ngerman]{report}
% \documentclass[11pt,xcolor=dvipsnames]{beamer}

% für deutsche zeichen äüö ohne kile auto-ersetzen
% \usepackage[utf8x]{inputenc}

% kile auto-ersetzen: einstellungen->latex:general-> hacken bei special
% characters
% \usepackage[ansinew]{inputenc}
% \usepackage[UKenglish]{babel}          %Englisch
\usepackage[ngerman]{babel}          %Deutsch


%%%%%%%%%%
%%% Geometry
%%%%%%%%%%
% \usepackage{showframe}
\usepackage[scale=0.8, hmarginratio=4:2]{geometry}
  \geometry{textheight=1.05\textheight, textwidth=.95\textwidth,
            marginparwidth=25 pt}



%%%%%%%%%%
%%% Packages (aus header datei)
%%%%%%%%%%
\IfFileExists{header_TobiasBrauell-DOCUMENT.tex}{
    % Copyright © 2014 Tobias Brauell <tobiasbrauell@gmail.com>

% This is my general purpose LaTeX header file for writing German documents.
% Ideally, you include this using a simple ``\input{header.tex}`` in your main
% document and start with ``\title`` and ``\begin{document}`` afterwards.

% If you need to add additional packages, I recommend not doing this in this
% file, but in your main document. That way, you can just drop in a new
% ``header.tex`` and get all the new commands without having to merge manually.

%%%%%%%%%%%%%%%%%%%%%%%%%%%%%
%%% Locale, date
%%%%%%%%%%%%%%%%%%%%%%%%%%%%%
\usepackage[german]{isodate}



%%%%%%%%%%%%%%%%%%%%%%%%%%%%%
%%% Margins and other spacing
%%%%%%%%%%%%%%%%%%%%%%%%%%%%%
\usepackage[activate]{pdfcprot}
% \usepackage[parfill]{parskip}
\usepackage{setspace}
  \setlength{\columnsep}{2 cm}
  \setlength{\parindent}{0 pt}


%%%%%%%%%%%%%%%%%%%%%%%%%%%%%
%%% Input encoding
%%%%%%%%%%%%%%%%%%%%%%%%%%%%%
\usepackage[T1]{fontenc}
\usepackage[utf8x]{inputenc}



%%%%%%%%%%%%%%%%%%%%%%%%%%%%%
%%% Indexing
%%%%%%%%%%%%%%%%%%%%%%%%%%%%%
\usepackage{makeidx}
  \makeindex



%%%%%%%%%%%%%%%%%%%%%%%%%%%%%
%%% Blindtext
%%%%%%%%%%%%%%%%%%%%%%%%%%%%%
\usepackage{blindtext}


%%%%%%%%%%%%%%%%%%%%%%%%%%%%%
%%% Global Counter
%%%%%%%%%%%%%%%%%%%%%%%%%%%%%



%%%%%%%%%%%%%%%%%%%%%%%%%%%%%
%%% Geometry
%%%%%%%%%%%%%%%%%%%%%%%%%%%%%
\usepackage{layout}
% \usepackage[scale=0.8]{geometry}
%   \geometry{textheight=1.05\textheight, marginparwidth=50 pt}

% \usepackage{multirow}
% \usepackage{dcolumn}



%%%%%%%%%%%%%%%%%%%%%%%%%%%%%
%%% Pagestyle
%%%%%%%%%%%%%%%%%%%%%%%%%%%%%
% \usepackage{fancyhdr}
% \usepackage{microtype} 

% \pagestyle{fancy}



%%%%%%%%%%%%%%%%%%%%%%%%%%%%%
%%% Fonts/Colors
%%%%%%%%%%%%%%%%%%%%%%%%%%%%%
\usepackage{lmodern}
\usepackage{xcolor}
% This replaces all fonts with Bitstream Charter, Bitstream Vera Sans and
% Bitstream Vera Mono. Math will be rendered in Charter.
% \usepackage[charter, greekuppercase=italicized]{mathdesign}
% \usepackage{beramono}
% \usepackage{berasans}

% Bold, sans-serif tensors. This fragment is taken from “egreg” from
% http://tex.stackexchange.com/a/82747/8945 and licensed under `CC-BY-SA
% <https://creativecommons.org/licenses/by-sa/3.0/>`_.
% \usepackage{bm}
%   \DeclareMathAlphabet{\mathsfit}{\encodingdefault}{\sfdefault}{m}{sl}
%   \SetMathAlphabet{\mathsfit}{bold}{\encodingdefault}{\sfdefault}{bx}{sl}
%   \newcommand{\tens}[1]{\bm{\mathsfit{#1}}}

% Bold vectors.
% \renewcommand{\vec}[1]{\boldsymbol{#1}}



%%%%%%%%%%%%%%%%%%%%%%%%%%%%%
%%% Code/Listings
%%%%%%%%%%%%%%%%%%%%%%%%%%%%%
\usepackage{listings}



%%%%%%%%%%%%%%%%%%%%%%%%%%%%%
%%% Enumerations
%%%%%%%%%%%%%%%%%%%%%%%%%%%%%
\usepackage{enumitem}
% \usepackage{paralist}


%%%%%%%%%%%%%%%%%%%%%%%%%%%%%
%%% Figures
%%%%%%%%%%%%%%%%%%%%%%%%%%%%%
% \usepackage[pdftex]{graphicx}
\usepackage{graphicx}
\usepackage{epsfig}
\usepackage{epstopdf}
\usepackage{subfigure}
\usepackage{wrapfig}
\makeatletter \newcommand\hyper@makecurrent[1]{} \makeatother
\usepackage{caption}
% \usepackage{subcaption}

\addto\captionsUKenglish{\renewcommand{\figurename}{Fig.}}
\addto\captionsngerman{\renewcommand{\figurename}{Abb.}}



%%%%%%%%%%%%%%%%%%%%%%%%%%%%%
%%% PDF Pages
%%%%%%%%%%%%%%%%%%%%%%%%%%%%%
\usepackage{pdfpages}



%%%%%%%%%%%%%%%%%%%%%%%%%%%%%
%%% Personal Graphics
%%%%%%%%%%%%%%%%%%%%%%%%%%%%%
\usepackage{tikz}
% \usepackage{tikz-3dplot}
  \usetikzlibrary{calc}
  \usetikzlibrary{decorations.markings}



%%%%%%%%%%%%%%%%%%%%%%%%%%%%%
%%% Math
%%%%%%%%%%%%%%%%%%%%%%%%%%%%%
\usepackage{amsmath}
\usepackage{amssymb}
\usepackage{mathtools}
\usepackage{dcolumn}
\usepackage[
    separate-uncertainty  = true,
    uncertainty-separator =  {\,}, 
%   mode = text, 
    output-decimal-marker ={,}, 
    multi-part-units      = brackets, 
    range-units           = brackets, 
    range-phrase          = {\,--\,}
  ]{siunitx}
% \usepackage{feynmf}



%%%%%%%%%%%%%%%%%%%%%%%%%%%%%
%%% Referenzen
%%%%%%%%%%%%%%%%%%%%%%%%%%%%%
\usepackage{hyperref}
\usepackage{url}
% \usepackage{cleveref}%\label{abc}--\cref{abc} \Cref{abc[,def]}-und \crefrange{abc}{def}-bis
\usepackage[ngerman]{cleveref}%\label{abc}--\cref{abc} \Cref{abc[,def]}-und \crefrange{abc}{def}-bis



%%%%%%%%%%%%%%%%%%%%%%%%%%%%%
%%% Table's
%%%%%%%%%%%%%%%%%%%%%%%%%%%%%
\usepackage{rotating}
\usepackage{longtable}
\usepackage{multirow}
\usepackage{tabularx}
  \newcolumntype{L}[1]{>{\raggedright\arraybackslash}p{#1}} % linksbündig mit Breitenangabe
  \newcolumntype{C}[1]{>{\centering\arraybackslash}p{#1}} % zentriert mit Breitenangabe
  \newcolumntype{R}[1]{>{\raggedleft\arraybackslash}p{#1}} % rechtsbündig mit Breitenangabe



%%%%%%%%%%%%%%%%%%%%%%%%%%%%%
%%% Todo's
%%%%%%%%%%%%%%%%%%%%%%%%%%%%%
% \usepackage{xkeyval}
\usepackage{todonotes} %\todo{text} oder \todo[inline]{text}
%   \presetkeys{todonotes}{inline}{}
%   \let\todox\todo
%   \renewcommand\todo{1}{\todox[inline]{#1}}


%%%%%%%%%%%%%%%%%%%%%%%%%%%%%%%%%%%%%%%%%%%%%%%%%%%%%%%%%%
%%% Settings
%%%%%%%%%%%%%%%%%%%%%%%%%%%%%%%%%%%%%%%%%%%%%%%%%%%%%%%%%%
\usepackage{cancel}

\newcommand{\HRule}{\rule{\linewidth}{0.5mm}}



%%%%%%%%%%%%%%%%%%%%%%%%%%%%%
%%% Theme
%%%%%%%%%%%%%%%%%%%%%%%%%%%%%



%%%%%%%%%%%%%%%%%%%%%%%%%%%%%
%%% header
%%%%%%%%%%%%%%%%%%%%%%%%%%%%%
% \lhead{text}
% \chead{text}
% \rhead{text}



%%%%%%%%%%%%%%%%%%%%%%%%%%%%%
%%% footer
%%%%%%%%%%%%%%%%%%%%%%%%%%%%%
%%% Tobias Brauell       	Versuch....		Ruth Jacobs
% \renewcommand\footrulewidth{.4pt}
% \lfoot{\scriptsize Ruth Jacobs - Tobias Brauell \\ {\ \ \ \ \ \ \ \ \ \ } Gruppe $\alpha 9$} 
% \cfoot{\thepage\ / \ \pageref{LastPage}}
% \rfoot{\scriptsize Versuch 518: Höhenstrahlung \\ Tutor: Christoph Krieger {\ \ \ } } 



%%%%%%%%%%%%%%%%%%%%%%%%%%%%%
%%% Title Page
%%%%%%%%%%%%%%%%%%%%%%%%%%%%%
% \title[ITER { } International Thermonuclear Experimental Reactor]{\huge{\bf{ITER}} \\ \large{\bf{International Thermonuclear Experimental Reactor}}}
% \author[T. Brauell]{Tobias Brauell}
% \institute{Universität Bonn}
% 
% \date{09.~Dez.~2013}
% \logo{\includegraphics[width=.15\textwidth]{Figures/toplogo.png}}


}{
    % Copyright © 2014 Tobias Brauell <tobiasbrauell@gmail.com>

% This is my general purpose LaTeX header file for writing German documents.
% Ideally, you include this using a simple ``\input{header.tex}`` in your main
% document and start with ``\title`` and ``\begin{document}`` afterwards.

% If you need to add additional packages, I recommend not doing this in this
% file, but in your main document. That way, you can just drop in a new
% ``header.tex`` and get all the new commands without having to merge manually.

%%%%%%%%%%%%%%%%%%%%%%%%%%%%%
%%% Locale, date
%%%%%%%%%%%%%%%%%%%%%%%%%%%%%
\usepackage[german]{isodate}



%%%%%%%%%%%%%%%%%%%%%%%%%%%%%
%%% Margins and other spacing
%%%%%%%%%%%%%%%%%%%%%%%%%%%%%
\usepackage[activate]{pdfcprot}
% \usepackage[parfill]{parskip}
\usepackage{setspace}
  \setlength{\columnsep}{2 cm}
  \setlength{\parindent}{0 pt}


%%%%%%%%%%%%%%%%%%%%%%%%%%%%%
%%% Input encoding
%%%%%%%%%%%%%%%%%%%%%%%%%%%%%
\usepackage[T1]{fontenc}
\usepackage[utf8x]{inputenc}



%%%%%%%%%%%%%%%%%%%%%%%%%%%%%
%%% Indexing
%%%%%%%%%%%%%%%%%%%%%%%%%%%%%
\usepackage{makeidx}
  \makeindex



%%%%%%%%%%%%%%%%%%%%%%%%%%%%%
%%% Blindtext
%%%%%%%%%%%%%%%%%%%%%%%%%%%%%
\usepackage{blindtext}


%%%%%%%%%%%%%%%%%%%%%%%%%%%%%
%%% Global Counter
%%%%%%%%%%%%%%%%%%%%%%%%%%%%%



%%%%%%%%%%%%%%%%%%%%%%%%%%%%%
%%% Geometry
%%%%%%%%%%%%%%%%%%%%%%%%%%%%%
\usepackage{layout}
% \usepackage[scale=0.8]{geometry}
%   \geometry{textheight=1.05\textheight, marginparwidth=50 pt}

% \usepackage{multirow}
% \usepackage{dcolumn}



%%%%%%%%%%%%%%%%%%%%%%%%%%%%%
%%% Pagestyle
%%%%%%%%%%%%%%%%%%%%%%%%%%%%%
% \usepackage{fancyhdr}
% \usepackage{microtype} 

% \pagestyle{fancy}



%%%%%%%%%%%%%%%%%%%%%%%%%%%%%
%%% Fonts/Colors
%%%%%%%%%%%%%%%%%%%%%%%%%%%%%
\usepackage{lmodern}
\usepackage{xcolor}
% This replaces all fonts with Bitstream Charter, Bitstream Vera Sans and
% Bitstream Vera Mono. Math will be rendered in Charter.
% \usepackage[charter, greekuppercase=italicized]{mathdesign}
% \usepackage{beramono}
% \usepackage{berasans}

% Bold, sans-serif tensors. This fragment is taken from “egreg” from
% http://tex.stackexchange.com/a/82747/8945 and licensed under `CC-BY-SA
% <https://creativecommons.org/licenses/by-sa/3.0/>`_.
% \usepackage{bm}
%   \DeclareMathAlphabet{\mathsfit}{\encodingdefault}{\sfdefault}{m}{sl}
%   \SetMathAlphabet{\mathsfit}{bold}{\encodingdefault}{\sfdefault}{bx}{sl}
%   \newcommand{\tens}[1]{\bm{\mathsfit{#1}}}

% Bold vectors.
% \renewcommand{\vec}[1]{\boldsymbol{#1}}



%%%%%%%%%%%%%%%%%%%%%%%%%%%%%
%%% Code/Listings
%%%%%%%%%%%%%%%%%%%%%%%%%%%%%
\usepackage{listings}



%%%%%%%%%%%%%%%%%%%%%%%%%%%%%
%%% Enumerations
%%%%%%%%%%%%%%%%%%%%%%%%%%%%%
\usepackage{enumitem}
% \usepackage{paralist}


%%%%%%%%%%%%%%%%%%%%%%%%%%%%%
%%% Figures
%%%%%%%%%%%%%%%%%%%%%%%%%%%%%
% \usepackage[pdftex]{graphicx}
\usepackage{graphicx}
\usepackage{epsfig}
\usepackage{epstopdf}
\usepackage{subfigure}
\usepackage{wrapfig}
\makeatletter \newcommand\hyper@makecurrent[1]{} \makeatother
\usepackage{caption}
% \usepackage{subcaption}

\addto\captionsUKenglish{\renewcommand{\figurename}{Fig.}}
\addto\captionsngerman{\renewcommand{\figurename}{Abb.}}



%%%%%%%%%%%%%%%%%%%%%%%%%%%%%
%%% PDF Pages
%%%%%%%%%%%%%%%%%%%%%%%%%%%%%
\usepackage{pdfpages}



%%%%%%%%%%%%%%%%%%%%%%%%%%%%%
%%% Personal Graphics
%%%%%%%%%%%%%%%%%%%%%%%%%%%%%
\usepackage{tikz}
% \usepackage{tikz-3dplot}
  \usetikzlibrary{calc}
  \usetikzlibrary{decorations.markings}



%%%%%%%%%%%%%%%%%%%%%%%%%%%%%
%%% Math
%%%%%%%%%%%%%%%%%%%%%%%%%%%%%
\usepackage{amsmath}
\usepackage{amssymb}
\usepackage{mathtools}
\usepackage{dcolumn}
\usepackage[
    separate-uncertainty  = true,
    uncertainty-separator =  {\,}, 
%   mode = text, 
    output-decimal-marker ={,}, 
    multi-part-units      = brackets, 
    range-units           = brackets, 
    range-phrase          = {\,--\,}
  ]{siunitx}
% \usepackage{feynmf}



%%%%%%%%%%%%%%%%%%%%%%%%%%%%%
%%% Referenzen
%%%%%%%%%%%%%%%%%%%%%%%%%%%%%
\usepackage{hyperref}
\usepackage{url}
% \usepackage{cleveref}%\label{abc}--\cref{abc} \Cref{abc[,def]}-und \crefrange{abc}{def}-bis
\usepackage[ngerman]{cleveref}%\label{abc}--\cref{abc} \Cref{abc[,def]}-und \crefrange{abc}{def}-bis



%%%%%%%%%%%%%%%%%%%%%%%%%%%%%
%%% Table's
%%%%%%%%%%%%%%%%%%%%%%%%%%%%%
\usepackage{rotating}
\usepackage{longtable}
\usepackage{multirow}
\usepackage{tabularx}
  \newcolumntype{L}[1]{>{\raggedright\arraybackslash}p{#1}} % linksbündig mit Breitenangabe
  \newcolumntype{C}[1]{>{\centering\arraybackslash}p{#1}} % zentriert mit Breitenangabe
  \newcolumntype{R}[1]{>{\raggedleft\arraybackslash}p{#1}} % rechtsbündig mit Breitenangabe



%%%%%%%%%%%%%%%%%%%%%%%%%%%%%
%%% Todo's
%%%%%%%%%%%%%%%%%%%%%%%%%%%%%
% \usepackage{xkeyval}
\usepackage{todonotes} %\todo{text} oder \todo[inline]{text}
%   \presetkeys{todonotes}{inline}{}
%   \let\todox\todo
%   \renewcommand\todo{1}{\todox[inline]{#1}}


%%%%%%%%%%%%%%%%%%%%%%%%%%%%%%%%%%%%%%%%%%%%%%%%%%%%%%%%%%
%%% Settings
%%%%%%%%%%%%%%%%%%%%%%%%%%%%%%%%%%%%%%%%%%%%%%%%%%%%%%%%%%
\usepackage{cancel}

\newcommand{\HRule}{\rule{\linewidth}{0.5mm}}



%%%%%%%%%%%%%%%%%%%%%%%%%%%%%
%%% Theme
%%%%%%%%%%%%%%%%%%%%%%%%%%%%%



%%%%%%%%%%%%%%%%%%%%%%%%%%%%%
%%% header
%%%%%%%%%%%%%%%%%%%%%%%%%%%%%
% \lhead{text}
% \chead{text}
% \rhead{text}



%%%%%%%%%%%%%%%%%%%%%%%%%%%%%
%%% footer
%%%%%%%%%%%%%%%%%%%%%%%%%%%%%
%%% Tobias Brauell       	Versuch....		Ruth Jacobs
% \renewcommand\footrulewidth{.4pt}
% \lfoot{\scriptsize Ruth Jacobs - Tobias Brauell \\ {\ \ \ \ \ \ \ \ \ \ } Gruppe $\alpha 9$} 
% \cfoot{\thepage\ / \ \pageref{LastPage}}
% \rfoot{\scriptsize Versuch 518: Höhenstrahlung \\ Tutor: Christoph Krieger {\ \ \ } } 



%%%%%%%%%%%%%%%%%%%%%%%%%%%%%
%%% Title Page
%%%%%%%%%%%%%%%%%%%%%%%%%%%%%
% \title[ITER { } International Thermonuclear Experimental Reactor]{\huge{\bf{ITER}} \\ \large{\bf{International Thermonuclear Experimental Reactor}}}
% \author[T. Brauell]{Tobias Brauell}
% \institute{Universität Bonn}
% 
% \date{09.~Dez.~2013}
% \logo{\includegraphics[width=.15\textwidth]{Figures/toplogo.png}}


}



%%%%%%%%%%
%%%%%%%%%%
%%%%%%%%%%
\begin{document}
%   \layout
  
  
  
  %%%%%%%%%%%%%%%%%%%%
  %%%%%%%%%%%%%%%%%%%%
  %%%%%%%%%%%%%%%%%%%%
  \input{./Titlepage-Versuch443.tex}
  %%%%%%%%%%%%%%%%%%%%
  
  
  
%   \setcounter{page}{2}
  
  \begin{chapter}*{Abstract}
    Ziel des Versuchs ist es kernmagnetische Resonanzen zu untersuchen und dabei wichtige Kenngrößen wie die Gitter- und Spin-Spin Relaxationszeiten zu bestimmen.
    Hierzu werden Protonen, in Form von Wasserstoff in leichtem Mineralöl, in einem magnetischen Wechselfeld angeregt und die daraus resultierende Magnetisierung beobachtet.
    
    \todo[inline]{TO-DO}
    
  \end{chapter}
  
  \tableofcontents
  
  
  
  %%%%%%%%%%%%%%%%%%%%
  %%%%%%%%%%%%%%%%%%%%
  %%%%%%%%%%%%%%%%%%%%
  \begin{chapter}{Theorie des Versuchs}
    \label{chp:Theorie}
    
    \begin{section}{Protonen im homogene Magnetfeld}
        In diesen Versuch experementieren wir mit Probe, welche Protonen bereitstellt.
        Protonen haben einen Spin $ s = \frac 12$ mit einer z-Komponente $m_s = -\frac 12 , \frac 12$.
        Aufgrund des Spins besitzen Protonen ein magnetisches Moment $\mu$, welches parallel zum Spin steht.
        \[
            \mu = \gamma\hbar s
        \]
        Wobei $\gamma$ das gyromagnetische Verhältnis ist und von der Art des Teilchens abhängig ist.
        Das magnetisch Moment bewirkt im externen Magnetfeld $\vec B$ ein Drehmoment $\vec M$ auf das Proton:
        \[
            \vec M = \vec \mu \times \vec B
        \]
        Darauf hin kommt es zu einer Präzession des mangetischen Moments, in unseren Fall um die z-Achse, wobei die Frequenz gegeben ist durch die sogenannte Lamorfrequenz 
        \[
            \omega_L = \gamma B_z
        \]

        Das Drehmoment führt außerdem zu einer Einestellenergie:
        \[
            E = -\langle \mu ,B_z\rangle
        \]
        Man erhält zwei Zustände \todo{spin up down}, da $m_s$ nur zwei Werte annehmen kann.
        So kommt es durch die Einstellenergie zu zwei unterschiedlichen Energieniveaus, welche ohne externes Feld entartet wären.
        \todo{bild}

    \end{section}
    
    \begin{section}{Thermisches Gleichgewicht}

        In dem Fall, dass das externe Magnetfeld abgestellt ist, befinden sich die Spins in einem beliebigen Zustand.
        Somit liegt keine Magnetisierung vor.

        Besteht allerdings ein Magnetfeld, so bildet sich ein anderes thermisches Gleichgewicht aus, wobei $N_+>N_-$, da sich mehr Protonen im unteren Energiniveau befinden.
        Der Besatzungszahlunterschied ist zwar recht gering, (\todo{begrúndung}), erzeugt dennoch eine makroskopische Magnetisierung entlang der z-Achse.
        Um diesen Besetzungszahlunterschied zu erreichen muss das System, für das im unmagnetisierten Zustand $N_+=N_-$ gilt, die Energie die bei der Erniedrigung der Spins frei wird an die Umgebung, das sogenannte Gitter, abgeben.
        Die Boltzmanverteilung gibt das Verhältnis zwischen $N_-$ und $N_+$ an:
        \[
            \frac{N_-}{N_+} = \mathrm e^{\frac{\Delta U}{k_\text BT}}
        \]
        Die Magnetisierung in Z-Richtung ergibt sich aus:
        \[
            M_Z = (N_+-N_-)\mu
        \]
        und für die Magnetisierung im thermischen Gleichgewicht erhält man
        \[
            M_0 \approx N\frac{\mu^2B}{k_\text BT}
        \]
        mit $N=N_-+N_+$.

    \end{section}

    \begin{section}{Longitudinale Relaxation}

        Um die Magnetisierung zu beschreiben, kann man die Differentialgleichung \todo
       % \[
        %    \dod{M_Z(t)}{t} = \frac{M_0-M_Z}{T_1}
        %\]
        mit der Randbedingung $M_{Z(0)} =0$ lösen und erhält die Zeitentwicklung:
        \[
            M(t) = M_0\big (1-\mathrm e^{-\frac tT_1}\big)
        \]
        Hierbei ist $T_1$ die Spin-Gitter-Relaxationszeit, welche beschreibt, wie lange der Prozess der Magnetisierung, beziehungsweise das zurück drehen der Spins in die Ausgangslage nach abschalten des externen Magnetfeldes dauert.

    \end{section}

    \begin{section}{Bei zusätzlichen Gepulst Feld}
        Zusätzlich zu dem homogenen Feld in Z-Richtung nutzen wir im Verlauf des Versuches noch ein gepulstet Magnetfeld, welches sich in der X-Y-Ebene befindet.

        \begin{subsection}{Blochkugel}
            Um sich das Zwei-Niveau-System der Spins im externen Magnetfeld vor zu stellen, bietet sich die Darstellung als Blochkugel an.
            Die Spins stellen hierbei die Vektoren dar, die auf eine Einheitskugel zeigen.
            Dabei wird der Spin up durch den Vektor (0,0,1) und Spin down durch (0,0,-1) Dargestellt.
            Es gibt allerdings auch Superpositions, welche durch die Formel 
            \[
                |\theta,\phi\rangle = \cos\big (\frac \theta 2 \big ) |\uparrow\rangle + \mathrm e^{\mathrm i\phi}\sin\big (\frac \theta 2\big )|\downarrow\rangle
            \]
            wobei die Mischzustände durch die Winkel $\theta$ und $\phi$ bestimmt wetrden.
            Diese Superpositionen ermöglich es auch Spins die nich Parallel zum Magnetfeld stehen zu beschreiben, auch wenn diese energetisch günstiger sind.

        \end{subsection}

        \begin{subsection}{Das rotierende Bezugssystem und das gepulst Magnetfeld}
            Im weiteren Verlauf nutzen wir statt des ruhenden Laborsystem ein rotierendes Inertialsystem, welches wie das magnetische Moment mit der Lamorfrequenz um die Z-Achse präzessiert.

            Wenn nun von außen ein gepulstes transversales Wechselfeld erzeugt, welches auch mit der Lamorfrequenz angelegt wird, so kann dieses die magnetische Flussdicht so verändern, dass diese parallel zur neuen $x*$-Achse liegt.
            Wenn dieser Puls bestimmte Länge und Stärke hat, so wir das magnetische Moment genau so lange mit der $x*$-Achse mitgeführt, das es von der $z*$-Achse auf $y*$ gerät.
            Diese Kippung beträgt dann \SI{90}{\degree}, also $\frac \pi 2$, so dass das magnetische Moment nun in der $x*-y*$-Ebene liegt.
            Wenn man nun die makroskopische Magnetisierung des Laborsystems mit Hilfe einer Messspule beobachtest, so stellt man fest, dass die Magnetisierung auf der festen $x$-Achse osziliert. 
            Diese Oszillation mit der Lamorfrequenz erhält man, da das System weiterhin um die $z$-Achse präzessiert.

            Man kann diesen Impuls auch so verlängern, dass das magnetische Moment nicht nur um $\frac \pi 2$ sondern um $\pi$ gekippt wird. \todo
            
        \end{subsection}

        \begin{subsection}{Transversale Relaxation}
            Nicht alle Spins präzessieren mit der gleichen Geschwindigkeit.
            Dieser unterschiede sind rcht klein, reichen aber aus, damit die Spins nach einiger Zeit "auseinander laufen". 
            Hierdurch nimmt die gemessene Magnetisierung mit der Zeit ab und verschwindet irgendwann.
            Man nennt diesen zerlaufen der Magnetisierung freien Induktionszerfall, FID.
            Dieser Effekt tritt auf, da die Lamorfrequenz, auf Grund von leichten Inhomogenitäten, ortsabhängig ist und die magnetischen Momente der Spins sich untereinander beeinflussen. 
            Der Anteil der durch das äußere Magnetfeld entsteht, kann, da diese Inhomognität konstant ist, später umgekehrt werden.
            Der zweite Effekt lässt sich nicht umkehren, ist aber für jede Probe charakteristisch.
            Um diese beide Effekte beschreiben zu können nutzen wir zwei Relaxationszeiten, die homogene transversale Relaxationszeit $T_2$ und die effektive transversale Relaxationszeit $T_2*$.

            Die effektive transversale Relaxationszeit $T_2*$ beschreibt den exponentiellen Abfall der Magnetisierung. Sie setzt sich wie folgt zusammen:
            \[
                \frac 1 {T_2*} = \frac 1T_2 + \frac 1{T_{2,\text{inhom}}}
            \]
            Der Anteil der inhomogenen Flussdichte ist größer (der exponentielle Abfall schneller) als der von $T_2$, weswegen $T_2*$ kleiner ist als $T_2$.  
            
            Außerdem gilt $T_1 \gg T_2$.
        \end{subsection}

        \begin{subsection}{Rabi-Oszillation}

        \end{subsection}
    \end{section}


    
  \end{chapter}
  %%%%%%%%%%%%%%%%%%%%
         
         
         
  %%%%%%%%%%%%%%%%%%%%
  %%%%%%%%%%%%%%%%%%%%
  %%%%%%%%%%%%%%%%%%%%
  \begin{chapter}{Erster Versuchsteil - Photoeffekt}
    \label{chp:Photoeffekt}
   
   
   
    %%%%%%%%%%%%%%%%%%%%%%%%%%%%%%
    %%%%%%%%%%%%%%%%%%%%%%%%%%%%%%
    %%%%%%%%%%%%%%%%%%%%%%%%%%%%%%
    \begin{section}{Aufbau und Justage}
      \label{chp:photoeffekt:sec:AufbauJustage}
      
      
      
    \end{section}
    %%%%%%%%%%%%%%%%%%%%%%%%%%%%%
   
   
   
    %%%%%%%%%%%%%%%%%%%%%%%%%%%%%
    %%%%%%%%%%%%%%%%%%%%%%%%%%%%%
    %%%%%%%%%%%%%%%%%%%%%%%%%%%%%
    \begin{section}{Durchführung}
      \label{chp:Aufbau:sec:ERSTERTEIL:subsec:UNTERTEIL}
      
      
      
    \end{section}
    %%%%%%%%%%%%%%%%%%%%%%%%%%%%%
   
   
   
    %%%%%%%%%%%%%%%%%%%%%%%%%%%%%%
    %%%%%%%%%%%%%%%%%%%%%%%%%%%%%%
    %%%%%%%%%%%%%%%%%%%%%%%%%%%%%%
    \begin{section}{Auswertung des ersten Versuchstages}
      \label{chp:Photoeffekt:sec:Auswertung}
      
      
      
    \end{section}
    %%%%%%%%%%%%%%%%%%%%%%%%%%%%%%
   
   
   
    %%%%%%%%%%%%%%%%%%%%%%%%%%%%%%
    %%%%%%%%%%%%%%%%%%%%%%%%%%%%%%
    %%%%%%%%%%%%%%%%%%%%%%%%%%%%%%
    \begin{section}{Fazit - Photoeffekt}
      \label{chp:Photoeffekt:sec:Fazit}
      
      
      
    \end{section}
    %%%%%%%%%%%%%%%%%%%%%%%%%%%%%%
   
  \end{chapter}
  %%%%%%%%%%%%%%%%%%%%
 
 
 
  %%%%%%%%%%%%%%%%%%%%
  %%%%%%%%%%%%%%%%%%%%
  %%%%%%%%%%%%%%%%%%%%
  \begin{chapter}{Zweiter Versuchsteil - Balmer-Serie}
    \label{chp:Balmer}
 
 
    %%%%%%%%%%%%%%%%%%%%%%%%%%%%%%
    %%%%%%%%%%%%%%%%%%%%%%%%%%%%%%
    %%%%%%%%%%%%%%%%%%%%%%%%%%%%%%
    \begin{section}{Aufbau}
      \label{chp:Balmer:sec:Aufbau}
      
      
      
    \end{section}
    %%%%%%%%%%%%%%%%%%%%%%%%%%%%%%
   
   
   
    %%%%%%%%%%%%%%%%%%%%%%%%%%%%%%
    %%%%%%%%%%%%%%%%%%%%%%%%%%%%%%
    %%%%%%%%%%%%%%%%%%%%%%%%%%%%%%
    \begin{section}{Justierung und Durchführung}
      \label{chp:Balmer:sec:JusitierungDurchfuehrung}
     
     
    
    \end{section}
    %%%%%%%%%%%%%%%%%%%%%%%%%%%%%%


    

    %%%%%%%%%%%%%%%%%%%%%%%%%%%%%%
    %%%%%%%%%%%%%%%%%%%%%%%%%%%%%%
    %%%%%%%%%%%%%%%%%%%%%%%%%%%%%%
    \begin{section}{Auswertung}
      \label{chp:Balmer:sec:Auswertung}
      
      
     
    \end{section}
    %%%%%%%%%%%%%%%%%%%%%%%%%%%%%%
   
   
   
    %%%%%%%%%%%%%%%%%%%%%%%%%%%%%%
    %%%%%%%%%%%%%%%%%%%%%%%%%%%%%%
    %%%%%%%%%%%%%%%%%%%%%%%%%%%%%%
    \begin{section}{Fazit}
      \label{chp:Balmer:sec:Fazit}
      
      
      
    \end{section}
    %%%%%%%%%%%%%%%%%%%%%%%%%%%%%%
   
  \end{chapter}
  %%%%%%%%%%%%%%%%%%%%
  
  
  
  %%%%%%%%%%%%%%%%%%%%
  %%%%%%%%%%%%%%%%%%%%
  %%%%%%%%%%%%%%%%%%%%
  %%%%%%%%%%%%%%%%%%%%
%%%%%%%%%%%%%%%%%%%%
%%%%%%%%%%%%%%%%%%%%
\begin{appendix}
  \label{chpAnhang}
  
  %%%%%%%%%%%%%%%%%%%%%%%%%%%%%%
  %%%%%%%%%%%%%%%%%%%%%%%%%%%%%%
  %%%%%%%%%%%%%%%%%%%%%%%%%%%%%%
  \begin{chapter}{Daten Plots}
    \label{chpAnhangPlots}
    
    %%%%%%%%%%%%%%%%%%%%%%%%%%%%%%
    %%%%%%%%%%%%%%%%%%%%%%%%%%%%%%
    %%%%%%%%%%%%%%%%%%%%%%%%%%%%%%
    \begin{section}{Offset der Spannungen}
      \label{chpAnhangOffset}
      \begin{figure}[hb]
        \centering
        \includegraphics[width=\textwidth]{Figures/Offset.png}
        \caption{Messdaten \textbf{ohne} eingeschalteten Puls zur
          \textit{Offset} - Messung.}
        \label{AnhangfigOffset}
      \end{figure}
      
    \end{section}
    %%%%%%%%%%%%%%%%%%%%%%%%%%%%%
    
    \newpage
    %%%%%%%%%%%%%%%%%%%%%%%%%%%%%%
    %%%%%%%%%%%%%%%%%%%%%%%%%%%%%%
    %%%%%%%%%%%%%%%%%%%%%%%%%%%%%%
    \begin{section}{Free Induction Decay}
      \label{chpAnhangFID}
      \begin{figure}[htb!]
        \centering
        \begin{minipage}{.48\textwidth}
          \centering
          \includegraphics[width=\textwidth]{Figures/FID_env1.png}
          \caption{\textit{Free Induction Decay} Antwort - Signal und angepasste
            Zerfalls - Funktion zu beginn des Versuches.}
          \label{AnhangfigFIDenv1}
        \end{minipage}\quad
        \begin{minipage}{.48\textwidth}
          \centering
          \includegraphics[width=\textwidth]{Figures/FID_env2.png}
          \caption{\textit{Free Induction Decay} Antwort - Signal und angepasste
            Zerfalls - Funktion nach der erneuten Einstellung der
            Resonanzfrequenz.}
          \label{AnhangfigFIDenv2}
        \end{minipage}\\
        \begin{minipage}{\textwidth}
          \centering
          \includegraphics[width=\textwidth]{Figures/FID_env_Q_I0.png}
          \caption{\textit{Free Induction Decay} Antwort-, Q- und I - Signal.}
          \label{AnhangfigFIDenv3}
        \end{minipage}
      \end{figure}
      
    \end{section}
    %%%%%%%%%%%%%%%%%%%%%%%%%%%%%
    
    \newpage
    %%%%%%%%%%%%%%%%%%%%%%%%%%%%%%
    %%%%%%%%%%%%%%%%%%%%%%%%%%%%%%
    %%%%%%%%%%%%%%%%%%%%%%%%%%%%%%
    \begin{section}{Rabi--Oszillation}
      \label{chpAnhangRabi}
      \begin{figure}[htb]
        \centering
        \begin{minipage}{\textwidth}
          \centering
          \includegraphics[width=\textwidth]{Figures/Rabi_freq12.png}
          \caption{Rabi--Oszillationen des Antwort - und In--Phase - Signales
            beider Frequenzen.}
          \label{AnhangfigRabi12}
        \end{minipage}\\
        \begin{minipage}{.48\textwidth}
          \centering
          \includegraphics[width=\textwidth]{Figures/Rabi_freq1.png}
          \caption{Rabi--Oszillationen des Antwort - und In--Phase - Signales
            bei Resonanzfrequenz.}
          \label{AnhangfigRabi1}
        \end{minipage}\quad
        \begin{minipage}{.48\textwidth}
          \centering
          \includegraphics[width=\textwidth]{Figures/Rabi_freq2.png}
          \caption{Rabi--Oszillationen des Antwort - und In--Phase - Signales
            bei veränderter Frequenz.}
          \label{AnhangfigRabi2}
        \end{minipage}
      \end{figure}
      
    \end{section}
    %%%%%%%%%%%%%%%%%%%%%%%%%%%%%
    
    \newpage
    %%%%%%%%%%%%%%%%%%%%%%%%%%%%%%
    %%%%%%%%%%%%%%%%%%%%%%%%%%%%%%
    %%%%%%%%%%%%%%%%%%%%%%%%%%%%%%
    \begin{section}{Longitudinale Relaxationszeit}
      \label{chpAnhangLong}
      \begin{figure}[htb!]
        \centering
        \begin{minipage}{\textwidth}
          \centering
          \includegraphics[width=\textwidth]
          {Figures/SaettigungsZurueckgewinnung.png}
          \caption{Verlauf der maximalen Spannung des Antwort - Signales bei
            der Sättigungs--Zurückgewinnungs - Methode.}
          \label{AnhangfigSaettigung}
        \end{minipage}\\
        \begin{minipage}{\textwidth}
          \centering
          \includegraphics[width=\textwidth]
          {Figures/PolarisationsZurueckgewinnung.png}
          \caption{Verlauf der maximalen Spannung des Antwort - Signales bei
            der Polarisations--Zurückgewinnungs - Methode.}
          \label{AnhangfigPolarisation}
        \end{minipage}
      \end{figure}
      
    \end{section}
    %%%%%%%%%%%%%%%%%%%%%%%%%%%%%
    
    \newpage
    %%%%%%%%%%%%%%%%%%%%%%%%%%%%%%
    %%%%%%%%%%%%%%%%%%%%%%%%%%%%%%
    %%%%%%%%%%%%%%%%%%%%%%%%%%%%%%
    \begin{section}{Homogene Transversale Relaxationszeit}
      \label{chpAnhangTrans}
      
      %%%%%%%%%%%%%%%%%%%%%%%%%%%%%%%%%%%%%%%
      %%%%%%%%%%%%%%%%%%%%%%%%%%%%%%%%%%%%%%%
      %%%%%%%%%%%%%%%%%%%%%%%%%%%%%%%%%%%%%%%
      \begin{subsection}*{Hahn--Spinecho - Sequenz}
        \label{chpAnhangTransHahn}
        \begin{figure}[htb]
          \centering
          \includegraphics[width=\textwidth]
          {Figures/HomoTransRelax_Hahn.png}
          \caption{Verlauf der maximalen Spannung des Echo - Signales bei der
            Hahn--Spinecho - Sequenz.}
          \label{AnhangfigHahn}
        \end{figure}
        \begin{figure}[htb]
          \centering
          \begin{minipage}{.48\textwidth}
            \centering
            \includegraphics[width=\textwidth]
            {Figures/HomoTransRelax_Hahn_beispiel0.png}
            \caption{Beispiel eines Antwort - Signales bei der
              Hahn--Spinecho - Sequenz.}
            \label{AnhangfigHahnBsp1}
          \end{minipage}\quad
          \begin{minipage}{.48\textwidth}
            \centering
            \includegraphics[width=\textwidth]
            {Figures/HomoTransRelax_Hahn_beispiel1.png}
            \caption{Beispiel eines Antwort - Signales bei der
              Hahn--Spinecho - Sequenz.}
            \label{AnhangfigHahnBsp2}
          \end{minipage}
        \end{figure}
        
      \end{subsection}
      %%%%%%%%%%%%%%%%%%%%%%%%%%%%%%%%%%%%%%%
      
      \newpage
      %%%%%%%%%%%%%%%%%%%%%%%%%%%%%%%%%%%%%%%
      %%%%%%%%%%%%%%%%%%%%%%%%%%%%%%%%%%%%%%%
      %%%%%%%%%%%%%%%%%%%%%%%%%%%%%%%%%%%%%%%
      \begin{subsection}*{Carr--Purcell - Sequenz}
        \label{chpAnhangTransCarr}
        \begin{figure}[htb!]
          \centering
          \begin{minipage}{.48\textwidth}
            \centering
            \includegraphics[width=\textwidth]
            {Figures/HomoTransRelax_Carr0.png}
            \caption{Verzögerungszeit von
              $\tau = \SI{1.7}{\micro\second}$.}
            \label{AnhangfigCarr0}
          \end{minipage}\quad
          \begin{minipage}{.48\textwidth}
            \centering
            \includegraphics[width=\textwidth]
            {Figures/HomoTransRelax_Carr1.png}
            \caption{Verzögerungszeit von
              $\tau = \SI{2.0}{\micro\second}$.}
            \label{AnhangfigCarr1}
          \end{minipage}\\
          \begin{minipage}{.48\textwidth}
            \centering
            \includegraphics[width=\textwidth]
            {Figures/HomoTransRelax_Carr2.png}
            \caption{Verzögerungszeit von
              $\tau = \SI{2.6}{\micro\second}$.}
            \label{AnhangfigCarr2}
          \end{minipage}\\
          \begin{minipage}{.48\textwidth}
            \centering
            \includegraphics[width=\textwidth]
            {Figures/HomoTransRelax_Carr3.png}
            \caption{Verzögerungszeit von
              $\tau = \SI{2.6}{\micro\second}$ und umgepolten Z-Gradienten.}
            \label{AnhangfigCarr3}
          \end{minipage}\quad
          \begin{minipage}{.48\textwidth}
            \centering
            \includegraphics[width=\textwidth]
            {Figures/HomoTransRelax_Carr4.png}
            \caption{Verzögerungszeit von
              $\tau = \SI{2.6}{\micro\second}$ und umgepolten X-Gradienten.}
            \label{AnhangfigCarr4}
          \end{minipage}
          \caption{Verlauf der lokalen Maxima des Antwort - Signales für die
            Carr--Purcell - Sequenz.}
        \end{figure}
        
      \end{subsection}
      %%%%%%%%%%%%%%%%%%%%%%%%%%%%%%%%%%%%%%%
      
      \newpage
      %%%%%%%%%%%%%%%%%%%%%%%%%%%%%%%%%%%%%%%
      %%%%%%%%%%%%%%%%%%%%%%%%%%%%%%%%%%%%%%%
      %%%%%%%%%%%%%%%%%%%%%%%%%%%%%%%%%%%%%%%
      \begin{subsection}*{Meiboom--Gill - Sequenz}
        \label{chpAnhangTransMG}
        \begin{figure}[htb!]
          \centering
          \begin{minipage}{.48\textwidth}
            \centering
            \includegraphics[width=\textwidth]
            {Figures/HomoTransRelax_MG_env0.png}
            \caption{Verlauf der lokalen Maxima des Antwort - Signales für die
              Carr--Purcell - Sequenz bei einer Verzögerungszeit von
              $\tau = \SI{1.6}{\micro\second}$.}
            \label{AnhangfigMG_env0}
          \end{minipage}\quad
          \begin{minipage}{.48\textwidth}
            \centering
            \includegraphics[width=\textwidth]
            {Figures/HomoTransRelax_MG_env1.png}
            \caption{Verlauf der lokalen Maxima des Antwort - Signales für die
              Meiboom--Gill - Sequenz bei einer Verzögerungszeit von
              $\tau = \SI{1.6}{\micro\second}$.}
            \label{AnhangfigMG_env1}
          \end{minipage}
        \end{figure}
        \begin{figure}[htb!]
          \begin{minipage}{.48\textwidth}
            \centering
            \includegraphics[width=\textwidth]
            {Figures/HomoTransRelax_MG_env2.png}
            \caption{Verlauf der lokalen Maxima des Antwort - Signales für die
              Carr--Purcell - Sequenz bei einer Verzögerungszeit von
              $\tau = \SI{3.0}{\micro\second}$.}
            \label{AnhangfigMG_env2}
          \end{minipage}\quad
          \begin{minipage}{.48\textwidth}
            \centering
            \includegraphics[width=\textwidth]
            {Figures/HomoTransRelax_MG_env3.png}
            \caption{Verlauf der lokalen Maxima des Antwort - Signales für die
              Meiboom--Gill - Sequenz bei einer Verzögerungszeit von
              $\tau = \SI{3.0}{\micro\second}$.}
            \label{AnhangfigMG_env3}
          \end{minipage}
        \end{figure}
        \newpage
        \begin{figure}[htb!]
          \centering
          \begin{minipage}{.48\textwidth}
            \centering
            \includegraphics[width=\textwidth]
            {Figures/HomoTransRelax_MG_env_Q_I0.png}
            \caption{Verlauf der Spannungen des Antwort-, Q- und
              In--Phase - Signales bei der Carr--Purcell - Sequenz
              bei einer Verzögerungszeit von $\tau = \SI{1.6}{\micro\second}$.}
            \label{AnhangfigMG_envQI0}
          \end{minipage}\quad
          \begin{minipage}{.48\textwidth}
            \centering
            \includegraphics[width=\textwidth]
            {Figures/HomoTransRelax_MG_env_Q_I1.png}
            \caption{Verlauf der Spannungen des Antwort-, Q- und
              In--Phase - Signales bei der Meiboom--Gill - Sequenz
              bei einer Verzögerungszeit von $\tau = \SI{1.6}{\micro\second}$.}
            \label{AnhangfigMG_envQI1}
          \end{minipage}
        \end{figure}
        \begin{figure}[htb!]
          \begin{minipage}{.48\textwidth}
            \centering
            \includegraphics[width=\textwidth]
            {Figures/HomoTransRelax_MG_env_Q_I2.png}
            \caption{Verlauf der Spannungen des Antwort-, Q- und
              In--Phase - Signales bei der Carr--Purcell - Sequenz
              bei einer Verzögerungszeit von $\tau = \SI{3.0}{\micro\second}$.}
            \label{AnhangfigMG_envQI2}
          \end{minipage}\quad
          \begin{minipage}{.48\textwidth}
            \centering
            \includegraphics[width=\textwidth]
            {Figures/HomoTransRelax_MG_env_Q_I3.png}
            \caption{Verlauf der Spannungen des Antwort-, Q- und
              In--Phase - Signales bei der Meiboom--Gill - Sequenz
              bei einer Verzögerungszeit von $\tau = \SI{3.0}{\micro\second}$.}
            \label{AnhangfigMG_envQI3}
          \end{minipage}
        \end{figure}
        
      \end{subsection}
      %%%%%%%%%%%%%%%%%%%%%%%%%%%%%%%%%%%%%%%
      
    \end{section}
    %%%%%%%%%%%%%%%%%%%%%%%%%%%%%
    
  \end{chapter}
  %%%%%%%%%%%%%%%%%%%%%%%%%%%%%%
  
\end{appendix}
%%%%%%%%%%%%%%%%%%%%
 
  %%%%%%%%%%%%%%%%%%%%
  
  
  
  %%%%%%%%%%%%%%%%%%%%
  %%%%%%%%%%%%%%%%%%%%
  %%%%%%%%%%%%%%%%%%%%
  \begin{thebibliography}{99}
    \scriptsize
    \bibitem{bib:Anleitung}\url{http://www.praktika.physik.uni-bonn.de/module/physik412/downloads/p441d}
\bibitem{bib:Theorieteil}\textit{Weißlichtspektroskopie an metallischen
Nanostrukturen - Erstellung eines Versuchs für ein Praktikum im Rahmen des
Physikstudiums an der Universität Bonn} von Roberto Röll 2013
\bibitem{bib:Wiki}\url{http://en.wikipedia.org/wiki/Plasmon}



  \end{thebibliography}
  %%%%%%%%%%%%%%%%%%%%
 
\end{document}
%%%%%%%%%%
