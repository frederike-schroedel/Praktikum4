%%%%%%%%%%%%%%%%%%%%
%%%%%%%%%%%%%%%%%%%%
%%%%%%%%%%%%%%%%%%%%
\begin{appendix}
  \label{Anhang}
  
  
  
  %%%%%%%%%%%%%%%%%%%%%%%%%%%%%%
  %%%%%%%%%%%%%%%%%%%%%%%%%%%%%%
  %%%%%%%%%%%%%%%%%%%%%%%%%%%%%%
  \begin{chapter}{ERSTER TEIL - Zeeman-Effekt}
    \label{Anhang:chp:Zeeman}
    
    
    
    %%%%%%%%%%%%%%%%%%%%%%%%%%%%%%%%%%%%%%%%
    %%%%%%%%%%%%%%%%%%%%%%%%%%%%%%%%%%%%%%%%
    %%%%%%%%%%%%%%%%%%%%%%%%%%%%%%%%%%%%%%%%
    \begin{section}{Magnetfeldkalibrierung}
      \label{Anhang:chp:Zeemanmagnetfeld}
      
      \begin{table}[htbp]
        \centering
        \footnotesize
        \begin{tabular}{|c|c|c|c|}
          \hline
          $I_{A}^{1}/A$ & $B_{B}^{2}/mT$ & $I_{A}^{2}/A$ & $B_{B}^{2}/mT$ 
              \\ \hline
          % & "Mfeld & Kalib" & 1 & "Mfeld & Kalib" & 2 \\ \hline
% & "I_A1 & / & A" & "B_B1 & / & mT" & "I_A1 & / & A" & "B_B1 & / & mT" \\ \hline
0 & 0 & 0 & 1 \\ \hline
0.33 & 27 & 0.72 & 60 \\ \hline
0.84 & 67 & 0.71 & 61 \\ \hline
1.26 & 107 & 1.86 & 167 \\ \hline
1.92 & 164 & 1.61 & 144 \\ \hline
2.19 & 189 & 2.22 & 199 \\ \hline
2.75 & 237 & 2.25 & 200 \\ \hline
2.78 & 239 & 2.45 & 219 \\ \hline
3.06 & 265 & 3.02 & 269 \\ \hline
3.44 & 298 & 3.11 & 279 \\ \hline
3.8 & 328 & 3.59 & 321 \\ \hline
4.11 & 355 & 3.89 & 350 \\ \hline
4.14 & 358 & 4.05 & 364 \\ \hline
4.38 & 378 & 4.53 & 406 \\ \hline
4.64 & 398 & 4.97 & 441 \\ \hline
4.85 & 416 & 5.12 & 453 \\ \hline
4.94 & 424 & 5.33 & 468 \\ \hline
5.22 & 444 & 5.73 & 494 \\ \hline
5.45 & 459 & 5.97 & 510 \\ \hline
5.64 & 473 & 6.47 & 534 \\ \hline
5.72 & 476 & 6.93 & 554 \\ \hline
5.91 & 488 & 7.43 & 572 \\ \hline
6.11 & 498 & 7.8 & 584 \\ \hline
6.24 & 505 & 8.1 & 594 \\ \hline
6.41 & 513 & 8.19 & 597 \\ \hline
6.72 & 527 & 8.45 & 605 \\ \hline
6.92 & 535 & 8.78 & 614 \\ \hline
7.16 & 544 & 8.9 & 618 \\ \hline
7.37 & 551 & 8.91 & 617 \\ \hline
7.68 & 562 & 8.91 & 617 \\ \hline
8.03 & 573 & 8.91 & 617 \\ \hline
8.31 & 581 & 8.91 & 617 \\ \hline
8.79 & 594 & 8.91 & 617 \\ \hline
9.27 & 607 & 8.91 & 617 \\ \hline
9.71 & 617 & 8.91 & 617 \\ \hline
10.17 & 626 & 8.91 & 617 \\ \hline

        \end{tabular}
        \caption{Messdaten der ersten und zweiten Messung der 
            Magnetfeldkalibrierung.}
        \label{tab:Magnetfeldkalibrierung}
      \end{table}
      
    \end{section}
    %%%%%%%%%%%%%%%%%%%%%%%%%%%%%%%%%%%%%%%%
    
    
    
    %%%%%%%%%%%%%%%%%%%%%%%%%%%%%%%%%%%%%%%%
    %%%%%%%%%%%%%%%%%%%%%%%%%%%%%%%%%%%%%%%%
    %%%%%%%%%%%%%%%%%%%%%%%%%%%%%%%%%%%%%%%%
    \begin{section}{CCD Kamera}
      \label{Anhang:chp:Zeemanccd}
      
      \begin{figure}[htbp!]
        \centering
        \begin{minipage}{0.48\textwidth}
          \centering
          \includegraphics[width=\textwidth]
        {Figures/Versuch401-FPEAusschnitt-0_0AMagnetstrom_Pixel_Intensitaet.png}
          \caption{Angepasste Gauss-Funktion für einen Magnetstrom von 
              $\SI{0.0}{\ampere}$.}
          \label{fig:AnhangZM00}
        \end{minipage} \quad
        \begin{minipage}{0.48\textwidth}
          \centering
          \includegraphics[width=\textwidth]
        {Figures/Versuch401-FPEAusschnitt-1_9AMagnetstrom_Pixel_Intensitaet.png}
          \caption{Angepasste Gauss-Funktion für einen Magnetstrom von 
              $\SI{1.9}{\ampere}$.}
          \label{fig:AnhangZM19}
        \end{minipage}
      \end{figure}
      \begin{figure}[htbp!]
        \begin{minipage}{0.48\textwidth}
          \centering
          \includegraphics[width=\textwidth]
        {Figures/Versuch401-FPEAusschnitt-3_0AMagnetstrom_Pixel_Intensitaet.png}
          \caption{Angepasste Gauss-Funktion für einen Magnetstrom von 
              $\SI{3.0}{\ampere}$.}
          \label{fig:AnhangZM30}
        \end{minipage} \quad
        \begin{minipage}{0.48\textwidth}
          \centering
          \includegraphics[width=\textwidth]
        {Figures/Versuch401-FPEAusschnitt-3_8AMagnetstrom_Pixel_Intensitaet.png}
          \caption{Angepasste Gauss-Funktion für einen Magnetstrom von 
              $\SI{3.8}{\ampere}$.}
          \label{fig:AnhangZM38}
        \end{minipage}
      \end{figure}
      \begin{figure}[htbp!]
        \begin{minipage}{0.48\textwidth}
          \centering
          \includegraphics[width=\textwidth]
        {Figures/Versuch401-FPEAusschnitt-4_4AMagnetstrom_Pixel_Intensitaet.png}
          \caption{Angepasste Gauss-Funktion für einen Magnetstrom von 
              $\SI{4.4}{\ampere}$.}
          \label{fig:AnhangZM44}
        \end{minipage}\quad
        \begin{minipage}{0.48\textwidth}
          \centering
          \includegraphics[width=\textwidth]
        {Figures/Versuch401-FPEAusschnitt-5_1AMagnetstrom_Pixel_Intensitaet.png}
          \caption{Angepasste Gauss-Funktion für einen Magnetstrom von 
              $\SI{5.1}{\ampere}$.}
          \label{fig:AnhangZM51}
        \end{minipage}
      \end{figure}
      \newpage
      \begin{figure}[htbp!]
        \centering
        \begin{minipage}{0.48\textwidth}
          \centering
          \includegraphics[width=\textwidth]
        {Figures/Versuch401-FPEAusschnitt-6_1AMagnetstrom_Pixel_Intensitaet.png}
          \caption{Angepasste Gauss-Funktion für einen Magnetstrom von 
              $\SI{6.1}{\ampere}$.}
          \label{fig:AnhangZM61}
        \end{minipage} \quad
        \begin{minipage}{0.48\textwidth}
          \centering
          \includegraphics[width=\textwidth]
        {Figures/Versuch401-FPEAusschnitt-6_8AMagnetstrom_Pixel_Intensitaet.png}
          \caption{Angepasste Gauss-Funktion für einen Magnetstrom von 
              $\SI{6.8}{\ampere}$.}
          \label{fig:AnhangZM68}
        \end{minipage} \\
        \begin{minipage}{0.48\textwidth}
          \centering
          \includegraphics[width=\textwidth]
        {Figures/Versuch401-FPEAusschnitt-7_5AMagnetstrom_Pixel_Intensitaet.png}
          \caption{Angepasste Gauss-Funktion für einen Magnetstrom von 
              $\SI{7.5}{\ampere}$.}
          \label{fig:AnhangZM75}
        \end{minipage} \quad
        \begin{minipage}{0.48\textwidth}
          \centering
          \includegraphics[width=\textwidth]
        {Figures/Versuch401-FPEAusschnitt-8_0AMagnetstrom_Pixel_Intensitaet.png}
          \caption{Angepasste Gauss-Funktion für einen Magnetstrom von 
              $\SI{8.0}{\ampere}$.}
          \label{fig:AnhangZM80}
        \end{minipage} \\
        \begin{minipage}{0.48\textwidth}
          \centering
          \includegraphics[width=\textwidth]
        {Figures/Versuch401-FPEAusschnitt-8_7AMagnetstrom_Pixel_Intensitaet.png}
          \caption{Angepasste Gauss-Funktion für einen Magnetstrom von 
              $\SI{8.7}{\ampere}$.}
          \label{fig:AnhangZM87}
        \end{minipage}
      \end{figure}
      
      \begin{scriptsize}
        \begin{longtable}[htbp]{|c|c|c|c|c|c|c|c|c|c|c|c|c|}
          \hline
          \multicolumn{2}{|c|}{Position} &\multicolumn{11}{|c|}{Magnetstrom/A}\\
          \hline
          Pixel & Winkel & $0.0$ & $1.9$ & $3.0$ & $3.8$ & $4.4$ & $5.1$ & 
              $6.1$ & $6.8$ & $7.5$ & $8.0$ & $8.7$ \\ \hline\hline \endhead
          % & Ampere & 0 & 1.9 & 3 & 3.8 & 4.4 & 5.1 & 6.1 & 6.8 & 7.5 & 8 & 8.7 \\ \hline
% & Pixel & "&a & / & °" & "I1 & / & %" \\ \hline
0 & 5.459 & 0 & 0.1 & 0.1 & 0.4 & 0.4 & 0.4 & 0.5 & 0.6 & 0.8 & 0.7 & 0.7 \\ \hline
1 & 5.454 & 0 & 0.1 & 0.1 & 0.4 & 0.4 & 0.4 & 0.5 & 0.6 & 0.8 & 0.7 & 0.7 \\ \hline
2 & 5.449 & 0 & 0.1 & 0.1 & 0.4 & 0.4 & 0.4 & 0.5 & 0.6 & 0.8 & 0.7 & 0.7 \\ \hline
3 & 5.443 & 0 & 0.1 & 0.1 & 0.4 & 0.4 & 0.4 & 0.5 & 0.6 & 0.8 & 0.7 & 0.7 \\ \hline
4 & 5.438 & 0 & 0.1 & 0.1 & 0.4 & 0.4 & 0.4 & 0.5 & 0.6 & 0.8 & 0.7 & 0.7 \\ \hline
5 & 5.433 & 0 & 0.1 & 0.1 & 0.4 & 0.4 & 0.4 & 0.5 & 0.6 & 0.8 & 0.7 & 0.7 \\ \hline
6 & 5.428 & 0 & 0.1 & 0.1 & 0.4 & 0.4 & 0.4 & 0.5 & 0.6 & 0.8 & 0.7 & 0.7 \\ \hline
7 & 5.422 & 0 & 0.1 & 0.1 & 0.4 & 0.4 & 0.4 & 0.5 & 0.6 & 0.8 & 0.7 & 0.7 \\ \hline
8 & 5.417 & 0 & 0.1 & 0.1 & 0.4 & 0.4 & 0.4 & 0.5 & 0.6 & 0.8 & 0.7 & 0.7 \\ \hline
9 & 5.412 & 0 & 0.1 & 0.1 & 0.4 & 0.4 & 0.4 & 0.5 & 0.6 & 0.8 & 0.7 & 0.7 \\ \hline
10 & 5.406 & 0 & 0.1 & 0.1 & 0.4 & 0.4 & 0.4 & 0.5 & 0.6 & 0.8 & 0.7 & 0.7 \\ \hline
11 & 5.401 & 0 & 0.1 & 0.1 & 0.4 & 0.4 & 0.4 & 0.5 & 0.6 & 0.8 & 0.7 & 0.7 \\ \hline
12 & 5.396 & 0 & 0.1 & 0.1 & 0.4 & 0.4 & 0.4 & 0.5 & 0.6 & 0.8 & 0.7 & 0.7 \\ \hline
13 & 5.39 & 0 & 0.1 & 0.1 & 0.4 & 0.4 & 0.4 & 0.5 & 0.6 & 0.8 & 0.7 & 0.7 \\ \hline
14 & 5.385 & 0 & 0.1 & 0.1 & 0.4 & 0.4 & 0.4 & 0.5 & 0.6 & 0.8 & 0.7 & 0.7 \\ \hline
15 & 5.38 & 0 & 0.1 & 0.1 & 0.4 & 0.4 & 0.4 & 0.5 & 0.6 & 0.8 & 0.7 & 0.7 \\ \hline
16 & 5.375 & 0 & 0.1 & 0.1 & 0.4 & 0.4 & 0.4 & 0.5 & 0.6 & 0.8 & 0.7 & 0.7 \\ \hline
17 & 5.369 & 0 & 0.1 & 0.1 & 0.4 & 0.4 & 0.4 & 0.5 & 0.6 & 0.8 & 0.7 & 0.7 \\ \hline
18 & 5.364 & 0 & 0.1 & 0.1 & 0.4 & 0.4 & 0.4 & 0.5 & 0.6 & 0.8 & 0.7 & 0.7 \\ \hline
19 & 5.359 & 0 & 0.1 & 0.1 & 0.4 & 0.4 & 0.4 & 0.5 & 0.6 & 0.8 & 0.7 & 0.7 \\ \hline
20 & 5.353 & 0 & 0.1 & 0.1 & 0.4 & 0.4 & 0.4 & 0.5 & 0.6 & 0.8 & 0.7 & 0.7 \\ \hline
21 & 5.348 & 0 & 0.1 & 0.1 & 0.4 & 0.4 & 0.4 & 0.5 & 0.6 & 0.8 & 0.7 & 0.7 \\ \hline
22 & 5.343 & 0 & 0.1 & 0.1 & 0.4 & 0.4 & 0.4 & 0.5 & 0.6 & 0.8 & 0.7 & 0.7 \\ \hline
23 & 5.337 & 0 & 0.1 & 0.1 & 0.4 & 0.4 & 0.4 & 0.5 & 0.6 & 0.8 & 0.7 & 0.7 \\ \hline
24 & 5.332 & 0 & 0.1 & 0.1 & 0.4 & 0.4 & 0.4 & 0.5 & 0.6 & 0.8 & 0.7 & 0.7 \\ \hline
25 & 5.327 & 0 & 0.1 & 0.1 & 0.4 & 0.4 & 0.4 & 0.5 & 0.7 & 0.8 & 0.7 & 0.7 \\ \hline
26 & 5.322 & 0 & 0.1 & 0.1 & 0.4 & 0.4 & 0.4 & 0.5 & 0.7 & 0.8 & 0.7 & 0.7 \\ \hline
27 & 5.316 & 0 & 0.1 & 0.1 & 0.4 & 0.4 & 0.4 & 0.6 & 0.8 & 0.8 & 0.8 & 0.8 \\ \hline
28 & 5.311 & 0 & 0.1 & 0.1 & 0.4 & 0.4 & 0.4 & 0.6 & 0.8 & 0.8 & 0.8 & 0.8 \\ \hline
29 & 5.306 & 0 & 0.1 & 0.1 & 0.4 & 0.4 & 0.4 & 0.7 & 0.8 & 0.8 & 0.8 & 1 \\ \hline
30 & 5.3 & 0 & 0.1 & 0.1 & 0.4 & 0.4 & 0.4 & 0.7 & 0.8 & 0.8 & 0.8 & 1 \\ \hline
31 & 5.295 & 0 & 0.1 & 0.1 & 0.4 & 0.4 & 0.4 & 0.8 & 0.8 & 1 & 1 & 1.1 \\ \hline
32 & 5.29 & 0 & 0.1 & 0.1 & 0.4 & 0.4 & 0.4 & 0.8 & 0.8 & 1 & 1.1 & 1.1 \\ \hline
33 & 5.284 & 0 & 0.1 & 0.1 & 0.4 & 0.4 & 0.5 & 0.8 & 0.8 & 1.1 & 1.1 & 1.1 \\ \hline
34 & 5.279 & 0 & 0.1 & 0.1 & 0.4 & 0.4 & 0.5 & 0.8 & 0.8 & 1.2 & 1.1 & 1.1 \\ \hline
35 & 5.274 & 0 & 0.1 & 0.1 & 0.4 & 0.4 & 0.6 & 0.9 & 1.1 & 1.2 & 1.1 & 1.2 \\ \hline
36 & 5.269 & 0 & 0.1 & 0.1 & 0.4 & 0.4 & 0.6 & 0.9 & 1.1 & 1.2 & 1.1 & 1.2 \\ \hline
37 & 5.263 & 0 & 0.1 & 0.2 & 0.4 & 0.4 & 0.8 & 1.1 & 1.2 & 1.2 & 1.3 & 1.4 \\ \hline
38 & 5.258 & 0 & 0.1 & 0.2 & 0.4 & 0.4 & 0.8 & 1.2 & 1.2 & 1.2 & 1.3 & 1.4 \\ \hline
39 & 5.253 & 0 & 0.1 & 0.4 & 0.6 & 0.5 & 1 & 1.2 & 1.3 & 1.5 & 1.5 & 1.5 \\ \hline
40 & 5.247 & 0 & 0.2 & 0.5 & 0.7 & 0.5 & 1 & 1.2 & 1.3 & 1.5 & 1.5 & 1.5 \\ \hline
41 & 5.242 & 0 & 0.4 & 0.5 & 0.9 & 0.7 & 1.2 & 1.3 & 1.5 & 1.6 & 1.5 & 1.5 \\ \hline
42 & 5.237 & 0 & 0.4 & 0.5 & 0.9 & 0.7 & 1.2 & 1.4 & 1.6 & 1.6 & 1.5 & 1.5 \\ \hline
43 & 5.231 & 0 & 0.5 & 0.6 & 1 & 0.8 & 1.2 & 1.6 & 1.6 & 1.6 & 1.5 & 1.5 \\ \hline
44 & 5.226 & 0 & 0.5 & 0.6 & 1 & 0.8 & 1.2 & 1.6 & 1.6 & 1.6 & 1.5 & 1.5 \\ \hline
45 & 5.221 & 0 & 0.5 & 0.6 & 1 & 0.8 & 1.2 & 1.6 & 1.6 & 1.6 & 1.7 & 1.7 \\ \hline
46 & 5.216 & 0 & 0.5 & 0.5 & 0.9 & 0.8 & 1.2 & 1.6 & 1.6 & 1.6 & 1.7 & 1.7 \\ \hline
47 & 5.21 & 0 & 0.5 & 0.5 & 0.8 & 0.8 & 1.2 & 1.6 & 1.6 & 1.6 & 1.7 & 1.9 \\ \hline
48 & 5.205 & 0 & 0.5 & 0.5 & 0.8 & 0.8 & 1.2 & 1.6 & 1.6 & 1.6 & 1.7 & 1.9 \\ \hline
49 & 5.2 & 0 & 0.5 & 0.5 & 0.8 & 0.8 & 1.2 & 1.6 & 1.6 & 1.8 & 1.9 & 1.9 \\ \hline
50 & 5.194 & 0 & 0.5 & 0.5 & 0.8 & 0.8 & 1.2 & 1.6 & 1.6 & 1.9 & 1.9 & 1.9 \\ \hline
51 & 5.189 & 0 & 0.5 & 0.6 & 1 & 0.9 & 1.4 & 1.6 & 1.7 & 2 & 1.9 & 2 \\ \hline
52 & 5.184 & 0 & 0.5 & 0.8 & 1.2 & 1.1 & 1.6 & 1.6 & 1.9 & 2 & 1.9 & 2 \\ \hline
53 & 5.178 & 0.2 & 0.9 & 1.1 & 1.5 & 1.2 & 1.6 & 2 & 2 & 2.2 & 2.2 & 2.3 \\ \hline
54 & 5.173 & 0.4 & 1.1 & 1.3 & 1.6 & 1.5 & 1.8 & 2 & 2.3 & 2.4 & 2.3 & 2.4 \\ \hline
55 & 5.168 & 0.8 & 1.3 & 1.4 & 1.7 & 1.6 & 2 & 2.4 & 2.5 & 2.7 & 2.7 & 2.7 \\ \hline
56 & 5.162 & 0.9 & 1.3 & 1.5 & 2 & 1.6 & 2.1 & 2.4 & 2.6 & 2.8 & 2.7 & 2.8 \\ \hline
57 & 5.157 & 0.9 & 1.3 & 1.6 & 2 & 1.6 & 2.1 & 2.4 & 2.7 & 2.8 & 2.7 & 2.8 \\ \hline
58 & 5.152 & 0.8 & 1.3 & 1.6 & 2 & 1.6 & 2.1 & 2.4 & 2.7 & 2.8 & 2.7 & 2.8 \\ \hline
59 & 5.147 & 0.8 & 1.3 & 1.5 & 2 & 1.6 & 2.1 & 2.4 & 2.7 & 2.8 & 2.7 & 2.8 \\ \hline
60 & 5.141 & 0.5 & 1 & 1.3 & 1.9 & 1.6 & 2.1 & 2.4 & 2.8 & 2.8 & 3 & 3.1 \\ \hline
61 & 5.136 & 0.4 & 0.9 & 1.3 & 1.8 & 1.6 & 2.1 & 2.4 & 2.8 & 2.9 & 3.1 & 3.1 \\ \hline
62 & 5.131 & 0.4 & 0.7 & 1.1 & 1.6 & 1.6 & 2 & 2.4 & 2.8 & 2.9 & 3.1 & 3.1 \\ \hline
63 & 5.125 & 0.4 & 0.7 & 1.1 & 1.6 & 1.6 & 2 & 2.4 & 2.8 & 3 & 3.1 & 3.1 \\ \hline
64 & 5.12 & 0.4 & 0.7 & 1.1 & 1.6 & 1.6 & 2 & 2.4 & 2.8 & 3.1 & 3.1 & 3.2 \\ \hline
65 & 5.115 & 0.4 & 0.9 & 1.2 & 1.6 & 1.6 & 2.3 & 2.8 & 3.2 & 3.3 & 3.4 & 3.5 \\ \hline
66 & 5.109 & 0.4 & 1 & 1.3 & 1.9 & 1.8 & 2.5 & 3.1 & 3.2 & 3.4 & 3.5 & 3.5 \\ \hline
67 & 5.104 & 0.4 & 1.3 & 1.7 & 2.3 & 2 & 2.8 & 3.2 & 3.2 & 3.4 & 3.5 & 3.5 \\ \hline
68 & 5.099 & 0.6 & 1.7 & 2 & 2.4 & 2.1 & 2.8 & 3.2 & 3.2 & 3.4 & 3.5 & 3.5 \\ \hline
69 & 5.094 & 1.2 & 2.1 & 2.4 & 2.8 & 2.4 & 3.1 & 3.6 & 3.6 & 3.8 & 3.9 & 3.9 \\ \hline
70 & 5.088 & 1.6 & 2.5 & 2.8 & 3.2 & 2.8 & 3.4 & 4 & 4 & 4.2 & 4.3 & 4.3 \\ \hline
71 & 5.083 & 2 & 2.7 & 2.9 & 3.2 & 2.9 & 3.5 & 4 & 4.1 & 4.4 & 4.3 & 4.3 \\ \hline
72 & 5.078 & 1.9 & 2.5 & 2.9 & 3.2 & 2.8 & 3.5 & 4 & 4.1 & 4.4 & 4.3 & 4.3 \\ \hline
73 & 5.072 & 1.6 & 2.1 & 2.6 & 3.2 & 2.8 & 3.5 & 4 & 4.1 & 4.4 & 4.3 & 4.3 \\ \hline
74 & 5.067 & 1.2 & 1.7 & 2.3 & 3.2 & 2.8 & 3.5 & 4 & 4.4 & 4.4 & 4.3 & 4.6 \\ \hline
75 & 5.062 & 0.8 & 1.7 & 2.1 & 2.8 & 2.8 & 3.5 & 4 & 4.4 & 4.7 & 4.7 & 5 \\ \hline
76 & 5.056 & 0.8 & 1.3 & 1.7 & 2.4 & 2.4 & 3.1 & 3.7 & 4.3 & 4.6 & 4.7 & 4.9 \\ \hline
77 & 5.051 & 0.8 & 1.3 & 1.7 & 2 & 2.1 & 2.8 & 3.6 & 4 & 4.4 & 4.6 & 4.7 \\ \hline
78 & 5.046 & 0.5 & 1.3 & 1.7 & 2 & 2 & 2.8 & 3.6 & 4 & 4.4 & 4.6 & 4.7 \\ \hline
79 & 5.04 & 0.5 & 1.3 & 1.7 & 2 & 2 & 3 & 3.9 & 4.3 & 4.7 & 4.9 & 5.1 \\ \hline
80 & 5.035 & 0.5 & 1.3 & 1.8 & 2.4 & 2.4 & 3.4 & 4.3 & 4.7 & 5.1 & 5.1 & 5.1 \\ \hline
81 & 5.03 & 0.6 & 1.7 & 2.3 & 2.9 & 2.9 & 3.8 & 4.6 & 5 & 5.2 & 5.1 & 5.1 \\ \hline
82 & 5.025 & 0.9 & 2.4 & 2.9 & 3.5 & 3.3 & 4.1 & 4.7 & 5 & 5.2 & 5.1 & 5.1 \\ \hline
83 & 5.019 & 1.6 & 3.2 & 3.7 & 4 & 3.7 & 4.6 & 5.1 & 5.2 & 5.5 & 5.5 & 5.5 \\ \hline
84 & 5.014 & 2.4 & 4 & 4.3 & 4.8 & 4.2 & 5 & 5.6 & 6 & 6.2 & 6.3 & 6.3 \\ \hline
85 & 5.009 & 3.2 & 4.5 & 4.7 & 5.2 & 4.6 & 5.3 & 6 & 6.3 & 6.6 & 6.7 & 6.7 \\ \hline
86 & 5.003 & 3.2 & 4.4 & 4.6 & 5.2 & 4.5 & 5.3 & 5.6 & 6.1 & 6.4 & 6.3 & 6.3 \\ \hline
87 & 4.998 & 2.8 & 4.1 & 4.5 & 5.2 & 4.5 & 5.3 & 5.6 & 5.7 & 6 & 5.9 & 5.9 \\ \hline
88 & 4.993 & 2.4 & 3.5 & 4.1 & 4.8 & 4.5 & 5.3 & 5.6 & 5.8 & 6 & 5.9 & 5.9 \\ \hline
89 & 4.987 & 2 & 3 & 3.7 & 4.5 & 4.4 & 5.3 & 6 & 6.2 & 6.4 & 6.3 & 6.3 \\ \hline
90 & 4.982 & 1.6 & 2.5 & 2.9 & 4 & 4 & 4.9 & 5.6 & 6.1 & 6.4 & 6.4 & 6.6 \\ \hline
91 & 4.977 & 1.2 & 2.1 & 2.5 & 3.6 & 3.6 & 4.4 & 5.2 & 5.7 & 6.3 & 6.3 & 6.6 \\ \hline
92 & 4.971 & 0.8 & 1.7 & 2.1 & 3.2 & 3.2 & 4 & 4.8 & 5.4 & 5.9 & 6.1 & 6.3 \\ \hline
93 & 4.966 & 0.8 & 1.7 & 2.1 & 2.8 & 2.9 & 3.8 & 4.8 & 5.4 & 5.9 & 6.2 & 6.6 \\ \hline
94 & 4.961 & 0.8 & 1.7 & 2.1 & 2.8 & 2.9 & 4 & 5.2 & 5.8 & 6.3 & 6.6 & 6.7 \\ \hline
95 & 4.956 & 0.8 & 2 & 2.5 & 3.2 & 3.6 & 4.7 & 5.6 & 6.2 & 6.7 & 6.7 & 6.7 \\ \hline
96 & 4.95 & 1 & 2.5 & 3.3 & 4 & 4.1 & 5.1 & 6 & 6.4 & 6.7 & 6.7 & 6.7 \\ \hline
97 & 4.945 & 1.4 & 3.4 & 4.1 & 4.8 & 4.8 & 5.5 & 6.3 & 6.5 & 6.8 & 6.7 & 6.7 \\ \hline
98 & 4.94 & 2.2 & 4.5 & 5 & 5.6 & 5.3 & 6.1 & 6.8 & 7.1 & 7.3 & 7.5 & 7.5 \\ \hline
99 & 4.934 & 3.6 & 5.7 & 5.8 & 6.4 & 6.1 & 7 & 7.6 & 8.2 & 8.4 & 8.6 & 8.7 \\ \hline
100 & 4.929 & 4.4 & 6.4 & 6.5 & 7.1 & 6.5 & 7.5 & 8.3 & 8.7 & 9.1 & 9.1 & 9.1 \\ \hline
101 & 4.924 & 4.8 & 6.5 & 6.9 & 7.4 & 6.8 & 7.6 & 8.3 & 8.7 & 9.1 & 9.1 & 9.1 \\ \hline
102 & 4.918 & 4.4 & 5.8 & 6.6 & 7.3 & 6.8 & 7.5 & 8 & 8.4 & 8.6 & 8.7 & 8.7 \\ \hline
103 & 4.913 & 3.6 & 5 & 5.8 & 7 & 6.8 & 7.5 & 8 & 8.4 & 8.4 & 8.5 & 8.4 \\ \hline
104 & 4.908 & 2.8 & 4.2 & 5 & 6.2 & 6 & 7.1 & 8 & 8.4 & 8.5 & 8.6 & 8.7 \\ \hline
105 & 4.902 & 2.3 & 3.4 & 4.2 & 5.4 & 5.3 & 6.5 & 7.6 & 8 & 8.4 & 8.6 & 8.7 \\ \hline
106 & 4.897 & 1.7 & 2.7 & 3.4 & 4.5 & 4.5 & 5.7 & 6.8 & 7.3 & 7.6 & 8.1 & 8.3 \\ \hline
107 & 4.892 & 1.3 & 2.3 & 2.9 & 4 & 4 & 5.1 & 6 & 6.8 & 7.2 & 7.5 & 7.9 \\ \hline
108 & 4.887 & 1 & 2.1 & 2.6 & 3.6 & 3.6 & 4.7 & 5.6 & 6.8 & 7.2 & 7.6 & 8 \\ \hline
109 & 4.881 & 1 & 2.1 & 2.9 & 3.6 & 4 & 5.1 & 6.4 & 7.5 & 8 & 8.3 & 8.8 \\ \hline
110 & 4.876 & 1.1 & 2.5 & 3.3 & 4.2 & 4.8 & 5.9 & 7.2 & 8 & 8.4 & 8.7 & 9.1 \\ \hline
111 & 4.871 & 1.5 & 3.3 & 4.1 & 5.2 & 5.6 & 6.7 & 7.6 & 8.4 & 8.7 & 8.8 & 9.1 \\ \hline
112 & 4.865 & 1.9 & 4.3 & 5.3 & 6 & 6.2 & 7.1 & 8 & 8.4 & 8.7 & 8.8 & 9.1 \\ \hline
113 & 4.86 & 3.1 & 5.9 & 6.6 & 7.2 & 7 & 7.9 & 8.8 & 9.2 & 9.5 & 9.6 & 9.9 \\ \hline
114 & 4.855 & 4.8 & 7.5 & 8.1 & 8.4 & 8.1 & 9.2 & 10.1 & 10.4 & 10.8 & 11.1 & 11.4 \\ \hline
115 & 4.849 & 6.2 & 8.9 & 9.1 & 9.6 & 8.9 & 10.2 & 10.9 & 11.5 & 12 & 12.2 & 12.3 \\ \hline
116 & 4.844 & 6.4 & 9 & 9.4 & 9.8 & 9 & 10.2 & 10.9 & 11.5 & 11.9 & 11.9 & 11.9 \\ \hline
117 & 4.839 & 6 & 8.5 & 9 & 9.8 & 9 & 9.9 & 10.5 & 11 & 11.2 & 11.2 & 11.2 \\ \hline
118 & 4.833 & 5.2 & 7.3 & 8.2 & 9.4 & 8.8 & 9.9 & 10.4 & 10.8 & 10.8 & 10.8 & 10.8 \\ \hline
119 & 4.828 & 4.4 & 6.1 & 7.1 & 8.6 & 8.4 & 9.7 & 10.4 & 10.9 & 11.2 & 11.2 & 11.2 \\ \hline
120 & 4.823 & 3.3 & 5 & 6.1 & 7.4 & 7.6 & 8.9 & 10 & 10.8 & 11.2 & 11.3 & 11.5 \\ \hline
121 & 4.818 & 2.6 & 4.2 & 5.2 & 6.4 & 6.4 & 7.9 & 9.2 & 10 & 10.7 & 10.9 & 11.1 \\ \hline
122 & 4.812 & 2.1 & 3.4 & 4.4 & 5.6 & 5.6 & 7.1 & 8.4 & 9.2 & 9.9 & 10.3 & 10.7 \\ \hline
123 & 4.807 & 1.7 & 3 & 4 & 5.1 & 5.2 & 6.7 & 8 & 8.9 & 9.9 & 10.3 & 11 \\ \hline
124 & 4.802 & 1.6 & 3 & 3.8 & 4.8 & 5.2 & 6.7 & 8.3 & 9.3 & 10.3 & 10.9 & 11.4 \\ \hline
125 & 4.796 & 1.6 & 3.1 & 4.1 & 5.2 & 5.6 & 7.5 & 9.1 & 10 & 10.8 & 11.2 & 11.5 \\ \hline
126 & 4.791 & 1.6 & 3.7 & 4.9 & 6 & 6.4 & 8.2 & 9.5 & 10.2 & 10.8 & 11 & 11.1 \\ \hline
127 & 4.786 & 2 & 4.9 & 6.3 & 7.2 & 7.2 & 8.8 & 9.8 & 10.3 & 10.8 & 11 & 11.1 \\ \hline
128 & 4.78 & 3.2 & 6.9 & 7.9 & 8.4 & 8.3 & 9.6 & 10.5 & 11.1 & 11.2 & 11.4 & 11.8 \\ \hline
129 & 4.775 & 5.2 & 9.2 & 9.7 & 10 & 9.9 & 11.2 & 12.2 & 12.8 & 13.1 & 13.2 & 13.6 \\ \hline
130 & 4.77 & 7.2 & 10.9 & 11.3 & 11.6 & 11.1 & 12.6 & 13.6 & 14.2 & 14.7 & 14.7 & 15 \\ \hline
131 & 4.764 & 8 & 11.7 & 12.1 & 12.4 & 11.5 & 13 & 14 & 14.6 & 15 & 15.1 & 15.2 \\ \hline
132 & 4.759 & 7.6 & 11.3 & 12.1 & 12.7 & 11.6 & 13 & 13.6 & 14.1 & 14.4 & 14.6 & 14.6 \\ \hline
133 & 4.754 & 6.8 & 10.2 & 11.3 & 12.4 & 11.6 & 13 & 13.5 & 13.7 & 14 & 13.9 & 13.9 \\ \hline
134 & 4.748 & 5.6 & 8.6 & 9.9 & 11.6 & 11.2 & 12.6 & 13.5 & 13.7 & 13.9 & 13.9 & 13.9 \\ \hline
135 & 4.743 & 4.7 & 7.1 & 8.4 & 10.3 & 10.2 & 11.9 & 13.1 & 13.6 & 14 & 14.3 & 14.3 \\ \hline
136 & 4.738 & 3.7 & 5.9 & 7 & 8.7 & 8.9 & 10.7 & 12 & 12.8 & 13.6 & 13.9 & 14.2 \\ \hline
137 & 4.733 & 2.9 & 4.9 & 6 & 7.5 & 7.7 & 9.5 & 10.8 & 12 & 12.8 & 13.1 & 13.8 \\ \hline
138 & 4.727 & 2.4 & 4.1 & 5.2 & 6.4 & 6.6 & 8.3 & 10 & 11.2 & 12 & 12.5 & 13.4 \\ \hline
139 & 4.722 & 2 & 3.7 & 4.8 & 6 & 6.1 & 7.9 & 10 & 11.2 & 12.1 & 12.9 & 13.9 \\ \hline
140 & 4.717 & 2 & 3.7 & 4.8 & 5.9 & 6.4 & 8.3 & 10.5 & 11.8 & 12.8 & 13.4 & 14.2 \\ \hline
141 & 4.711 & 2 & 4.1 & 5.5 & 6.8 & 7.3 & 9.5 & 11.2 & 12.4 & 13.2 & 13.5 & 14.1 \\ \hline
142 & 4.706 & 2.4 & 5.1 & 6.7 & 8 & 8.5 & 10.3 & 11.6 & 12.4 & 13.2 & 13.3 & 13.7 \\ \hline
143 & 4.701 & 3.2 & 7 & 8.6 & 9.6 & 9.6 & 11.1 & 12.3 & 12.8 & 13.4 & 13.5 & 13.9 \\ \hline
144 & 4.695 & 4.8 & 9.4 & 10.6 & 11.2 & 10.9 & 12.6 & 13.6 & 14.2 & 14.7 & 14.8 & 15.2 \\ \hline
145 & 4.69 & 7.5 & 12 & 12.7 & 13.2 & 12.8 & 14.6 & 15.6 & 16.2 & 16.8 & 17 & 17.4 \\ \hline
146 & 4.685 & 9.3 & 13.7 & 14.1 & 14.6 & 14 & 15.7 & 16.8 & 17.2 & 17.9 & 18.2 & 18.3 \\ \hline
147 & 4.679 & 9.7 & 14.1 & 14.9 & 15.3 & 14.4 & 15.9 & 16.8 & 17.2 & 17.8 & 18.1 & 18.2 \\ \hline
148 & 4.674 & 9 & 13.4 & 14.5 & 15.4 & 14.4 & 15.9 & 16.4 & 16.8 & 17.1 & 17.2 & 17.4 \\ \hline
149 & 4.669 & 8 & 11.8 & 13.4 & 15 & 14.4 & 15.9 & 16.4 & 16.7 & 16.9 & 16.9 & 17 \\ \hline
150 & 4.663 & 6.8 & 10.1 & 11.8 & 13.6 & 13.6 & 15.3 & 16.4 & 16.7 & 16.9 & 17 & 17.1 \\ \hline
151 & 4.658 & 5.6 & 8.4 & 9.9 & 12 & 12.1 & 14.1 & 15.6 & 16.3 & 16.8 & 17 & 17.1 \\ \hline
152 & 4.653 & 4.4 & 6.8 & 8.3 & 10.1 & 10.5 & 12.5 & 14.3 & 15.2 & 16 & 16.3 & 16.7 \\ \hline
153 & 4.648 & 3.6 & 5.7 & 7 & 8.6 & 9.2 & 11.1 & 13 & 14 & 15.1 & 15.5 & 16.3 \\ \hline
154 & 4.642 & 2.8 & 4.9 & 6.1 & 7.5 & 8 & 9.9 & 12 & 13.3 & 14.6 & 15.1 & 16.3 \\ \hline
155 & 4.637 & 2.4 & 4.5 & 5.7 & 7 & 7.6 & 9.7 & 12 & 13.6 & 14.9 & 15.5 & 16.7 \\ \hline
156 & 4.632 & 2.4 & 4.5 & 5.7 & 7.1 & 8 & 10.3 & 12.8 & 14.2 & 15.3 & 15.9 & 16.7 \\ \hline
157 & 4.626 & 2.4 & 5.1 & 6.8 & 8.4 & 9.2 & 11.4 & 13.6 & 14.6 & 15.4 & 15.9 & 16.3 \\ \hline
158 & 4.621 & 2.8 & 6.5 & 8.5 & 10 & 10.4 & 12.3 & 14 & 14.8 & 15.4 & 15.8 & 16 \\ \hline
159 & 4.616 & 4 & 8.9 & 10.9 & 11.8 & 12 & 13.5 & 15 & 15.6 & 16.2 & 16.5 & 16.8 \\ \hline
160 & 4.61 & 6.4 & 11.9 & 13.3 & 14 & 13.7 & 15.4 & 16.7 & 17.6 & 18 & 18.4 & 18.8 \\ \hline
161 & 4.605 & 9.2 & 14.8 & 15.7 & 16 & 15.8 & 17.5 & 18.9 & 19.7 & 20.3 & 20.6 & 21 \\ \hline
162 & 4.6 & 11.2 & 16.4 & 17.2 & 17.6 & 17 & 18.6 & 20 & 20.6 & 21.2 & 21.4 & 21.8 \\ \hline
163 & 4.594 & 11.6 & 16.8 & 17.8 & 18.4 & 17.6 & 19 & 20 & 20.6 & 21.1 & 21.1 & 21.4 \\ \hline
164 & 4.589 & 10.8 & 15.7 & 17.3 & 18.4 & 17.6 & 19 & 19.6 & 20.1 & 20.3 & 20.3 & 20.6 \\ \hline
165 & 4.584 & 9.6 & 13.8 & 16 & 17.6 & 17.2 & 18.8 & 19.6 & 19.9 & 20 & 19.9 & 20.2 \\ \hline
166 & 4.578 & 8 & 11.8 & 13.8 & 16 & 16 & 18 & 19 & 19.7 & 20 & 19.9 & 20.2 \\ \hline
167 & 4.573 & 6.8 & 9.8 & 11.8 & 14 & 14.4 & 16.5 & 18.2 & 18.9 & 19.6 & 19.9 & 20.2 \\ \hline
168 & 4.568 & 5.6 & 8.2 & 9.8 & 12 & 12.8 & 14.9 & 16.7 & 17.7 & 18.5 & 19.1 & 19.6 \\ \hline
169 & 4.563 & 4.4 & 6.7 & 8.2 & 10.4 & 11.2 & 13.2 & 15.1 & 16.5 & 17.5 & 18.3 & 18.8 \\ \hline
170 & 4.557 & 3.6 & 5.9 & 7.1 & 9.1 & 9.8 & 12 & 14.2 & 15.7 & 17.1 & 17.9 & 18.8 \\ \hline
171 & 4.552 & 3.2 & 5.4 & 6.6 & 8.4 & 9.2 & 11.6 & 14.4 & 16.1 & 17.5 & 18.3 & 19.3 \\ \hline
172 & 4.547 & 2.8 & 5.3 & 6.6 & 8.4 & 9.6 & 12.3 & 15.1 & 16.6 & 17.9 & 18.7 & 19.3 \\ \hline
173 & 4.541 & 2.9 & 5.9 & 7.7 & 9.6 & 10.8 & 13.4 & 15.6 & 17 & 17.9 & 18.3 & 18.9 \\ \hline
174 & 4.536 & 3.5 & 7.4 & 9.7 & 11.2 & 12 & 14.2 & 16 & 17 & 17.7 & 18.1 & 18.5 \\ \hline
175 & 4.531 & 4.8 & 10.1 & 12.1 & 13.2 & 13.6 & 15.4 & 17 & 17.6 & 18.4 & 18.7 & 19.1 \\ \hline
176 & 4.525 & 7.2 & 12.9 & 14.5 & 15.3 & 15.4 & 17.3 & 18.6 & 19.3 & 20 & 20.3 & 20.7 \\ \hline
177 & 4.52 & 10 & 15.6 & 16.9 & 17.4 & 17.3 & 19.1 & 20.6 & 21.3 & 21.9 & 22.3 & 22.7 \\ \hline
178 & 4.515 & 11.8 & 17.4 & 18.5 & 18.8 & 18.5 & 20.2 & 21.6 & 22.3 & 22.8 & 23.1 & 23.5 \\ \hline
179 & 4.509 & 12.6 & 18.2 & 19.6 & 20 & 19.5 & 20.9 & 22 & 22.6 & 23.1 & 23.2 & 23.5 \\ \hline
180 & 4.504 & 12.2 & 17.6 & 19.6 & 20.7 & 20.1 & 21.4 & 22.1 & 22.6 & 22.8 & 22.9 & 23.1 \\ \hline
181 & 4.499 & 11.2 & 16.2 & 18.5 & 20.3 & 20.1 & 21.7 & 22.5 & 22.9 & 23 & 23 & 23.1 \\ \hline
182 & 4.493 & 9.7 & 14 & 16.5 & 18.8 & 19.2 & 21 & 22.2 & 22.9 & 23.1 & 23.1 & 23.3 \\ \hline
183 & 4.488 & 8.1 & 11.9 & 14.1 & 16.8 & 17.5 & 19.5 & 21.1 & 22.1 & 22.6 & 22.8 & 23.2 \\ \hline
184 & 4.483 & 6.7 & 9.9 & 11.9 & 14.4 & 15.4 & 17.5 & 19.5 & 20.6 & 21.4 & 21.9 & 22.3 \\ \hline
185 & 4.478 & 5.5 & 8.2 & 10 & 12.4 & 13.3 & 15.5 & 17.7 & 19 & 19.9 & 20.7 & 21.5 \\ \hline
186 & 4.472 & 4.4 & 6.9 & 8.5 & 10.7 & 11.6 & 13.6 & 16.2 & 17.9 & 19.1 & 20.3 & 21.2 \\ \hline
187 & 4.467 & 3.8 & 6.1 & 7.7 & 9.6 & 10.8 & 13 & 16.1 & 18 & 19.5 & 20.7 & 21.7 \\ \hline
188 & 4.462 & 3.4 & 5.7 & 7.4 & 9.5 & 10.8 & 13.5 & 16.7 & 18.6 & 19.9 & 20.8 & 21.7 \\ \hline
189 & 4.456 & 3.4 & 6.1 & 8.3 & 10.6 & 12 & 14.6 & 17.4 & 18.9 & 19.9 & 20.7 & 21.3 \\ \hline
190 & 4.451 & 3.6 & 7.6 & 9.9 & 12 & 13.2 & 15.4 & 17.8 & 19 & 19.8 & 20.3 & 20.9 \\ \hline
191 & 4.446 & 4.5 & 10.3 & 12.5 & 14 & 14.8 & 16.7 & 18.7 & 19.6 & 20.3 & 20.7 & 21.2 \\ \hline
192 & 4.44 & 7 & 13.6 & 15.5 & 16.6 & 17 & 19 & 20.7 & 21.6 & 22.2 & 22.7 & 23.1 \\ \hline
193 & 4.435 & 10.6 & 17.1 & 18.4 & 19.4 & 19.4 & 21.5 & 23.2 & 24.1 & 24.7 & 25.1 & 25.6 \\ \hline
194 & 4.43 & 13.2 & 19.5 & 20.5 & 21.1 & 21 & 23 & 24.5 & 25.3 & 25.9 & 26.3 & 26.8 \\ \hline
195 & 4.424 & 14.4 & 20.6 & 22 & 22.4 & 22 & 23.7 & 24.9 & 25.7 & 26.2 & 26.5 & 26.9 \\ \hline
196 & 4.419 & 14.4 & 20.2 & 22.1 & 23.2 & 22.8 & 23.9 & 24.8 & 25.2 & 25.6 & 25.9 & 26.1 \\ \hline
197 & 4.414 & 13.2 & 18.6 & 21 & 22.8 & 22.8 & 24.3 & 24.9 & 24.9 & 25.2 & 25.1 & 25.4 \\ \hline
198 & 4.408 & 11.6 & 16.5 & 19 & 21.3 & 21.9 & 23.6 & 24.8 & 25.1 & 25.3 & 25.2 & 25.4 \\ \hline
199 & 4.403 & 10 & 14.1 & 16.8 & 19.3 & 20 & 22.2 & 24 & 24.7 & 25 & 25.2 & 25.6 \\ \hline
200 & 4.398 & 8.4 & 11.9 & 14.3 & 17 & 18 & 20.3 & 22.4 & 23.4 & 24.1 & 24.4 & 25.1 \\ \hline
201 & 4.392 & 6.8 & 10.1 & 12.2 & 14.7 & 16 & 18.3 & 20.5 & 21.8 & 22.8 & 23.4 & 24.3 \\ \hline
202 & 4.387 & 5.6 & 8.5 & 10.4 & 12.7 & 14 & 16.3 & 18.8 & 20.3 & 21.6 & 22.5 & 23.5 \\ \hline
203 & 4.382 & 4.8 & 7.3 & 9.1 & 11.2 & 12.4 & 14.8 & 17.9 & 19.9 & 21.3 & 22.5 & 23.7 \\ \hline
204 & 4.377 & 4 & 6.5 & 8.3 & 10.4 & 11.7 & 14.6 & 18 & 20.2 & 21.6 & 22.7 & 23.8 \\ \hline
205 & 4.371 & 3.6 & 6.5 & 8.4 & 10.8 & 12.4 & 15.4 & 18.6 & 20.5 & 21.6 & 22.5 & 23.3 \\ \hline
206 & 4.366 & 3.6 & 7.2 & 9.6 & 12 & 13.2 & 16 & 18.6 & 20.1 & 21.2 & 21.8 & 22.4 \\ \hline
207 & 4.361 & 4.2 & 9.2 & 11.8 & 13.6 & 14.4 & 16.8 & 19 & 20.1 & 21.1 & 21.4 & 22 \\ \hline
208 & 4.355 & 5.7 & 12.4 & 14.5 & 15.7 & 16.4 & 18.4 & 20.3 & 21.3 & 22.1 & 22.4 & 23.1 \\ \hline
209 & 4.35 & 8.9 & 16.1 & 17.8 & 18.8 & 19.2 & 21.3 & 23.1 & 24 & 24.8 & 25.1 & 25.8 \\ \hline
210 & 4.345 & 12.1 & 18.8 & 20 & 20.9 & 21.2 & 23.1 & 24.8 & 25.7 & 26.4 & 26.7 & 27.4 \\ \hline
211 & 4.339 & 13.5 & 19.8 & 20.8 & 21.3 & 21.6 & 23.1 & 24.7 & 25.5 & 26 & 26.3 & 27 \\ \hline
212 & 4.334 & 13.5 & 19.1 & 20.6 & 21 & 21.1 & 22.2 & 23.5 & 24 & 24.5 & 24.7 & 25.3 \\ \hline
213 & 4.329 & 12.6 & 17.4 & 19.3 & 20.4 & 20.4 & 21.4 & 22.3 & 22.5 & 22.9 & 23.1 & 23.3 \\ \hline
214 & 4.323 & 11.2 & 15.5 & 17.7 & 19.3 & 19.9 & 21.1 & 21.9 & 22.1 & 22.3 & 22.3 & 22.5 \\ \hline
215 & 4.318 & 10 & 13.9 & 16.4 & 18.5 & 19.3 & 21.1 & 22.5 & 22.9 & 23.4 & 23.5 & 23.5 \\ \hline
216 & 4.313 & 8.8 & 12.6 & 15 & 17.5 & 18.5 & 20.5 & 22.5 & 23.4 & 24.1 & 24.5 & 24.7 \\ \hline
217 & 4.307 & 7.9 & 11.4 & 13.7 & 16.3 & 17.6 & 19.5 & 21.9 & 23.1 & 24 & 24.6 & 25.1 \\ \hline
218 & 4.302 & 6.8 & 10.1 & 12.2 & 14.7 & 16 & 18.3 & 20.7 & 22.1 & 23.2 & 24 & 24.7 \\ \hline
219 & 4.297 & 6 & 8.9 & 10.8 & 13.1 & 14.4 & 16.7 & 19.5 & 21 & 22.4 & 23.6 & 24.7 \\ \hline
220 & 4.291 & 5.2 & 7.7 & 9.6 & 11.7 & 13.2 & 15.6 & 18.8 & 21 & 22.7 & 24 & 25.5 \\ \hline
221 & 4.286 & 4.6 & 7.3 & 9 & 11.2 & 13 & 16 & 19.6 & 22.1 & 23.6 & 24.7 & 25.9 \\ \hline
222 & 4.281 & 4.2 & 7 & 9.3 & 11.8 & 13.9 & 17.2 & 20.5 & 22.6 & 23.8 & 24.7 & 25.5 \\ \hline
223 & 4.275 & 4.2 & 8.1 & 11.1 & 13.7 & 15.5 & 18.3 & 20.9 & 22.5 & 23.5 & 24.3 & 24.8 \\ \hline
224 & 4.27 & 4.8 & 10.5 & 13.5 & 15.5 & 16.6 & 19.1 & 21.2 & 22.5 & 23.3 & 23.9 & 24.4 \\ \hline
225 & 4.265 & 6.4 & 14.1 & 16.4 & 17.6 & 18.4 & 20.7 & 22.6 & 23.6 & 24.4 & 25 & 25.6 \\ \hline
226 & 4.26 & 10 & 18.1 & 19.8 & 20.8 & 21.4 & 23.6 & 25.5 & 26.5 & 27.3 & 27.9 & 28.5 \\ \hline
227 & 4.254 & 14.2 & 21.4 & 22.6 & 23.7 & 24 & 26.3 & 28.3 & 29.3 & 30 & 30.5 & 31.1 \\ \hline
228 & 4.249 & 16.5 & 23.5 & 24.3 & 24.9 & 25.2 & 27.1 & 29 & 30 & 30.4 & 31 & 31.5 \\ \hline
229 & 4.244 & 17 & 23.9 & 25.6 & 25.9 & 25.6 & 27.1 & 28.6 & 29.5 & 30 & 30.3 & 30.7 \\ \hline
230 & 4.238 & 16.3 & 22.6 & 25.3 & 26.7 & 26.4 & 27.3 & 28.2 & 28.6 & 29 & 29.1 & 29.5 \\ \hline
231 & 4.233 & 14.8 & 20.6 & 23.7 & 26.3 & 26.7 & 27.9 & 28.6 & 28.7 & 28.9 & 28.7 & 29 \\ \hline
232 & 4.228 & 12.9 & 18.2 & 21.3 & 24.4 & 25.6 & 27.4 & 28.9 & 29.3 & 29.5 & 29.5 & 29.6 \\ \hline
233 & 4.222 & 11.2 & 15.8 & 18.9 & 22 & 23.5 & 25.7 & 27.9 & 28.8 & 29.5 & 29.9 & 30.2 \\ \hline
234 & 4.217 & 9.5 & 13.5 & 16.4 & 19.2 & 20.9 & 23.3 & 25.8 & 27.2 & 28 & 28.7 & 29.2 \\ \hline
235 & 4.212 & 8 & 11.5 & 14 & 16.8 & 18.5 & 20.8 & 23.4 & 24.8 & 26 & 26.9 & 27.6 \\ \hline
236 & 4.206 & 6.8 & 9.8 & 12.1 & 14.4 & 16.2 & 18.4 & 21.1 & 22.8 & 24.3 & 25.3 & 26.5 \\ \hline
237 & 4.201 & 5.8 & 8.6 & 10.5 & 12.8 & 14.5 & 16.8 & 19.8 & 22 & 23.9 & 25.2 & 26.7 \\ \hline
238 & 4.196 & 5 & 7.7 & 9.7 & 11.7 & 13.7 & 16.4 & 20.3 & 22.8 & 24.7 & 25.9 & 27.2 \\ \hline
239 & 4.19 & 4.6 & 7.4 & 9.7 & 12.1 & 14.4 & 17.9 & 21.6 & 23.6 & 25.2 & 26.2 & 27.2 \\ \hline
240 & 4.185 & 4.4 & 8 & 11.1 & 13.8 & 16 & 19.1 & 22.1 & 23.6 & 24.8 & 25.6 & 26.3 \\ \hline
241 & 4.18 & 4.8 & 10 & 13.5 & 16 & 17.4 & 19.9 & 22.3 & 23.6 & 24.4 & 25.1 & 25.6 \\ \hline
242 & 4.174 & 6 & 13.5 & 16.4 & 18 & 18.8 & 21.1 & 23.1 & 24 & 24.8 & 25.5 & 25.9 \\ \hline
243 & 4.169 & 9.1 & 17.9 & 20 & 20.9 & 21.6 & 23.8 & 25.7 & 26.8 & 27.6 & 28.3 & 28.9 \\ \hline
244 & 4.164 & 14 & 21.9 & 23.4 & 24.4 & 24.8 & 27.1 & 29.2 & 30.4 & 31.2 & 31.8 & 32.5 \\ \hline
245 & 4.158 & 17.5 & 24.5 & 25.4 & 26.4 & 26.6 & 28.8 & 30.8 & 31.9 & 32.7 & 33.2 & 33.7 \\ \hline
246 & 4.153 & 18.3 & 25.2 & 26.5 & 26.8 & 26.9 & 28.5 & 30.3 & 31.2 & 31.9 & 32.1 & 32.7 \\ \hline
247 & 4.148 & 17.4 & 23.9 & 26.2 & 27.1 & 26.9 & 27.8 & 29 & 29.6 & 30.1 & 30.4 & 30.7 \\ \hline
248 & 4.143 & 15.6 & 21.5 & 24.6 & 26.5 & 26.8 & 27.6 & 28.1 & 28.4 & 28.6 & 28.7 & 28.9 \\ \hline
249 & 4.137 & 13.6 & 19 & 22.2 & 25.1 & 25.8 & 27.4 & 28.4 & 28.6 & 28.6 & 28.6 & 28.6 \\ \hline
250 & 4.132 & 11.6 & 16.5 & 19.6 & 22.8 & 24.2 & 26.4 & 28.3 & 29 & 29.4 & 29.5 & 29.7 \\ \hline
251 & 4.127 & 10 & 14.3 & 17.3 & 20.4 & 22.2 & 24.6 & 27.2 & 28.5 & 29.4 & 29.9 & 30.3 \\ \hline
252 & 4.121 & 8.7 & 12.6 & 15.3 & 18.2 & 20.2 & 22.5 & 25.4 & 27 & 28.2 & 29.1 & 29.8 \\ \hline
253 & 4.116 & 7.6 & 11.1 & 13.3 & 16.3 & 18.2 & 20.4 & 23.4 & 25.1 & 26.5 & 27.5 & 28.6 \\ \hline
254 & 4.111 & 6.5 & 9.8 & 11.8 & 14.4 & 16.4 & 18.4 & 21.6 & 23.6 & 25.3 & 26.7 & 28.1 \\ \hline
255 & 4.105 & 6 & 8.9 & 10.6 & 13.2 & 15.1 & 17.4 & 21.2 & 23.6 & 25.6 & 27.2 & 29 \\ \hline
256 & 4.1 & 5.2 & 8.1 & 10.1 & 12.4 & 14.8 & 17.8 & 22.4 & 24.8 & 26.8 & 28 & 29.4 \\ \hline
257 & 4.095 & 4.8 & 8.1 & 10.8 & 13.6 & 16 & 19.7 & 23.6 & 25.6 & 27.1 & 27.9 & 28.8 \\ \hline
258 & 4.089 & 4.8 & 9.2 & 12.8 & 16 & 18 & 21.1 & 24 & 25.5 & 26.5 & 27.1 & 27.7 \\ \hline
259 & 4.084 & 5.5 & 12 & 15.7 & 18.1 & 19.5 & 21.9 & 24.2 & 25.3 & 26.1 & 26.6 & 27 \\ \hline
260 & 4.079 & 7.1 & 15.8 & 18.8 & 20.2 & 21.2 & 23.1 & 25.1 & 26.1 & 26.9 & 27.4 & 28 \\ \hline
261 & 4.073 & 10.8 & 20.1 & 22.4 & 23.2 & 24 & 26 & 28 & 29.1 & 29.9 & 30.6 & 31.2 \\ \hline
262 & 4.068 & 15.6 & 23.7 & 25.2 & 26.2 & 26.8 & 29 & 31.1 & 32.3 & 33.1 & 33.7 & 34.3 \\ \hline
263 & 4.063 & 18.8 & 26.1 & 26.9 & 27.8 & 28.2 & 30.3 & 32.4 & 33.5 & 34.3 & 34.9 & 35.4 \\ \hline
264 & 4.057 & 19.6 & 26.7 & 28.2 & 28.5 & 28.6 & 30.3 & 31.9 & 33 & 33.6 & 34.1 & 34.6 \\ \hline
265 & 4.052 & 18.8 & 25.6 & 28.5 & 29.4 & 29.2 & 30 & 31.1 & 31.8 & 32.3 & 32.5 & 32.7 \\ \hline
266 & 4.047 & 16.8 & 23.4 & 27.1 & 29.6 & 29.9 & 30.4 & 31 & 31.1 & 31.3 & 31.3 & 31.5 \\ \hline
267 & 4.041 & 14.8 & 20.6 & 24.7 & 28.1 & 29.4 & 30.8 & 32 & 32.1 & 32 & 31.9 & 31.9 \\ \hline
268 & 4.036 & 12.8 & 17.8 & 21.7 & 25.5 & 27.4 & 29.6 & 31.9 & 32.7 & 33.2 & 33.1 & 33.5 \\ \hline
269 & 4.031 & 10.9 & 15.4 & 18.9 & 22.4 & 24.7 & 27.2 & 30.2 & 31.7 & 32.7 & 33.1 & 33.8 \\ \hline
270 & 4.025 & 9.3 & 13.4 & 16.2 & 19.5 & 21.8 & 24.1 & 27.4 & 29.2 & 30.6 & 31.5 & 32.3 \\ \hline
271 & 4.02 & 8.1 & 11.7 & 14 & 17.1 & 19.2 & 21.3 & 24.6 & 26.6 & 28.2 & 29.3 & 30.4 \\ \hline
272 & 4.015 & 7.1 & 10.2 & 12.4 & 15 & 17 & 19.2 & 22.5 & 24.7 & 26.4 & 28 & 29.6 \\ \hline
273 & 4.009 & 6.3 & 9.2 & 11.3 & 13.5 & 15.6 & 18 & 22 & 24.7 & 26.8 & 28.7 & 30.5 \\ \hline
274 & 4.004 & 5.6 & 8.5 & 10.6 & 12.8 & 15.2 & 18.6 & 23.2 & 26.1 & 28.2 & 29.8 & 31.2 \\ \hline
275 & 3.999 & 5.2 & 8.5 & 11 & 13.7 & 16.8 & 20.6 & 24.8 & 27.1 & 28.6 & 29.7 & 30.6 \\ \hline
276 & 3.994 & 5.2 & 9.3 & 13 & 16 & 18.8 & 22.1 & 25.2 & 26.7 & 27.9 & 28.5 & 29.1 \\ \hline
277 & 3.988 & 5.6 & 11.9 & 16.1 & 18.5 & 20.4 & 22.9 & 25.1 & 26.3 & 27.1 & 27.6 & 28.1 \\ \hline
278 & 3.983 & 6.9 & 16.1 & 19.3 & 20.9 & 22 & 24 & 25.9 & 26.8 & 27.5 & 28 & 28.7 \\ \hline
279 & 3.978 & 10.8 & 21.2 & 23.7 & 24.4 & 25.2 & 27.2 & 29.1 & 30.1 & 30.9 & 31.6 & 32.4 \\ \hline
280 & 3.972 & 16.8 & 25.8 & 27.5 & 28.4 & 29 & 31.3 & 33.4 & 34.6 & 35.6 & 36.3 & 37.2 \\ \hline
281 & 3.967 & 21.2 & 28.7 & 29.6 & 30.5 & 31 & 33.4 & 35.6 & 36.9 & 38 & 38.4 & 39.2 \\ \hline
282 & 3.962 & 22.3 & 29.6 & 30.7 & 30.8 & 31 & 33.1 & 35.2 & 36.3 & 37.2 & 37.6 & 38.2 \\ \hline
283 & 3.956 & 21.1 & 28.4 & 30.8 & 31.2 & 31 & 32 & 33.5 & 34.3 & 34.9 & 35.3 & 35.7 \\ \hline
284 & 3.951 & 18.7 & 25.6 & 29.2 & 31.2 & 31.5 & 31.8 & 32.3 & 32.7 & 32.9 & 33.1 & 33.3 \\ \hline
285 & 3.946 & 16 & 22.5 & 26.6 & 30 & 31.2 & 32.2 & 32.8 & 32.8 & 32.7 & 32.7 & 32.6 \\ \hline
286 & 3.94 & 13.6 & 19.3 & 23.4 & 27.2 & 29.2 & 31.5 & 33.3 & 33.9 & 34 & 33.9 & 34 \\ \hline
287 & 3.935 & 11.6 & 16.6 & 20.2 & 24 & 26.4 & 29.3 & 32.1 & 33.6 & 34.4 & 34.7 & 35.2 \\ \hline
288 & 3.93 & 10 & 14.2 & 17.3 & 20.8 & 23.2 & 26.1 & 29.5 & 31.5 & 32.8 & 33.5 & 34.4 \\ \hline
289 & 3.924 & 8.7 & 12.5 & 15.1 & 18 & 20.4 & 23 & 26.5 & 28.7 & 30.3 & 31.2 & 32.5 \\ \hline
290 & 3.919 & 7.6 & 10.9 & 13.3 & 15.7 & 18 & 20.3 & 23.7 & 26 & 27.9 & 29.1 & 30.6 \\ \hline
291 & 3.914 & 6.7 & 9.8 & 12 & 14.2 & 16.4 & 18.7 & 22.4 & 24.9 & 27.1 & 28.8 & 30.7 \\ \hline
292 & 3.908 & 6 & 8.9 & 11.2 & 13.1 & 15.6 & 18.4 & 23.1 & 26.1 & 28.5 & 30.3 & 32.3 \\ \hline
293 & 3.903 & 5.6 & 8.6 & 11.2 & 13.5 & 16.4 & 20.3 & 25.2 & 28 & 29.9 & 31.2 & 32.6 \\ \hline
294 & 3.898 & 5.5 & 9 & 12.3 & 15.5 & 18.8 & 22.6 & 26.5 & 28.6 & 29.9 & 30.7 & 31.6 \\ \hline
295 & 3.892 & 5.5 & 10.7 & 15.1 & 18.4 & 21.1 & 24 & 26.8 & 28.1 & 29 & 29.5 & 30.2 \\ \hline
296 & 3.887 & 6.3 & 14.3 & 18.5 & 20.8 & 22.4 & 24.8 & 26.6 & 27.7 & 28.3 & 28.7 & 29.3 \\ \hline
297 & 3.882 & 8.5 & 19 & 22.1 & 23.2 & 24.4 & 26.3 & 28.1 & 28.9 & 29.6 & 30.1 & 30.9 \\ \hline
298 & 3.876 & 13.3 & 24.1 & 26.5 & 27 & 27.9 & 29.9 & 31.9 & 32.9 & 33.8 & 34.5 & 35.4 \\ \hline
299 & 3.871 & 19.5 & 28.2 & 29.7 & 30.6 & 31.3 & 33.6 & 36 & 37.2 & 38.1 & 38.7 & 39.5 \\ \hline
300 & 3.866 & 22.9 & 30.4 & 31 & 32 & 32.5 & 34.8 & 37.2 & 38.4 & 39.3 & 39.9 & 40.6 \\ \hline
301 & 3.86 & 23.3 & 30.6 & 32.1 & 32.3 & 32.4 & 34.1 & 36.2 & 37.3 & 38 & 38.6 & 39 \\ \hline
302 & 3.855 & 21.5 & 28.7 & 31.8 & 32.4 & 32.4 & 32.9 & 34.2 & 34.9 & 35.5 & 35.9 & 36.2 \\ \hline
303 & 3.85 & 18.8 & 25.8 & 29.8 & 32.4 & 32.8 & 32.9 & 33.4 & 33.4 & 33.6 & 33.8 & 33.8 \\ \hline
304 & 3.844 & 16 & 22.2 & 26.6 & 30.6 & 32 & 33.2 & 33.9 & 33.8 & 33.6 & 33.6 & 33.4 \\ \hline
305 & 3.839 & 13.6 & 19 & 23 & 27.6 & 29.6 & 32 & 34.2 & 34.8 & 35 & 35.1 & 35.1 \\ \hline
306 & 3.834 & 11.3 & 16.2 & 19.6 & 23.9 & 26.4 & 29.2 & 32.5 & 34 & 34.9 & 35.5 & 35.9 \\ \hline
307 & 3.829 & 9.7 & 13.8 & 16.9 & 20.5 & 23.2 & 26 & 29.5 & 31.6 & 32.9 & 33.9 & 34.8 \\ \hline
308 & 3.823 & 8.4 & 12 & 14.5 & 17.6 & 20 & 22.8 & 26.2 & 28.4 & 30.1 & 31.4 & 32.5 \\ \hline
309 & 3.818 & 7.4 & 10.7 & 12.9 & 15.6 & 17.6 & 20.2 & 23.6 & 26 & 27.7 & 29.4 & 30.9 \\ \hline
310 & 3.813 & 6.5 & 9.6 & 11.7 & 14 & 16 & 18.5 & 22.3 & 24.9 & 27.1 & 29.1 & 31.1 \\ \hline
311 & 3.807 & 6 & 9 & 11.2 & 13.3 & 15.6 & 18.4 & 23.2 & 26.4 & 29 & 30.7 & 32.8 \\ \hline
312 & 3.802 & 5.6 & 8.8 & 11.2 & 13.7 & 16.5 & 20.5 & 25.7 & 28.7 & 30.8 & 32 & 33.4 \\ \hline
313 & 3.797 & 5.6 & 9.3 & 12.7 & 16 & 19.3 & 23.4 & 27.6 & 29.7 & 30.9 & 31.8 & 32.6 \\ \hline
314 & 3.791 & 5.6 & 10.9 & 15.7 & 19.1 & 21.9 & 24.9 & 27.9 & 29.3 & 30 & 30.6 & 31.1 \\ \hline
315 & 3.786 & 6.4 & 14.4 & 19 & 21.6 & 23.2 & 25.6 & 27.6 & 28.5 & 29.2 & 29.7 & 30.1 \\ \hline
316 & 3.781 & 8.3 & 18.8 & 22.3 & 23.6 & 24.8 & 26.7 & 28.4 & 29.2 & 30 & 30.3 & 31 \\ \hline
317 & 3.775 & 12.7 & 23.9 & 26.4 & 26.9 & 28 & 29.8 & 31.6 & 32.6 & 33.5 & 33.9 & 35 \\ \hline
318 & 3.77 & 18.7 & 28.1 & 29.8 & 30.5 & 31.6 & 33.6 & 35.7 & 37 & 37.9 & 38.4 & 39.4 \\ \hline
319 & 3.765 & 23.2 & 30.9 & 31.8 & 32.6 & 33.5 & 35.6 & 38 & 39.3 & 40.2 & 40.7 & 41.4 \\ \hline
320 & 3.759 & 24.4 & 31.9 & 32.9 & 33.1 & 33.5 & 35.3 & 37.6 & 38.8 & 39.6 & 40 & 40.6 \\ \hline
321 & 3.754 & 23.1 & 30.6 & 33.3 & 33.5 & 33.2 & 34.1 & 35.6 & 36.6 & 37.2 & 37.6 & 38 \\ \hline
322 & 3.749 & 20.3 & 27.5 & 31.7 & 33.6 & 33.6 & 33.7 & 34.1 & 34.5 & 34.8 & 35.1 & 35.2 \\ \hline
323 & 3.743 & 17.3 & 24 & 28.8 & 32.4 & 33.6 & 34.4 & 34.6 & 34.5 & 34.4 & 34.3 & 34.3 \\ \hline
324 & 3.738 & 14.5 & 20.4 & 25 & 29.6 & 31.6 & 34 & 35.5 & 35.8 & 35.9 & 35.7 & 35.6 \\ \hline
325 & 3.733 & 12.1 & 17.4 & 21.4 & 26 & 28.5 & 31.7 & 34.8 & 36.1 & 36.7 & 37.1 & 37.4 \\ \hline
326 & 3.727 & 10.1 & 14.6 & 18.1 & 22 & 24.9 & 28.2 & 32 & 34.2 & 35.5 & 36.3 & 37.2 \\ \hline
327 & 3.722 & 8.8 & 12.6 & 15.7 & 18.9 & 21.6 & 24.6 & 28.6 & 31 & 32.7 & 33.9 & 35.2 \\ \hline
328 & 3.717 & 7.6 & 11 & 13.7 & 16.5 & 18.8 & 21.4 & 25.3 & 27.8 & 29.5 & 31.1 & 32.6 \\ \hline
329 & 3.711 & 6.8 & 10.1 & 12.3 & 14.8 & 16.8 & 19.3 & 23.2 & 25.8 & 27.9 & 29.6 & 31.8 \\ \hline
330 & 3.706 & 6 & 9.3 & 11.4 & 13.6 & 15.6 & 18.4 & 22.8 & 25.9 & 28.5 & 30.6 & 33 \\ \hline
331 & 3.701 & 5.6 & 8.9 & 11.1 & 13.5 & 16 & 19.6 & 24.9 & 28.3 & 30.8 & 32.6 & 34.5 \\ \hline
332 & 3.695 & 5.6 & 9 & 11.8 & 14.7 & 18 & 22.4 & 27.5 & 30.2 & 32 & 33.1 & 34.3 \\ \hline
333 & 3.69 & 5.6 & 10.1 & 14.2 & 17.9 & 21.2 & 25.2 & 28.9 & 30.6 & 31.9 & 32.6 & 33.2 \\ \hline
334 & 3.685 & 6 & 12.5 & 17.8 & 21.2 & 23.6 & 26.4 & 29.1 & 30.2 & 31 & 31.4 & 31.8 \\ \hline
335 & 3.679 & 7.2 & 16.9 & 21.7 & 23.9 & 25.5 & 27.4 & 29.3 & 30.1 & 30.6 & 31.1 & 31.4 \\ \hline
336 & 3.674 & 10.3 & 22.5 & 26 & 26.8 & 28 & 29.6 & 31.3 & 32.1 & 32.8 & 33.3 & 34 \\ \hline
337 & 3.669 & 16.4 & 28.5 & 30.9 & 31.2 & 32.4 & 34.2 & 36.1 & 37.3 & 38.2 & 38.6 & 39.7 \\ \hline
338 & 3.663 & 23.2 & 32.9 & 34.1 & 35.2 & 36.1 & 38.3 & 40.6 & 41.9 & 42.9 & 43.4 & 44.4 \\ \hline
339 & 3.658 & 27.2 & 35.1 & 35.7 & 36.5 & 37.3 & 39.6 & 42.1 & 43.4 & 44.3 & 44.8 & 45.6 \\ \hline
340 & 3.653 & 27.2 & 34.8 & 36.1 & 36.1 & 36.4 & 38.1 & 40.4 & 41.7 & 42.4 & 42.8 & 43.4 \\ \hline
341 & 3.647 & 24.7 & 32.3 & 35.3 & 35.8 & 35.6 & 36.1 & 37.6 & 38.5 & 39.1 & 39.4 & 39.8 \\ \hline
342 & 3.642 & 21 & 28.3 & 32.7 & 35.2 & 35.5 & 35.2 & 35.6 & 35.8 & 36 & 36.2 & 36.4 \\ \hline
343 & 3.637 & 17.4 & 24 & 29 & 33.2 & 34.7 & 35.5 & 35.9 & 35.6 & 35.3 & 35.2 & 35.2 \\ \hline
344 & 3.632 & 14.2 & 20.1 & 24.9 & 29.9 & 32.3 & 34.7 & 36.5 & 36.6 & 36.8 & 36.6 & 36.6 \\ \hline
345 & 3.626 & 11.9 & 16.9 & 20.9 & 26 & 28.8 & 32.2 & 35.5 & 36.9 & 37.6 & 38 & 38.4 \\ \hline
346 & 3.621 & 10 & 14.2 & 17.7 & 22 & 24.9 & 28.4 & 32.7 & 34.8 & 36.4 & 37.2 & 38.2 \\ \hline
347 & 3.616 & 8.4 & 12.3 & 15.3 & 18.8 & 21.6 & 24.8 & 29 & 31.6 & 33.5 & 34.8 & 36.1 \\ \hline
348 & 3.61 & 7.3 & 10.9 & 13.3 & 16.4 & 18.8 & 21.6 & 25.6 & 28 & 30.3 & 31.8 & 33.4 \\ \hline
349 & 3.605 & 6.5 & 9.8 & 12.1 & 14.7 & 16.8 & 19.4 & 23.2 & 26 & 28.3 & 30.2 & 32.2 \\ \hline
350 & 3.6 & 6 & 9 & 11.3 & 13.3 & 15.6 & 18.4 & 22.7 & 26 & 28.7 & 30.7 & 33.1 \\ \hline
351 & 3.594 & 5.6 & 8.7 & 10.9 & 13.2 & 15.6 & 19.2 & 24.4 & 27.9 & 30.4 & 32.2 & 34.2 \\ \hline
352 & 3.589 & 5.6 & 8.7 & 11.3 & 14 & 17.2 & 21.5 & 26.6 & 29.5 & 31.3 & 32.5 & 33.9 \\ \hline
353 & 3.584 & 5.6 & 9.5 & 13.3 & 16.7 & 20.1 & 24.1 & 28.1 & 30.1 & 31.2 & 32 & 32.7 \\ \hline
354 & 3.578 & 6 & 11.6 & 16.5 & 20 & 22.8 & 25.7 & 28.5 & 29.9 & 30.4 & 31 & 31.5 \\ \hline
355 & 3.573 & 7.1 & 15.8 & 20.6 & 23.2 & 25.1 & 27.1 & 28.9 & 29.9 & 30.3 & 30.6 & 31.1 \\ \hline
356 & 3.568 & 9.5 & 21.4 & 25.3 & 26.4 & 27.6 & 29.1 & 30.7 & 31.5 & 32 & 32.5 & 33.1 \\ \hline
357 & 3.562 & 15.1 & 28 & 30.8 & 31.1 & 32 & 33.7 & 35.5 & 36.5 & 37.2 & 37.9 & 38.8 \\ \hline
358 & 3.557 & 22.4 & 33.2 & 34.9 & 35.6 & 36.5 & 38.7 & 40.8 & 42 & 42.9 & 43.6 & 44.5 \\ \hline
359 & 3.552 & 27.6 & 36.5 & 37 & 38 & 38.9 & 41.2 & 43.6 & 45 & 46 & 46.6 & 47.3 \\ \hline
360 & 3.546 & 29.1 & 37.3 & 37.8 & 38.1 & 38.5 & 40.7 & 43.1 & 44.3 & 45.3 & 45.8 & 46.2 \\ \hline
361 & 3.541 & 27.2 & 35.3 & 37.4 & 37.6 & 37.3 & 38.5 & 40.3 & 41.2 & 42 & 42.4 & 42.7 \\ \hline
362 & 3.536 & 23.5 & 31.2 & 35.2 & 36.9 & 36.8 & 36.6 & 37.3 & 37.6 & 38.1 & 38.3 & 38.6 \\ \hline
363 & 3.53 & 19.4 & 26.6 & 31.5 & 35.2 & 36 & 36.3 & 36.3 & 36 & 36 & 35.9 & 36 \\ \hline
364 & 3.525 & 15.6 & 22.2 & 27.1 & 32.2 & 34.1 & 36 & 36.8 & 36.7 & 36.5 & 36.3 & 36.1 \\ \hline
365 & 3.52 & 12.8 & 18.4 & 22.9 & 28.4 & 30.9 & 34.2 & 36.8 & 37.6 & 37.9 & 38 & 38.1 \\ \hline
366 & 3.514 & 10.6 & 15.3 & 19.2 & 24.1 & 27.1 & 30.9 & 34.9 & 36.8 & 38 & 38.6 & 39.3 \\ \hline
367 & 3.509 & 9 & 13 & 16.3 & 20.5 & 23.4 & 26.9 & 31.5 & 34.1 & 36 & 37.1 & 38.3 \\ \hline
368 & 3.504 & 7.8 & 11.4 & 13.9 & 17.6 & 20.1 & 23.3 & 27.7 & 30.5 & 32.6 & 34 & 35.6 \\ \hline
369 & 3.498 & 7 & 10.2 & 12.5 & 15.6 & 17.7 & 20.6 & 24.6 & 27.5 & 29.6 & 31.4 & 33.2 \\ \hline
370 & 3.493 & 6.2 & 9.4 & 11.4 & 14 & 16.1 & 18.9 & 22.9 & 26 & 28.4 & 30.6 & 32.9 \\ \hline
371 & 3.488 & 5.8 & 9 & 11 & 13.5 & 15.6 & 18.6 & 23.7 & 27.2 & 30 & 32.3 & 34.7 \\ \hline
372 & 3.482 & 5.6 & 8.9 & 11.3 & 13.7 & 16.4 & 20.5 & 26.4 & 29.9 & 32.4 & 34.3 & 36 \\ \hline
373 & 3.477 & 5.6 & 9.4 & 12.5 & 15.6 & 19.2 & 24 & 29.2 & 32 & 33.9 & 35.1 & 36.3 \\ \hline
374 & 3.472 & 6 & 10.6 & 15.3 & 19.1 & 22.8 & 26.8 & 30.8 & 32.7 & 33.9 & 34.5 & 35.1 \\ \hline
375 & 3.466 & 6.4 & 13.5 & 19.3 & 22.7 & 25.6 & 28.4 & 30.9 & 32.2 & 32.8 & 33.2 & 33.5 \\ \hline
376 & 3.461 & 7.6 & 17.9 & 23.2 & 25.5 & 27.3 & 29.3 & 30.9 & 31.6 & 32.1 & 32.4 & 32.8 \\ \hline
377 & 3.456 & 10.7 & 23.6 & 27.7 & 28.6 & 29.7 & 31.2 & 32.5 & 33.3 & 33.8 & 34.3 & 35 \\ \hline
378 & 3.45 & 16.4 & 29.7 & 32.5 & 32.7 & 33.7 & 35.3 & 37 & 38 & 38.7 & 39.3 & 40.3 \\ \hline
379 & 3.445 & 23.7 & 34.9 & 36.5 & 37.2 & 38.1 & 40 & 42.2 & 43.4 & 44.3 & 44.9 & 46 \\ \hline
380 & 3.44 & 29 & 38.2 & 38.6 & 39.6 & 40.4 & 42.6 & 45.1 & 46.5 & 47.3 & 48 & 48.8 \\ \hline
381 & 3.434 & 30.8 & 39.1 & 39.6 & 39.9 & 40.2 & 42.3 & 44.7 & 46.1 & 46.9 & 47.5 & 48 \\ \hline
382 & 3.429 & 28.8 & 36.8 & 39.1 & 39.1 & 38.8 & 39.9 & 41.6 & 42.8 & 43.5 & 43.9 & 44.3 \\ \hline
383 & 3.424 & 24.8 & 32.8 & 36.9 & 38.3 & 38 & 38 & 38.4 & 39.1 & 39.5 & 39.7 & 39.9 \\ \hline
384 & 3.418 & 20.4 & 27.8 & 33.1 & 36.8 & 37.4 & 37.5 & 37.3 & 37.1 & 37.1 & 36.9 & 36.9 \\ \hline
385 & 3.413 & 16.4 & 23 & 28.5 & 34 & 35.7 & 37.3 & 37.9 & 37.6 & 37.2 & 37.1 & 36.9 \\ \hline
386 & 3.408 & 13.2 & 18.8 & 23.7 & 29.5 & 32.3 & 35.4 & 37.6 & 38.4 & 38.7 & 38.7 & 38.6 \\ \hline
387 & 3.402 & 10.8 & 15.5 & 19.7 & 24.9 & 28 & 31.9 & 35.8 & 37.7 & 38.7 & 39.4 & 39.9 \\ \hline
388 & 3.397 & 8.9 & 13.1 & 16.5 & 20.9 & 23.8 & 27.7 & 32.3 & 35 & 36.7 & 37.9 & 39 \\ \hline
389 & 3.392 & 7.7 & 11.4 & 14.1 & 17.7 & 20.4 & 23.7 & 28.4 & 31.2 & 33.4 & 35 & 36.4 \\ \hline
390 & 3.386 & 6.6 & 10.1 & 12.4 & 15.3 & 17.6 & 20.5 & 24.8 & 27.8 & 30 & 31.8 & 33.6 \\ \hline
391 & 3.381 & 6 & 9.3 & 11.4 & 13.7 & 16 & 18.8 & 22.8 & 25.9 & 28.4 & 30.5 & 32.7 \\ \hline
392 & 3.376 & 5.6 & 8.9 & 10.9 & 12.9 & 15.2 & 18.3 & 22.9 & 26.4 & 29.3 & 31.6 & 34.3 \\ \hline
393 & 3.37 & 5.6 & 8.9 & 11 & 13.2 & 15.8 & 19.7 & 25.3 & 29.2 & 32 & 34.1 & 36.4 \\ \hline
394 & 3.365 & 5.6 & 9.3 & 11.8 & 14.6 & 18.2 & 22.9 & 28.7 & 32 & 34.4 & 35.7 & 37.6 \\ \hline
395 & 3.36 & 5.9 & 10.3 & 14.2 & 17.8 & 22 & 26.6 & 31.4 & 33.8 & 35.4 & 36.2 & 37.3 \\ \hline
396 & 3.354 & 6.3 & 12.5 & 18 & 21.8 & 25.6 & 29.2 & 32.6 & 34 & 34.8 & 35.2 & 35.7 \\ \hline
397 & 3.349 & 7.4 & 16.5 & 22.4 & 25.5 & 28.2 & 30.6 & 32.6 & 33.3 & 33.7 & 34 & 34.4 \\ \hline
398 & 3.344 & 9.5 & 21.8 & 26.9 & 28.7 & 30.2 & 31.7 & 33.1 & 33.6 & 34 & 34.4 & 34.9 \\ \hline
399 & 3.338 & 13.9 & 28.3 & 32.1 & 32.4 & 33.5 & 34.7 & 36.2 & 36.9 & 37.6 & 38 & 38.9 \\ \hline
400 & 3.333 & 21 & 34.4 & 36.8 & 36.8 & 38 & 39.4 & 41.4 & 42.3 & 43.2 & 43.8 & 44.9 \\ \hline
401 & 3.328 & 27.9 & 38.7 & 39.7 & 40.4 & 41.6 & 43.3 & 45.8 & 46.9 & 48 & 48.5 & 49.6 \\ \hline
402 & 3.322 & 31.7 & 40.7 & 40.9 & 41.6 & 42.4 & 44.5 & 47.1 & 48.5 & 49.6 & 50 & 50.8 \\ \hline
403 & 3.317 & 31.9 & 40 & 40.9 & 40.8 & 41.2 & 42.8 & 45.3 & 46.5 & 47.5 & 47.9 & 48.4 \\ \hline
404 & 3.312 & 28.8 & 36.8 & 39.4 & 39.6 & 39.2 & 39.8 & 41.3 & 42.4 & 43.1 & 43.5 & 43.7 \\ \hline
405 & 3.306 & 24.4 & 32.2 & 36.6 & 38.4 & 38.4 & 38.1 & 38.3 & 38.8 & 39.1 & 39.2 & 39.3 \\ \hline
406 & 3.301 & 19.6 & 27 & 32.6 & 36.8 & 37.6 & 37.7 & 37.6 & 37.3 & 37.1 & 36.9 & 36.8 \\ \hline
407 & 3.296 & 15.6 & 22.2 & 27.8 & 33.9 & 35.9 & 37.6 & 38.5 & 38.3 & 38 & 37.7 & 37.6 \\ \hline
408 & 3.291 & 12.8 & 18.2 & 23 & 29.3 & 32.4 & 36 & 38.8 & 39.5 & 39.9 & 39.8 & 40 \\ \hline
409 & 3.285 & 10.4 & 15 & 19 & 24.6 & 28 & 32.4 & 36.9 & 39 & 40.1 & 40.7 & 41.4 \\ \hline
410 & 3.28 & 8.8 & 12.6 & 15.8 & 20.5 & 23.6 & 27.8 & 32.9 & 35.8 & 37.7 & 39.1 & 40.2 \\ \hline
411 & 3.275 & 7.6 & 10.9 & 13.7 & 17.3 & 20 & 23.7 & 28.5 & 31.6 & 33.8 & 35.5 & 37.1 \\ \hline
412 & 3.269 & 6.4 & 9.7 & 12.1 & 14.8 & 17.2 & 20.5 & 24.8 & 27.7 & 30 & 31.9 & 33.8 \\ \hline
413 & 3.264 & 6 & 8.9 & 10.9 & 13.2 & 15.6 & 18.4 & 22.5 & 25.5 & 28 & 29.9 & 32.3 \\ \hline
414 & 3.259 & 5.6 & 8.5 & 10.4 & 12.4 & 14.8 & 17.6 & 22.1 & 25.5 & 28.4 & 30.3 & 33.1 \\ \hline
415 & 3.253 & 5.3 & 8.5 & 10.5 & 12.7 & 15.2 & 18.8 & 24.1 & 27.9 & 30.7 & 32.6 & 35 \\ \hline
416 & 3.248 & 5.3 & 8.7 & 11 & 13.9 & 17.2 & 21.6 & 27.4 & 30.7 & 33.1 & 34.8 & 36.6 \\ \hline
417 & 3.243 & 5.4 & 9.7 & 13.1 & 16.7 & 20.8 & 25.5 & 30.5 & 33.2 & 34.9 & 36 & 37 \\ \hline
418 & 3.237 & 6 & 11.7 & 16.9 & 20.7 & 24.8 & 28.7 & 32.4 & 34.1 & 35.1 & 35.6 & 36 \\ \hline
419 & 3.232 & 7 & 15.4 & 21.4 & 24.8 & 28 & 30.7 & 33.1 & 33.9 & 34.4 & 34.6 & 34.7 \\ \hline
420 & 3.227 & 8.8 & 20.6 & 26.1 & 28.4 & 30.4 & 31.9 & 33.4 & 33.9 & 34.2 & 34.4 & 34.7 \\ \hline
421 & 3.221 & 12.7 & 27.1 & 31.5 & 32 & 33.3 & 34.5 & 35.7 & 36.3 & 36.7 & 37.2 & 37.9 \\ \hline
422 & 3.216 & 19.2 & 33.7 & 36.6 & 36.4 & 37.6 & 38.8 & 40.5 & 41.4 & 42 & 42.7 & 43.6 \\ \hline
423 & 3.211 & 26.5 & 38.9 & 40.2 & 40.4 & 41.8 & 43.3 & 45.4 & 46.6 & 47.6 & 48.2 & 49.3 \\ \hline
424 & 3.205 & 32 & 42 & 42.1 & 42.8 & 43.9 & 45.9 & 48.3 & 49.8 & 50.8 & 51.4 & 52.4 \\ \hline
425 & 3.2 & 34 & 42.5 & 42.7 & 42.9 & 43.5 & 45.5 & 48.1 & 49.6 & 50.4 & 51.1 & 51.6 \\ \hline
426 & 3.195 & 32.1 & 40.2 & 41.7 & 41.6 & 41.4 & 42.7 & 44.8 & 46 & 46.8 & 47.3 & 47.6 \\ \hline
427 & 3.189 & 27.8 & 36 & 39.4 & 40.4 & 40 & 39.8 & 40.8 & 41.4 & 41.9 & 42.3 & 42.4 \\ \hline
428 & 3.184 & 22.7 & 30.7 & 35.8 & 38.8 & 39.2 & 38.7 & 38.4 & 38.3 & 38.4 & 38.4 & 38.4 \\ \hline
429 & 3.179 & 18.3 & 25.4 & 31.4 & 36.7 & 38.3 & 38.7 & 38.7 & 38.1 & 37.8 & 37.5 & 37.2 \\ \hline
430 & 3.173 & 14.7 & 20.6 & 26.2 & 32.9 & 35.6 & 38 & 39.5 & 39.4 & 39.4 & 39 & 38.8 \\ \hline
431 & 3.168 & 11.9 & 17 & 21.7 & 28.2 & 31.6 & 35.8 & 39.3 & 40.6 & 41.1 & 41.2 & 41.3 \\ \hline
432 & 3.163 & 9.8 & 14.1 & 18 & 23.5 & 26.9 & 31.6 & 36.8 & 39.4 & 40.9 & 41.8 & 42.5 \\ \hline
433 & 3.157 & 8.3 & 12.1 & 15.2 & 19.7 & 22.7 & 27.2 & 32.7 & 35.9 & 38.1 & 39.7 & 41.2 \\ \hline
434 & 3.152 & 7.1 & 10.5 & 13.1 & 16.7 & 19.5 & 23.2 & 28.3 & 31.6 & 34.1 & 36 & 37.9 \\ \hline
435 & 3.147 & 6.3 & 9.4 & 11.7 & 14.6 & 17.1 & 20.3 & 24.8 & 28 & 30.7 & 32.8 & 35 \\ \hline
436 & 3.141 & 5.8 & 8.6 & 10.9 & 13.2 & 15.6 & 18.6 & 22.9 & 26.3 & 29.1 & 31.5 & 34.1 \\ \hline
437 & 3.136 & 5.6 & 8.5 & 10.6 & 12.7 & 15.2 & 18.4 & 23.2 & 27.1 & 30 & 32.7 & 35.4 \\ \hline
438 & 3.131 & 5.4 & 8.5 & 10.7 & 13.1 & 16 & 20 & 25.8 & 29.9 & 32.8 & 35.1 & 37.5 \\ \hline
439 & 3.125 & 5.4 & 8.9 & 11.8 & 14.7 & 18.4 & 23.5 & 29.4 & 33.1 & 35.6 & 37.5 & 39.3 \\ \hline
440 & 3.12 & 5.5 & 10.1 & 14 & 17.8 & 22.4 & 27.2 & 32.6 & 35.5 & 37.3 & 38.5 & 39.6 \\ \hline
441 & 3.115 & 6.2 & 12.1 & 17.8 & 22 & 26.4 & 30.4 & 34.6 & 36.3 & 37.3 & 37.9 & 38.3 \\ \hline
442 & 3.109 & 7.2 & 15.7 & 22.2 & 26 & 29.5 & 32.4 & 34.9 & 35.7 & 36 & 36.3 & 36.4 \\ \hline
443 & 3.104 & 9 & 20.9 & 27 & 29.6 & 31.8 & 33.5 & 34.9 & 35.2 & 35.5 & 35.7 & 35.9 \\ \hline
444 & 3.099 & 12.6 & 27.3 & 32.1 & 33.1 & 34.2 & 35.3 & 36.4 & 36.8 & 37.2 & 37.6 & 38.3 \\ \hline
445 & 3.093 & 18.8 & 34.1 & 37.4 & 37.2 & 38.4 & 39.4 & 40.8 & 41.5 & 42.1 & 42.8 & 43.8 \\ \hline
446 & 3.088 & 26.2 & 39.7 & 41.3 & 41.2 & 42.5 & 43.8 & 45.8 & 46.7 & 47.7 & 48.3 & 49.5 \\ \hline
447 & 3.083 & 32.2 & 43.3 & 43.7 & 44 & 45.2 & 47 & 49.5 & 50.7 & 51.8 & 52.4 & 53.5 \\ \hline
448 & 3.077 & 35.4 & 44.4 & 44.4 & 44.7 & 45.3 & 47.5 & 50.1 & 51.5 & 52.5 & 53.1 & 53.9 \\ \hline
449 & 3.072 & 34.6 & 42.8 & 43.6 & 43.6 & 43.6 & 45.2 & 47.5 & 48.7 & 49.6 & 50.2 & 50.6 \\ \hline
450 & 3.067 & 30.5 & 38.9 & 41.5 & 41.9 & 41.2 & 41.5 & 42.9 & 43.9 & 44.4 & 45 & 45.2 \\ \hline
451 & 3.061 & 25.3 & 33.6 & 38.3 & 40.3 & 40 & 39.4 & 39.5 & 39.6 & 40 & 40.2 & 40.2 \\ \hline
452 & 3.056 & 20.3 & 27.9 & 33.9 & 38.3 & 39 & 38.9 & 38.4 & 37.9 & 37.6 & 37.5 & 37.4 \\ \hline
453 & 3.051 & 16 & 22.7 & 28.8 & 35.2 & 37.3 & 38.8 & 39.2 & 38.7 & 38.4 & 37.9 & 37.7 \\ \hline
454 & 3.045 & 12.8 & 18.5 & 23.8 & 30.8 & 34 & 37.5 & 39.7 & 40.2 & 40.3 & 39.9 & 40.1 \\ \hline
455 & 3.04 & 10.4 & 15.1 & 19.7 & 26 & 29.5 & 34.3 & 38.7 & 40.6 & 41.6 & 41.9 & 42.5 \\ \hline
456 & 3.035 & 8.8 & 12.7 & 16.5 & 21.6 & 25 & 29.9 & 35.5 & 38.6 & 40.5 & 41.7 & 42.9 \\ \hline
457 & 3.029 & 7.6 & 11.1 & 14.1 & 18.4 & 21.3 & 25.6 & 31.2 & 34.8 & 37.3 & 39.1 & 40.8 \\ \hline
458 & 3.024 & 6.4 & 9.9 & 12.5 & 15.9 & 18.4 & 22 & 27.2 & 30.5 & 33.3 & 35.4 & 37.4 \\ \hline
459 & 3.019 & 6 & 9.1 & 11.3 & 14 & 16.4 & 19.6 & 24.3 & 27.6 & 30.4 & 32.7 & 35 \\ \hline
460 & 3.013 & 5.6 & 8.5 & 10.6 & 12.9 & 15.2 & 18.4 & 23 & 26.4 & 29.5 & 31.9 & 34.7 \\ \hline
461 & 3.008 & 5.5 & 8.5 & 10.5 & 12.7 & 15.2 & 18.7 & 23.9 & 27.7 & 31.1 & 33.5 & 36.4 \\ \hline
462 & 3.003 & 5.4 & 8.5 & 10.9 & 13.2 & 16.4 & 20.7 & 26.8 & 30.6 & 33.9 & 36 & 38.4 \\ \hline
463 & 2.997 & 5.4 & 9.1 & 12.1 & 15.1 & 19.2 & 24.2 & 30.3 & 33.8 & 36.6 & 38.3 & 40.2 \\ \hline
464 & 2.992 & 5.7 & 10.3 & 14.5 & 18.3 & 23 & 27.9 & 33.5 & 36.2 & 38.2 & 39.2 & 40.2 \\ \hline
465 & 2.987 & 6.3 & 12.4 & 18.2 & 22.3 & 27 & 31.1 & 35.5 & 37 & 38.1 & 38.5 & 38.7 \\ \hline
466 & 2.981 & 7.2 & 16 & 22.6 & 26.4 & 30.2 & 33.2 & 35.8 & 36.4 & 36.8 & 36.9 & 36.9 \\ \hline
467 & 2.976 & 9.1 & 21.1 & 27.4 & 30.3 & 32.6 & 34.4 & 35.6 & 35.9 & 36 & 36.1 & 36.2 \\ \hline
468 & 2.971 & 12.5 & 27.4 & 32.7 & 33.8 & 35 & 36 & 37 & 37.3 & 37.6 & 37.9 & 38.4 \\ \hline
469 & 2.965 & 18.5 & 34.5 & 38.1 & 37.8 & 38.9 & 39.7 & 41 & 41.6 & 42.2 & 42.7 & 43.7 \\ \hline
470 & 2.96 & 25.8 & 40.3 & 42.1 & 41.6 & 43 & 44.2 & 45.9 & 46.8 & 47.7 & 48.3 & 49.5 \\ \hline
471 & 2.955 & 32 & 44.2 & 44.5 & 44.5 & 45.9 & 47.5 & 49.8 & 51 & 52.1 & 52.6 & 53.8 \\ \hline
472 & 2.949 & 35.8 & 45.7 & 45.3 & 45.6 & 46.5 & 48.5 & 51.1 & 52.6 & 53.6 & 54.2 & 55.1 \\ \hline
473 & 2.944 & 36.3 & 44.5 & 44.7 & 44.8 & 45 & 46.8 & 49.2 & 50.7 & 51.6 & 52.2 & 52.7 \\ \hline
474 & 2.939 & 33.2 & 41.3 & 42.9 & 42.9 & 42.5 & 43.3 & 45.1 & 46.3 & 47 & 47.5 & 47.8 \\ \hline
475 & 2.933 & 28.3 & 36.5 & 40.4 & 41.5 & 40.9 & 40.4 & 41 & 41.6 & 42 & 42.3 & 42.5 \\ \hline
476 & 2.928 & 22.8 & 30.9 & 36.8 & 40 & 40.2 & 39.6 & 39.1 & 38.8 & 38.8 & 38.7 & 38.7 \\ \hline
477 & 2.923 & 18.2 & 25.4 & 32 & 38 & 39.4 & 40 & 39.6 & 39 & 38.6 & 38 & 38 \\ \hline
478 & 2.917 & 14.5 & 20.6 & 26.8 & 34.3 & 37 & 39.6 & 40.8 & 40.6 & 40.4 & 40 & 39.8 \\ \hline
479 & 2.912 & 11.7 & 17 & 22.2 & 29.5 & 33.1 & 37.7 & 41.2 & 42.2 & 42.6 & 42.6 & 42.7 \\ \hline
480 & 2.907 & 9.6 & 14.2 & 18.2 & 24.7 & 28.3 & 33.7 & 39.2 & 41.7 & 43.1 & 43.9 & 44.7 \\ \hline
481 & 2.901 & 8.1 & 12.2 & 15.4 & 20.7 & 24 & 29.1 & 35.2 & 38.9 & 41.2 & 42.7 & 44.2 \\ \hline
482 & 2.896 & 7 & 10.6 & 13.3 & 17.5 & 20.4 & 24.7 & 30.5 & 34.4 & 37.2 & 39.2 & 41.3 \\ \hline
483 & 2.891 & 6.2 & 9.5 & 11.9 & 15.2 & 17.7 & 21.5 & 26.6 & 30.3 & 33.2 & 35.4 & 37.8 \\ \hline
484 & 2.885 & 5.8 & 8.7 & 11 & 13.6 & 16 & 19.2 & 23.9 & 27.5 & 30.4 & 32.8 & 35.6 \\ \hline
485 & 2.88 & 5.5 & 8.3 & 10.6 & 12.8 & 15.2 & 18.4 & 23.1 & 26.9 & 30 & 32.7 & 35.8 \\ \hline
486 & 2.875 & 5.2 & 8.2 & 10.5 & 12.7 & 15.2 & 18.8 & 24.4 & 28.5 & 31.8 & 34.4 & 37.3 \\ \hline
487 & 2.869 & 5.2 & 8.6 & 10.9 & 13.5 & 16.8 & 21.2 & 27.6 & 31.6 & 34.6 & 37.1 & 39.6 \\ \hline
488 & 2.864 & 5.2 & 9.2 & 12.2 & 15.5 & 19.6 & 24.8 & 31.2 & 34.7 & 37.5 & 39.5 & 41.4 \\ \hline
489 & 2.859 & 5.6 & 10.4 & 14.8 & 18.8 & 23.6 & 28.8 & 34.5 & 37.5 & 39.5 & 40.7 & 41.8 \\ \hline
490 & 2.853 & 6.2 & 12.5 & 18.5 & 22.8 & 27.5 & 32.1 & 36.6 & 38.5 & 39.5 & 40 & 40.3 \\ \hline
491 & 2.848 & 7.2 & 16.1 & 22.9 & 26.8 & 30.9 & 34.4 & 37 & 37.9 & 38.2 & 38.3 & 38.2 \\ \hline
492 & 2.843 & 8.8 & 20.9 & 27.7 & 30.8 & 33.5 & 35.4 & 36.6 & 36.9 & 37 & 37.1 & 37.1 \\ \hline
493 & 2.837 & 12.1 & 27.2 & 33 & 34.3 & 35.9 & 36.8 & 37.6 & 37.8 & 38 & 38.3 & 38.7 \\ \hline
494 & 2.832 & 17.6 & 34.1 & 38.2 & 38.1 & 39.3 & 39.9 & 40.8 & 41.4 & 41.9 & 42.4 & 43.3 \\ \hline
495 & 2.827 & 24.9 & 40.6 & 42.8 & 42.1 & 43.5 & 44.4 & 45.9 & 46.6 & 47.6 & 48.2 & 49.3 \\ \hline
496 & 2.821 & 32 & 45.6 & 46.1 & 45.7 & 47.2 & 48.6 & 50.7 & 51.8 & 52.9 & 53.7 & 54.9 \\ \hline
497 & 2.816 & 37.3 & 48.5 & 48.1 & 48 & 49.3 & 51.3 & 53.9 & 55.3 & 56.4 & 57.3 & 58.3 \\ \hline
498 & 2.811 & 39.5 & 48.4 & 48.1 & 48.1 & 48.8 & 50.8 & 53.5 & 55 & 56 & 56.8 & 57.4 \\ \hline
499 & 2.805 & 37.5 & 45.6 & 46.6 & 46.5 & 46.4 & 47.5 & 49.9 & 51.1 & 52 & 52.7 & 53 \\ \hline
500 & 2.8 & 32.6 & 40.9 & 43.8 & 44.3 & 43.8 & 43.7 & 44.8 & 45.7 & 46.3 & 46.7 & 46.9 \\ \hline
501 & 2.795 & 26.8 & 35.2 & 40.4 & 42.4 & 42.3 & 41.4 & 41.2 & 41.3 & 41.5 & 41.6 & 41.7 \\ \hline
502 & 2.789 & 21.2 & 29.2 & 35.7 & 40.4 & 41.2 & 40.8 & 40 & 39.4 & 39.1 & 38.9 & 38.8 \\ \hline
503 & 2.784 & 16.7 & 23.6 & 30.5 & 37.6 & 39.6 & 40.8 & 40.9 & 40.2 & 39.7 & 39.3 & 39 \\ \hline
504 & 2.779 & 13.2 & 19.1 & 25.3 & 33.2 & 36.4 & 40 & 41.8 & 42 & 41.8 & 41.5 & 41.4 \\ \hline
505 & 2.773 & 10.8 & 15.7 & 20.8 & 28.4 & 32 & 37.2 & 41.4 & 42.8 & 43.6 & 43.9 & 44.2 \\ \hline
506 & 2.768 & 8.8 & 13.2 & 17.3 & 23.5 & 27.2 & 32.8 & 38.7 & 41.6 & 43.3 & 44.3 & 45.3 \\ \hline
507 & 2.763 & 7.6 & 11.4 & 14.6 & 19.7 & 23 & 28.2 & 34.6 & 38.4 & 40.9 & 42.7 & 44.3 \\ \hline
508 & 2.757 & 6.4 & 10.1 & 12.8 & 16.8 & 19.6 & 24 & 30.1 & 34 & 36.9 & 39.1 & 41.2 \\ \hline
509 & 2.752 & 6 & 9.3 & 11.6 & 14.8 & 17.2 & 21 & 26.4 & 30 & 33 & 35.4 & 37.9 \\ \hline
510 & 2.747 & 5.6 & 8.5 & 10.8 & 13.2 & 15.6 & 19 & 23.9 & 27.2 & 30.4 & 33 & 35.8 \\ \hline
511 & 2.741 & 5.3 & 8.2 & 10.4 & 12.5 & 15 & 18.2 & 23.2 & 26.9 & 30.3 & 32.9 & 36 \\ \hline
512 & 2.736 & 5.1 & 8.2 & 10.5 & 12.1 & 14.9 & 18.5 & 24.3 & 28.4 & 31.9 & 34.5 & 37.5 \\ \hline
513 & 2.731 & 5.2 & 8.5 & 10.9 & 13.1 & 16.4 & 20.9 & 27.2 & 31.3 & 34.6 & 37 & 39.6 \\ \hline
514 & 2.725 & 5.2 & 9 & 12.1 & 15.1 & 19.2 & 24.4 & 30.8 & 34.7 & 37.6 & 39.7 & 41.6 \\ \hline
515 & 2.72 & 5.6 & 10.2 & 14.5 & 18.3 & 23.2 & 28.5 & 34.5 & 37.8 & 40.1 & 41.6 & 42.7 \\ \hline
516 & 2.715 & 6.1 & 12.2 & 18.1 & 22.3 & 27.3 & 32.3 & 37.2 & 39.5 & 40.8 & 41.5 & 41.8 \\ \hline
517 & 2.709 & 7.2 & 15.5 & 22.5 & 26.7 & 31.2 & 35 & 38.4 & 39.4 & 39.9 & 39.9 & 39.8 \\ \hline
518 & 2.704 & 8.6 & 19.9 & 27.2 & 30.8 & 34.1 & 36.4 & 38 & 38.3 & 38.4 & 38.3 & 38.2 \\ \hline
519 & 2.699 & 11.3 & 25.6 & 32.2 & 34.5 & 36.5 & 37.6 & 38.3 & 38.3 & 38.4 & 38.4 & 38.6 \\ \hline
520 & 2.693 & 15.9 & 32.3 & 37.6 & 38.1 & 39.3 & 39.8 & 40.5 & 40.7 & 41.2 & 41.4 & 42.1 \\ \hline
521 & 2.688 & 22.6 & 39.2 & 42.6 & 42 & 43.2 & 43.7 & 44.9 & 45.5 & 46.2 & 46.6 & 47.7 \\ \hline
522 & 2.683 & 29.8 & 44.8 & 46.2 & 45.5 & 47 & 47.8 & 49.7 & 50.7 & 51.7 & 52.3 & 53.5 \\ \hline
523 & 2.677 & 35.8 & 48.7 & 48.6 & 48.3 & 49.8 & 51.2 & 53.7 & 55 & 56.1 & 56.8 & 57.9 \\ \hline
524 & 2.672 & 39.3 & 49.7 & 49.1 & 49.1 & 50.3 & 52 & 54.8 & 56.3 & 57.3 & 58.1 & 59 \\ \hline
525 & 2.667 & 39.2 & 47.7 & 47.8 & 47.7 & 48.2 & 49.6 & 52.3 & 53.7 & 54.6 & 55.4 & 55.9 \\ \hline
526 & 2.661 & 35.9 & 43.6 & 45 & 45.3 & 45 & 45.6 & 47.6 & 48.8 & 49.5 & 50.2 & 50.4 \\ \hline
527 & 2.656 & 30.7 & 38.6 & 41.8 & 42.8 & 42.3 & 41.9 & 42.7 & 43.5 & 43.9 & 44.4 & 44.4 \\ \hline
528 & 2.651 & 25.1 & 33 & 38.2 & 40.8 & 40.7 & 39.7 & 39.5 & 39.6 & 39.6 & 39.9 & 39.7 \\ \hline
529 & 2.645 & 20.1 & 27.8 & 34.5 & 39.2 & 40 & 39.7 & 39.1 & 38.4 & 38.1 & 37.9 & 37.7 \\ \hline
530 & 2.64 & 16 & 22.9 & 29.7 & 36.8 & 38.9 & 40.2 & 40.3 & 39.6 & 39.2 & 38.7 & 38.4 \\ \hline
531 & 2.635 & 12.8 & 18.9 & 25 & 33.2 & 36.4 & 40 & 42 & 42 & 42 & 41.6 & 41.5 \\ \hline
532 & 2.629 & 10.4 & 15.6 & 20.8 & 28.6 & 32.4 & 37.6 & 42 & 43.6 & 44.3 & 44.4 & 44.7 \\ \hline
533 & 2.624 & 8.8 & 13.3 & 17.6 & 24.1 & 28 & 33.7 & 39.8 & 42.9 & 44.7 & 45.5 & 46.4 \\ \hline
534 & 2.619 & 7.6 & 11.6 & 15 & 20.4 & 23.7 & 29.2 & 36 & 40 & 42.6 & 44.3 & 45.8 \\ \hline
535 & 2.613 & 6.8 & 10.4 & 13.3 & 17.6 & 20.4 & 25.2 & 31.7 & 35.9 & 38.9 & 41.1 & 43.2 \\ \hline
536 & 2.608 & 6 & 9.5 & 12.1 & 15.6 & 18 & 22 & 27.8 & 31.9 & 34.9 & 37.4 & 40 \\ \hline
537 & 2.602 & 5.6 & 9 & 11.3 & 14.3 & 16.4 & 20 & 25.2 & 29.1 & 32.2 & 34.8 & 37.7 \\ \hline
538 & 2.597 & 5.4 & 8.6 & 10.9 & 13.3 & 15.6 & 18.9 & 24 & 27.9 & 31.2 & 34 & 37.2 \\ \hline
539 & 2.592 & 5.3 & 8.6 & 10.8 & 13.2 & 15.6 & 19.1 & 24.8 & 28.8 & 32.4 & 35.1 & 38.2 \\ \hline
540 & 2.586 & 5.2 & 8.5 & 10.9 & 13.2 & 16.4 & 20.6 & 26.8 & 30.8 & 34.2 & 36.7 & 39.3 \\ \hline
541 & 2.581 & 5.2 & 8.8 & 11.3 & 14.1 & 18 & 22.9 & 29 & 32.7 & 35.7 & 37.8 & 40 \\ \hline
542 & 2.576 & 5.2 & 9.2 & 12.5 & 15.7 & 20.4 & 25.2 & 30.8 & 34 & 36.4 & 38.1 & 39.6 \\ \hline
543 & 2.57 & 5.3 & 10.1 & 14.5 & 18 & 22.7 & 27.2 & 32.3 & 34.8 & 36.6 & 37.7 & 38.4 \\ \hline
544 & 2.565 & 5.7 & 11.7 & 16.9 & 20.8 & 25.1 & 29.1 & 33.2 & 35.1 & 36 & 36.5 & 36.8 \\ \hline
545 & 2.56 & 6.5 & 14.4 & 20.2 & 23.9 & 27.8 & 31.1 & 33.8 & 34.7 & 35.2 & 35.2 & 35.1 \\ \hline
546 & 2.554 & 8 & 18.3 & 24.5 & 27.6 & 30.6 & 32.7 & 34.1 & 34.3 & 34.4 & 34.3 & 34.3 \\ \hline
547 & 2.549 & 10.5 & 23.8 & 29.8 & 31.8 & 33.4 & 34.4 & 35.1 & 35.2 & 35.2 & 35.3 & 35.5 \\ \hline
548 & 2.544 & 14.9 & 30.5 & 35.6 & 36 & 36.8 & 37.3 & 38 & 38.3 & 38.5 & 38.9 & 39.5 \\ \hline
549 & 2.538 & 21.6 & 37.9 & 41.2 & 40.4 & 41.3 & 42 & 43.2 & 43.7 & 44.4 & 44.9 & 45.9 \\ \hline
550 & 2.533 & 28.8 & 44.2 & 45.6 & 44.6 & 45.7 & 46.8 & 48.7 & 49.6 & 50.4 & 51.2 & 52.4 \\ \hline
551 & 2.528 & 35.3 & 48.7 & 48.5 & 48 & 49.3 & 51 & 53.4 & 54.8 & 55.8 & 56.7 & 57.7 \\ \hline
552 & 2.522 & 39.6 & 50.7 & 49.9 & 49.8 & 50.9 & 52.9 & 55.7 & 57.2 & 58.3 & 59.3 & 60.1 \\ \hline
553 & 2.517 & 41.2 & 50.1 & 49.8 & 49.8 & 50.1 & 52.1 & 55 & 56.5 & 57.5 & 58.4 & 59 \\ \hline
554 & 2.512 & 39.2 & 47.2 & 48.1 & 48.2 & 47.7 & 48.9 & 51.3 & 52.8 & 53.6 & 54.3 & 54.6 \\ \hline
555 & 2.506 & 34.5 & 42.8 & 45.5 & 46 & 45.3 & 45.3 & 46.6 & 47.6 & 48.1 & 48.7 & 48.7 \\ \hline
556 & 2.501 & 28.5 & 37.1 & 42.1 & 44 & 43.6 & 42.8 & 42.7 & 42.9 & 43.1 & 43.4 & 43.3 \\ \hline
557 & 2.496 & 22.9 & 31.3 & 38.1 & 42.3 & 42.7 & 42.1 & 41.3 & 40.7 & 40.3 & 40.2 & 40 \\ \hline
558 & 2.49 & 18.1 & 25.7 & 33.2 & 39.9 & 41.5 & 42.1 & 41.6 & 40.7 & 40.2 & 39.7 & 39.3 \\ \hline
559 & 2.485 & 14.5 & 21 & 28.2 & 36.3 & 39.2 & 41.9 & 42.8 & 42.3 & 42 & 41.6 & 41.2 \\ \hline
560 & 2.48 & 11.7 & 17.3 & 23.4 & 31.9 & 35.6 & 40.3 & 43.3 & 44 & 44.4 & 44.2 & 44.2 \\ \hline
561 & 2.474 & 9.7 & 14.5 & 19.5 & 27.2 & 31.1 & 37.1 & 42.4 & 44.4 & 45.6 & 46.2 & 46.6 \\ \hline
562 & 2.469 & 8.2 & 12.5 & 16.3 & 22.8 & 26.4 & 32.7 & 39.3 & 42.7 & 44.8 & 46.2 & 47.4 \\ \hline
563 & 2.464 & 7.2 & 11 & 13.9 & 19.2 & 22.4 & 28.3 & 35.2 & 39.2 & 42 & 44.1 & 45.8 \\ \hline
564 & 2.458 & 6.4 & 9.8 & 12.4 & 16.6 & 19.2 & 24.3 & 30.8 & 35 & 38.1 & 40.6 & 42.9 \\ \hline
565 & 2.453 & 6 & 9 & 11.4 & 14.8 & 17.1 & 21.2 & 27.1 & 31.2 & 34.4 & 37 & 39.7 \\ \hline
566 & 2.448 & 5.6 & 8.6 & 10.6 & 13.4 & 15.6 & 19.2 & 24.7 & 28.4 & 31.7 & 34.5 & 37.5 \\ \hline
567 & 2.442 & 5.4 & 8.3 & 10.5 & 12.8 & 15 & 18.4 & 23.7 & 27.6 & 31.2 & 34.1 & 37.3 \\ \hline
568 & 2.437 & 5.2 & 8.2 & 10.5 & 12.6 & 15 & 18.8 & 24.5 & 28.8 & 32.3 & 35.2 & 38.5 \\ \hline
569 & 2.432 & 5.2 & 8.5 & 10.9 & 13.1 & 16 & 20.5 & 27 & 31.2 & 34.8 & 37.5 & 40.5 \\ \hline
570 & 2.426 & 5.2 & 8.9 & 11.5 & 14.3 & 18 & 23.4 & 30.2 & 34.4 & 37.6 & 40 & 42.7 \\ \hline
571 & 2.421 & 5.3 & 9.7 & 13.1 & 16.6 & 21.2 & 27.1 & 33.8 & 37.6 & 40.5 & 42.5 & 44.7 \\ \hline
572 & 2.416 & 5.7 & 10.6 & 15.5 & 19.9 & 25.2 & 30.7 & 37 & 40.4 & 42.5 & 44 & 45.3 \\ \hline
573 & 2.41 & 6.4 & 12.7 & 19.1 & 23.7 & 29.1 & 34.1 & 39.4 & 42 & 43.2 & 43.9 & 44.4 \\ \hline
574 & 2.405 & 7.2 & 15.8 & 23.1 & 27.7 & 32.7 & 36.9 & 40.5 & 41.9 & 42.3 & 42.3 & 42.2 \\ \hline
575 & 2.4 & 8.5 & 20.1 & 27.6 & 31.7 & 35.7 & 38.5 & 40.4 & 40.7 & 40.6 & 40.3 & 40.2 \\ \hline
576 & 2.394 & 10.8 & 25 & 32.4 & 35.3 & 38 & 39.3 & 40 & 39.9 & 39.8 & 39.5 & 39.5 \\ \hline
577 & 2.389 & 14.4 & 31.1 & 37.6 & 38.8 & 40.4 & 40.7 & 41.1 & 41.1 & 41.2 & 41.2 & 41.6 \\ \hline
578 & 2.384 & 20 & 37.6 & 42.5 & 42.1 & 43.3 & 43.4 & 44.1 & 44.4 & 44.9 & 45.3 & 46.1 \\ \hline
579 & 2.378 & 26.8 & 43.7 & 46.3 & 45.4 & 46.7 & 47 & 48.4 & 49.1 & 49.9 & 50.5 & 51.6 \\ \hline
580 & 2.373 & 33.2 & 48.5 & 49.1 & 48.2 & 49.7 & 50.6 & 52.6 & 53.6 & 54.7 & 55.6 & 56.8 \\ \hline
581 & 2.368 & 38.4 & 51.8 & 51.1 & 50.6 & 52.1 & 53.5 & 56.1 & 57.6 & 58.7 & 59.7 & 60.8 \\ \hline
582 & 2.362 & 42 & 52.9 & 51.8 & 51.7 & 52.8 & 54.7 & 57.5 & 59.2 & 60.3 & 61.3 & 62.3 \\ \hline
583 & 2.357 & 42.8 & 51.9 & 51.4 & 51.3 & 51.6 & 53.5 & 56.3 & 58 & 59.1 & 60 & 60.6 \\ \hline
584 & 2.352 & 40.8 & 48.8 & 49.6 & 49.6 & 49.2 & 50.3 & 52.7 & 54 & 55.1 & 55.7 & 56.2 \\ \hline
585 & 2.346 & 36.1 & 44.4 & 47 & 47.6 & 46.5 & 46.7 & 48 & 48.8 & 49.6 & 50.1 & 50.3 \\ \hline
586 & 2.341 & 30.2 & 38.8 & 43.6 & 45.4 & 44.8 & 44 & 44 & 44.1 & 44.4 & 44.6 & 44.7 \\ \hline
587 & 2.336 & 24.2 & 33 & 39.6 & 43.5 & 43.6 & 42.9 & 42 & 41.5 & 41.2 & 41 & 41 \\ \hline
588 & 2.33 & 19.2 & 27.3 & 34.9 & 41.2 & 42.4 & 42.8 & 42.1 & 41.1 & 40.5 & 39.9 & 39.7 \\ \hline
589 & 2.325 & 15.4 & 22.4 & 30.1 & 38 & 40.6 & 42.8 & 43.1 & 42.4 & 42 & 41.4 & 41 \\ \hline
590 & 2.32 & 12.4 & 18.4 & 25.2 & 34 & 37.4 & 41.6 & 44 & 44.2 & 44.2 & 43.8 & 43.7 \\ \hline
591 & 2.314 & 10.4 & 15.4 & 21 & 29.2 & 33.2 & 39.1 & 43.6 & 45.1 & 45.8 & 46.1 & 46.5 \\ \hline
592 & 2.309 & 8.8 & 13 & 17.7 & 24.8 & 28.7 & 35.1 & 41.3 & 44.3 & 45.8 & 46.9 & 47.7 \\ \hline
593 & 2.304 & 7.6 & 11.4 & 15.3 & 21.1 & 24.6 & 30.8 & 37.8 & 41.8 & 44.2 & 45.9 & 47.4 \\ \hline
594 & 2.298 & 6.7 & 10.2 & 13.3 & 18 & 21 & 26.6 & 33.6 & 38 & 40.9 & 43.2 & 45.3 \\ \hline
595 & 2.293 & 6 & 9.4 & 12.1 & 15.7 & 18.3 & 23.1 & 29.6 & 34 & 37.2 & 39.7 & 42.1 \\ \hline
596 & 2.288 & 5.6 & 8.8 & 11.2 & 14.1 & 16.4 & 20.5 & 26.3 & 30.4 & 33.6 & 36.3 & 39.2 \\ \hline
597 & 2.282 & 5.6 & 8.4 & 10.7 & 13 & 15.2 & 18.9 & 24.3 & 28.3 & 31.6 & 34.4 & 37.5 \\ \hline
598 & 2.277 & 5.3 & 8.2 & 10.4 & 12.4 & 14.8 & 18.2 & 23.5 & 27.7 & 31.2 & 34 & 37.4 \\ \hline
599 & 2.272 & 5.2 & 8.2 & 10.4 & 12.5 & 14.8 & 18.7 & 24.5 & 28.8 & 32.4 & 35.2 & 38.5 \\ \hline
600 & 2.266 & 5.2 & 8.4 & 10.7 & 13 & 15.8 & 20.3 & 26.8 & 31.2 & 34.7 & 37.3 & 40.3 \\ \hline
601 & 2.261 & 5.2 & 8.9 & 11.5 & 14.2 & 17.9 & 23.2 & 30 & 34.2 & 37.5 & 39.9 & 42.5 \\ \hline
602 & 2.256 & 5.2 & 9.5 & 12.8 & 16.2 & 20.8 & 26.5 & 33.2 & 37.2 & 40.1 & 42.3 & 44.5 \\ \hline
603 & 2.25 & 5.6 & 10.5 & 15 & 19.3 & 24.4 & 30.1 & 36.5 & 40 & 42.5 & 44.2 & 45.7 \\ \hline
604 & 2.245 & 6 & 12.1 & 18.1 & 22.8 & 28.1 & 33.5 & 39.2 & 42 & 43.7 & 44.7 & 45.4 \\ \hline
605 & 2.24 & 6.8 & 14.9 & 22.1 & 26.7 & 31.8 & 36.6 & 41 & 42.8 & 43.5 & 43.8 & 43.8 \\ \hline
606 & 2.234 & 8 & 18.4 & 26.1 & 30.4 & 35 & 38.6 & 41.3 & 42 & 42 & 41.8 & 41.4 \\ \hline
607 & 2.229 & 9.6 & 22.8 & 30.5 & 34 & 37.6 & 39.7 & 40.8 & 40.8 & 40.5 & 40.3 & 39.9 \\ \hline
608 & 2.223 & 12.3 & 27.9 & 35.1 & 37.2 & 39.6 & 40.4 & 40.7 & 40.4 & 40.4 & 40.3 & 40.3 \\ \hline
609 & 2.218 & 16.5 & 33.9 & 40 & 40.4 & 42 & 42 & 42.3 & 42.1 & 42.4 & 42.6 & 43.2 \\ \hline
610 & 2.213 & 22.3 & 40.1 & 44.4 & 43.6 & 44.8 & 44.9 & 45.6 & 45.8 & 46.5 & 47 & 48 \\ \hline
611 & 2.207 & 28.8 & 45.7 & 48 & 46.8 & 48 & 48.4 & 49.9 & 50.6 & 51.5 & 52.2 & 53.4 \\ \hline
612 & 2.202 & 34.8 & 50.5 & 50.7 & 49.5 & 51 & 52 & 54 & 55.2 & 56.3 & 57.1 & 58.5 \\ \hline
613 & 2.197 & 40 & 53.5 & 52.6 & 51.8 & 53.3 & 54.8 & 57.5 & 59 & 60.2 & 61.1 & 62.3 \\ \hline
614 & 2.191 & 43.2 & 54.5 & 53.3 & 53 & 54 & 55.8 & 58.8 & 60.6 & 61.8 & 62.8 & 63.8 \\ \hline
615 & 2.186 & 44.4 & 53.6 & 52.9 & 52.7 & 52.9 & 54.9 & 57.7 & 59.7 & 60.8 & 61.7 & 62.4 \\ \hline
616 & 2.181 & 42.6 & 50.7 & 51 & 51.1 & 50.6 & 52 & 54.5 & 56.1 & 57.1 & 57.8 & 58.3 \\ \hline
617 & 2.175 & 38.3 & 46.4 & 48.5 & 49 & 48 & 48.4 & 50 & 51.1 & 51.9 & 52.3 & 52.5 \\ \hline
618 & 2.17 & 32.5 & 41 & 45.3 & 46.7 & 46 & 45.2 & 45.5 & 46 & 46.4 & 46.6 & 46.6 \\ \hline
619 & 2.165 & 26.5 & 35.4 & 41.4 & 44.7 & 44.5 & 43.6 & 42.9 & 42.5 & 42.5 & 42.4 & 42.1 \\ \hline
620 & 2.159 & 21.3 & 29.8 & 37.2 & 42.5 & 43.2 & 43.1 & 42.2 & 41.2 & 40.8 & 40.4 & 39.9 \\ \hline
621 & 2.154 & 17.2 & 24.7 & 32.8 & 40 & 42 & 43.1 & 42.9 & 42 & 41.3 & 40.7 & 40.2 \\ \hline
622 & 2.149 & 13.9 & 20.4 & 28 & 36.8 & 39.6 & 42.7 & 44 & 43.6 & 43.2 & 42.6 & 42.2 \\ \hline
623 & 2.143 & 11.5 & 17.2 & 23.7 & 32.8 & 36.4 & 41.3 & 44.6 & 45.2 & 45.4 & 45.3 & 45.2 \\ \hline
624 & 2.138 & 9.6 & 14.4 & 20.1 & 28.4 & 32.3 & 38.6 & 43.8 & 45.7 & 46.7 & 47.2 & 47.6 \\ \hline
625 & 2.133 & 8.4 & 12.6 & 17.2 & 24.4 & 28.2 & 35 & 41.7 & 44.8 & 46.6 & 47.8 & 48.7 \\ \hline
626 & 2.127 & 7.5 & 11.4 & 14.9 & 20.8 & 24.3 & 30.9 & 38.3 & 42.4 & 44.9 & 46.6 & 48.3 \\ \hline
627 & 2.122 & 6.8 & 10.5 & 13.3 & 18 & 21.1 & 27 & 34.4 & 39 & 42.1 & 44.3 & 46.6 \\ \hline
628 & 2.117 & 6.1 & 9.7 & 12.2 & 16 & 18.7 & 23.7 & 30.5 & 35 & 38.6 & 41.1 & 43.8 \\ \hline
629 & 2.111 & 5.7 & 9.3 & 11.4 & 14.8 & 16.8 & 21.1 & 27.4 & 31.7 & 35.2 & 38 & 41 \\ \hline
630 & 2.106 & 5.6 & 8.9 & 10.9 & 13.7 & 15.6 & 19.5 & 25.1 & 29.3 & 32.8 & 35.7 & 39 \\ \hline
631 & 2.101 & 5.6 & 8.6 & 10.6 & 13.1 & 15.2 & 18.8 & 24.2 & 28.4 & 32 & 35 & 38.6 \\ \hline
632 & 2.095 & 5.6 & 8.6 & 10.6 & 12.9 & 15.1 & 18.8 & 24.7 & 29.1 & 32.8 & 35.8 & 39.3 \\ \hline
633 & 2.09 & 5.6 & 8.6 & 10.9 & 13.2 & 15.7 & 20 & 26.5 & 30.9 & 34.7 & 37.4 & 40.6 \\ \hline
634 & 2.085 & 5.6 & 8.9 & 11.3 & 14 & 16.9 & 22.2 & 29 & 33.3 & 36.8 & 39.4 & 42.2 \\ \hline
635 & 2.079 & 5.6 & 9.4 & 12.2 & 15.4 & 19.4 & 25.1 & 32 & 36.1 & 39.3 & 41.7 & 44.3 \\ \hline
636 & 2.074 & 5.7 & 10.1 & 13.8 & 17.6 & 22.6 & 28.5 & 35.2 & 39.1 & 42 & 44.1 & 46.2 \\ \hline
637 & 2.069 & 6.1 & 11.3 & 16.3 & 20.8 & 26.2 & 32 & 38.4 & 41.7 & 44.2 & 45.7 & 47.1 \\ \hline
638 & 2.063 & 6.5 & 13 & 19.5 & 24.2 & 29.8 & 35.2 & 40.7 & 43.4 & 45 & 46 & 46.5 \\ \hline
639 & 2.058 & 7.3 & 15.5 & 23.1 & 27.7 & 33.1 & 37.9 & 42.3 & 43.8 & 44.6 & 44.8 & 44.7 \\ \hline
640 & 2.053 & 8.3 & 18.9 & 26.9 & 31.2 & 36 & 39.9 & 42.6 & 43.1 & 43.1 & 42.8 & 42.4 \\ \hline
641 & 2.047 & 10 & 23.1 & 31.1 & 34.8 & 38.7 & 41.1 & 42.3 & 42 & 41.7 & 41.3 & 41 \\ \hline
642 & 2.042 & 12.4 & 28.1 & 35.6 & 38.4 & 40.8 & 41.9 & 42 & 41.6 & 41.3 & 41.1 & 41.1 \\ \hline
643 & 2.037 & 16 & 33.9 & 40.5 & 41.6 & 43.1 & 43.1 & 43 & 42.9 & 43 & 43.1 & 43.6 \\ \hline
644 & 2.031 & 21.4 & 39.8 & 44.9 & 44.6 & 45.6 & 45.4 & 45.8 & 46 & 46.5 & 46.9 & 47.8 \\ \hline
645 & 2.026 & 27.6 & 45.3 & 48.4 & 47.4 & 48.4 & 48.5 & 49.6 & 50.2 & 51 & 51.7 & 52.8 \\ \hline
646 & 2.021 & 33.3 & 49.8 & 50.9 & 49.5 & 50.8 & 51.4 & 53.2 & 54.3 & 55.3 & 56.1 & 57.5 \\ \hline
647 & 2.015 & 38.4 & 53.1 & 52.6 & 51.5 & 52.9 & 54.1 & 56.6 & 58 & 59.2 & 60.2 & 61.5 \\ \hline
648 & 2.01 & 42.3 & 55 & 53.7 & 53 & 54.4 & 56 & 59 & 60.4 & 61.9 & 63 & 64.1 \\ \hline
649 & 2.005 & 44.6 & 55.3 & 54.1 & 53.7 & 54.5 & 56.4 & 59.6 & 61.3 & 62.5 & 63.7 & 64.6 \\ \hline
650 & 1.999 & 44.6 & 53.5 & 53 & 52.8 & 52.9 & 54.8 & 57.8 & 59.4 & 60.5 & 61.6 & 62.2 \\ \hline
651 & 1.994 & 42.4 & 50.3 & 51 & 51.2 & 50.5 & 51.9 & 54.3 & 55.8 & 56.8 & 57.5 & 57.9 \\ \hline
652 & 1.989 & 38.4 & 46.1 & 48.4 & 49.2 & 48 & 48.4 & 50 & 51.1 & 51.9 & 52.4 & 52.6 \\ \hline
653 & 1.983 & 33.2 & 41.5 & 45.5 & 47.1 & 46 & 45.6 & 46.1 & 46.6 & 47.1 & 47.3 & 47.3 \\ \hline
654 & 1.978 & 27.6 & 36.5 & 42.5 & 45.3 & 44.7 & 44 & 43.3 & 43.3 & 43.2 & 43.2 & 43 \\ \hline
655 & 1.972 & 22.4 & 31.3 & 38.9 & 43.7 & 44 & 43.7 & 42.6 & 41.9 & 41.5 & 41.1 & 40.7 \\ \hline
656 & 1.967 & 18 & 26.3 & 34.7 & 41.6 & 43 & 43.7 & 43.2 & 42.3 & 41.7 & 41 & 40.6 \\ \hline
657 & 1.962 & 14.8 & 22.1 & 30.2 & 38.7 & 41.2 & 43.7 & 44.4 & 43.9 & 43.3 & 42.6 & 42.2 \\ \hline
658 & 1.956 & 12.3 & 18.5 & 25.8 & 35 & 38.4 & 42.6 & 45.1 & 45.4 & 45.3 & 44.9 & 44.7 \\ \hline
659 & 1.951 & 10.4 & 15.8 & 21.9 & 31.2 & 35 & 40.7 & 45 & 46.3 & 46.8 & 46.9 & 47.1 \\ \hline
660 & 1.946 & 9.1 & 13.8 & 18.9 & 27.3 & 31.2 & 38 & 43.8 & 46.3 & 47.6 & 48.2 & 48.9 \\ \hline
661 & 1.94 & 8 & 12.4 & 16.5 & 23.7 & 27.5 & 34.6 & 41.8 & 45.2 & 47.3 & 48.6 & 49.8 \\ \hline
662 & 1.935 & 7.2 & 11.3 & 14.8 & 20.5 & 23.9 & 30.8 & 38.6 & 42.8 & 45.6 & 47.5 & 49.3 \\ \hline
663 & 1.93 & 6.8 & 10.5 & 13.3 & 18.1 & 21 & 27.1 & 34.7 & 39.4 & 42.7 & 45.1 & 47.4 \\ \hline
664 & 1.924 & 6.3 & 9.8 & 12.4 & 16.1 & 18.6 & 23.9 & 30.9 & 35.6 & 39.1 & 41.9 & 44.5 \\ \hline
665 & 1.919 & 6 & 9.4 & 11.7 & 14.8 & 16.9 & 21.5 & 27.7 & 32.3 & 35.9 & 38.7 & 41.7 \\ \hline
666 & 1.914 & 5.6 & 9 & 11.2 & 13.9 & 15.7 & 19.9 & 25.5 & 29.8 & 33.4 & 36.3 & 39.5 \\ \hline
667 & 1.908 & 5.6 & 8.9 & 10.9 & 13.5 & 15.2 & 19.1 & 24.4 & 28.6 & 32.2 & 35.3 & 38.7 \\ \hline
668 & 1.903 & 5.6 & 8.9 & 10.9 & 13.2 & 15.2 & 18.9 & 24.4 & 28.9 & 32.6 & 35.7 & 39.2 \\ \hline
669 & 1.898 & 5.6 & 9 & 11.2 & 13.3 & 15.6 & 19.8 & 26 & 30.5 & 34.3 & 37.2 & 40.6 \\ \hline
670 & 1.892 & 5.6 & 9.2 & 11.4 & 13.9 & 16.7 & 21.5 & 28.4 & 32.9 & 36.6 & 39.4 & 42.5 \\ \hline
671 & 1.887 & 5.6 & 9.6 & 12.2 & 15.1 & 18.7 & 24.3 & 31.5 & 35.9 & 39.3 & 42 & 44.9 \\ \hline
672 & 1.882 & 5.9 & 10.1 & 13.3 & 16.7 & 21.4 & 27.4 & 34.4 & 38.7 & 41.7 & 44.2 & 46.9 \\ \hline
673 & 1.876 & 6.2 & 10.9 & 15 & 19.1 & 24.5 & 30.6 & 37.2 & 41.2 & 44 & 46.1 & 48.1 \\ \hline
674 & 1.871 & 6.4 & 12 & 17.3 & 21.9 & 27.7 & 33.3 & 39.5 & 42.9 & 45.2 & 46.9 & 48.2 \\ \hline
675 & 1.866 & 6.8 & 13.6 & 20.1 & 24.8 & 30.5 & 35.7 & 41.2 & 44 & 45.6 & 46.6 & 47.2 \\ \hline
676 & 1.86 & 7.3 & 15.6 & 22.9 & 27.6 & 32.9 & 37.5 & 42 & 44 & 44.8 & 45.1 & 45 \\ \hline
677 & 1.855 & 8.1 & 18.4 & 26.2 & 30.5 & 35.2 & 39.2 & 42.4 & 43.3 & 43.3 & 43.1 & 42.6 \\ \hline
678 & 1.85 & 9.3 & 21.9 & 29.7 & 33.6 & 37.6 & 40.5 & 42.1 & 42.1 & 41.7 & 41.3 & 40.7 \\ \hline
679 & 1.844 & 11.3 & 26.1 & 33.7 & 36.9 & 40 & 41.6 & 41.8 & 41.3 & 40.9 & 40.6 & 40.3 \\ \hline
680 & 1.839 & 14.1 & 30.9 & 38.2 & 40.3 & 42.1 & 42.5 & 42.2 & 41.7 & 41.6 & 41.5 & 41.7 \\ \hline
681 & 1.834 & 18.4 & 36.6 & 43 & 43.6 & 44.5 & 44.3 & 44.2 & 44.1 & 44.3 & 44.5 & 45.2 \\ \hline
682 & 1.828 & 24 & 42.4 & 47.2 & 46.5 & 47.3 & 47 & 47.5 & 47.7 & 48.3 & 48.9 & 49.9 \\ \hline
683 & 1.823 & 30 & 47.7 & 50.4 & 49 & 50 & 50.1 & 51.5 & 52.1 & 52.9 & 53.6 & 54.9 \\ \hline
684 & 1.818 & 35.5 & 52.1 & 52.8 & 51.2 & 52.5 & 53.1 & 55.2 & 56.2 & 57.3 & 58.3 & 59.6 \\ \hline
685 & 1.812 & 40.3 & 55.5 & 54.5 & 53.2 & 54.6 & 55.9 & 58.6 & 60.1 & 61.2 & 62.3 & 63.7 \\ \hline
686 & 1.807 & 43.8 & 57.3 & 55.4 & 54.5 & 56 & 57.8 & 60.8 & 62.6 & 63.9 & 65.1 & 66.4 \\ \hline
687 & 1.802 & 46.1 & 57.5 & 55.8 & 55.3 & 56.4 & 58.4 & 61.7 & 63.6 & 64.9 & 66.1 & 67.2 \\ \hline
688 & 1.796 & 46.7 & 56.1 & 55.1 & 54.9 & 55.3 & 57.5 & 60.6 & 62.4 & 63.7 & 64.7 & 65.6 \\ \hline
689 & 1.791 & 45.5 & 53.5 & 53.7 & 53.9 & 53.3 & 55.1 & 57.9 & 59.6 & 60.7 & 61.5 & 62.2 \\ \hline
690 & 1.786 & 42.3 & 49.9 & 51.3 & 52 & 50.9 & 51.9 & 54 & 55.4 & 56.2 & 56.9 & 57.3 \\ \hline
691 & 1.78 & 37.6 & 45.7 & 48.7 & 50 & 48.8 & 48.7 & 49.8 & 50.7 & 51.4 & 51.7 & 51.9 \\ \hline
692 & 1.775 & 32 & 41 & 45.8 & 47.8 & 47 & 46.3 & 46.2 & 46.5 & 46.7 & 46.9 & 46.8 \\ \hline
693 & 1.769 & 26.6 & 36 & 42.6 & 46 & 45.8 & 45.1 & 44.2 & 43.8 & 43.5 & 43.5 & 43.2 \\ \hline
694 & 1.764 & 21.8 & 31.1 & 39.1 & 44.3 & 44.8 & 44.6 & 43.6 & 42.7 & 42.1 & 41.8 & 41.3 \\ \hline
695 & 1.759 & 18 & 26.7 & 35.4 & 42.4 & 44 & 44.7 & 44.4 & 43.2 & 42.5 & 41.9 & 41.4 \\ \hline
696 & 1.753 & 14.8 & 22.7 & 31.4 & 40 & 42.4 & 44.8 & 45.5 & 44.7 & 44.1 & 43.5 & 42.9 \\ \hline
697 & 1.748 & 12.8 & 19.5 & 27.4 & 37.3 & 40.5 & 44.4 & 46.7 & 46.6 & 46.3 & 45.9 & 45.6 \\ \hline
698 & 1.743 & 10.9 & 17 & 23.8 & 33.8 & 37.6 & 43.2 & 47.1 & 48 & 48.3 & 48.2 & 48.3 \\ \hline
699 & 1.737 & 9.6 & 15 & 20.7 & 30 & 34.3 & 41.1 & 46.6 & 48.7 & 49.6 & 50.1 & 50.6 \\ \hline
700 & 1.732 & 8.6 & 13.4 & 18.2 & 26.4 & 30.4 & 38.1 & 44.9 & 48 & 49.8 & 50.8 & 51.8 \\ \hline
701 & 1.727 & 7.8 & 12.2 & 16.2 & 23.2 & 26.8 & 34.4 & 42.4 & 46.4 & 48.9 & 50.4 & 51.9 \\ \hline
702 & 1.721 & 7.2 & 11.3 & 14.6 & 20.4 & 23.5 & 30.6 & 38.8 & 43.6 & 46.7 & 48.7 & 50.7 \\ \hline
703 & 1.716 & 6.8 & 10.6 & 13.4 & 18.1 & 20.8 & 27.1 & 35.1 & 40.2 & 43.8 & 46.3 & 48.7 \\ \hline
704 & 1.711 & 6.4 & 10.2 & 12.5 & 16.4 & 18.8 & 24.1 & 31.4 & 36.4 & 40.2 & 43.1 & 46 \\ \hline
705 & 1.705 & 6.3 & 9.8 & 12.1 & 15.2 & 17.3 & 22 & 28.5 & 33.2 & 37 & 40 & 43.2 \\ \hline
706 & 1.7 & 5.9 & 9.4 & 11.7 & 14.4 & 16.3 & 20.2 & 26.2 & 30.7 & 34.4 & 37.6 & 41 \\ \hline
707 & 1.695 & 5.8 & 9.3 & 11.4 & 14 & 15.7 & 19.3 & 25 & 29.4 & 33.2 & 36.4 & 40 \\ \hline
708 & 1.689 & 5.6 & 9.3 & 11.3 & 13.6 & 15.5 & 19.2 & 24.8 & 29.2 & 33.2 & 36.4 & 40 \\ \hline
709 & 1.684 & 5.6 & 9.3 & 11.3 & 13.6 & 15.9 & 19.7 & 25.8 & 30.4 & 34.4 & 37.6 & 41.2 \\ \hline
710 & 1.679 & 5.6 & 9.4 & 11.3 & 14 & 16.5 & 20.9 & 27.7 & 32.4 & 36.4 & 39.4 & 42.7 \\ \hline
711 & 1.673 & 5.8 & 9.8 & 12.1 & 14.7 & 17.8 & 23 & 30.3 & 34.8 & 38.7 & 41.5 & 44.6 \\ \hline
712 & 1.668 & 6 & 10.2 & 12.8 & 15.8 & 19.8 & 25.7 & 33.2 & 37.6 & 41.1 & 43.7 & 46.6 \\ \hline
713 & 1.663 & 6.3 & 10.6 & 14 & 17.5 & 22.5 & 28.7 & 36 & 40.3 & 43.5 & 46 & 48.5 \\ \hline
714 & 1.657 & 6.4 & 11.1 & 15.5 & 19.7 & 25.4 & 31.6 & 38.6 & 42.7 & 45.7 & 47.9 & 50 \\ \hline
715 & 1.652 & 6.8 & 12.2 & 17.6 & 22.4 & 28.5 & 34.5 & 41.2 & 45 & 47.6 & 49.5 & 50.9 \\ \hline
716 & 1.647 & 7.2 & 13.6 & 20.3 & 25.3 & 31.5 & 37.2 & 43.3 & 46.7 & 48.8 & 49.9 & 50.5 \\ \hline
717 & 1.641 & 7.6 & 15.6 & 23.3 & 28.4 & 34.4 & 39.8 & 45.2 & 47.6 & 48.8 & 49.2 & 49.2 \\ \hline
718 & 1.636 & 8.4 & 18.1 & 26.5 & 31.3 & 37.1 & 41.8 & 46 & 47.3 & 47.6 & 47.2 & 46.6 \\ \hline
719 & 1.631 & 9.2 & 21.2 & 29.7 & 34.4 & 39.6 & 43.3 & 45.7 & 46 & 45.6 & 45 & 44.2 \\ \hline
720 & 1.625 & 10.4 & 24.5 & 33 & 37.2 & 41.5 & 43.9 & 44.8 & 44.3 & 43.6 & 43 & 42.4 \\ \hline
721 & 1.62 & 12.4 & 28.4 & 36.6 & 40 & 43.2 & 44.2 & 44 & 43.3 & 42.7 & 42.2 & 42 \\ \hline
722 & 1.615 & 15 & 32.7 & 40.3 & 42.4 & 44.6 & 44.4 & 43.8 & 43.2 & 42.9 & 42.8 & 42.8 \\ \hline
723 & 1.609 & 18.7 & 37.4 & 44.1 & 44.6 & 45.9 & 45.1 & 44.8 & 44.5 & 44.7 & 44.9 & 45.5 \\ \hline
724 & 1.604 & 23.2 & 42.1 & 47.3 & 46.6 & 47.6 & 46.8 & 47.1 & 47.3 & 47.8 & 48.2 & 49.2 \\ \hline
725 & 1.599 & 28.4 & 46.5 & 50 & 48.4 & 49.6 & 49.1 & 50.2 & 50.8 & 51.6 & 52.2 & 53.5 \\ \hline
726 & 1.593 & 33.1 & 50.5 & 52 & 50 & 51.4 & 51.6 & 53.4 & 54.3 & 55.3 & 56.2 & 57.6 \\ \hline
727 & 1.588 & 37.5 & 54.1 & 53.7 & 51.9 & 53.4 & 54.2 & 56.6 & 57.9 & 59.1 & 60.2 & 61.6 \\ \hline
728 & 1.582 & 41.5 & 57.1 & 55.4 & 53.9 & 55.6 & 56.9 & 59.8 & 61.5 & 62.9 & 64.1 & 65.4 \\ \hline
729 & 1.577 & 45.3 & 59.2 & 57 & 55.9 & 57.6 & 59.4 & 62.7 & 64.6 & 66 & 67.3 & 68.5 \\ \hline
730 & 1.572 & 47.8 & 59.6 & 57.8 & 57.1 & 58.4 & 60.5 & 64 & 66 & 67.4 & 68.6 & 69.7 \\ \hline
731 & 1.566 & 49 & 58.7 & 57.6 & 57.3 & 57.9 & 60.1 & 63.5 & 65.6 & 66.9 & 68 & 68.9 \\ \hline
732 & 1.561 & 48.5 & 56.6 & 56.4 & 56.5 & 56.3 & 58.3 & 61.4 & 63.3 & 64.5 & 65.4 & 65.9 \\ \hline
733 & 1.556 & 46.1 & 53.6 & 54.4 & 55 & 54 & 55.4 & 57.9 & 59.6 & 60.5 & 61.2 & 61.6 \\ \hline
734 & 1.55 & 41.8 & 49.7 & 51.8 & 52.9 & 51.6 & 52 & 53.6 & 54.8 & 55.5 & 56 & 56.3 \\ \hline
735 & 1.545 & 36.7 & 45.4 & 49 & 50.7 & 49.6 & 49 & 49.6 & 50.2 & 50.7 & 50.9 & 51.1 \\ \hline
736 & 1.54 & 31.2 & 40.8 & 46.2 & 48.6 & 48 & 46.9 & 46.4 & 46.5 & 46.6 & 46.6 & 46.5 \\ \hline
737 & 1.534 & 26 & 36.2 & 43.4 & 46.9 & 46.8 & 45.9 & 44.7 & 44.1 & 43.8 & 43.7 & 43.4 \\ \hline
738 & 1.529 & 21.6 & 31.6 & 40.2 & 45.3 & 46 & 45.5 & 44.3 & 43.2 & 42.5 & 42.1 & 41.8 \\ \hline
739 & 1.524 & 18 & 27.4 & 36.6 & 43.6 & 44.9 & 45.5 & 44.8 & 43.6 & 42.8 & 42.2 & 41.7 \\ \hline
740 & 1.518 & 15.1 & 23.4 & 32.7 & 41.2 & 43.5 & 45.5 & 45.7 & 44.7 & 44 & 43.3 & 42.7 \\ \hline
741 & 1.513 & 13.1 & 20.3 & 29 & 38.8 & 41.8 & 45.1 & 46.6 & 46.3 & 45.8 & 45.3 & 44.8 \\ \hline
742 & 1.508 & 11.5 & 17.8 & 25.4 & 35.9 & 39.5 & 44.2 & 47 & 47.4 & 47.4 & 47.2 & 46.9 \\ \hline
743 & 1.502 & 10.3 & 15.8 & 22.3 & 32.5 & 36.7 & 42.9 & 47 & 48.3 & 48.8 & 49 & 49 \\ \hline
744 & 1.497 & 9.2 & 14.2 & 19.6 & 28.9 & 33.2 & 40.6 & 46.3 & 48.6 & 49.7 & 50.2 & 50.8 \\ \hline
745 & 1.492 & 8.4 & 13 & 17.4 & 25.6 & 29.6 & 37.7 & 44.7 & 48 & 49.8 & 50.8 & 51.8 \\ \hline
746 & 1.486 & 7.6 & 11.8 & 15.7 & 22.5 & 26.3 & 34.3 & 42.3 & 46.4 & 48.9 & 50.4 & 51.9 \\ \hline
747 & 1.481 & 7.2 & 11 & 14.5 & 20.1 & 23.5 & 31 & 39.4 & 44.1 & 47.2 & 49.2 & 51.2 \\ \hline
748 & 1.476 & 6.8 & 10.5 & 13.4 & 18.1 & 20.9 & 27.8 & 36.2 & 41.3 & 44.9 & 47.4 & 49.7 \\ \hline
749 & 1.47 & 6.5 & 10.1 & 12.6 & 16.8 & 19.2 & 25.2 & 33.1 & 38.4 & 42.3 & 45.1 & 47.9 \\ \hline
750 & 1.465 & 6.4 & 9.8 & 12.1 & 15.6 & 17.9 & 23.1 & 30.3 & 35.3 & 39.2 & 42.3 & 45.5 \\ \hline
751 & 1.46 & 6.3 & 9.7 & 11.7 & 14.8 & 17 & 21.5 & 27.9 & 32.7 & 36.4 & 39.7 & 43.1 \\ \hline
752 & 1.454 & 6 & 9.6 & 11.7 & 14.4 & 16.3 & 20.4 & 26.3 & 30.7 & 34.4 & 37.7 & 41.4 \\ \hline
753 & 1.449 & 6 & 9.6 & 11.7 & 14 & 16 & 19.9 & 25.4 & 29.8 & 33.6 & 36.9 & 40.6 \\ \hline
754 & 1.444 & 6 & 9.6 & 11.7 & 14 & 16 & 19.6 & 25.2 & 29.7 & 33.6 & 36.9 & 40.7 \\ \hline
755 & 1.438 & 6 & 9.7 & 11.7 & 14 & 16.2 & 20 & 25.8 & 30.4 & 34.4 & 37.8 & 41.6 \\ \hline
756 & 1.433 & 6 & 9.8 & 11.8 & 14.4 & 16.6 & 20.8 & 27.2 & 32 & 36 & 39.3 & 42.8 \\ \hline
757 & 1.428 & 6.2 & 10.1 & 12.2 & 14.8 & 17.4 & 22.3 & 29.2 & 34.1 & 38 & 41.2 & 44.4 \\ \hline
758 & 1.422 & 6.3 & 10.2 & 12.6 & 15.3 & 18.6 & 24.2 & 31.6 & 36.3 & 40 & 42.8 & 45.9 \\ \hline
759 & 1.417 & 6.4 & 10.6 & 13.4 & 16.4 & 20.5 & 26.6 & 34 & 38.5 & 42 & 44.6 & 47.5 \\ \hline
760 & 1.411 & 6.4 & 11 & 14.2 & 17.6 & 22.5 & 28.9 & 36.4 & 40.5 & 43.8 & 46.3 & 48.9 \\ \hline
761 & 1.406 & 6.7 & 11.4 & 15.4 & 19.6 & 25.2 & 31.5 & 38.7 & 42.7 & 45.9 & 48.2 & 50.4 \\ \hline
762 & 1.401 & 6.8 & 12.2 & 17 & 21.7 & 28 & 34 & 40.9 & 44.8 & 47.6 & 49.7 & 51.4 \\ \hline
763 & 1.395 & 7.1 & 13.1 & 19 & 24.1 & 30.6 & 36.4 & 43 & 46.6 & 48.8 & 50.5 & 51.5 \\ \hline
764 & 1.39 & 7.5 & 14.3 & 21.3 & 26.6 & 33 & 38.5 & 44.7 & 47.8 & 49.4 & 50.3 & 50.6 \\ \hline
765 & 1.385 & 7.9 & 16 & 24 & 29.2 & 35.2 & 40.6 & 46 & 48.2 & 49.1 & 49.1 & 48.9 \\ \hline
766 & 1.379 & 8.4 & 18.1 & 26.6 & 31.6 & 37.6 & 42.2 & 46.5 & 47.7 & 47.9 & 47.2 & 46.5 \\ \hline
767 & 1.374 & 9.2 & 20.6 & 29.4 & 34.1 & 39.6 & 43.6 & 46.3 & 46.5 & 45.9 & 45.2 & 44.3 \\ \hline
768 & 1.369 & 10 & 23.4 & 32.2 & 36.5 & 41.5 & 44.4 & 45.3 & 44.8 & 43.9 & 43.2 & 42.4 \\ \hline
769 & 1.363 & 11.4 & 26.6 & 35.4 & 39.1 & 43.1 & 44.5 & 44.4 & 43.5 & 42.7 & 42 & 41.6 \\ \hline
770 & 1.358 & 13.2 & 30.1 & 38.5 & 41.5 & 44.3 & 44.4 & 43.7 & 42.8 & 42.3 & 41.9 & 41.9 \\ \hline
771 & 1.353 & 15.6 & 34.1 & 42.1 & 43.9 & 45.6 & 44.8 & 43.9 & 43.2 & 43.1 & 43.1 & 43.6 \\ \hline
772 & 1.347 & 19 & 38.5 & 45.6 & 46.1 & 46.9 & 45.8 & 45.3 & 44.9 & 45.2 & 45.5 & 46.4 \\ \hline
773 & 1.342 & 23.3 & 43.3 & 49.1 & 48.1 & 48.7 & 47.7 & 47.7 & 48 & 48.4 & 49.1 & 50.2 \\ \hline
774 & 1.337 & 28.1 & 48 & 51.8 & 50 & 50.8 & 50.1 & 51 & 51.6 & 52.4 & 53.1 & 54.5 \\ \hline
775 & 1.331 & 33.3 & 52.1 & 54.1 & 51.9 & 53.1 & 52.9 & 54.5 & 55.4 & 56.5 & 57.5 & 58.8 \\ \hline
776 & 1.326 & 37.8 & 55.3 & 55.6 & 53.4 & 54.9 & 55.4 & 57.7 & 58.9 & 60.1 & 61.2 & 62.7 \\ \hline
777 & 1.321 & 41.7 & 58.1 & 56.8 & 54.9 & 56.6 & 57.8 & 60.6 & 62.1 & 63.5 & 64.7 & 66 \\ \hline
778 & 1.315 & 44.9 & 60.1 & 57.8 & 56.3 & 58.1 & 59.8 & 63.1 & 64.8 & 66.3 & 67.5 & 68.8 \\ \hline
779 & 1.31 & 47.6 & 61.3 & 58.7 & 57.6 & 59.5 & 61.3 & 65 & 67 & 68.4 & 69.8 & 71 \\ \hline
780 & 1.305 & 49.7 & 61.7 & 59.2 & 58.6 & 60.1 & 62.2 & 66.1 & 68.2 & 69.6 & 70.9 & 71.9 \\ \hline
781 & 1.299 & 51.2 & 61.2 & 59.3 & 59 & 59.9 & 62.3 & 66.1 & 68.2 & 69.6 & 70.8 & 71.6 \\ \hline
782 & 1.294 & 51.2 & 59.6 & 58.5 & 58.6 & 58.7 & 61.1 & 64.5 & 66.6 & 68 & 69 & 69.7 \\ \hline
783 & 1.289 & 49.9 & 57.3 & 57.1 & 57.6 & 56.8 & 58.8 & 61.9 & 63.8 & 65 & 65.8 & 66.3 \\ \hline
784 & 1.283 & 47 & 54.1 & 55 & 55.8 & 54.4 & 55.8 & 58.3 & 59.9 & 60.8 & 61.5 & 61.8 \\ \hline
785 & 1.278 & 43 & 50.7 & 52.7 & 53.7 & 52.2 & 52.6 & 54.3 & 55.6 & 56.3 & 56.7 & 56.9 \\ \hline
786 & 1.273 & 38.1 & 46.8 & 50.2 & 51.6 & 50.1 & 49.6 & 50.4 & 51.2 & 51.7 & 52 & 52.1 \\ \hline
787 & 1.267 & 33.1 & 42.9 & 47.9 & 49.7 & 48.5 & 47.5 & 47.2 & 47.5 & 47.7 & 47.9 & 47.9 \\ \hline
788 & 1.262 & 28.3 & 38.8 & 45.4 & 48.1 & 47.5 & 46.2 & 45.2 & 45 & 44.9 & 44.7 & 44.7 \\ \hline
789 & 1.256 & 24 & 34.8 & 43 & 46.8 & 47 & 45.8 & 44.4 & 43.5 & 43.2 & 42.8 & 42.7 \\ \hline
790 & 1.251 & 20.4 & 30.9 & 40.2 & 45.6 & 46.4 & 45.9 & 44.3 & 43.1 & 42.5 & 41.9 & 41.6 \\ \hline
791 & 1.246 & 17.6 & 27.3 & 37.2 & 44.4 & 46 & 46.3 & 45.1 & 43.8 & 43 & 42.3 & 41.7 \\ \hline
792 & 1.24 & 15.2 & 23.9 & 33.9 & 42.8 & 44.8 & 46.5 & 46.3 & 45.1 & 44.2 & 43.4 & 42.7 \\ \hline
793 & 1.235 & 13.5 & 21.1 & 30.7 & 40.8 & 43.7 & 46.6 & 47.5 & 46.7 & 46 & 45.3 & 44.7 \\ \hline
794 & 1.23 & 11.9 & 18.7 & 27.4 & 38.4 & 42 & 46.2 & 48.3 & 48.2 & 47.8 & 47.3 & 46.8 \\ \hline
795 & 1.224 & 10.7 & 16.8 & 24.4 & 35.6 & 39.6 & 45.5 & 48.9 & 49.6 & 49.5 & 49.3 & 49.1 \\ \hline
796 & 1.219 & 9.6 & 15.2 & 21.7 & 32.7 & 37 & 44.2 & 48.9 & 50.4 & 50.8 & 51 & 51.1 \\ \hline
797 & 1.214 & 8.8 & 14 & 19.7 & 29.7 & 34.2 & 42.4 & 48.5 & 50.7 & 51.9 & 52.4 & 52.8 \\ \hline
798 & 1.208 & 8.4 & 12.9 & 17.8 & 26.8 & 31.2 & 39.9 & 47.3 & 50.4 & 52 & 53 & 53.8 \\ \hline
799 & 1.203 & 7.8 & 12.2 & 16.5 & 24.1 & 28.4 & 37.1 & 45.5 & 49.4 & 51.6 & 53 & 54.4 \\ \hline
800 & 1.198 & 7.4 & 11.7 & 15.3 & 21.7 & 25.6 & 33.9 & 43 & 47.5 & 50.4 & 52.3 & 54 \\ \hline
801 & 1.192 & 7.2 & 11.3 & 14.5 & 20 & 23.2 & 31 & 40.1 & 45.3 & 48.7 & 51 & 53.1 \\ \hline
802 & 1.187 & 6.9 & 10.9 & 13.7 & 18.4 & 21.2 & 28.2 & 37 & 42.6 & 46.3 & 49 & 51.4 \\ \hline
803 & 1.182 & 6.9 & 10.8 & 13.3 & 17.2 & 19.6 & 25.9 & 34.1 & 39.7 & 43.6 & 46.5 & 49.4 \\ \hline
804 & 1.176 & 6.8 & 10.6 & 12.9 & 16.4 & 18.5 & 23.9 & 31.6 & 36.9 & 40.9 & 44.1 & 47.1 \\ \hline
805 & 1.171 & 6.8 & 10.6 & 12.7 & 15.6 & 17.6 & 22.4 & 29.5 & 34.5 & 38.5 & 41.6 & 44.9 \\ \hline
806 & 1.166 & 6.7 & 10.3 & 12.3 & 15.2 & 16.9 & 21.2 & 27.6 & 32.3 & 36.2 & 39.4 & 42.9 \\ \hline
807 & 1.16 & 6.6 & 10.2 & 12.2 & 14.8 & 16.5 & 20.4 & 26.1 & 30.7 & 34.5 & 37.8 & 41.4 \\ \hline
808 & 1.155 & 6.4 & 10.1 & 11.9 & 14.7 & 16.4 & 20 & 25.3 & 29.7 & 33.4 & 36.7 & 40.5 \\ \hline
809 & 1.15 & 6.4 & 10.1 & 11.9 & 14.6 & 16.4 & 20 & 25.1 & 29.4 & 33.3 & 36.6 & 40.5 \\ \hline
810 & 1.144 & 6.4 & 9.9 & 11.8 & 14.5 & 16.4 & 20 & 25.4 & 29.8 & 33.7 & 37.1 & 41 \\ \hline
811 & 1.139 & 6.4 & 9.9 & 11.8 & 14.4 & 16.5 & 20.4 & 26.2 & 30.8 & 34.9 & 38.3 & 42.2 \\ \hline
812 & 1.134 & 6.4 & 9.8 & 11.8 & 14.4 & 16.8 & 20.9 & 27.4 & 32.3 & 36.3 & 39.8 & 43.6 \\ \hline
813 & 1.128 & 6.4 & 9.8 & 12.1 & 14.6 & 17.5 & 22 & 29 & 34.2 & 38.3 & 41.5 & 45.1 \\ \hline
814 & 1.123 & 6.4 & 9.8 & 12.4 & 14.9 & 18.3 & 23.3 & 30.8 & 36 & 40.1 & 43.2 & 46.6 \\ \hline
815 & 1.118 & 6.4 & 10.1 & 12.8 & 15.5 & 19.3 & 24.9 & 32.8 & 38 & 41.9 & 44.8 & 48.1 \\ \hline
816 & 1.112 & 6.4 & 10.2 & 13.1 & 16.2 & 20.6 & 26.9 & 35.1 & 40 & 43.6 & 46.5 & 49.6 \\ \hline
817 & 1.107 & 6.4 & 10.6 & 13.8 & 17.2 & 22.6 & 29.1 & 37.4 & 42 & 45.6 & 48.3 & 51.2 \\ \hline
818 & 1.101 & 6.4 & 10.9 & 14.6 & 18.4 & 24.6 & 31.3 & 39.4 & 43.8 & 47.3 & 49.9 & 52.4 \\ \hline
819 & 1.096 & 6.4 & 11.3 & 15.7 & 20 & 26.8 & 33.6 & 41.4 & 45.7 & 48.9 & 51.3 & 53.5 \\ \hline
820 & 1.091 & 6.5 & 11.9 & 17 & 22 & 28.9 & 35.6 & 43.3 & 47.4 & 50.2 & 52.2 & 53.9 \\ \hline
821 & 1.085 & 6.8 & 12.7 & 18.8 & 24.1 & 31.3 & 37.8 & 44.9 & 48.7 & 51 & 52.6 & 53.6 \\ \hline
822 & 1.08 & 6.9 & 13.6 & 20.7 & 26.4 & 33.4 & 39.6 & 46.1 & 49.5 & 51.1 & 52 & 52.4 \\ \hline
823 & 1.075 & 7.3 & 14.9 & 22.9 & 28.7 & 35.4 & 41.2 & 47.1 & 49.7 & 50.7 & 51 & 50.8 \\ \hline
824 & 1.069 & 7.7 & 16.5 & 25.1 & 30.7 & 37.2 & 42.5 & 47.6 & 49.4 & 49.7 & 49.4 & 48.8 \\ \hline
825 & 1.064 & 8.3 & 18.4 & 27.4 & 32.8 & 39.1 & 43.9 & 47.8 & 48.6 & 48.4 & 47.7 & 46.7 \\ \hline
826 & 1.059 & 8.9 & 20.5 & 29.8 & 35.1 & 40.9 & 44.9 & 47.5 & 47.5 & 46.8 & 45.7 & 44.7 \\ \hline
827 & 1.053 & 9.8 & 23.1 & 32.6 & 37.5 & 42.7 & 45.8 & 47 & 46.3 & 45.2 & 44.2 & 43.2 \\ \hline
828 & 1.048 & 10.8 & 25.9 & 35.4 & 39.8 & 44.1 & 46.1 & 46.1 & 45 & 43.9 & 43 & 42.2 \\ \hline
829 & 1.043 & 12.3 & 29 & 38.2 & 41.8 & 45.3 & 46 & 45 & 43.9 & 43 & 42.2 & 41.8 \\ \hline
830 & 1.037 & 13.9 & 32.1 & 40.9 & 43.5 & 46 & 45.6 & 44.2 & 43.2 & 42.6 & 42.1 & 42 \\ \hline
831 & 1.032 & 15.9 & 35.3 & 43.5 & 45.1 & 46.5 & 45.4 & 44 & 43.3 & 43 & 42.9 & 43 \\ \hline
832 & 1.027 & 18.3 & 38.4 & 45.9 & 46.4 & 47.2 & 45.7 & 44.6 & 44.1 & 44.2 & 44.3 & 44.9 \\ \hline
833 & 1.021 & 21.1 & 41.6 & 48.2 & 47.8 & 48.3 & 46.5 & 45.9 & 45.8 & 46.2 & 46.5 & 47.3 \\ \hline
834 & 1.016 & 24.2 & 45 & 50.4 & 49 & 49.2 & 47.8 & 47.8 & 47.9 & 48.6 & 49.1 & 50.2 \\ \hline
835 & 1.011 & 27.8 & 48.5 & 52.4 & 50.4 & 50.8 & 49.7 & 50.4 & 50.9 & 51.8 & 52.4 & 53.7 \\ \hline
836 & 1.005 & 31.8 & 51.7 & 54.1 & 51.6 & 52.6 & 52 & 53.4 & 54.1 & 55.1 & 56 & 57.5 \\ \hline
837 & 1 & 35.9 & 54.5 & 55.4 & 53 & 54.4 & 54.4 & 56.3 & 57.4 & 58.6 & 59.6 & 61.1 \\ \hline
838 & 0.995 & 39.5 & 56.9 & 56.5 & 54.2 & 56 & 56.4 & 59 & 60.3 & 61.7 & 62.8 & 64.4 \\ \hline
839 & 0.989 & 42.8 & 59.1 & 57.6 & 55.5 & 57.6 & 58.7 & 61.7 & 63.2 & 64.7 & 66 & 67.5 \\ \hline
840 & 0.984 & 45.7 & 60.9 & 58.5 & 56.8 & 59.1 & 60.6 & 64 & 65.7 & 67.2 & 68.5 & 70 \\ \hline
841 & 0.979 & 48.4 & 62.3 & 59.4 & 58.2 & 60.4 & 62.4 & 66 & 67.9 & 69.5 & 70.9 & 72.3 \\ \hline
842 & 0.973 & 50.5 & 63.1 & 60.2 & 59.3 & 61.3 & 63.6 & 67.4 & 69.5 & 71.1 & 72.5 & 73.9 \\ \hline
843 & 0.968 & 52.3 & 63.4 & 60.7 & 60.2 & 61.9 & 64.4 & 68.3 & 70.6 & 72.1 & 73.5 & 74.7 \\ \hline
844 & 0.962 & 53.2 & 63 & 60.6 & 60.4 & 61.7 & 64.3 & 68.3 & 70.5 & 72.1 & 73.3 & 74.3 \\ \hline
845 & 0.957 & 53.2 & 61.8 & 59.9 & 60 & 60.7 & 63.4 & 67.3 & 69.5 & 70.9 & 72.1 & 73 \\ \hline
846 & 0.952 & 52.4 & 60 & 58.8 & 59.2 & 59.1 & 61.8 & 65.5 & 67.6 & 69 & 70.1 & 70.8 \\ \hline
847 & 0.946 & 51.1 & 58.1 & 57.6 & 58 & 57.5 & 59.8 & 63.2 & 65.2 & 66.5 & 67.4 & 68 \\ \hline
848 & 0.941 & 49.1 & 55.8 & 56.2 & 56.8 & 55.6 & 57.5 & 60.4 & 62.3 & 63.3 & 64.1 & 64.4 \\ \hline
849 & 0.936 & 46.1 & 53.4 & 54.6 & 55.2 & 53.6 & 54.8 & 57.3 & 58.8 & 59.7 & 60.3 & 60.5 \\ \hline
850 & 0.93 & 42.4 & 50.6 & 52.9 & 53.5 & 51.6 & 52 & 53.7 & 55 & 55.7 & 56.1 & 56.2 \\ \hline
851 & 0.925 & 38.4 & 47.5 & 51.1 & 51.9 & 50 & 49.5 & 50.5 & 51.5 & 52 & 52.2 & 52.4 \\ \hline
852 & 0.92 & 34.6 & 44.5 & 49.4 & 50.5 & 48.8 & 47.7 & 47.9 & 48.5 & 48.8 & 49 & 49.1 \\ \hline
853 & 0.914 & 31.1 & 41.8 & 47.8 & 49.5 & 48.1 & 46.6 & 46.1 & 46.2 & 46.4 & 46.4 & 46.4 \\ \hline
854 & 0.909 & 27.9 & 39.1 & 46.3 & 48.7 & 47.7 & 46 & 44.9 & 44.6 & 44.6 & 44.4 & 44.4 \\ \hline
855 & 0.904 & 25.1 & 36.6 & 44.9 & 48.3 & 47.7 & 46.2 & 44.4 & 43.8 & 43.5 & 43.2 & 43.2 \\ \hline
856 & 0.898 & 22.4 & 33.8 & 43.3 & 48 & 48 & 46.6 & 44.5 & 43.5 & 43 & 42.5 & 42.4 \\ \hline
857 & 0.893 & 20.1 & 31 & 41.4 & 47.5 & 48.1 & 47.3 & 45.2 & 43.7 & 43 & 42.4 & 42 \\ \hline
858 & 0.888 & 18 & 28.1 & 38.6 & 46.4 & 47.9 & 47.8 & 46 & 44.2 & 43.3 & 42.5 & 41.9 \\ \hline
859 & 0.882 & 16 & 25.3 & 35.7 & 44.9 & 47.2 & 48.1 & 46.9 & 45.2 & 44 & 43.1 & 42.3 \\ \hline
860 & 0.877 & 14.4 & 22.9 & 32.9 & 43 & 45.9 & 48.1 & 47.7 & 46.2 & 45.1 & 44.2 & 43.3 \\ \hline
861 & 0.872 & 13.2 & 20.9 & 30.4 & 41.2 & 44.5 & 47.7 & 48.5 & 47.3 & 46.4 & 45.4 & 44.7 \\ \hline
862 & 0.866 & 12 & 19 & 28.1 & 39.3 & 43.1 & 47.3 & 48.9 & 48.2 & 47.5 & 46.6 & 46 \\ \hline
863 & 0.861 & 11.2 & 17.6 & 25.8 & 37.7 & 41.8 & 47 & 49.6 & 49.4 & 48.8 & 48.3 & 47.7 \\ \hline
864 & 0.856 & 10.4 & 16.3 & 23.7 & 35.5 & 40 & 46.4 & 50.3 & 50.6 & 50.4 & 50 & 49.7 \\ \hline
865 & 0.85 & 9.6 & 15.3 & 21.7 & 33.2 & 37.8 & 45.2 & 50.4 & 51.7 & 52 & 51.9 & 51.8 \\ \hline
866 & 0.845 & 9 & 14.2 & 19.8 & 30.6 & 35.2 & 43.5 & 49.7 & 51.7 & 52.6 & 52.9 & 53.2 \\ \hline
867 & 0.84 & 8.6 & 13.4 & 18.5 & 28.4 & 33 & 41.9 & 48.8 & 51.5 & 52.9 & 53.6 & 54.1 \\ \hline
868 & 0.834 & 8.2 & 12.6 & 17.3 & 26.4 & 30.8 & 39.9 & 47.7 & 51 & 52.9 & 53.9 & 54.8 \\ \hline
869 & 0.829 & 7.9 & 12.2 & 16.4 & 24.4 & 28.7 & 38.1 & 46.6 & 50.6 & 52.9 & 54.4 & 55.6 \\ \hline
870 & 0.823 & 7.5 & 11.8 & 15.5 & 22.8 & 26.7 & 36.1 & 45.2 & 49.9 & 52.7 & 54.5 & 56 \\ \hline
871 & 0.818 & 7.2 & 11.4 & 14.9 & 21.2 & 25 & 34 & 43.6 & 48.8 & 52.1 & 54.3 & 56.3 \\ \hline
872 & 0.813 & 7.1 & 11 & 14.2 & 19.6 & 23.3 & 31.8 & 41.7 & 47.2 & 50.9 & 53.5 & 55.8 \\ \hline
873 & 0.807 & 6.9 & 10.8 & 13.7 & 18.7 & 21.9 & 29.8 & 39.6 & 45.5 & 49.4 & 52.3 & 54.9 \\ \hline
874 & 0.802 & 6.8 & 10.5 & 13.3 & 17.8 & 20.8 & 27.9 & 37.4 & 43.4 & 47.5 & 50.7 & 53.3 \\ \hline
875 & 0.797 & 6.8 & 10.5 & 13 & 17 & 19.8 & 26.3 & 35.4 & 41.3 & 45.5 & 48.7 & 51.7 \\ \hline
876 & 0.791 & 6.4 & 10.2 & 12.6 & 16.4 & 19 & 24.8 & 33.2 & 38.9 & 43.2 & 46.5 & 49.7 \\ \hline
877 & 0.786 & 6.4 & 10.1 & 12.6 & 16 & 18.2 & 23.6 & 31.2 & 36.8 & 41.1 & 44.5 & 47.7 \\ \hline
878 & 0.781 & 6.4 & 10.1 & 12.4 & 15.6 & 17.6 & 22.4 & 29.5 & 34.8 & 39 & 42.3 & 45.7 \\ \hline
879 & 0.775 & 6.4 & 10.2 & 12.4 & 15.2 & 17.2 & 21.6 & 28.3 & 33.2 & 37.4 & 40.7 & 44.2 \\ \hline
880 & 0.77 & 6.4 & 10.2 & 12.4 & 15 & 16.8 & 21 & 27.1 & 31.9 & 35.8 & 39.1 & 42.9 \\ \hline
881 & 0.765 & 6.4 & 10.5 & 12.6 & 14.9 & 16.8 & 20.7 & 26.5 & 31.1 & 34.9 & 38.3 & 42 \\ \hline
882 & 0.759 & 6.4 & 10.5 & 12.6 & 14.8 & 16.8 & 20.5 & 26.1 & 30.4 & 34.2 & 37.5 & 41.4 \\ \hline
883 & 0.754 & 6.8 & 10.6 & 12.7 & 14.8 & 16.8 & 20.4 & 25.8 & 30.1 & 33.9 & 37.2 & 41.1 \\ \hline
884 & 0.749 & 6.8 & 10.6 & 12.6 & 14.8 & 16.8 & 20.3 & 25.6 & 29.9 & 33.8 & 37.1 & 41.1 \\ \hline
885 & 0.743 & 6.8 & 10.7 & 12.7 & 14.8 & 16.8 & 20.4 & 25.7 & 30 & 33.9 & 37.2 & 41.4 \\ \hline
886 & 0.738 & 6.8 & 10.7 & 12.7 & 14.8 & 16.8 & 20.4 & 25.9 & 30.4 & 34.3 & 37.7 & 41.9 \\ \hline
887 & 0.733 & 6.8 & 10.7 & 12.9 & 14.8 & 16.9 & 20.5 & 26.3 & 30.8 & 35 & 38.4 & 42.6 \\ \hline
888 & 0.727 & 6.8 & 10.6 & 12.9 & 14.8 & 17.1 & 20.8 & 26.8 & 31.6 & 35.8 & 39.3 & 43.2 \\ \hline
889 & 0.722 & 6.8 & 10.6 & 12.9 & 15 & 17.4 & 21.3 & 27.7 & 32.7 & 37 & 40.5 & 44.3 \\ \hline
890 & 0.717 & 6.5 & 10.5 & 12.9 & 15.1 & 17.8 & 22 & 28.8 & 34 & 38.2 & 41.7 & 45.5 \\ \hline
891 & 0.711 & 6.4 & 10.5 & 12.9 & 15.2 & 18.1 & 22.9 & 30.1 & 35.5 & 39.8 & 43.2 & 46.8 \\ \hline
892 & 0.706 & 6.4 & 10.5 & 12.9 & 15.3 & 18.7 & 23.7 & 31.5 & 37 & 41.3 & 44.6 & 48 \\ \hline
893 & 0.701 & 6.4 & 10.5 & 12.9 & 15.6 & 19.3 & 24.9 & 33.1 & 38.6 & 42.7 & 46 & 49.4 \\ \hline
894 & 0.695 & 6.4 & 10.5 & 12.9 & 16 & 20.1 & 26.1 & 34.6 & 40.1 & 44.1 & 47.3 & 50.5 \\ \hline
895 & 0.69 & 6.4 & 10.5 & 13.3 & 16.4 & 21 & 27.4 & 36 & 41.6 & 45.4 & 48.6 & 51.7 \\ \hline
896 & 0.684 & 6.4 & 10.5 & 13.7 & 16.8 & 22.1 & 28.7 & 37.6 & 42.8 & 46.7 & 49.6 & 52.5 \\ \hline
897 & 0.679 & 6.4 & 10.7 & 14.1 & 17.6 & 23.3 & 30.1 & 39 & 44 & 47.8 & 50.5 & 53.3 \\ \hline
898 & 0.674 & 6.4 & 10.9 & 14.5 & 18.4 & 24.5 & 31.6 & 40.4 & 45.2 & 48.8 & 51.3 & 53.8 \\ \hline
899 & 0.668 & 6.4 & 11.3 & 15.3 & 19.4 & 26 & 33.2 & 41.7 & 46.4 & 49.8 & 52.2 & 54.4 \\ \hline
900 & 0.663 & 6.4 & 11.4 & 16 & 20.4 & 27.6 & 34.6 & 42.9 & 47.5 & 50.6 & 52.9 & 54.9 \\ \hline
901 & 0.658 & 6.5 & 11.8 & 16.9 & 21.6 & 29.2 & 36.2 & 44.2 & 48.7 & 51.6 & 53.7 & 55.3 \\ \hline
902 & 0.652 & 6.6 & 12.3 & 17.8 & 23.1 & 30.8 & 37.8 & 45.7 & 49.9 & 52.4 & 54.2 & 55.4 \\ \hline
903 & 0.647 & 6.8 & 12.9 & 19 & 24.7 & 32.4 & 39.5 & 47 & 50.8 & 53.1 & 54.5 & 55.3 \\ \hline
904 & 0.642 & 6.8 & 13.6 & 20.2 & 26.2 & 34.1 & 41 & 48.1 & 51.3 & 53.2 & 54.2 & 54.6 \\ \hline
905 & 0.636 & 7.2 & 14.3 & 21.8 & 27.9 & 35.7 & 42.3 & 48.9 & 51.7 & 53.2 & 53.8 & 53.8 \\ \hline
906 & 0.631 & 7.5 & 15.1 & 23 & 29.5 & 37.2 & 43.4 & 49.4 & 51.7 & 52.7 & 52.9 & 52.6 \\ \hline
907 & 0.626 & 7.8 & 16.1 & 24.6 & 31.1 & 38.7 & 44.4 & 49.8 & 51.6 & 52.1 & 51.9 & 51.4 \\ \hline
908 & 0.62 & 8 & 17 & 26.2 & 32.5 & 39.9 & 45.2 & 49.9 & 51 & 51.2 & 50.7 & 49.8 \\ \hline
909 & 0.615 & 8.4 & 18.2 & 27.9 & 34 & 41.1 & 45.9 & 49.8 & 50.5 & 50.3 & 49.5 & 48.5 \\ \hline
910 & 0.61 & 8.8 & 19.5 & 29.6 & 35.6 & 42.2 & 46.5 & 49.6 & 49.8 & 49.2 & 48.2 & 47.1 \\ \hline
911 & 0.604 & 9.2 & 21.1 & 31.4 & 37.2 & 43.2 & 46.9 & 49.2 & 49 & 48.1 & 47.1 & 45.9 \\ \hline
912 & 0.599 & 9.9 & 22.7 & 33 & 38.5 & 44.1 & 47.2 & 48.7 & 48.1 & 47 & 45.9 & 44.8 \\ \hline
913 & 0.594 & 10.7 & 24.3 & 34.6 & 39.8 & 44.9 & 47.5 & 48.2 & 47.3 & 46.1 & 44.8 & 43.9 \\ \hline
914 & 0.588 & 11.2 & 25.8 & 36 & 40.9 & 45.5 & 47.3 & 47.4 & 46.2 & 44.9 & 43.9 & 43.1 \\ \hline
915 & 0.583 & 11.8 & 27.4 & 37.5 & 42 & 46 & 47 & 46.4 & 45 & 44 & 43.1 & 42.3 \\ \hline
916 & 0.578 & 12.4 & 29 & 38.8 & 42.8 & 46 & 46.5 & 45.4 & 44.1 & 43.1 & 42.2 & 41.6 \\ \hline
917 & 0.572 & 13.2 & 30.6 & 40.2 & 43.5 & 46.3 & 46.1 & 44.7 & 43.5 & 42.6 & 41.9 & 41.5 \\ \hline
918 & 0.567 & 13.9 & 32.1 & 41.4 & 44.4 & 46.4 & 45.7 & 44.2 & 43 & 42.2 & 41.6 & 41.4 \\ \hline
919 & 0.561 & 14.6 & 33.7 & 42.8 & 45.2 & 46.7 & 45.6 & 43.8 & 42.7 & 42.1 & 41.7 & 41.6 \\ \hline
920 & 0.556 & 15.4 & 35.4 & 44.2 & 46 & 46.8 & 45.4 & 43.8 & 42.7 & 42.3 & 42.1 & 42.1 \\ \hline
921 & 0.551 & 16.6 & 37.5 & 45.7 & 46.8 & 47.2 & 45.5 & 44 & 43.2 & 43 & 42.9 & 43.3 \\ \hline
922 & 0.545 & 17.9 & 39.5 & 47 & 47.6 & 47.6 & 45.8 & 44.6 & 44 & 44.1 & 44.1 & 44.7 \\ \hline
923 & 0.54 & 19.5 & 41.5 & 48.4 & 48.2 & 48.1 & 46.3 & 45.5 & 45.2 & 45.4 & 45.7 & 46.4 \\ \hline
924 & 0.535 & 21.1 & 43.5 & 49.6 & 48.7 & 48.6 & 46.9 & 46.4 & 46.4 & 46.8 & 47.3 & 48.1 \\ \hline
925 & 0.529 & 22.8 & 45.7 & 50.9 & 49.5 & 49.4 & 47.7 & 47.7 & 48 & 48.4 & 49.1 & 50.1 \\ \hline
926 & 0.524 & 24.8 & 47.7 & 52.1 & 50.1 & 50.2 & 48.7 & 48.9 & 49.5 & 50.1 & 51.1 & 52.2 \\ \hline
927 & 0.519 & 27.1 & 49.7 & 53.4 & 50.9 & 51.1 & 49.9 & 50.6 & 51.2 & 52.1 & 53.1 & 54.3 \\ \hline
928 & 0.513 & 29.2 & 51.6 & 54.5 & 51.6 & 52 & 51.1 & 52.2 & 53 & 54.1 & 55 & 56.4 \\ \hline
929 & 0.508 & 31.6 & 53.5 & 55.4 & 52.4 & 53.2 & 52.6 & 54 & 54.9 & 56.1 & 57.2 & 58.7 \\ \hline
930 & 0.503 & 34 & 55 & 56.3 & 53.2 & 54.4 & 54 & 55.8 & 56.8 & 58.1 & 59.3 & 60.8 \\ \hline
931 & 0.497 & 36.5 & 56.6 & 57.1 & 54.1 & 55.6 & 55.6 & 57.7 & 58.9 & 60.3 & 61.5 & 63.2 \\ \hline
932 & 0.492 & 38.7 & 57.9 & 58 & 55 & 56.7 & 56.9 & 59.4 & 60.8 & 62.2 & 63.5 & 65.2 \\ \hline
933 & 0.487 & 40.9 & 59.4 & 58.8 & 55.8 & 57.9 & 58.4 & 61.2 & 62.7 & 64.3 & 65.5 & 67.2 \\ \hline
934 & 0.481 & 43 & 60.8 & 59.6 & 56.8 & 59.1 & 60 & 62.9 & 64.7 & 66.3 & 67.5 & 69.2 \\ \hline
935 & 0.476 & 45.2 & 62.2 & 60.5 & 58 & 60.4 & 61.6 & 64.9 & 66.6 & 68.3 & 69.6 & 71.2 \\ \hline
936 & 0.471 & 47.2 & 63.3 & 61.3 & 59.2 & 61.6 & 63.1 & 66.6 & 68.3 & 70 & 71.4 & 72.9 \\ \hline
937 & 0.465 & 49.2 & 64.4 & 62.1 & 60.1 & 62.8 & 64.5 & 68 & 69.9 & 71.6 & 73 & 74.4 \\ \hline
938 & 0.46 & 50.6 & 65 & 62.5 & 60.9 & 63.6 & 65.4 & 69.1 & 71.1 & 72.7 & 74 & 75.4 \\ \hline
939 & 0.455 & 51.9 & 65.4 & 62.5 & 61.6 & 64 & 66.2 & 69.9 & 71.9 & 73.5 & 74.8 & 76.1 \\ \hline
940 & 0.449 & 52.7 & 65.4 & 62.5 & 61.7 & 64.1 & 66.3 & 70.1 & 72 & 73.6 & 74.9 & 76.2 \\ \hline
941 & 0.444 & 53.4 & 65.4 & 62.5 & 61.8 & 64 & 66.3 & 70.1 & 72.1 & 73.7 & 75 & 76.2 \\ \hline
942 & 0.438 & 53.9 & 65.4 & 62.5 & 61.9 & 64 & 66.3 & 70.1 & 72.1 & 73.7 & 75 & 76.1 \\ \hline
943 & 0.433 & 54.7 & 65.7 & 62.5 & 62.4 & 64.1 & 66.7 & 70.5 & 72.7 & 74.2 & 75.4 & 76.6 \\ \hline
944 & 0.428 & 55.4 & 65.9 & 62.6 & 62.8 & 64.2 & 66.9 & 70.8 & 72.9 & 74.5 & 75.6 & 76.8 \\ \hline
945 & 0.422 & 55.8 & 65.6 & 62.3 & 62.6 & 64 & 66.6 & 70.4 & 72.7 & 74.2 & 75.3 & 76.3 \\ \hline
946 & 0.417 & 55.3 & 64.3 & 61.3 & 61.5 & 62.7 & 65.3 & 69 & 71.1 & 72.6 & 73.7 & 74.7 \\ \hline
947 & 0.412 & 54.1 & 62.6 & 59.7 & 59.9 & 60.8 & 63.3 & 67 & 69.1 & 70.5 & 71.5 & 72.4 \\ \hline
948 & 0.406 & 52.6 & 60.6 & 58.1 & 58.3 & 58.8 & 61.3 & 64.9 & 67 & 68.4 & 69.4 & 70.2 \\ \hline
949 & 0.401 & 51.8 & 59.3 & 57.3 & 57.4 & 57.6 & 60.1 & 63.7 & 65.8 & 67.2 & 68.1 & 68.8 \\ \hline
950 & 0.396 & 51.5 & 58.6 & 56.9 & 57.1 & 56.8 & 59.5 & 63.1 & 65.2 & 66.4 & 67.4 & 68.1 \\ \hline
951 & 0.39 & 51.6 & 58.6 & 56.9 & 57.3 & 56.7 & 59.4 & 62.9 & 64.9 & 66.2 & 67.1 & 67.7 \\ \hline
952 & 0.385 & 51.5 & 58.7 & 57.2 & 57.5 & 56.7 & 59.4 & 62.7 & 64.7 & 65.9 & 66.7 & 67.3 \\ \hline
953 & 0.38 & 51.3 & 58.7 & 57.6 & 57.9 & 56.7 & 59.4 & 62.5 & 64.4 & 65.5 & 66.2 & 66.8 \\ \hline
954 & 0.374 & 50.8 & 58.3 & 57.7 & 57.9 & 56.5 & 58.9 & 62 & 63.6 & 64.8 & 65.4 & 66 \\ \hline
955 & 0.369 & 50 & 57.5 & 57.4 & 57.6 & 56 & 58.2 & 61.1 & 62.8 & 63.9 & 64.3 & 64.8 \\ \hline
956 & 0.364 & 48.8 & 56.4 & 56.9 & 56.9 & 55.2 & 57.1 & 59.7 & 61.5 & 62.4 & 62.9 & 63.3 \\ \hline
957 & 0.358 & 47.5 & 55.4 & 56.4 & 56.4 & 54.3 & 56 & 58.5 & 60.2 & 61.1 & 61.5 & 62 \\ \hline
958 & 0.353 & 45.9 & 54.2 & 55.6 & 55.6 & 53.4 & 54.8 & 57.1 & 58.6 & 59.5 & 59.9 & 60.3 \\ \hline
959 & 0.348 & 44.4 & 52.9 & 54.9 & 54.8 & 52.5 & 53.6 & 55.8 & 57.1 & 58 & 58.3 & 58.6 \\ \hline
960 & 0.342 & 42.8 & 51.5 & 54.1 & 54 & 51.4 & 52.3 & 54.2 & 55.5 & 56.4 & 56.7 & 57 \\ \hline
961 & 0.337 & 41.4 & 50.4 & 53.5 & 53.6 & 50.8 & 51.3 & 53 & 54.2 & 54.9 & 55.2 & 55.5 \\ \hline
962 & 0.332 & 40 & 49.3 & 52.9 & 53.2 & 50.4 & 50.4 & 51.8 & 53 & 53.6 & 53.9 & 54.2 \\ \hline
963 & 0.326 & 38.4 & 48.3 & 52.3 & 52.6 & 50 & 49.6 & 50.7 & 51.6 & 52.3 & 52.5 & 52.7 \\ \hline
964 & 0.321 & 36.6 & 47 & 51.6 & 51.9 & 49.5 & 48.6 & 49.4 & 50.2 & 50.7 & 50.9 & 51.2 \\ \hline
965 & 0.316 & 34.7 & 45.4 & 50.8 & 51.3 & 48.9 & 47.7 & 48.1 & 48.7 & 49.2 & 49.4 & 49.5 \\ \hline
966 & 0.31 & 32.8 & 43.8 & 49.7 & 50.7 & 48.4 & 46.8 & 46.9 & 47.3 & 47.7 & 47.9 & 48 \\ \hline
967 & 0.305 & 31.2 & 42.2 & 48.6 & 49.9 & 48 & 46.1 & 45.8 & 46.1 & 46.4 & 46.6 & 46.7 \\ \hline
968 & 0.299 & 29.6 & 40.6 & 47.4 & 49.2 & 47.3 & 45.3 & 44.8 & 45 & 45.2 & 45.2 & 45.3 \\ \hline
969 & 0.294 & 28.4 & 39.5 & 46.6 & 48.7 & 47 & 45 & 44.3 & 44.3 & 44.4 & 44.4 & 44.5 \\ \hline
970 & 0.289 & 27.3 & 38.5 & 46.1 & 48.4 & 46.9 & 44.9 & 44 & 43.8 & 43.8 & 43.8 & 43.7 \\ \hline
971 & 0.283 & 26.6 & 37.8 & 45.9 & 48.7 & 47.2 & 45.2 & 43.9 & 43.6 & 43.5 & 43.4 & 43.3 \\ \hline
972 & 0.278 & 25.6 & 37.2 & 45.7 & 48.8 & 47.6 & 45.3 & 43.9 & 43.3 & 43.2 & 43 & 42.9 \\ \hline
973 & 0.273 & 24.9 & 36.7 & 45.7 & 49.2 & 48.1 & 45.8 & 43.9 & 43.3 & 43.1 & 42.8 & 42.7 \\ \hline
974 & 0.267 & 24.1 & 36 & 45.4 & 49.6 & 48.7 & 46.3 & 44 & 43.3 & 43 & 42.7 & 42.4 \\ \hline
975 & 0.262 & 23.3 & 35.3 & 45.2 & 49.9 & 49.2 & 46.8 & 44.3 & 43.3 & 42.9 & 42.6 & 42.3 \\ \hline
976 & 0.257 & 22.5 & 34.6 & 44.7 & 49.8 & 49.5 & 47.1 & 44.4 & 43.3 & 42.8 & 42.4 & 42 \\ \hline
977 & 0.251 & 21.9 & 33.9 & 44.2 & 49.8 & 49.8 & 47.6 & 44.8 & 43.4 & 42.8 & 42.4 & 42 \\ \hline
978 & 0.246 & 21.4 & 33.2 & 43.7 & 49.6 & 49.9 & 47.9 & 45.1 & 43.5 & 42.8 & 42.4 & 42 \\ \hline
979 & 0.241 & 20.9 & 32.5 & 43.2 & 49.6 & 50 & 48.2 & 45.4 & 43.6 & 42.9 & 42.4 & 42 \\ \hline
980 & 0.235 & 20.4 & 31.7 & 42.5 & 49.2 & 50 & 48.5 & 45.6 & 43.6 & 42.9 & 42.4 & 42 \\ \hline
981 & 0.23 & 20 & 31.1 & 41.9 & 49.1 & 50 & 48.9 & 46 & 44 & 43.3 & 42.6 & 42.2 \\ \hline
982 & 0.225 & 19.6 & 30.3 & 41.3 & 48.8 & 50 & 49.2 & 46.4 & 44.4 & 43.4 & 42.7 & 42.2 \\ \hline
983 & 0.219 & 18.9 & 29.6 & 40.6 & 48.5 & 50 & 49.4 & 46.7 & 44.7 & 43.6 & 42.8 & 42.2 \\ \hline
984 & 0.214 & 18.4 & 28.7 & 39.8 & 48.1 & 49.6 & 49.4 & 46.8 & 44.8 & 43.6 & 42.8 & 42.2 \\ \hline
985 & 0.209 & 18 & 27.9 & 39 & 47.7 & 49.5 & 49.5 & 47.1 & 45.1 & 43.6 & 42.9 & 42.2 \\ \hline
986 & 0.203 & 17.3 & 27 & 38.2 & 47.3 & 49.2 & 49.6 & 47.3 & 45.2 & 43.6 & 42.9 & 42.2 \\ \hline
987 & 0.198 & 16.8 & 26.2 & 37.5 & 46.9 & 49 & 49.7 & 47.6 & 45.4 & 43.8 & 43.1 & 42.3 \\ \hline
988 & 0.193 & 16.1 & 25.5 & 36.7 & 46.5 & 48.7 & 49.7 & 47.9 & 45.5 & 44 & 43.2 & 42.3 \\ \hline
989 & 0.187 & 15.7 & 24.9 & 35.9 & 46.1 & 48.5 & 49.9 & 48.2 & 45.9 & 44.4 & 43.5 & 42.6 \\ \hline
990 & 0.182 & 15.2 & 24.2 & 35.2 & 45.7 & 48.4 & 50 & 48.6 & 46.3 & 44.8 & 43.8 & 42.9 \\ \hline
991 & 0.176 & 14.8 & 23.7 & 34.6 & 45.5 & 48.3 & 50.2 & 49.1 & 46.9 & 45.4 & 44.3 & 43.4 \\ \hline
992 & 0.171 & 14.4 & 23.1 & 34 & 45.2 & 48 & 50.3 & 49.5 & 47.4 & 45.9 & 44.7 & 43.8 \\ \hline
993 & 0.166 & 14 & 22.6 & 33.3 & 44.7 & 47.9 & 50.6 & 49.9 & 48 & 46.4 & 45.3 & 44.3 \\ \hline
994 & 0.16 & 13.6 & 21.9 & 32.5 & 44.2 & 47.5 & 50.5 & 50.2 & 48.4 & 46.9 & 45.7 & 44.7 \\ \hline
995 & 0.155 & 13.3 & 21.4 & 31.7 & 43.6 & 47.1 & 50.5 & 50.5 & 48.8 & 47.3 & 46.2 & 45.2 \\ \hline
996 & 0.15 & 13 & 21 & 31.1 & 43 & 46.5 & 50.3 & 50.6 & 49 & 47.7 & 46.5 & 45.4 \\ \hline
997 & 0.144 & 12.7 & 20.6 & 30.5 & 42.4 & 46 & 50.1 & 50.6 & 49.2 & 47.9 & 46.7 & 45.7 \\ \hline
998 & 0.139 & 12.3 & 20.1 & 29.9 & 42 & 45.6 & 49.8 & 50.5 & 49.2 & 48 & 46.8 & 45.9 \\ \hline
999 & 0.134 & 12 & 19.8 & 29.4 & 41.6 & 45.2 & 49.7 & 50.8 & 49.5 & 48.3 & 47.2 & 46.2 \\ \hline
1000 & 0.128 & 12 & 19.4 & 28.9 & 41.1 & 44.7 & 49.6 & 50.9 & 49.8 & 48.7 & 47.5 & 46.6 \\ \hline
1001 & 0.123 & 12 & 19.2 & 28.5 & 40.8 & 44.5 & 49.6 & 51.2 & 50.1 & 49.1 & 48 & 47.1 \\ \hline
1002 & 0.118 & 12 & 18.9 & 28.1 & 40.4 & 44.1 & 49.6 & 51.4 & 50.4 & 49.4 & 48.4 & 47.5 \\ \hline
1003 & 0.112 & 11.9 & 18.8 & 27.8 & 40.4 & 44.1 & 49.8 & 51.8 & 50.9 & 49.9 & 48.9 & 48 \\ \hline
1004 & 0.107 & 11.6 & 18.4 & 27.5 & 40.1 & 44 & 49.8 & 52.1 & 51.3 & 50.3 & 49.4 & 48.5 \\ \hline
1005 & 0.102 & 11.5 & 18.2 & 27.3 & 39.9 & 44 & 50 & 52.5 & 51.7 & 50.9 & 50 & 49.2 \\ \hline
1006 & 0.096 & 11.2 & 18.1 & 26.8 & 39.5 & 43.8 & 50 & 52.8 & 52.2 & 51.3 & 50.4 & 49.6 \\ \hline
1007 & 0.091 & 11.2 & 17.9 & 26.5 & 39.1 & 43.6 & 50 & 52.9 & 52.6 & 51.7 & 50.8 & 50 \\ \hline
1008 & 0.086 & 11.1 & 17.6 & 26.1 & 38.8 & 43.2 & 49.6 & 52.9 & 52.6 & 51.8 & 51 & 50.1 \\ \hline
1009 & 0.08 & 11 & 17.5 & 26.1 & 38.6 & 43 & 49.6 & 52.9 & 52.7 & 52 & 51.1 & 50.3 \\ \hline
1010 & 0.075 & 10.9 & 17.4 & 25.8 & 38.5 & 42.9 & 49.6 & 52.9 & 52.7 & 52.1 & 51.2 & 50.4 \\ \hline
1011 & 0.07 & 10.8 & 17.4 & 25.8 & 38.5 & 42.9 & 49.7 & 53.1 & 53.1 & 52.4 & 51.5 & 50.7 \\ \hline
1012 & 0.064 & 10.8 & 17.4 & 25.7 & 38.2 & 42.8 & 49.6 & 53.2 & 53.1 & 52.6 & 51.7 & 50.9 \\ \hline
1013 & 0.059 & 10.8 & 17.4 & 25.5 & 37.9 & 42.7 & 49.6 & 53.3 & 53.3 & 52.8 & 52 & 51.3 \\ \hline
1014 & 0.053 & 10.8 & 17.3 & 25.3 & 37.6 & 42.3 & 49.2 & 53.1 & 53.3 & 52.8 & 52.1 & 51.4 \\ \hline
1015 & 0.048 & 10.8 & 17.4 & 25.3 & 37.4 & 41.9 & 48.9 & 53 & 53.3 & 52.8 & 52.2 & 51.5 \\ \hline
1016 & 0.043 & 10.8 & 17.4 & 25.2 & 37.2 & 41.6 & 48.6 & 52.8 & 53.2 & 52.8 & 52.2 & 51.5 \\ \hline
1017 & 0.037 & 10.8 & 17.5 & 25.2 & 37.2 & 41.6 & 48.6 & 52.8 & 53.2 & 52.9 & 52.2 & 51.6 \\ \hline
1018 & 0.032 & 10.9 & 17.6 & 25.2 & 37.2 & 41.6 & 48.7 & 52.9 & 53.4 & 53 & 52.5 & 51.9 \\ \hline
1019 & 0.027 & 11.2 & 17.8 & 25.2 & 37.2 & 41.6 & 49.1 & 53.4 & 54 & 53.7 & 53.1 & 52.5 \\ \hline
1020 & 0.021 & 11.5 & 17.8 & 25.2 & 37.2 & 41.7 & 49.3 & 53.8 & 54.4 & 54.2 & 53.7 & 53.1 \\ \hline
1021 & 0.016 & 11.6 & 17.8 & 25.3 & 37.2 & 41.9 & 49.6 & 54.2 & 54.8 & 54.7 & 54.2 & 53.7 \\ \hline
1022 & 0.011 & 11.6 & 17.8 & 25.2 & 37.2 & 41.9 & 49.6 & 54.3 & 55.1 & 55 & 54.4 & 53.9 \\ \hline
1023 & 0.005 & 11.6 & 17.7 & 25 & 37.1 & 41.7 & 49.5 & 54.4 & 55.2 & 55.1 & 54.5 & 54.1 \\ \hline
1024 & 0 & 11.6 & 17.6 & 24.7 & 36.8 & 41.3 & 49.2 & 54.2 & 55.1 & 55 & 54.6 & 54.1 \\ \hline
1025 & -0.005 & 11.3 & 17.3 & 24.4 & 36.4 & 40.9 & 48.8 & 53.8 & 54.7 & 54.7 & 54.2 & 53.7 \\ \hline
1026 & -0.011 & 10.9 & 16.9 & 23.8 & 35.8 & 40.4 & 48.1 & 53.1 & 54.1 & 54 & 53.6 & 53.2 \\ \hline
1027 & -0.016 & 10.6 & 16.5 & 23.4 & 35.1 & 39.6 & 47.3 & 52.3 & 53.3 & 53.3 & 52.9 & 52.4 \\ \hline
1028 & -0.021 & 10.3 & 16.1 & 22.9 & 34.5 & 38.8 & 46.5 & 51.4 & 52.5 & 52.5 & 52.1 & 51.6 \\ \hline
1029 & -0.027 & 10.2 & 15.9 & 22.9 & 34.5 & 38.7 & 46.1 & 50.9 & 52 & 52 & 51.7 & 51.2 \\ \hline
1030 & -0.032 & 10 & 15.8 & 22.8 & 34.5 & 38.8 & 46.1 & 50.5 & 51.6 & 51.6 & 51.1 & 50.6 \\ \hline
1031 & -0.037 & 10 & 15.8 & 22.9 & 34.8 & 39 & 46.5 & 50.9 & 51.7 & 51.6 & 51.1 & 50.6 \\ \hline
1032 & -0.043 & 10 & 15.8 & 22.9 & 35 & 39.2 & 46.9 & 51.4 & 52.1 & 52 & 51.5 & 51 \\ \hline
1033 & -0.048 & 10 & 15.8 & 22.9 & 35.2 & 39.6 & 47.4 & 52.1 & 52.9 & 52.8 & 52.3 & 51.8 \\ \hline
1034 & -0.053 & 10 & 15.7 & 22.9 & 35.2 & 39.7 & 47.6 & 52.5 & 53.3 & 53.2 & 52.8 & 52.3 \\ \hline
1035 & -0.059 & 10 & 15.8 & 22.9 & 35.2 & 39.8 & 47.8 & 52.8 & 53.6 & 53.5 & 53.1 & 52.6 \\ \hline
1036 & -0.064 & 10 & 15.8 & 22.9 & 35.2 & 39.9 & 47.8 & 52.8 & 53.7 & 53.7 & 53.3 & 52.8 \\ \hline
1037 & -0.07 & 10 & 15.8 & 23.1 & 35.3 & 40 & 47.9 & 52.9 & 53.9 & 53.8 & 53.4 & 52.9 \\ \hline
1038 & -0.075 & 10 & 15.8 & 23.2 & 35.4 & 40 & 48 & 52.9 & 53.9 & 53.8 & 53.4 & 52.9 \\ \hline
1039 & -0.08 & 10 & 15.9 & 23.3 & 35.6 & 40.1 & 48.1 & 53 & 53.9 & 53.8 & 53.4 & 53 \\ \hline
1040 & -0.086 & 10 & 15.9 & 23.3 & 35.7 & 40.1 & 48.1 & 53 & 53.9 & 53.8 & 53.4 & 52.9 \\ \hline
1041 & -0.091 & 10.1 & 16 & 23.4 & 35.8 & 40.3 & 48.1 & 53 & 53.8 & 53.7 & 53.3 & 52.8 \\ \hline
1042 & -0.096 & 10.1 & 16.1 & 23.4 & 35.9 & 40.3 & 48.1 & 52.9 & 53.6 & 53.4 & 53.1 & 52.5 \\ \hline
1043 & -0.102 & 10.3 & 16.2 & 23.6 & 36 & 40.4 & 48.1 & 52.9 & 53.5 & 53.3 & 52.9 & 52.3 \\ \hline
1044 & -0.107 & 10.3 & 16.2 & 23.7 & 36 & 40.4 & 48.1 & 52.7 & 53.3 & 53 & 52.6 & 51.9 \\ \hline
1045 & -0.112 & 10.4 & 16.3 & 24 & 36.2 & 40.7 & 48.2 & 52.7 & 53.2 & 52.9 & 52.4 & 51.7 \\ \hline
1046 & -0.118 & 10.4 & 16.5 & 24.1 & 36.4 & 40.8 & 48.3 & 52.7 & 53.1 & 52.7 & 52.1 & 51.5 \\ \hline
1047 & -0.123 & 10.4 & 16.6 & 24.2 & 36.8 & 41.2 & 48.6 & 52.8 & 53.1 & 52.7 & 52.1 & 51.4 \\ \hline
1048 & -0.128 & 10.4 & 16.6 & 24.3 & 36.9 & 41.4 & 48.7 & 52.8 & 53.1 & 52.5 & 51.9 & 51.2 \\ \hline
1049 & -0.134 & 10.6 & 16.9 & 24.7 & 37.2 & 41.7 & 48.9 & 52.8 & 53 & 52.5 & 51.8 & 51 \\ \hline
1050 & -0.139 & 10.7 & 17 & 25.1 & 37.4 & 42 & 49.1 & 52.8 & 53 & 52.3 & 51.6 & 50.7 \\ \hline
1051 & -0.144 & 10.8 & 17.4 & 25.5 & 38 & 42.4 & 49.4 & 52.9 & 52.9 & 52.1 & 51.3 & 50.4 \\ \hline
1052 & -0.15 & 10.9 & 17.7 & 25.9 & 38.4 & 42.9 & 49.7 & 52.9 & 52.8 & 51.9 & 51.1 & 50.2 \\ \hline
1053 & -0.155 & 11.2 & 17.8 & 26.5 & 39.1 & 43.3 & 50 & 53.1 & 52.8 & 51.8 & 51 & 50 \\ \hline
1054 & -0.16 & 11.2 & 18 & 26.9 & 39.5 & 43.7 & 50.1 & 53.1 & 52.7 & 51.6 & 50.6 & 49.7 \\ \hline
1055 & -0.166 & 11.6 & 18.5 & 27.3 & 40 & 44.2 & 50.5 & 53.1 & 52.5 & 51.4 & 50.4 & 49.4 \\ \hline
1056 & -0.171 & 11.6 & 18.9 & 27.7 & 40.3 & 44.6 & 50.6 & 53 & 52.1 & 51 & 50 & 49 \\ \hline
1057 & -0.176 & 11.6 & 19.3 & 28.1 & 40.7 & 44.8 & 50.6 & 52.8 & 51.8 & 50.6 & 49.6 & 48.6 \\ \hline
1058 & -0.182 & 11.6 & 19.4 & 28.5 & 40.8 & 44.9 & 50.4 & 52.4 & 51.4 & 50.1 & 49.1 & 48.1 \\ \hline
1059 & -0.187 & 12 & 19.8 & 28.9 & 41 & 45.1 & 50.4 & 52 & 50.9 & 49.7 & 48.7 & 47.6 \\ \hline
1060 & -0.193 & 12.4 & 20.1 & 29.2 & 41.1 & 45.1 & 50.1 & 51.5 & 50.2 & 48.9 & 47.9 & 46.9 \\ \hline
1061 & -0.198 & 12.8 & 20.4 & 29.4 & 41.1 & 45.1 & 49.7 & 50.8 & 49.6 & 48.3 & 47.2 & 46.1 \\ \hline
1062 & -0.203 & 12.8 & 20.5 & 29.5 & 41.1 & 44.9 & 49.3 & 50.3 & 48.8 & 47.5 & 46.4 & 45.4 \\ \hline
1063 & -0.209 & 13.2 & 20.8 & 29.7 & 41.1 & 44.9 & 49.1 & 49.9 & 48.4 & 47 & 45.9 & 44.9 \\ \hline
1064 & -0.214 & 13.6 & 21.1 & 30.1 & 41.1 & 44.8 & 48.8 & 49.4 & 47.8 & 46.4 & 45.3 & 44.3 \\ \hline
1065 & -0.219 & 13.9 & 21.5 & 30.5 & 41.4 & 44.8 & 48.4 & 48.8 & 47.2 & 45.8 & 44.6 & 43.8 \\ \hline
1066 & -0.225 & 14 & 21.8 & 30.8 & 41.6 & 44.8 & 48.1 & 48.1 & 46.4 & 45 & 44 & 43 \\ \hline
1067 & -0.23 & 14.4 & 22.1 & 31.3 & 42 & 44.8 & 48 & 47.6 & 45.9 & 44.4 & 43.5 & 42.5 \\ \hline
1068 & -0.235 & 14.5 & 22.4 & 31.7 & 42.2 & 44.9 & 47.7 & 47.1 & 45.3 & 43.8 & 42.8 & 41.9 \\ \hline
1069 & -0.241 & 14.8 & 22.8 & 32.2 & 42.5 & 45.1 & 47.5 & 46.7 & 44.6 & 43.4 & 42.3 & 41.4 \\ \hline
1070 & -0.246 & 14.8 & 23 & 32.6 & 42.8 & 45.1 & 47.2 & 46 & 44 & 42.6 & 41.6 & 40.8 \\ \hline
1071 & -0.251 & 15.2 & 23.4 & 33 & 43.1 & 45.2 & 47 & 45.5 & 43.5 & 42.2 & 41.2 & 40.4 \\ \hline
1072 & -0.257 & 15.3 & 23.8 & 33.5 & 43.5 & 45.2 & 46.8 & 45 & 43.1 & 41.6 & 40.7 & 39.9 \\ \hline
1073 & -0.262 & 15.6 & 24.1 & 34.1 & 43.6 & 45.5 & 46.7 & 44.5 & 42.6 & 41.2 & 40.3 & 39.5 \\ \hline
1074 & -0.267 & 15.7 & 24.3 & 34.6 & 43.6 & 45.6 & 46.3 & 44.1 & 42.2 & 40.8 & 39.9 & 39.1 \\ \hline
1075 & -0.273 & 16 & 24.7 & 34.9 & 43.8 & 45.6 & 46 & 43.7 & 41.7 & 40.4 & 39.6 & 38.9 \\ \hline
1076 & -0.278 & 16.1 & 25.1 & 35.3 & 43.8 & 45.6 & 45.6 & 43.2 & 41.3 & 40.1 & 39.2 & 38.7 \\ \hline
1077 & -0.283 & 16.5 & 25.7 & 35.8 & 44.2 & 45.6 & 45.5 & 43 & 41.1 & 40 & 39.2 & 38.7 \\ \hline
1078 & -0.289 & 16.9 & 26.4 & 36.5 & 44.5 & 45.6 & 45.2 & 42.7 & 40.9 & 40 & 39.2 & 38.7 \\ \hline
1079 & -0.294 & 17.6 & 27.2 & 37.4 & 45 & 46 & 45.2 & 42.7 & 41 & 40.1 & 39.6 & 39 \\ \hline
1080 & -0.299 & 18.4 & 28.2 & 38.5 & 45.7 & 46.4 & 45.2 & 42.7 & 41.3 & 40.5 & 39.9 & 39.4 \\ \hline
1081 & -0.305 & 19.2 & 29.3 & 39.6 & 46.5 & 46.8 & 45.6 & 43.1 & 41.7 & 40.9 & 40.4 & 40.1 \\ \hline
1082 & -0.31 & 19.9 & 30.4 & 40.6 & 47.3 & 47.2 & 45.9 & 43.3 & 42.1 & 41.3 & 40.9 & 40.5 \\ \hline
1083 & -0.316 & 20.7 & 31.4 & 41.7 & 48.1 & 47.7 & 46.1 & 43.6 & 42.4 & 41.8 & 41.4 & 41.1 \\ \hline
1084 & -0.321 & 21.3 & 32.2 & 42.5 & 48.5 & 48 & 46.1 & 43.6 & 42.7 & 42.1 & 41.8 & 41.6 \\ \hline
1085 & -0.326 & 22 & 33.1 & 43.1 & 48.8 & 48.1 & 46 & 43.6 & 42.7 & 42.4 & 42.2 & 42 \\ \hline
1086 & -0.332 & 22.8 & 33.8 & 43.7 & 48.8 & 48 & 45.6 & 43.4 & 42.8 & 42.4 & 42.3 & 42.2 \\ \hline
1087 & -0.337 & 23.6 & 34.8 & 44.5 & 49 & 48 & 45.5 & 43.5 & 42.9 & 42.7 & 42.6 & 42.6 \\ \hline
1088 & -0.342 & 24.4 & 35.7 & 45 & 49 & 47.7 & 45.2 & 43.5 & 42.9 & 42.9 & 42.8 & 42.8 \\ \hline
1089 & -0.348 & 25.2 & 36.6 & 45.7 & 49 & 47.5 & 45 & 43.5 & 43.1 & 43.2 & 43.1 & 43.2 \\ \hline
1090 & -0.353 & 26 & 37.4 & 46.1 & 49 & 47.2 & 44.8 & 43.5 & 43.3 & 43.5 & 43.5 & 43.5 \\ \hline
1091 & -0.358 & 27.1 & 38.4 & 46.6 & 49.1 & 46.9 & 44.8 & 43.6 & 43.7 & 43.9 & 44 & 44.1 \\ \hline
1092 & -0.364 & 28.3 & 39.5 & 47.1 & 49.2 & 46.8 & 44.9 & 44.2 & 44.4 & 44.7 & 44.7 & 44.9 \\ \hline
1093 & -0.369 & 29.6 & 40.7 & 47.7 & 49.2 & 46.8 & 45.1 & 44.7 & 45.1 & 45.5 & 45.5 & 45.7 \\ \hline
1094 & -0.374 & 30.8 & 41.9 & 48.5 & 49.2 & 46.8 & 45.4 & 45.4 & 45.9 & 46.3 & 46.4 & 46.8 \\ \hline
1095 & -0.38 & 32.2 & 43.3 & 49.3 & 49.6 & 47 & 46.1 & 46.4 & 47.1 & 47.5 & 47.7 & 48 \\ \hline
1096 & -0.385 & 33.6 & 44.6 & 50.1 & 50.2 & 47.3 & 46.8 & 47.4 & 48.2 & 48.7 & 48.9 & 49.2 \\ \hline
1097 & -0.39 & 35.1 & 45.9 & 51 & 50.8 & 47.8 & 47.7 & 48.6 & 49.4 & 49.9 & 50.2 & 50.5 \\ \hline
1098 & -0.396 & 36.3 & 47.2 & 51.7 & 51.2 & 48.3 & 48.5 & 49.5 & 50.6 & 51.2 & 51.5 & 51.7 \\ \hline
1099 & -0.401 & 37.9 & 48.5 & 52.5 & 51.8 & 48.9 & 49.4 & 50.8 & 51.9 & 52.6 & 53 & 53.3 \\ \hline
1100 & -0.406 & 39.3 & 49.7 & 53 & 52.3 & 49.6 & 50.2 & 51.9 & 53.2 & 53.9 & 54.3 & 54.6 \\ \hline
1101 & -0.412 & 40.7 & 51 & 53.8 & 52.9 & 50.3 & 51.3 & 53.2 & 54.5 & 55.2 & 55.6 & 56 \\ \hline
1102 & -0.417 & 42.2 & 52.1 & 54.4 & 53.5 & 51 & 52.4 & 54.5 & 55.8 & 56.6 & 56.9 & 57.4 \\ \hline
1103 & -0.422 & 43.7 & 53.5 & 55.1 & 54.2 & 51.9 & 53.5 & 55.7 & 57.2 & 58 & 58.4 & 58.9 \\ \hline
1104 & -0.428 & 45.1 & 54.6 & 55.7 & 54.9 & 52.7 & 54.6 & 56.9 & 58.5 & 59.3 & 59.7 & 60.2 \\ \hline
1105 & -0.433 & 46.4 & 55.8 & 56.2 & 55.6 & 53.6 & 55.7 & 58.2 & 59.9 & 60.8 & 61.3 & 61.7 \\ \hline
1106 & -0.438 & 47.6 & 56.9 & 56.8 & 56.1 & 54.4 & 56.8 & 59.5 & 61.2 & 62.1 & 62.7 & 63.1 \\ \hline
1107 & -0.444 & 48.8 & 58.1 & 57.4 & 56.9 & 55.5 & 57.9 & 60.7 & 62.5 & 63.3 & 64.2 & 64.7 \\ \hline
1108 & -0.449 & 50 & 59 & 57.8 & 57.6 & 56.2 & 58.8 & 61.8 & 63.6 & 64.5 & 65.4 & 65.9 \\ \hline
1109 & -0.455 & 51 & 59.9 & 58.4 & 58.3 & 57.2 & 59.8 & 63 & 64.8 & 66 & 66.7 & 67.4 \\ \hline
1110 & -0.46 & 52 & 60.8 & 58.9 & 58.9 & 58.1 & 60.8 & 64.2 & 66.1 & 67.2 & 68 & 68.8 \\ \hline
1111 & -0.465 & 53.2 & 61.7 & 59.4 & 59.7 & 59.3 & 61.9 & 65.4 & 67.4 & 68.6 & 69.5 & 70.4 \\ \hline
1112 & -0.471 & 54.1 & 62.5 & 60 & 60.4 & 60.1 & 62.9 & 66.6 & 68.5 & 69.8 & 70.7 & 71.6 \\ \hline
1113 & -0.476 & 54.9 & 63.3 & 60.4 & 61.1 & 61.1 & 63.9 & 67.7 & 69.7 & 71 & 71.9 & 72.9 \\ \hline
1114 & -0.481 & 55.5 & 63.8 & 60.9 & 61.5 & 61.9 & 64.8 & 68.5 & 70.6 & 71.9 & 72.9 & 73.9 \\ \hline
1115 & -0.487 & 56 & 64.3 & 61.3 & 62 & 62.6 & 65.5 & 69.3 & 71.5 & 72.8 & 73.9 & 74.8 \\ \hline
1116 & -0.492 & 56.1 & 64.5 & 61.6 & 62.1 & 63.1 & 66 & 69.8 & 71.9 & 73.3 & 74.5 & 75.3 \\ \hline
1117 & -0.497 & 56.1 & 64.8 & 61.9 & 62.3 & 63.6 & 66.4 & 70.3 & 72.4 & 73.9 & 75.1 & 76.1 \\ \hline
1118 & -0.503 & 55.9 & 64.8 & 62 & 62.4 & 63.9 & 66.6 & 70.5 & 72.6 & 74.1 & 75.4 & 76.5 \\ \hline
1119 & -0.508 & 55.6 & 65 & 62.4 & 62.4 & 64.3 & 66.8 & 70.7 & 72.9 & 74.4 & 75.5 & 76.8 \\ \hline
1120 & -0.513 & 54.8 & 64.8 & 62.4 & 62.2 & 64.2 & 66.7 & 70.6 & 72.6 & 74.1 & 75.4 & 76.7 \\ \hline
1121 & -0.519 & 54 & 64.6 & 62.4 & 62 & 64.2 & 66.4 & 70.3 & 72.3 & 73.9 & 75.1 & 76.4 \\ \hline
1122 & -0.524 & 52.8 & 64.2 & 62.1 & 61.5 & 63.7 & 65.9 & 69.6 & 71.5 & 73.1 & 74.4 & 75.7 \\ \hline
1123 & -0.529 & 51.4 & 63.5 & 61.9 & 60.9 & 63.2 & 65.1 & 68.8 & 70.6 & 72.2 & 73.6 & 74.9 \\ \hline
1124 & -0.535 & 49.6 & 62.9 & 61.5 & 60 & 62.5 & 64.2 & 67.6 & 69.5 & 70.9 & 72.4 & 73.8 \\ \hline
1125 & -0.54 & 47.6 & 62.1 & 60.9 & 59.2 & 61.7 & 63.1 & 66.4 & 68.1 & 69.7 & 70.9 & 72.4 \\ \hline
1126 & -0.545 & 45.3 & 61.1 & 60.2 & 58.1 & 60.6 & 61.8 & 64.8 & 66.4 & 68 & 69.4 & 70.8 \\ \hline
1127 & -0.551 & 43.1 & 60 & 59.5 & 57.2 & 59.4 & 60.4 & 63.2 & 64.8 & 66.3 & 67.7 & 69.2 \\ \hline
1128 & -0.556 & 40.5 & 58.6 & 58.5 & 55.9 & 58 & 58.7 & 61.3 & 62.8 & 64.2 & 65.5 & 67.1 \\ \hline
1129 & -0.561 & 37.9 & 57.2 & 57.3 & 54.5 & 56.7 & 57.1 & 59.2 & 60.6 & 62 & 63.3 & 64.8 \\ \hline
1130 & -0.567 & 35.1 & 55.5 & 56 & 53.1 & 55 & 55.1 & 57.1 & 58.3 & 59.5 & 60.8 & 62.3 \\ \hline
1131 & -0.572 & 32.6 & 53.9 & 54.8 & 51.8 & 53.4 & 53.4 & 55.1 & 56.2 & 57.4 & 58.6 & 60 \\ \hline
1132 & -0.578 & 30 & 52.1 & 53.5 & 50.6 & 51.8 & 51.7 & 53.1 & 54 & 55.2 & 56.3 & 57.6 \\ \hline
1133 & -0.583 & 27.6 & 50.5 & 52.3 & 49.6 & 50.5 & 50.1 & 51.2 & 52 & 53.1 & 54.1 & 55.3 \\ \hline
1134 & -0.588 & 25.4 & 48.6 & 51.1 & 48.5 & 49.2 & 48.5 & 49.5 & 50.1 & 51.1 & 52 & 53.1 \\ \hline
1135 & -0.594 & 23.4 & 46.9 & 49.9 & 47.6 & 48 & 47.4 & 48 & 48.6 & 49.3 & 50.1 & 51.1 \\ \hline
1136 & -0.599 & 21.5 & 45 & 48.6 & 46.7 & 46.9 & 46.2 & 46.6 & 47 & 47.6 & 48.3 & 49.2 \\ \hline
1137 & -0.604 & 19.9 & 43 & 47.5 & 46 & 46.1 & 45.1 & 45.3 & 45.6 & 46.1 & 46.7 & 47.4 \\ \hline
1138 & -0.61 & 18.3 & 41 & 46.3 & 45.2 & 45.3 & 44.4 & 44.3 & 44.4 & 44.8 & 45.2 & 45.7 \\ \hline
1139 & -0.615 & 16.8 & 39 & 45.2 & 44.6 & 44.8 & 43.9 & 43.5 & 43.6 & 43.7 & 44 & 44.5 \\ \hline
1140 & -0.62 & 15.6 & 37 & 44.1 & 44.2 & 44.3 & 43.5 & 43 & 42.8 & 42.8 & 43 & 43.3 \\ \hline
1141 & -0.626 & 14.4 & 35.2 & 43 & 43.7 & 44 & 43.2 & 42.6 & 42.3 & 42.2 & 42.2 & 42.5 \\ \hline
1142 & -0.631 & 13.2 & 33 & 41.8 & 43 & 43.7 & 42.8 & 42.2 & 41.8 & 41.5 & 41.4 & 41.6 \\ \hline
1143 & -0.636 & 12.4 & 31 & 40.4 & 42.3 & 43.4 & 42.7 & 42 & 41.4 & 41.1 & 41 & 40.9 \\ \hline
1144 & -0.642 & 11.6 & 29 & 38.8 & 41.4 & 43.1 & 42.6 & 42 & 41.4 & 40.9 & 40.6 & 40.5 \\ \hline
1145 & -0.647 & 10.8 & 27 & 37.3 & 40.7 & 43 & 42.8 & 42.4 & 41.7 & 41.2 & 40.7 & 40.5 \\ \hline
1146 & -0.652 & 10.1 & 25 & 35.7 & 39.9 & 42.8 & 43 & 42.8 & 42.1 & 41.5 & 41 & 40.6 \\ \hline
1147 & -0.658 & 9.6 & 23.4 & 34.5 & 39.1 & 42.8 & 43.4 & 43.3 & 42.9 & 42.3 & 41.6 & 41.1 \\ \hline
1148 & -0.663 & 9.2 & 21.8 & 33 & 38.3 & 42.8 & 43.8 & 44.1 & 43.6 & 43.1 & 42.4 & 41.9 \\ \hline
1149 & -0.668 & 8.8 & 20.6 & 31.7 & 37.5 & 42.8 & 44.3 & 45 & 44.5 & 44 & 43.3 & 42.7 \\ \hline
1150 & -0.674 & 8.4 & 19.3 & 30.1 & 36.3 & 42.4 & 44.4 & 45.6 & 45.4 & 44.8 & 44.2 & 43.6 \\ \hline
1151 & -0.679 & 8.1 & 18.1 & 28.5 & 35 & 41.7 & 44.5 & 46.3 & 46.3 & 45.9 & 45.4 & 44.7 \\ \hline
1152 & -0.684 & 7.7 & 16.9 & 26.8 & 33.4 & 40.8 & 44.3 & 46.7 & 47 & 46.8 & 46.3 & 45.7 \\ \hline
1153 & -0.69 & 7.5 & 15.9 & 25.1 & 31.8 & 39.6 & 44.1 & 47.2 & 47.8 & 47.8 & 47.4 & 46.9 \\ \hline
1154 & -0.695 & 7.2 & 15 & 23.4 & 30 & 38.3 & 43.6 & 47.5 & 48.4 & 48.6 & 48.3 & 48 \\ \hline
1155 & -0.701 & 7.2 & 14.2 & 21.8 & 28.3 & 36.9 & 42.9 & 47.7 & 49 & 49.4 & 49.2 & 49.1 \\ \hline
1156 & -0.706 & 6.8 & 13.4 & 20.3 & 26.4 & 35.2 & 41.9 & 47.5 & 49.3 & 49.9 & 49.9 & 49.9 \\ \hline
1157 & -0.711 & 6.8 & 12.6 & 19.1 & 24.8 & 33.6 & 40.7 & 47.1 & 49.4 & 50.3 & 50.6 & 50.7 \\ \hline
1158 & -0.717 & 6.4 & 12.1 & 17.8 & 23.2 & 31.6 & 39.2 & 46.4 & 49.2 & 50.4 & 51.1 & 51.4 \\ \hline
1159 & -0.722 & 6.4 & 11.7 & 16.9 & 21.9 & 29.8 & 37.7 & 45.6 & 48.9 & 50.6 & 51.5 & 52.2 \\ \hline
1160 & -0.727 & 6.3 & 11.3 & 15.9 & 20.5 & 28 & 35.9 & 44.6 & 48.5 & 50.4 & 51.7 & 52.7 \\ \hline
1161 & -0.733 & 6.3 & 10.9 & 15.2 & 19.3 & 26.4 & 34.2 & 43.5 & 47.9 & 50.4 & 51.9 & 53.2 \\ \hline
1162 & -0.738 & 6 & 10.6 & 14.5 & 18.4 & 24.8 & 32.6 & 42.2 & 47.1 & 50 & 51.8 & 53.4 \\ \hline
1163 & -0.743 & 6 & 10.5 & 14.1 & 17.6 & 23.5 & 31 & 40.6 & 46 & 49.4 & 51.6 & 53.5 \\ \hline
1164 & -0.749 & 5.7 & 10.2 & 13.7 & 16.8 & 22.3 & 29.3 & 38.9 & 44.5 & 48.4 & 50.9 & 53.2 \\ \hline
1165 & -0.754 & 5.7 & 10.2 & 13.3 & 16.4 & 21.1 & 27.7 & 37.1 & 43 & 47.2 & 50.1 & 52.8 \\ \hline
1166 & -0.759 & 5.6 & 10.1 & 13 & 15.9 & 20.1 & 26.1 & 35.3 & 41.3 & 45.7 & 49 & 52 \\ \hline
1167 & -0.765 & 5.6 & 10.1 & 12.9 & 15.6 & 19.2 & 24.9 & 33.5 & 39.3 & 44.1 & 47.7 & 51 \\ \hline
1168 & -0.77 & 5.6 & 9.9 & 12.7 & 15.2 & 18.5 & 23.7 & 31.7 & 37.5 & 42.2 & 46 & 49.7 \\ \hline
1169 & -0.775 & 5.6 & 9.9 & 12.6 & 14.9 & 18 & 22.7 & 30.2 & 35.7 & 40.5 & 44.4 & 48.2 \\ \hline
1170 & -0.781 & 5.6 & 9.9 & 12.5 & 14.8 & 17.6 & 21.9 & 28.7 & 34 & 38.5 & 42.5 & 46.6 \\ \hline
1171 & -0.786 & 5.7 & 9.9 & 12.4 & 14.8 & 17.2 & 21.2 & 27.5 & 32.5 & 36.9 & 40.8 & 45 \\ \hline
1172 & -0.791 & 5.6 & 9.9 & 12.3 & 14.6 & 16.8 & 20.4 & 26.4 & 31.2 & 35.3 & 39.1 & 43.4 \\ \hline
1173 & -0.797 & 5.9 & 10.1 & 12.3 & 14.6 & 16.4 & 20 & 25.6 & 30.1 & 34.1 & 37.8 & 41.9 \\ \hline
1174 & -0.802 & 6 & 10.1 & 12.3 & 14.5 & 16.4 & 19.7 & 25.2 & 29.4 & 33.2 & 36.7 & 40.7 \\ \hline
1175 & -0.807 & 6 & 10.2 & 12.5 & 14.5 & 16.4 & 19.7 & 24.8 & 29.1 & 32.8 & 36 & 39.9 \\ \hline
1176 & -0.813 & 6.1 & 10.3 & 12.5 & 14.5 & 16.4 & 19.7 & 24.8 & 29 & 32.5 & 35.7 & 39.5 \\ \hline
1177 & -0.818 & 6.4 & 10.6 & 12.6 & 14.8 & 16.4 & 20 & 25.2 & 29.1 & 32.5 & 35.7 & 39.5 \\ \hline
1178 & -0.823 & 6.5 & 10.6 & 12.7 & 14.9 & 16.6 & 20.3 & 25.6 & 29.5 & 32.9 & 36.1 & 39.7 \\ \hline
1179 & -0.829 & 6.8 & 10.9 & 12.9 & 15.3 & 16.9 & 20.7 & 26.2 & 30.3 & 33.7 & 36.7 & 40.3 \\ \hline
1180 & -0.834 & 6.8 & 11.2 & 13 & 15.6 & 17.2 & 21.2 & 27 & 31.2 & 34.8 & 37.8 & 41.3 \\ \hline
1181 & -0.84 & 7.2 & 11.3 & 13.3 & 16 & 17.6 & 22 & 28 & 32.4 & 36 & 39.1 & 42.5 \\ \hline
1182 & -0.845 & 7.4 & 11.3 & 13.5 & 16.4 & 18 & 22.8 & 29.1 & 33.6 & 37.4 & 40.4 & 43.8 \\ \hline
1183 & -0.85 & 7.7 & 11.4 & 13.8 & 16.8 & 18.8 & 23.6 & 30.3 & 35.2 & 39 & 42 & 45.3 \\ \hline
1184 & -0.856 & 8 & 11.5 & 14.1 & 17.5 & 19.2 & 24.7 & 31.7 & 36.8 & 40.6 & 43.6 & 47 \\ \hline
1185 & -0.861 & 8.1 & 11.8 & 14.5 & 18.2 & 20 & 25.9 & 33.4 & 38.6 & 42.5 & 45.5 & 48.8 \\ \hline
1186 & -0.866 & 8.3 & 12.2 & 14.8 & 18.9 & 20.8 & 27.2 & 35.1 & 40.5 & 44.4 & 47.4 & 50.5 \\ \hline
1187 & -0.872 & 8.6 & 12.6 & 15.3 & 19.8 & 21.8 & 28.8 & 37.1 & 42.6 & 46.4 & 49.4 & 52.4 \\ \hline
1188 & -0.877 & 8.8 & 12.8 & 15.7 & 20.6 & 22.8 & 30.3 & 39 & 44.6 & 48.4 & 51.2 & 54 \\ \hline
1189 & -0.882 & 8.9 & 13 & 16.1 & 21.6 & 24 & 31.9 & 41 & 46.5 & 50 & 52.7 & 55.3 \\ \hline
1190 & -0.888 & 9 & 13.1 & 16.5 & 22.7 & 25.2 & 33.5 & 42.7 & 48.1 & 51.6 & 53.9 & 56.2 \\ \hline
1191 & -0.893 & 9.3 & 13.5 & 17.3 & 23.9 & 26.8 & 35.4 & 44.6 & 49.7 & 52.8 & 54.9 & 56.8 \\ \hline
1192 & -0.898 & 9.6 & 13.9 & 18.1 & 25.3 & 28.4 & 37.3 & 46.2 & 50.9 & 53.6 & 55.3 & 56.8 \\ \hline
1193 & -0.904 & 9.8 & 14.4 & 18.9 & 26.8 & 30.2 & 39.3 & 47.8 & 52 & 54 & 55.4 & 56.5 \\ \hline
1194 & -0.909 & 10.1 & 14.9 & 20 & 28.5 & 32 & 41.2 & 49.1 & 52.5 & 54.1 & 55 & 55.6 \\ \hline
1195 & -0.914 & 10.5 & 15.7 & 21.3 & 30.5 & 34.3 & 43.1 & 50.2 & 52.8 & 53.7 & 54.1 & 54.2 \\ \hline
1196 & -0.92 & 10.9 & 16.5 & 22.7 & 32.5 & 36.5 & 44.8 & 50.8 & 52.4 & 52.7 & 52.6 & 52.2 \\ \hline
1197 & -0.925 & 11.3 & 17.4 & 24.3 & 34.7 & 38.8 & 46.4 & 50.9 & 51.5 & 51.2 & 50.7 & 50 \\ \hline
1198 & -0.93 & 12 & 18.5 & 25.9 & 36.8 & 40.8 & 47.5 & 50.5 & 50.2 & 49.3 & 48.6 & 47.6 \\ \hline
1199 & -0.936 & 12.9 & 19.8 & 27.9 & 39.1 & 42.8 & 48.3 & 49.8 & 48.7 & 47.4 & 46.7 & 45.6 \\ \hline
1200 & -0.941 & 13.8 & 21.3 & 30 & 41.1 & 44.4 & 48.6 & 48.6 & 47 & 45.6 & 44.7 & 43.7 \\ \hline
1201 & -0.946 & 15 & 23 & 32.2 & 43 & 45.9 & 48.5 & 47.3 & 45.4 & 44 & 43.2 & 42.4 \\ \hline
1202 & -0.952 & 16.2 & 24.9 & 34.5 & 44.7 & 46.8 & 48 & 45.7 & 44 & 42.8 & 42 & 41.4 \\ \hline
1203 & -0.957 & 17.9 & 27.2 & 37.1 & 46.3 & 47.6 & 47.3 & 44.8 & 43.2 & 42.2 & 41.6 & 41.1 \\ \hline
1204 & -0.962 & 19.6 & 29.5 & 39.6 & 47.5 & 47.9 & 46.5 & 44 & 42.8 & 42 & 41.6 & 41.2 \\ \hline
1205 & -0.968 & 21.6 & 32.1 & 42 & 48.4 & 47.8 & 45.8 & 43.7 & 42.8 & 42.4 & 42 & 41.9 \\ \hline
1206 & -0.973 & 23.9 & 34.6 & 44.2 & 48.8 & 47.3 & 45.4 & 43.7 & 43.2 & 43.2 & 42.8 & 42.8 \\ \hline
1207 & -0.979 & 26.4 & 37.6 & 46.5 & 49.2 & 47 & 45.4 & 44.5 & 44.4 & 44.5 & 44.4 & 44.5 \\ \hline
1208 & -0.984 & 29.2 & 40.5 & 48.1 & 49.4 & 46.8 & 45.9 & 45.7 & 46 & 46.3 & 46.3 & 46.5 \\ \hline
1209 & -0.989 & 32.1 & 43.7 & 49.7 & 49.8 & 47 & 46.8 & 47.4 & 48 & 48.4 & 48.7 & 48.9 \\ \hline
1210 & -0.995 & 34.9 & 46.5 & 50.8 & 50 & 47.4 & 48 & 49.2 & 50.1 & 50.7 & 51 & 51.3 \\ \hline
1211 & -1 & 37.8 & 49.1 & 51.7 & 50.4 & 48.2 & 49.4 & 51.2 & 52.5 & 53.1 & 53.7 & 54 \\ \hline
1212 & -1.005 & 40.8 & 51.4 & 52.4 & 51.2 & 49.3 & 51.2 & 53.6 & 55 & 55.7 & 56.4 & 56.8 \\ \hline
1213 & -1.011 & 43.8 & 53.7 & 53.2 & 52.4 & 51.2 & 53.5 & 56.3 & 57.9 & 58.8 & 59.5 & 60 \\ \hline
1214 & -1.016 & 46.5 & 55.5 & 53.9 & 53.9 & 53.1 & 55.6 & 58.9 & 60.6 & 61.6 & 62.5 & 63.2 \\ \hline
1215 & -1.021 & 49 & 57.3 & 54.9 & 55.3 & 55.2 & 58 & 61.5 & 63.4 & 64.6 & 65.5 & 66.3 \\ \hline
1216 & -1.027 & 50.9 & 58.5 & 56 & 56.8 & 57.2 & 60.2 & 63.9 & 65.9 & 67 & 68.2 & 69.1 \\ \hline
1217 & -1.032 & 52.4 & 59.7 & 57 & 58 & 59.2 & 62.1 & 66 & 68 & 69.3 & 70.5 & 71.6 \\ \hline
1218 & -1.037 & 52.6 & 60.1 & 57.8 & 58.8 & 60.4 & 63.2 & 67.2 & 69.2 & 70.6 & 71.8 & 73.1 \\ \hline
1219 & -1.043 & 51.8 & 60.3 & 58.5 & 59 & 61.2 & 63.7 & 67.7 & 69.6 & 71.1 & 72.5 & 73.8 \\ \hline
1220 & -1.048 & 49.8 & 59.9 & 58.6 & 58.4 & 60.9 & 63.2 & 67 & 69 & 70.4 & 71.9 & 73.4 \\ \hline
1221 & -1.053 & 46.7 & 59 & 58.2 & 57.3 & 59.9 & 61.8 & 65.4 & 67.4 & 68.9 & 70.5 & 71.9 \\ \hline
1222 & -1.059 & 42.6 & 57.4 & 57 & 55.2 & 57.9 & 59.4 & 62.6 & 64.5 & 66 & 67.6 & 69.2 \\ \hline
1223 & -1.064 & 38 & 55.3 & 55.2 & 52.8 & 55.2 & 56.3 & 59.2 & 60.9 & 62.4 & 64 & 65.5 \\ \hline
1224 & -1.069 & 33.2 & 52.6 & 52.8 & 50 & 52.1 & 52.8 & 55.2 & 56.8 & 58.1 & 59.6 & 61.1 \\ \hline
1225 & -1.075 & 28.7 & 49.5 & 50.4 & 47.6 & 49.1 & 49.4 & 51.3 & 52.7 & 53.8 & 55.2 & 56.4 \\ \hline
1226 & -1.08 & 24.4 & 46.3 & 47.9 & 45.4 & 46.3 & 46.3 & 47.7 & 48.8 & 49.7 & 50.9 & 52 \\ \hline
1227 & -1.085 & 20.9 & 43.3 & 45.7 & 43.8 & 44.3 & 44 & 44.9 & 45.6 & 46.4 & 47.3 & 48.2 \\ \hline
1228 & -1.091 & 18 & 40.1 & 43.9 & 42.5 & 42.7 & 42.2 & 42.6 & 43.1 & 43.6 & 44.4 & 45 \\ \hline
1229 & -1.096 & 15.6 & 36.9 & 42.3 & 41.7 & 41.8 & 41.2 & 41.3 & 41.5 & 41.8 & 42.3 & 42.6 \\ \hline
1230 & -1.101 & 13.6 & 33.5 & 40.6 & 40.9 & 41.2 & 40.6 & 40.5 & 40.4 & 40.5 & 40.7 & 41 \\ \hline
1231 & -1.107 & 12 & 30.2 & 38.8 & 40.1 & 41.1 & 40.5 & 40.4 & 40 & 39.9 & 40 & 40.1 \\ \hline
1232 & -1.112 & 10.8 & 26.7 & 36.5 & 39.2 & 40.9 & 40.5 & 40.4 & 40 & 39.9 & 39.7 & 39.7 \\ \hline
1233 & -1.118 & 9.8 & 23.7 & 34.1 & 38 & 40.8 & 41 & 41.1 & 40.8 & 40.4 & 40.1 & 40 \\ \hline
1234 & -1.123 & 9 & 20.9 & 31.4 & 36.4 & 40.5 & 41.4 & 41.9 & 41.6 & 41.2 & 40.9 & 40.8 \\ \hline
1235 & -1.128 & 8.3 & 18.6 & 28.6 & 34.4 & 40 & 41.9 & 42.8 & 42.8 & 42.5 & 42.3 & 42 \\ \hline
1236 & -1.134 & 7.6 & 16.6 & 25.8 & 32.1 & 38.9 & 42.1 & 44 & 44.1 & 44 & 43.8 & 43.5 \\ \hline
1237 & -1.139 & 7.2 & 15.1 & 23.3 & 29.7 & 37.3 & 42 & 44.8 & 45.3 & 45.5 & 45.3 & 45.1 \\ \hline
1238 & -1.144 & 6.8 & 13.8 & 20.9 & 27.1 & 35.2 & 41.2 & 45.2 & 46.4 & 46.8 & 46.8 & 46.7 \\ \hline
1239 & -1.15 & 6.6 & 12.6 & 18.9 & 24.4 & 32.6 & 39.6 & 45.3 & 47.2 & 47.9 & 48.2 & 48.3 \\ \hline
1240 & -1.155 & 6.4 & 11.8 & 17 & 22 & 29.7 & 37.3 & 44.5 & 47.2 & 48.5 & 49.1 & 49.6 \\ \hline
1241 & -1.16 & 6.1 & 11.1 & 15.6 & 20 & 26.8 & 34.6 & 43 & 46.8 & 48.7 & 49.6 & 50.6 \\ \hline
1242 & -1.166 & 5.7 & 10.5 & 14.4 & 18 & 24 & 31.6 & 40.9 & 45.4 & 48.2 & 49.8 & 51.1 \\ \hline
1243 & -1.171 & 5.6 & 10.1 & 13.5 & 16.7 & 21.7 & 28.9 & 38.4 & 43.7 & 47.2 & 49.5 & 51.4 \\ \hline
1244 & -1.176 & 5.6 & 9.7 & 12.5 & 15.5 & 19.7 & 26.2 & 35.5 & 41.3 & 45.4 & 48.3 & 50.7 \\ \hline
1245 & -1.182 & 5.6 & 9.4 & 12.1 & 14.7 & 18.3 & 24 & 32.7 & 38.5 & 43.1 & 46.6 & 49.8 \\ \hline
1246 & -1.187 & 5.6 & 9 & 11.6 & 14 & 17.1 & 22.2 & 30.1 & 35.8 & 40.6 & 44.4 & 48.1 \\ \hline
1247 & -1.192 & 5.6 & 9 & 11.3 & 13.6 & 16.3 & 21 & 28.1 & 33.4 & 38.1 & 42.1 & 46 \\ \hline
1248 & -1.198 & 5.6 & 8.9 & 10.9 & 13.2 & 15.6 & 20 & 26.4 & 31.4 & 35.9 & 39.7 & 43.7 \\ \hline
1249 & -1.203 & 5.6 & 8.8 & 10.9 & 13.2 & 15.2 & 19.4 & 25.3 & 29.9 & 34.1 & 37.7 & 41.7 \\ \hline
1250 & -1.208 & 5.6 & 8.8 & 10.9 & 13.2 & 15.1 & 19.1 & 24.7 & 29 & 32.8 & 36.2 & 40.1 \\ \hline
1251 & -1.214 & 5.6 & 8.8 & 10.9 & 13.2 & 15.1 & 19.1 & 24.7 & 28.8 & 32.4 & 35.5 & 39.2 \\ \hline
1252 & -1.219 & 5.6 & 8.8 & 10.9 & 13.4 & 15.2 & 19.5 & 24.9 & 29 & 32.4 & 35.5 & 38.8 \\ \hline
1253 & -1.224 & 5.6 & 8.9 & 11.3 & 13.9 & 15.6 & 20.1 & 25.6 & 29.8 & 33.2 & 36.1 & 39.4 \\ \hline
1254 & -1.23 & 5.6 & 9 & 11.3 & 14.3 & 16 & 20.7 & 26.5 & 30.9 & 34.4 & 37.2 & 40.3 \\ \hline
1255 & -1.235 & 6 & 9.4 & 11.7 & 15 & 16.8 & 21.8 & 28 & 32.5 & 36 & 38.7 & 41.8 \\ \hline
1256 & -1.24 & 6 & 9.6 & 12.1 & 15.6 & 17.6 & 23 & 29.6 & 34.3 & 37.9 & 40.7 & 43.6 \\ \hline
1257 & -1.246 & 6.4 & 10 & 12.5 & 16.5 & 18.7 & 24.6 & 31.6 & 36.5 & 40.2 & 43 & 45.8 \\ \hline
1258 & -1.251 & 6.5 & 10.3 & 13.2 & 17.5 & 19.9 & 26.3 & 33.9 & 38.9 & 42.6 & 45.3 & 47.9 \\ \hline
1259 & -1.256 & 6.8 & 10.8 & 14 & 19 & 21.4 & 28.5 & 36.4 & 41.5 & 45.2 & 47.7 & 50.3 \\ \hline
1260 & -1.262 & 7.2 & 11.2 & 14.9 & 20.6 & 23 & 30.9 & 39.2 & 44.2 & 47.6 & 50 & 52.3 \\ \hline
1261 & -1.267 & 7.6 & 12 & 16.1 & 22.6 & 25.2 & 33.6 & 42.1 & 46.9 & 49.9 & 52 & 53.9 \\ \hline
1262 & -1.273 & 8 & 12.8 & 17.3 & 24.6 & 27.6 & 36.3 & 44.7 & 49 & 51.5 & 53.1 & 54.4 \\ \hline
1263 & -1.278 & 8.8 & 13.8 & 18.9 & 27 & 30.4 & 39.2 & 47.1 & 50.5 & 52.3 & 53.3 & 54 \\ \hline
1264 & -1.283 & 9.6 & 14.9 & 20.5 & 29.6 & 33.2 & 41.8 & 48.7 & 51.1 & 52 & 52.3 & 52.3 \\ \hline
1265 & -1.289 & 10.4 & 16.4 & 22.7 & 32.4 & 36.2 & 44.3 & 49.6 & 50.7 & 50.8 & 50.5 & 49.9 \\ \hline
1266 & -1.294 & 11.2 & 18 & 24.9 & 35.1 & 39.1 & 46 & 49.4 & 49.3 & 48.7 & 47.9 & 47.1 \\ \hline
1267 & -1.299 & 12.7 & 19.9 & 27.7 & 38 & 41.7 & 47.1 & 48.4 & 47.3 & 46.3 & 45.5 & 44.5 \\ \hline
1268 & -1.305 & 14.1 & 22 & 30.5 & 40.7 & 43.8 & 47.2 & 46.7 & 45.1 & 44 & 43.1 & 42.3 \\ \hline
1269 & -1.31 & 15.7 & 24.5 & 33.4 & 43 & 45.2 & 46.4 & 44.7 & 43.2 & 42.1 & 41.5 & 40.8 \\ \hline
1270 & -1.315 & 17.7 & 27.1 & 36.3 & 44.6 & 45.6 & 45.2 & 43 & 41.7 & 40.9 & 40.5 & 40 \\ \hline
1271 & -1.321 & 20.1 & 30.1 & 39.3 & 45.6 & 45.3 & 44 & 42 & 41.2 & 40.6 & 40.4 & 40.2 \\ \hline
1272 & -1.326 & 22.8 & 33.3 & 42.1 & 46 & 44.5 & 43.1 & 41.8 & 41.5 & 41.2 & 41.2 & 41.1 \\ \hline
1273 & -1.331 & 26 & 36.9 & 44.6 & 46.3 & 44 & 43 & 42.6 & 42.7 & 42.5 & 42.7 & 42.9 \\ \hline
1274 & -1.337 & 29.2 & 40.5 & 46.6 & 46.4 & 43.9 & 43.6 & 44 & 44.6 & 44.7 & 44.9 & 45.2 \\ \hline
1275 & -1.342 & 32.8 & 44.3 & 48.2 & 47.2 & 44.7 & 45.2 & 46.3 & 47.3 & 47.5 & 47.9 & 48.4 \\ \hline
1276 & -1.347 & 36.4 & 47.7 & 49.4 & 48.2 & 46.1 & 47.4 & 49.1 & 50.2 & 50.8 & 51.2 & 51.8 \\ \hline
1277 & -1.353 & 40 & 50.8 & 50.6 & 49.9 & 48.3 & 50.2 & 52.4 & 53.7 & 54.5 & 55 & 55.7 \\ \hline
1278 & -1.358 & 43.6 & 53.2 & 51.7 & 51.6 & 50.9 & 53 & 55.6 & 57.2 & 58 & 58.7 & 59.4 \\ \hline
1279 & -1.363 & 46.6 & 55.1 & 53.2 & 53.6 & 53.7 & 55.9 & 59 & 60.7 & 61.6 & 62.4 & 63.3 \\ \hline
1280 & -1.369 & 48.4 & 56.4 & 54.6 & 55.3 & 56.1 & 58.4 & 61.7 & 63.4 & 64.4 & 65.3 & 66.3 \\ \hline
1281 & -1.374 & 48.8 & 57.1 & 55.7 & 56.4 & 57.7 & 60 & 63.3 & 65.1 & 66.3 & 67.3 & 68.3 \\ \hline
1282 & -1.379 & 47.2 & 56.7 & 56 & 56 & 58 & 60 & 63.2 & 65 & 66.3 & 67.4 & 68.6 \\ \hline
1283 & -1.385 & 43.6 & 55.8 & 55.5 & 54.5 & 56.8 & 58.5 & 61.5 & 63.2 & 64.6 & 65.8 & 67 \\ \hline
1284 & -1.39 & 38.4 & 53.8 & 53.7 & 52 & 54.2 & 55.5 & 58.2 & 59.7 & 61 & 62.3 & 63.6 \\ \hline
1285 & -1.395 & 32.8 & 51 & 51.4 & 49.1 & 50.8 & 51.8 & 54 & 55.3 & 56.5 & 57.8 & 59 \\ \hline
1286 & -1.401 & 27.6 & 47.7 & 48.5 & 46 & 47.2 & 47.8 & 49.6 & 50.6 & 51.6 & 52.8 & 53.8 \\ \hline
1287 & -1.406 & 22.8 & 44.3 & 46 & 43.6 & 44.4 & 44.5 & 45.7 & 46.5 & 47.2 & 48.2 & 48.9 \\ \hline
1288 & -1.411 & 19 & 40.5 & 43.6 & 41.7 & 42 & 41.7 & 42.5 & 43.1 & 43.6 & 44.2 & 44.8 \\ \hline
1289 & -1.417 & 16.1 & 36.5 & 41.4 & 40.4 & 40.5 & 40.1 & 40.4 & 40.8 & 41.1 & 41.4 & 41.8 \\ \hline
1290 & -1.422 & 13.7 & 32.4 & 39.3 & 39.3 & 39.8 & 39.3 & 39.3 & 39.4 & 39.4 & 39.7 & 39.8 \\ \hline
1291 & -1.428 & 12 & 28.4 & 37 & 38.4 & 39.7 & 39.3 & 39.3 & 39.2 & 39.1 & 39.1 & 39.1 \\ \hline
1292 & -1.433 & 10.7 & 24.8 & 34.4 & 37.3 & 39.9 & 39.8 & 39.9 & 39.6 & 39.5 & 39.4 & 39.2 \\ \hline
1293 & -1.438 & 9.6 & 21.9 & 31.5 & 35.9 & 40 & 40.6 & 41.1 & 40.9 & 40.7 & 40.5 & 40.3 \\ \hline
1294 & -1.444 & 8.8 & 19.4 & 28.4 & 33.7 & 39.4 & 41.3 & 42.4 & 42.4 & 42.3 & 42.1 & 41.8 \\ \hline
1295 & -1.449 & 8.4 & 17.4 & 25.3 & 30.9 & 38 & 41.6 & 43.7 & 44 & 44 & 43.8 & 43.6 \\ \hline
1296 & -1.454 & 8 & 15.8 & 22.5 & 27.7 & 35.4 & 40.8 & 44.5 & 45.3 & 45.6 & 45.5 & 45.5 \\ \hline
1297 & -1.46 & 7.6 & 14.5 & 20.1 & 24.8 & 32.2 & 38.9 & 44.5 & 46.2 & 46.9 & 47.2 & 47.4 \\ \hline
1298 & -1.465 & 7.2 & 13.4 & 18.1 & 22 & 28.8 & 36.1 & 43.4 & 46.3 & 47.7 & 48.4 & 48.9 \\ \hline
1299 & -1.47 & 7.2 & 12.6 & 16.5 & 19.9 & 25.6 & 32.9 & 41.3 & 45.4 & 47.7 & 49.1 & 50.1 \\ \hline
1300 & -1.476 & 7 & 12.1 & 15.3 & 18.1 & 22.8 & 29.5 & 38.3 & 43.4 & 46.6 & 48.9 & 50.6 \\ \hline
1301 & -1.481 & 7 & 11.7 & 14.5 & 16.9 & 20.8 & 26.7 & 35.1 & 40.6 & 44.6 & 47.7 & 50.2 \\ \hline
1302 & -1.486 & 6.9 & 11.3 & 13.7 & 16 & 19.2 & 24.3 & 32 & 37.4 & 41.8 & 45.3 & 48.6 \\ \hline
1303 & -1.492 & 6.9 & 11 & 13.3 & 15.5 & 18.1 & 22.6 & 29.5 & 34.6 & 39 & 42.7 & 46.3 \\ \hline
1304 & -1.497 & 6.9 & 10.9 & 12.9 & 15.1 & 17.3 & 21.4 & 27.5 & 32.2 & 36.3 & 39.9 & 43.6 \\ \hline
1305 & -1.502 & 6.9 & 10.9 & 12.9 & 14.9 & 16.9 & 20.7 & 26.3 & 30.6 & 34.3 & 37.7 & 41.3 \\ \hline
1306 & -1.508 & 6.9 & 10.9 & 12.9 & 14.9 & 16.7 & 20.4 & 25.8 & 29.8 & 33.2 & 36.3 & 39.7 \\ \hline
1307 & -1.513 & 6.9 & 10.9 & 12.9 & 15.2 & 16.8 & 20.7 & 25.8 & 29.7 & 32.9 & 35.9 & 39 \\ \hline
1308 & -1.518 & 7 & 10.9 & 12.9 & 15.5 & 17.1 & 21.1 & 26.3 & 30.3 & 33.3 & 36.2 & 39.3 \\ \hline
1309 & -1.524 & 7.3 & 11.1 & 13 & 16 & 17.6 & 21.9 & 27.4 & 31.5 & 34.5 & 37.4 & 40.4 \\ \hline
1310 & -1.529 & 7.6 & 11.3 & 13.4 & 16.7 & 18.3 & 23 & 28.9 & 33.1 & 36.2 & 39 & 42 \\ \hline
1311 & -1.534 & 8 & 11.5 & 14 & 17.5 & 19.3 & 24.6 & 30.9 & 35.3 & 38.7 & 41.4 & 44.3 \\ \hline
1312 & -1.54 & 8.2 & 12 & 14.6 & 18.7 & 20.5 & 26.6 & 33.4 & 38 & 41.3 & 44.1 & 46.9 \\ \hline
1313 & -1.545 & 8.6 & 12.6 & 15.6 & 20.4 & 22.3 & 29 & 36.4 & 41 & 44.4 & 47 & 49.6 \\ \hline
1314 & -1.55 & 9.1 & 13.3 & 16.7 & 22.2 & 24.4 & 31.7 & 39.5 & 44.1 & 47.2 & 49.4 & 51.6 \\ \hline
1315 & -1.556 & 9.9 & 14.2 & 18.1 & 24.4 & 27 & 34.7 & 42.7 & 46.9 & 49.5 & 51.1 & 52.7 \\ \hline
1316 & -1.561 & 10.5 & 15.4 & 19.7 & 27 & 29.8 & 37.9 & 45.5 & 48.7 & 50.6 & 51.5 & 52.3 \\ \hline
1317 & -1.566 & 11.4 & 16.6 & 21.7 & 29.9 & 33.2 & 41 & 47.5 & 49.5 & 50.4 & 50.6 & 50.7 \\ \hline
1318 & -1.572 & 12.5 & 18.2 & 24.1 & 33 & 36.6 & 43.6 & 48.2 & 48.8 & 48.7 & 48.5 & 47.9 \\ \hline
1319 & -1.577 & 14 & 20.3 & 26.9 & 36.4 & 39.9 & 45.5 & 47.6 & 47.1 & 46.3 & 45.9 & 45.1 \\ \hline
1320 & -1.582 & 15.6 & 22.7 & 30.1 & 39.6 & 42.5 & 46 & 46 & 45 & 44 & 43.5 & 42.7 \\ \hline
1321 & -1.588 & 17.6 & 25.6 & 33.6 & 42.4 & 44.3 & 45.5 & 44.3 & 43.1 & 42.3 & 41.8 & 41.4 \\ \hline
1322 & -1.593 & 19.9 & 28.7 & 37 & 44.4 & 44.7 & 44.4 & 42.7 & 41.9 & 41.4 & 41.1 & 40.9 \\ \hline
1323 & -1.599 & 22.6 & 32.1 & 40.2 & 45.3 & 44.3 & 43.2 & 42 & 41.6 & 41.4 & 41.4 & 41.3 \\ \hline
1324 & -1.604 & 25.4 & 35.7 & 43 & 45.3 & 43.6 & 42.5 & 42.2 & 42.3 & 42.3 & 42.4 & 42.5 \\ \hline
1325 & -1.609 & 28.8 & 39.5 & 45 & 45.2 & 43.2 & 42.9 & 43.3 & 43.9 & 44.2 & 44.4 & 44.6 \\ \hline
1326 & -1.615 & 32.4 & 43.2 & 46.4 & 45.6 & 43.6 & 44.2 & 45.5 & 46.4 & 46.9 & 47.2 & 47.7 \\ \hline
1327 & -1.62 & 36.5 & 46.7 & 47.7 & 46.8 & 45.2 & 46.7 & 48.6 & 49.8 & 50.6 & 51 & 51.6 \\ \hline
1328 & -1.625 & 40.6 & 49.8 & 48.9 & 48.8 & 47.8 & 49.8 & 52.3 & 53.8 & 54.6 & 55.1 & 55.9 \\ \hline
1329 & -1.631 & 44.5 & 52.5 & 50.8 & 51.2 & 51.2 & 53.3 & 56.4 & 58 & 58.8 & 59.6 & 60.4 \\ \hline
1330 & -1.636 & 47.1 & 54.3 & 52.6 & 53.5 & 54.3 & 56.4 & 59.6 & 61.4 & 62.4 & 63.3 & 64.3 \\ \hline
1331 & -1.641 & 47.6 & 55.3 & 54.1 & 54.7 & 56.3 & 58.5 & 61.7 & 63.4 & 64.6 & 65.7 & 66.7 \\ \hline
1332 & -1.647 & 45.2 & 54.9 & 54.2 & 54.1 & 56.4 & 58.1 & 61.4 & 63 & 64.3 & 65.4 & 66.6 \\ \hline
1333 & -1.652 & 40.4 & 53.3 & 52.9 & 52.1 & 54.4 & 55.8 & 58.8 & 60.3 & 61.5 & 62.8 & 64 \\ \hline
1334 & -1.657 & 34.4 & 50.4 & 50.5 & 48.8 & 50.8 & 51.8 & 54.3 & 55.7 & 56.8 & 58.1 & 59.2 \\ \hline
1335 & -1.663 & 28.4 & 46.9 & 47.4 & 45.4 & 46.8 & 47.5 & 49.4 & 50.5 & 51.6 & 52.6 & 53.6 \\ \hline
1336 & -1.668 & 23.2 & 42.7 & 44.2 & 42.4 & 43.3 & 43.6 & 44.9 & 45.6 & 46.5 & 47.3 & 48.1 \\ \hline
1337 & -1.673 & 19 & 38.2 & 41.4 & 40.1 & 40.8 & 40.5 & 41.4 & 42 & 42.4 & 43 & 43.6 \\ \hline
1338 & -1.679 & 15.6 & 33.4 & 38.6 & 38.4 & 39.1 & 38.5 & 39.1 & 39.3 & 39.6 & 39.9 & 40.3 \\ \hline
1339 & -1.684 & 13.2 & 28.6 & 35.8 & 37 & 38.3 & 37.7 & 38.1 & 38.1 & 38.1 & 38.3 & 38.4 \\ \hline
1340 & -1.689 & 11.6 & 24.3 & 32.7 & 35.5 & 37.9 & 37.7 & 38.1 & 38 & 37.8 & 37.9 & 37.8 \\ \hline
1341 & -1.695 & 10.4 & 20.9 & 29.3 & 33.7 & 37.5 & 38.4 & 39 & 39 & 38.8 & 38.7 & 38.6 \\ \hline
1342 & -1.7 & 9.4 & 18.2 & 25.8 & 31.2 & 36.5 & 38.9 & 40.3 & 40.6 & 40.4 & 40.3 & 40.1 \\ \hline
1343 & -1.705 & 8.8 & 16.2 & 22.8 & 28.3 & 34.6 & 39 & 41.8 & 42.4 & 42.5 & 42.3 & 42.3 \\ \hline
1344 & -1.711 & 8.4 & 14.6 & 20 & 25.1 & 31.8 & 37.8 & 42.5 & 43.9 & 44.4 & 44.5 & 44.6 \\ \hline
1345 & -1.716 & 8 & 13.3 & 17.8 & 22.2 & 28.3 & 35.3 & 42 & 44.4 & 45.7 & 46.3 & 46.7 \\ \hline
1346 & -1.721 & 7.6 & 12.5 & 16.1 & 19.6 & 24.9 & 31.7 & 39.6 & 43.6 & 45.7 & 47 & 48 \\ \hline
1347 & -1.727 & 7.2 & 11.7 & 14.7 & 17.8 & 22 & 28.1 & 36.4 & 41.3 & 44.5 & 46.6 & 48.4 \\ \hline
1348 & -1.732 & 7.2 & 11.2 & 13.7 & 16.4 & 19.7 & 25.1 & 32.9 & 38.1 & 41.9 & 44.8 & 47.5 \\ \hline
1349 & -1.737 & 7.2 & 10.9 & 13.1 & 15.6 & 18.2 & 22.9 & 29.8 & 34.8 & 38.8 & 42.2 & 45.4 \\ \hline
1350 & -1.743 & 7.1 & 10.5 & 12.5 & 14.9 & 17.2 & 21.3 & 27.4 & 31.9 & 35.8 & 39.2 & 42.6 \\ \hline
1351 & -1.748 & 7.1 & 10.5 & 12.3 & 14.8 & 16.7 & 20.5 & 26 & 30 & 33.5 & 36.8 & 40 \\ \hline
1352 & -1.753 & 7.1 & 10.5 & 12.1 & 14.8 & 16.4 & 20 & 25.2 & 28.9 & 32.2 & 35.2 & 38.2 \\ \hline
1353 & -1.759 & 7.1 & 10.5 & 12.3 & 14.9 & 16.4 & 20.2 & 25.2 & 28.9 & 31.9 & 34.7 & 37.6 \\ \hline
1354 & -1.764 & 7.1 & 10.5 & 12.5 & 15.3 & 16.8 & 20.7 & 25.8 & 29.6 & 32.5 & 35.1 & 37.9 \\ \hline
1355 & -1.769 & 7.3 & 10.7 & 12.9 & 16 & 17.4 & 21.7 & 27.1 & 31.1 & 34 & 36.5 & 39.2 \\ \hline
1356 & -1.775 & 7.6 & 11 & 13.3 & 16.8 & 18.2 & 23.1 & 29 & 33 & 36 & 38.6 & 41.3 \\ \hline
1357 & -1.78 & 8 & 11.4 & 14.1 & 18 & 19.6 & 25 & 31.5 & 35.8 & 38.8 & 41.2 & 44 \\ \hline
1358 & -1.786 & 8.4 & 12.1 & 14.9 & 19.6 & 21.3 & 27.4 & 34.4 & 38.8 & 42 & 44.1 & 46.7 \\ \hline
1359 & -1.791 & 8.9 & 12.9 & 16.3 & 21.6 & 23.7 & 30.5 & 37.8 & 42 & 45 & 46.9 & 48.8 \\ \hline
1360 & -1.796 & 9.7 & 14.1 & 17.9 & 24.2 & 26.5 & 33.9 & 41.2 & 44.9 & 47.1 & 48.5 & 49.7 \\ \hline
1361 & -1.802 & 10.8 & 15.6 & 20 & 27.2 & 30 & 37.5 & 44.1 & 46.8 & 48 & 48.7 & 49 \\ \hline
1362 & -1.807 & 12 & 17.3 & 22.5 & 30.5 & 33.6 & 40.7 & 45.7 & 47 & 47.2 & 47.1 & 46.9 \\ \hline
1363 & -1.812 & 13.3 & 19.4 & 25.5 & 34.1 & 37.5 & 43.2 & 45.8 & 45.7 & 45.2 & 44.7 & 44.2 \\ \hline
1364 & -1.818 & 15 & 21.8 & 28.8 & 37.7 & 40.6 & 44.1 & 44.4 & 43.6 & 42.9 & 42.4 & 41.8 \\ \hline
1365 & -1.823 & 17.2 & 25 & 32.5 & 40.8 & 42.5 & 43.7 & 42.8 & 41.9 & 41.2 & 40.8 & 40.4 \\ \hline
1366 & -1.828 & 19.7 & 28.6 & 36.4 & 42.9 & 42.9 & 42.5 & 41.5 & 40.9 & 40.4 & 40.3 & 40.1 \\ \hline
1367 & -1.834 & 22.9 & 32.7 & 40.2 & 44 & 42.7 & 41.9 & 41.3 & 41.2 & 41.2 & 41.1 & 41.2 \\ \hline
1368 & -1.839 & 26.4 & 37.1 & 43.3 & 44.4 & 42.4 & 42.1 & 42.2 & 42.5 & 42.8 & 43 & 43.2 \\ \hline
1369 & -1.844 & 30.6 & 41.7 & 45.6 & 45.1 & 43.1 & 43.5 & 44.5 & 45.3 & 45.8 & 46.2 & 46.5 \\ \hline
1370 & -1.85 & 35.1 & 45.7 & 47.1 & 46.3 & 44.8 & 46 & 47.7 & 48.8 & 49.5 & 49.9 & 50.5 \\ \hline
1371 & -1.855 & 39.6 & 49.1 & 48.5 & 48.3 & 47.6 & 49.3 & 51.7 & 52.9 & 53.8 & 54.3 & 55 \\ \hline
1372 & -1.86 & 43.5 & 51.4 & 50 & 50.5 & 50.8 & 52.7 & 55.5 & 57 & 57.9 & 58.6 & 59.4 \\ \hline
1373 & -1.866 & 45.8 & 52.7 & 51.5 & 52.4 & 53.6 & 55.5 & 58.6 & 60 & 61.1 & 62 & 63 \\ \hline
1374 & -1.871 & 44.9 & 52.8 & 52.1 & 52.7 & 54.5 & 56.3 & 59.4 & 60.9 & 62.1 & 63.2 & 64.2 \\ \hline
1375 & -1.876 & 40.9 & 51.5 & 51.3 & 51.1 & 53.2 & 54.8 & 57.7 & 59.1 & 60.3 & 61.5 & 62.7 \\ \hline
1376 & -1.882 & 34.8 & 48.9 & 48.9 & 47.7 & 49.8 & 50.9 & 53.5 & 54.8 & 55.9 & 57.2 & 58.3 \\ \hline
1377 & -1.887 & 28.5 & 45.3 & 45.8 & 44.1 & 45.7 & 46.5 & 48.5 & 49.5 & 50.6 & 51.7 & 52.6 \\ \hline
1378 & -1.892 & 22.9 & 41 & 42.5 & 41.1 & 42.1 & 42.5 & 43.8 & 44.7 & 45.5 & 46.4 & 47.1 \\ \hline
1379 & -1.898 & 18.5 & 36.6 & 39.8 & 39.2 & 39.7 & 39.8 & 40.6 & 41.1 & 41.6 & 42.3 & 42.8 \\ \hline
1380 & -1.903 & 15.2 & 31.8 & 37.4 & 37.9 & 38.5 & 38.3 & 38.8 & 39.1 & 39.3 & 39.8 & 40 \\ \hline
1381 & -1.908 & 12.8 & 27 & 34.6 & 36.7 & 38.2 & 37.9 & 38.3 & 38.4 & 38.4 & 38.6 & 38.7 \\ \hline
1382 & -1.914 & 11.1 & 22.7 & 31.1 & 35 & 37.8 & 38.3 & 38.7 & 38.8 & 38.7 & 38.7 & 38.7 \\ \hline
1383 & -1.919 & 9.9 & 19.4 & 27.2 & 32.3 & 37 & 38.7 & 39.7 & 39.8 & 39.6 & 39.5 & 39.5 \\ \hline
1384 & -1.924 & 8.8 & 16.6 & 23.3 & 28.4 & 34.7 & 38.2 & 40.5 & 40.9 & 41 & 40.9 & 40.8 \\ \hline
1385 & -1.93 & 8.1 & 14.6 & 20.1 & 24.8 & 31.2 & 36.6 & 40.9 & 42.1 & 42.4 & 42.5 & 42.6 \\ \hline
1386 & -1.935 & 7.6 & 13 & 17.3 & 21.4 & 27.2 & 33.4 & 39.7 & 42.1 & 43.2 & 43.9 & 44.3 \\ \hline
1387 & -1.94 & 7.2 & 12 & 15.3 & 18.8 & 23.6 & 29.8 & 37.2 & 40.9 & 43.1 & 44.5 & 45.6 \\ \hline
1388 & -1.946 & 7 & 11.3 & 14.1 & 16.8 & 20.7 & 26.2 & 33.8 & 38.4 & 41.6 & 43.8 & 45.7 \\ \hline
1389 & -1.951 & 6.9 & 10.9 & 13.2 & 15.6 & 18.8 & 23.6 & 30.5 & 35.3 & 39 & 41.9 & 44.6 \\ \hline
1390 & -1.956 & 6.8 & 10.5 & 12.5 & 14.8 & 17.4 & 21.6 & 27.7 & 32.2 & 35.9 & 39.1 & 42.2 \\ \hline
1391 & -1.962 & 6.8 & 10.5 & 12.1 & 14.6 & 16.7 & 20.5 & 26 & 30 & 33.5 & 36.5 & 39.7 \\ \hline
1392 & -1.967 & 6.8 & 10.4 & 11.9 & 14.6 & 16.3 & 20 & 25.2 & 28.8 & 31.9 & 34.7 & 37.8 \\ \hline
1393 & -1.972 & 6.8 & 10.4 & 12.2 & 14.8 & 16.4 & 20.3 & 25.1 & 28.7 & 31.5 & 34.2 & 37 \\ \hline
1394 & -1.978 & 6.8 & 10.5 & 12.5 & 15.3 & 16.8 & 20.7 & 25.7 & 29.4 & 32.2 & 34.7 & 37.2 \\ \hline
1395 & -1.983 & 7.2 & 10.9 & 12.9 & 16.1 & 17.6 & 21.9 & 27.3 & 31 & 33.9 & 36.3 & 39 \\ \hline
1396 & -1.989 & 7.6 & 11.3 & 13.7 & 17.2 & 18.8 & 23.6 & 29.4 & 33.4 & 36.4 & 38.8 & 41.4 \\ \hline
1397 & -1.994 & 8 & 12 & 14.6 & 18.8 & 20.4 & 26 & 32.4 & 36.6 & 39.6 & 42 & 44.6 \\ \hline
1398 & -1.999 & 8.6 & 12.8 & 15.8 & 20.8 & 22.4 & 28.9 & 35.8 & 40.2 & 43 & 45.1 & 47.4 \\ \hline
1399 & -2.005 & 9.3 & 13.8 & 17.4 & 23.2 & 25.2 & 32.3 & 39.4 & 43.4 & 45.7 & 47.4 & 48.8 \\ \hline
1400 & -2.01 & 10.1 & 14.9 & 19 & 25.7 & 28.4 & 35.6 & 42.3 & 45.2 & 46.7 & 47.5 & 48 \\ \hline
1401 & -2.015 & 10.9 & 16.5 & 21.1 & 28.6 & 32 & 38.8 & 43.8 & 45.2 & 45.4 & 45.6 & 45.3 \\ \hline
1402 & -2.021 & 12.1 & 18.1 & 23.5 & 31.5 & 35.1 & 40.4 & 43 & 42.9 & 42.6 & 42.3 & 41.7 \\ \hline
1403 & -2.026 & 13.6 & 20.4 & 26.5 & 34.4 & 37.3 & 40.4 & 40.7 & 40 & 39.4 & 39.1 & 38.6 \\ \hline
1404 & -2.031 & 15.6 & 23.3 & 29.8 & 36.7 & 38.1 & 38.8 & 38.3 & 37.6 & 37.2 & 37 & 36.6 \\ \hline
1405 & -2.037 & 18.4 & 26.9 & 33.4 & 38.3 & 37.7 & 37.6 & 37.2 & 36.9 & 36.8 & 36.7 & 36.6 \\ \hline
1406 & -2.042 & 22 & 31.3 & 37.2 & 39.2 & 37.6 & 37.6 & 37.7 & 38 & 38.1 & 38.3 & 38.3 \\ \hline
1407 & -2.047 & 26.2 & 36.5 & 40.4 & 40.4 & 38.4 & 39.1 & 40 & 40.6 & 40.9 & 41.2 & 41.5 \\ \hline
1408 & -2.053 & 30.8 & 41.3 & 42.7 & 42 & 40.4 & 41.7 & 43.4 & 44.4 & 44.9 & 45.3 & 45.8 \\ \hline
1409 & -2.058 & 35.7 & 45.3 & 44.8 & 44.6 & 43.7 & 45.5 & 47.7 & 48.9 & 49.7 & 50.2 & 50.9 \\ \hline
1410 & -2.063 & 40 & 47.9 & 46.7 & 47.3 & 47.4 & 49.3 & 52 & 53.3 & 54.1 & 54.8 & 55.7 \\ \hline
1411 & -2.069 & 42.4 & 49.4 & 48.3 & 49.3 & 50.5 & 52.4 & 55.2 & 56.6 & 57.6 & 58.4 & 59.3 \\ \hline
1412 & -2.074 & 41.4 & 49.5 & 49 & 49.5 & 51.4 & 53.2 & 56 & 57.4 & 58.5 & 59.5 & 60.5 \\ \hline
1413 & -2.079 & 36.9 & 47.9 & 48 & 47.5 & 49.8 & 51.3 & 53.8 & 55.2 & 56.3 & 57.5 & 58.5 \\ \hline
1414 & -2.085 & 30.4 & 44.8 & 45.2 & 44 & 46 & 46.9 & 49.2 & 50.4 & 51.5 & 52.6 & 53.5 \\ \hline
1415 & -2.09 & 24.1 & 40.8 & 42 & 40.7 & 41.8 & 42.5 & 44.2 & 45.2 & 46 & 47 & 47.8 \\ \hline
1416 & -2.095 & 18.9 & 36.4 & 39 & 38 & 38.8 & 39 & 40.2 & 40.8 & 41.5 & 42.2 & 42.7 \\ \hline
1417 & -2.101 & 14.9 & 31.6 & 36.5 & 36.5 & 37.2 & 37.1 & 37.8 & 38.1 & 38.6 & 39 & 39.4 \\ \hline
1418 & -2.106 & 12.1 & 26.6 & 33.7 & 35.2 & 36.7 & 36.4 & 36.9 & 36.9 & 37 & 37.3 & 37.4 \\ \hline
1419 & -2.111 & 10.1 & 22.1 & 30.1 & 33.6 & 36.4 & 36.5 & 37.1 & 37 & 37 & 37 & 37.1 \\ \hline
1420 & -2.117 & 8.5 & 18.2 & 25.9 & 30.7 & 35.2 & 36.7 & 37.8 & 37.9 & 37.9 & 37.8 & 37.8 \\ \hline
1421 & -2.122 & 7.6 & 15.4 & 21.9 & 27 & 32.8 & 36.3 & 38.8 & 39.3 & 39.4 & 39.4 & 39.4 \\ \hline
1422 & -2.127 & 6.8 & 13.4 & 18.5 & 23 & 29.2 & 34.3 & 38.8 & 40.2 & 40.9 & 41.1 & 41.4 \\ \hline
1423 & -2.133 & 6.4 & 11.9 & 16.1 & 19.8 & 25.2 & 31.1 & 37.6 & 40.2 & 41.8 & 42.5 & 43.4 \\ \hline
1424 & -2.138 & 6 & 10.9 & 14.1 & 17.3 & 21.6 & 27.3 & 34.7 & 38.6 & 41.2 & 42.9 & 44.3 \\ \hline
1425 & -2.143 & 5.6 & 10.2 & 12.9 & 15.7 & 19 & 24.1 & 31.1 & 35.6 & 39 & 41.5 & 43.8 \\ \hline
1426 & -2.149 & 5.6 & 9.7 & 12.1 & 14.5 & 17.1 & 21.6 & 27.8 & 32.1 & 35.8 & 38.7 & 41.5 \\ \hline
1427 & -2.154 & 5.6 & 9.4 & 11.7 & 14 & 16 & 20 & 25.4 & 29.3 & 32.8 & 35.5 & 38.7 \\ \hline
1428 & -2.159 & 5.6 & 9.2 & 11.4 & 13.6 & 15.4 & 19.2 & 24 & 27.7 & 30.8 & 33.1 & 36.3 \\ \hline
1429 & -2.165 & 5.6 & 9.3 & 11.4 & 14 & 15.4 & 19.2 & 23.9 & 27.2 & 30.1 & 32.3 & 35.1 \\ \hline
1430 & -2.17 & 5.6 & 9.3 & 11.4 & 14.4 & 15.8 & 19.8 & 24.6 & 27.9 & 30.5 & 32.8 & 35.4 \\ \hline
1431 & -2.175 & 6 & 9.7 & 12.2 & 15.5 & 16.7 & 21.1 & 26.2 & 29.6 & 32.4 & 34.7 & 37.1 \\ \hline
1432 & -2.181 & 6.4 & 10.2 & 13 & 16.8 & 18 & 23 & 28.6 & 32.4 & 35.1 & 37.5 & 39.9 \\ \hline
1433 & -2.186 & 7 & 10.9 & 14.2 & 18.7 & 20 & 25.8 & 31.9 & 35.8 & 38.7 & 40.7 & 43.1 \\ \hline
1434 & -2.191 & 7.7 & 12.1 & 15.7 & 21 & 22.7 & 28.9 & 35.5 & 39.4 & 42 & 43.7 & 45.6 \\ \hline
1435 & -2.197 & 8.7 & 13.5 & 17.7 & 23.7 & 26 & 32.7 & 39.2 & 42.4 & 44.3 & 45.3 & 46.3 \\ \hline
1436 & -2.202 & 9.9 & 15.3 & 20.1 & 27.1 & 29.8 & 36.4 & 41.7 & 43.6 & 44.3 & 44.5 & 44.7 \\ \hline
1437 & -2.207 & 11.6 & 17.5 & 23.1 & 30.9 & 33.8 & 39.6 & 42.5 & 42.8 & 42.7 & 42.3 & 42 \\ \hline
1438 & -2.213 & 13.3 & 20.3 & 26.6 & 34.8 & 37.3 & 40.8 & 41.3 & 40.8 & 40.1 & 39.9 & 39.4 \\ \hline
1439 & -2.218 & 15.7 & 23.7 & 30.6 & 38.1 & 39.3 & 40.4 & 39.6 & 39.1 & 38.5 & 38.3 & 38 \\ \hline
1440 & -2.223 & 18.5 & 27.5 & 34.6 & 40 & 39.6 & 39.4 & 38.5 & 38.3 & 38.1 & 38.1 & 38 \\ \hline
1441 & -2.229 & 22.1 & 32 & 38.3 & 40.8 & 39.3 & 39.1 & 39 & 39.2 & 39.3 & 39.4 & 39.6 \\ \hline
1442 & -2.234 & 26 & 36.3 & 40.5 & 40.9 & 39.5 & 39.9 & 40.6 & 41.2 & 41.6 & 41.6 & 42 \\ \hline
1443 & -2.24 & 30.3 & 40 & 41.7 & 41.7 & 40.8 & 41.8 & 43.2 & 44.2 & 44.8 & 45 & 45.6 \\ \hline
1444 & -2.245 & 34.4 & 42.5 & 42.5 & 42.9 & 42.9 & 44.3 & 46.4 & 47.7 & 48.3 & 48.8 & 49.4 \\ \hline
1445 & -2.25 & 37.5 & 43.9 & 43.6 & 44.6 & 45.6 & 47.2 & 49.7 & 51 & 51.9 & 52.4 & 53 \\ \hline
1446 & -2.256 & 37.9 & 44.5 & 44.4 & 45.4 & 47.1 & 48.8 & 51.4 & 52.7 & 53.8 & 54.4 & 55.2 \\ \hline
1447 & -2.261 & 34.8 & 44.1 & 44.3 & 44.6 & 46.7 & 48.1 & 50.7 & 51.9 & 53 & 53.9 & 54.8 \\ \hline
1448 & -2.266 & 29.2 & 41.8 & 42.4 & 41.8 & 43.7 & 45 & 47.2 & 48.4 & 49.5 & 50.5 & 51.4 \\ \hline
1449 & -2.272 & 23.2 & 38.3 & 39.5 & 38.6 & 40 & 40.9 & 42.7 & 43.6 & 44.6 & 45.5 & 46.2 \\ \hline
1450 & -2.277 & 17.6 & 33.9 & 36.3 & 35.6 & 36.8 & 37.1 & 38.3 & 39.1 & 39.8 & 40.4 & 41 \\ \hline
1451 & -2.282 & 13.6 & 28.8 & 33.5 & 33.6 & 34.8 & 34.8 & 35.6 & 36.1 & 36.6 & 37 & 37.4 \\ \hline
1452 & -2.288 & 10.8 & 23.4 & 30.3 & 32 & 33.6 & 33.6 & 34.4 & 34.7 & 35 & 35.2 & 35.4 \\ \hline
1453 & -2.293 & 8.8 & 19 & 26.4 & 29.9 & 33.1 & 33.6 & 34.5 & 34.8 & 35.1 & 35.1 & 35.2 \\ \hline
1454 & -2.298 & 7.5 & 15.4 & 22.1 & 26.8 & 31.6 & 33.7 & 35.3 & 35.7 & 36 & 36 & 36.1 \\ \hline
1455 & -2.304 & 6.4 & 13 & 18.5 & 23.2 & 28.8 & 33 & 36.4 & 37.3 & 37.7 & 37.9 & 38.1 \\ \hline
1456 & -2.309 & 6 & 11.1 & 15.7 & 19.6 & 25.1 & 30.8 & 36.2 & 38.2 & 39.2 & 39.8 & 40.1 \\ \hline
1457 & -2.314 & 5.6 & 9.9 & 13.7 & 16.8 & 21.5 & 27.4 & 34.2 & 37.6 & 39.7 & 40.9 & 41.8 \\ \hline
1458 & -2.32 & 5.2 & 9.1 & 12.1 & 14.8 & 18.4 & 23.7 & 30.8 & 35 & 38 & 40.1 & 42 \\ \hline
1459 & -2.325 & 5.2 & 8.6 & 11.1 & 13.4 & 16.4 & 20.9 & 27.3 & 31.6 & 35.1 & 37.8 & 40.4 \\ \hline
1460 & -2.33 & 5.1 & 8.2 & 10.5 & 12.7 & 14.9 & 19.2 & 24.7 & 28.7 & 32 & 34.7 & 37.6 \\ \hline
1461 & -2.336 & 5.2 & 8.1 & 10.3 & 12.6 & 14.4 & 18.3 & 23.3 & 27 & 30 & 32.5 & 35.2 \\ \hline
1462 & -2.341 & 5.2 & 8.1 & 10.3 & 12.9 & 14.3 & 18.3 & 23 & 26.5 & 29.2 & 31.6 & 34 \\ \hline
1463 & -2.346 & 5.2 & 8.5 & 10.6 & 13.5 & 14.8 & 19 & 23.7 & 27.1 & 29.8 & 32 & 34.4 \\ \hline
1464 & -2.352 & 5.6 & 8.9 & 11 & 14.4 & 15.6 & 20.2 & 25.2 & 28.7 & 31.3 & 33.5 & 35.9 \\ \hline
1465 & -2.357 & 6.1 & 9.7 & 12.1 & 16 & 16.9 & 22.2 & 27.6 & 31.3 & 34 & 36.2 & 38.4 \\ \hline
1466 & -2.362 & 6.8 & 10.5 & 13.3 & 18 & 18.9 & 24.8 & 30.8 & 34.4 & 37.2 & 39 & 41.1 \\ \hline
1467 & -2.368 & 7.6 & 11.7 & 15.3 & 20.5 & 21.8 & 28.2 & 34.4 & 37.8 & 40.1 & 41.4 & 43 \\ \hline
1468 & -2.373 & 8.8 & 13.3 & 17.4 & 23.6 & 25.4 & 32 & 37.7 & 40.2 & 41.6 & 42.3 & 43 \\ \hline
1469 & -2.378 & 10.4 & 15.7 & 20.4 & 27.3 & 29.7 & 35.7 & 39.8 & 40.8 & 41.2 & 41.2 & 41.2 \\ \hline
1470 & -2.384 & 12.1 & 18.2 & 23.7 & 31.2 & 33.6 & 38 & 39.5 & 39.5 & 39.2 & 39 & 38.8 \\ \hline
1471 & -2.389 & 14.4 & 21.5 & 27.7 & 34.9 & 36.4 & 38.4 & 38.3 & 37.9 & 37.6 & 37.4 & 37.2 \\ \hline
1472 & -2.394 & 17.2 & 25.4 & 31.7 & 37.2 & 37.2 & 37.5 & 37.2 & 37.1 & 37 & 37 & 36.9 \\ \hline
1473 & -2.4 & 20.7 & 29.7 & 35.7 & 38.4 & 37.2 & 37.3 & 37.3 & 37.8 & 37.9 & 38.1 & 38.3 \\ \hline
1474 & -2.405 & 24.6 & 34.2 & 38.3 & 38.8 & 37.2 & 38.1 & 38.9 & 39.8 & 40.1 & 40.4 & 40.9 \\ \hline
1475 & -2.41 & 28.9 & 38.3 & 39.9 & 40 & 38.9 & 40.2 & 41.7 & 42.9 & 43.4 & 43.9 & 44.5 \\ \hline
1476 & -2.416 & 33.2 & 41 & 41.1 & 41.6 & 41.3 & 43 & 45.3 & 46.5 & 47.3 & 47.9 & 48.5 \\ \hline
1477 & -2.421 & 36.2 & 42.6 & 42.3 & 43.4 & 44.2 & 46 & 48.5 & 49.9 & 50.9 & 51.5 & 52.4 \\ \hline
1478 & -2.426 & 35.8 & 42.8 & 42.9 & 43.6 & 45.4 & 47 & 49.6 & 51.1 & 52.1 & 53 & 54 \\ \hline
1479 & -2.432 & 31.6 & 41.3 & 41.7 & 41.6 & 43.7 & 45.1 & 47.6 & 49 & 50.1 & 51.1 & 52 \\ \hline
1480 & -2.437 & 25.4 & 38 & 38.9 & 38 & 39.9 & 40.9 & 43 & 44.2 & 45.3 & 46.3 & 47.1 \\ \hline
1481 & -2.442 & 19.4 & 33.8 & 35.4 & 34.6 & 36 & 36.7 & 38.3 & 39.3 & 40.1 & 40.9 & 41.6 \\ \hline
1482 & -2.448 & 14.6 & 29 & 32.5 & 32.3 & 33.5 & 33.9 & 35 & 35.7 & 36.2 & 36.7 & 37.3 \\ \hline
1483 & -2.453 & 11.4 & 24.2 & 29.7 & 30.8 & 32.3 & 32.4 & 33.4 & 34 & 34.3 & 34.7 & 34.9 \\ \hline
1484 & -2.458 & 9.1 & 19.4 & 26.5 & 29.2 & 31.6 & 32.3 & 33.2 & 33.6 & 33.9 & 34.1 & 34.2 \\ \hline
1485 & -2.464 & 7.6 & 15.8 & 22.5 & 26.7 & 30.8 & 32.5 & 34 & 34.4 & 34.7 & 34.9 & 35 \\ \hline
1486 & -2.469 & 6.4 & 13 & 18.5 & 23.2 & 28.4 & 31.9 & 34.8 & 35.7 & 36 & 36.2 & 36.6 \\ \hline
1487 & -2.474 & 6 & 11.1 & 15.6 & 19.6 & 24.8 & 29.8 & 34.7 & 36.5 & 37.5 & 38 & 38.6 \\ \hline
1488 & -2.48 & 5.6 & 9.8 & 13.3 & 16.8 & 20.9 & 26.4 & 32.8 & 35.9 & 37.9 & 39 & 40.2 \\ \hline
1489 & -2.485 & 5.3 & 9 & 11.8 & 14.8 & 18 & 23 & 29.6 & 33.5 & 36.4 & 38.4 & 40.4 \\ \hline
1490 & -2.49 & 5.2 & 8.5 & 11 & 13.6 & 16 & 20.4 & 26.3 & 30.4 & 33.6 & 36.1 & 38.7 \\ \hline
1491 & -2.496 & 5.2 & 8.3 & 10.6 & 13.1 & 14.9 & 18.8 & 24.1 & 27.9 & 30.9 & 33.4 & 36.2 \\ \hline
1492 & -2.501 & 5.2 & 8.3 & 10.5 & 13.1 & 14.5 & 18.3 & 23 & 26.5 & 29.2 & 31.5 & 34.2 \\ \hline
1493 & -2.506 & 5.2 & 8.6 & 10.8 & 13.6 & 14.7 & 18.6 & 23.2 & 26.4 & 29.1 & 31.1 & 33.5 \\ \hline
1494 & -2.512 & 5.6 & 8.9 & 11.2 & 14.4 & 15.5 & 19.6 & 24.4 & 27.5 & 30 & 31.9 & 34.3 \\ \hline
1495 & -2.517 & 6.1 & 9.7 & 12.1 & 15.6 & 16.8 & 21.4 & 26.4 & 29.7 & 32.1 & 34 & 36.3 \\ \hline
1496 & -2.522 & 6.7 & 10.5 & 13.3 & 17.5 & 18.7 & 23.8 & 29.2 & 32.5 & 35 & 36.9 & 39 \\ \hline
1497 & -2.528 & 7.6 & 11.7 & 14.9 & 19.9 & 21.3 & 27 & 32.8 & 35.9 & 38.2 & 39.7 & 41.4 \\ \hline
1498 & -2.533 & 8.8 & 13.3 & 17.1 & 22.8 & 24.4 & 30.6 & 36.2 & 38.7 & 40.2 & 41.1 & 41.8 \\ \hline
1499 & -2.538 & 10.3 & 15.5 & 19.9 & 26.3 & 28.4 & 34.1 & 38.4 & 39.6 & 40.2 & 40.3 & 40.2 \\ \hline
1500 & -2.544 & 12 & 18.1 & 23 & 30 & 32.4 & 36.6 & 38.5 & 38.5 & 38.5 & 38.3 & 37.9 \\ \hline
1501 & -2.549 & 14.3 & 21.3 & 26.9 & 33.6 & 35.1 & 37 & 37.4 & 36.9 & 36.8 & 36.6 & 36.3 \\ \hline
1502 & -2.554 & 17.1 & 24.9 & 30.9 & 36 & 36.1 & 36.6 & 36.5 & 36.2 & 36.2 & 36.1 & 36 \\ \hline
1503 & -2.56 & 20.5 & 29.3 & 34.9 & 37.2 & 36.3 & 36.6 & 36.9 & 37 & 37.3 & 37.3 & 37.5 \\ \hline
1504 & -2.565 & 24.4 & 33.8 & 37.4 & 38 & 36.8 & 37.7 & 38.5 & 39.1 & 39.6 & 39.8 & 40.2 \\ \hline
1505 & -2.57 & 28.7 & 37.8 & 39.1 & 39.3 & 38.8 & 40 & 41.4 & 42.4 & 43.1 & 43.4 & 43.9 \\ \hline
1506 & -2.576 & 32.7 & 40.3 & 40.3 & 41.1 & 41.3 & 42.8 & 44.9 & 46 & 46.7 & 47.2 & 47.8 \\ \hline
1507 & -2.581 & 35 & 41.6 & 41.5 & 42.4 & 43.7 & 45.3 & 47.6 & 48.8 & 49.6 & 50.3 & 51 \\ \hline
1508 & -2.586 & 33.4 & 41.1 & 41.5 & 42 & 44 & 45.3 & 47.6 & 48.9 & 49.9 & 50.7 & 51.5 \\ \hline
1509 & -2.592 & 28.3 & 38.8 & 39.6 & 39.3 & 41.4 & 42.4 & 44.5 & 45.7 & 46.8 & 47.6 & 48.3 \\ \hline
1510 & -2.597 & 22 & 35.1 & 36.5 & 35.8 & 37.5 & 38.1 & 39.8 & 40.9 & 41.6 & 42.4 & 43 \\ \hline
1511 & -2.602 & 16.8 & 30.7 & 33.3 & 33 & 34.3 & 34.5 & 35.9 & 36.7 & 37.2 & 37.9 & 38.3 \\ \hline
1512 & -2.608 & 12.8 & 25.8 & 30.5 & 31.1 & 32.4 & 32.5 & 33.4 & 33.9 & 34.3 & 34.8 & 35.1 \\ \hline
1513 & -2.613 & 10.1 & 21 & 27.3 & 29.4 & 31.6 & 31.7 & 32.6 & 33 & 33.2 & 33.6 & 33.8 \\ \hline
1514 & -2.619 & 8.4 & 17 & 23.3 & 26.8 & 30.4 & 31.3 & 32.7 & 33.1 & 33.4 & 33.6 & 33.8 \\ \hline
1515 & -2.624 & 7.2 & 13.8 & 19.3 & 23.2 & 28 & 30.9 & 33.3 & 34.1 & 34.5 & 34.7 & 35 \\ \hline
1516 & -2.629 & 6.4 & 11.8 & 16.1 & 19.6 & 24.4 & 28.9 & 33.2 & 34.8 & 35.7 & 36 & 36.6 \\ \hline
1517 & -2.635 & 6 & 10.6 & 13.7 & 16.8 & 20.8 & 25.7 & 31.5 & 34.4 & 36.1 & 37.2 & 38.2 \\ \hline
1518 & -2.64 & 5.6 & 9.8 & 12.2 & 14.9 & 18 & 22.5 & 28.4 & 32.2 & 34.8 & 36.7 & 38.5 \\ \hline
1519 & -2.645 & 5.6 & 9.3 & 11.4 & 14 & 16.2 & 20.1 & 25.6 & 29.3 & 32.4 & 34.7 & 37 \\ \hline
1520 & -2.651 & 5.6 & 9 & 11 & 13.6 & 15.2 & 18.8 & 23.6 & 27 & 30 & 32.3 & 34.7 \\ \hline
1521 & -2.656 & 6 & 9.2 & 11.2 & 13.6 & 14.9 & 18.4 & 23.1 & 26.1 & 28.8 & 31.1 & 33.3 \\ \hline
1522 & -2.661 & 6.3 & 9.5 & 11.4 & 14.2 & 15.3 & 19.1 & 23.6 & 26.5 & 29 & 31.1 & 33.1 \\ \hline
1523 & -2.667 & 6.7 & 10.1 & 12.2 & 15.3 & 16.4 & 20.5 & 25.1 & 28.2 & 30.6 & 32.4 & 34.5 \\ \hline
1524 & -2.672 & 7.2 & 10.9 & 13.3 & 16.8 & 18 & 22.5 & 27.5 & 30.5 & 33 & 34.8 & 36.7 \\ \hline
1525 & -2.677 & 7.9 & 12 & 14.8 & 18.8 & 20.3 & 25.3 & 30.6 & 33.6 & 36 & 37.6 & 39.4 \\ \hline
1526 & -2.683 & 8.8 & 13.3 & 16.4 & 21.3 & 23.1 & 28.4 & 33.8 & 36.5 & 38.4 & 39.5 & 40.6 \\ \hline
1527 & -2.688 & 10 & 15.2 & 18.9 & 24.5 & 26.5 & 31.9 & 36.5 & 38.2 & 39.2 & 39.5 & 39.8 \\ \hline
1528 & -2.693 & 11.6 & 17.5 & 22 & 28.1 & 30.2 & 34.7 & 37.3 & 37.8 & 38 & 37.8 & 37.6 \\ \hline
1529 & -2.699 & 13.6 & 20.5 & 25.6 & 31.7 & 33.5 & 36 & 36.6 & 36.4 & 36.3 & 36 & 35.8 \\ \hline
1530 & -2.704 & 16.1 & 23.8 & 29.4 & 34.4 & 34.9 & 35.5 & 35.5 & 35.4 & 35.3 & 35.1 & 35.1 \\ \hline
1531 & -2.709 & 19.2 & 27.8 & 33.1 & 35.6 & 35.1 & 35.2 & 35.5 & 35.7 & 35.9 & 35.9 & 36.1 \\ \hline
1532 & -2.715 & 22.8 & 31.8 & 35.6 & 36.2 & 35.4 & 35.9 & 36.7 & 37.3 & 37.7 & 37.9 & 38.2 \\ \hline
1533 & -2.72 & 26.8 & 35.6 & 37 & 37.2 & 36.8 & 37.9 & 39.3 & 40.2 & 40.8 & 41.1 & 41.6 \\ \hline
1534 & -2.725 & 30.4 & 38 & 38.2 & 38.8 & 39.2 & 40.5 & 42.5 & 43.6 & 44.4 & 44.7 & 45.4 \\ \hline
1535 & -2.731 & 32.4 & 39.1 & 39.3 & 40.1 & 41.6 & 42.8 & 45.2 & 46.4 & 47.2 & 47.8 & 48.5 \\ \hline
1536 & -2.736 & 31.6 & 39.2 & 39.8 & 40.3 & 42.8 & 44 & 46.6 & 47.7 & 48.8 & 49.6 & 50.4 \\ \hline
1537 & -2.741 & 25.7 & 36.3 & 37.3 & 37 & 39.2 & 40.3 & 42.5 & 43.6 & 44.6 & 45.5 & 46.2 \\ \hline
1538 & -2.747 & 19.6 & 32.2 & 33.8 & 33.3 & 35.1 & 35.7 & 37.6 & 38.4 & 39.2 & 40 & 40.7 \\ \hline
1539 & -2.752 & 14.5 & 27.5 & 30.6 & 30.5 & 32 & 32.5 & 33.9 & 34.5 & 35.2 & 35.7 & 36.2 \\ \hline
1540 & -2.757 & 10.8 & 22.4 & 27.4 & 28.7 & 30.4 & 30.7 & 31.9 & 32.3 & 32.8 & 33.2 & 33.5 \\ \hline
1541 & -2.763 & 8.4 & 17.8 & 24 & 26.7 & 29.6 & 30.3 & 31.5 & 31.6 & 32.2 & 32.4 & 32.7 \\ \hline
1542 & -2.768 & 6.8 & 14.2 & 20 & 23.7 & 28 & 30 & 31.9 & 32.1 & 32.7 & 32.9 & 33.1 \\ \hline
1543 & -2.773 & 6 & 11.7 & 16.4 & 20.1 & 24.9 & 28.8 & 32.2 & 33.2 & 34 & 34.3 & 34.7 \\ \hline
1544 & -2.779 & 5.3 & 9.9 & 13.6 & 16.9 & 21.2 & 26.1 & 31.2 & 33.4 & 34.8 & 35.5 & 36.2 \\ \hline
1545 & -2.784 & 4.9 & 9 & 11.9 & 14.5 & 18 & 22.8 & 28.6 & 31.9 & 34.3 & 35.8 & 37.1 \\ \hline
1546 & -2.789 & 4.8 & 8.2 & 10.7 & 13.2 & 15.6 & 20 & 25.5 & 29.1 & 32 & 34.2 & 36.2 \\ \hline
1547 & -2.795 & 4.8 & 8.1 & 10.2 & 12.4 & 14.4 & 18.4 & 23.2 & 26.7 & 29.6 & 31.8 & 34.2 \\ \hline
1548 & -2.8 & 5.2 & 8.1 & 10.2 & 12.5 & 14 & 17.8 & 22.2 & 25.4 & 28 & 30.1 & 32.4 \\ \hline
1549 & -2.805 & 5.2 & 8.5 & 10.6 & 13.2 & 14.4 & 18.2 & 22.5 & 25.5 & 27.9 & 29.9 & 32 \\ \hline
1550 & -2.811 & 5.6 & 8.9 & 11.1 & 14.4 & 15.2 & 19.4 & 23.9 & 26.8 & 29.2 & 31.1 & 33 \\ \hline
1551 & -2.816 & 6.4 & 9.8 & 12.3 & 16 & 16.8 & 21.5 & 26.3 & 29.2 & 31.6 & 33.4 & 35.1 \\ \hline
1552 & -2.821 & 7.2 & 10.9 & 13.8 & 18 & 19.2 & 24.1 & 29.2 & 32.3 & 34.4 & 36.1 & 37.6 \\ \hline
1553 & -2.827 & 8.2 & 12.5 & 15.8 & 20.8 & 22.3 & 27.5 & 32.7 & 35.4 & 37.1 & 38.1 & 39.1 \\ \hline
1554 & -2.832 & 9.5 & 14.5 & 18.4 & 24 & 25.8 & 30.9 & 35.1 & 36.9 & 37.7 & 38.1 & 38.3 \\ \hline
1555 & -2.837 & 11.3 & 17 & 21.5 & 27.6 & 29.7 & 33.7 & 36 & 36.5 & 36.5 & 36.5 & 36.3 \\ \hline
1556 & -2.843 & 13.4 & 20.2 & 25.2 & 31.2 & 32.5 & 34.6 & 35.2 & 35.2 & 35 & 35 & 34.7 \\ \hline
1557 & -2.848 & 16.2 & 23.8 & 29.1 & 33.6 & 33.7 & 34.2 & 34.4 & 34.5 & 34.6 & 34.6 & 34.7 \\ \hline
1558 & -2.853 & 19.3 & 27.8 & 32.5 & 34.4 & 33.7 & 34.1 & 34.7 & 35.1 & 35.4 & 35.6 & 35.9 \\ \hline
1559 & -2.859 & 22.9 & 31.8 & 34.6 & 34.9 & 34.4 & 35.2 & 36.3 & 37.1 & 37.6 & 38 & 38.4 \\ \hline
1560 & -2.864 & 26.5 & 34.7 & 35.6 & 35.9 & 36 & 37.1 & 38.9 & 39.9 & 40.7 & 41 & 41.6 \\ \hline
1561 & -2.869 & 29.6 & 36.4 & 36.5 & 37.3 & 38.4 & 39.6 & 41.8 & 43 & 43.8 & 44.3 & 45.1 \\ \hline
1562 & -2.875 & 29.6 & 36.5 & 36.9 & 37.3 & 39.3 & 40.6 & 42.9 & 44.2 & 45.2 & 45.9 & 46.7 \\ \hline
1563 & -2.88 & 25.6 & 34.9 & 35.6 & 35.5 & 37.7 & 38.8 & 40.9 & 42.2 & 43.2 & 44.1 & 44.8 \\ \hline
1564 & -2.885 & 20 & 31.4 & 32.6 & 32.3 & 34 & 34.9 & 36.8 & 37.8 & 38.7 & 39.6 & 40.2 \\ \hline
1565 & -2.891 & 14.9 & 27.3 & 29.5 & 29.3 & 30.8 & 31.4 & 32.9 & 33.8 & 34.5 & 35.2 & 35.5 \\ \hline
1566 & -2.896 & 11.1 & 22.5 & 26.6 & 27.3 & 28.8 & 29.3 & 30.5 & 31.3 & 31.7 & 32.4 & 32.6 \\ \hline
1567 & -2.901 & 8.7 & 17.8 & 23.3 & 25.6 & 28 & 28.5 & 29.7 & 30.4 & 30.8 & 31.2 & 31.5 \\ \hline
1568 & -2.907 & 7.1 & 14.2 & 19.5 & 23 & 26.8 & 28.4 & 30.1 & 30.8 & 31.2 & 31.5 & 31.9 \\ \hline
1569 & -2.912 & 6.2 & 11.7 & 16.2 & 19.7 & 24.1 & 27.7 & 30.5 & 31.8 & 32.4 & 32.9 & 33.2 \\ \hline
1570 & -2.917 & 5.5 & 9.9 & 13.4 & 16.7 & 20.6 & 25.3 & 30 & 32.2 & 33.4 & 34.2 & 34.9 \\ \hline
1571 & -2.923 & 5.2 & 9 & 11.8 & 14.6 & 17.6 & 22.3 & 27.8 & 31 & 33 & 34.6 & 36 \\ \hline
1572 & -2.928 & 5.2 & 8.6 & 10.9 & 13.2 & 15.6 & 19.8 & 25 & 28.4 & 31.1 & 33.2 & 35.2 \\ \hline
1573 & -2.933 & 5.2 & 8.5 & 10.5 & 12.9 & 14.4 & 18.4 & 23 & 26.2 & 28.8 & 31.1 & 33.2 \\ \hline
1574 & -2.939 & 5.2 & 8.6 & 10.6 & 13.2 & 14.3 & 18 & 22.2 & 25.2 & 27.6 & 29.5 & 31.6 \\ \hline
1575 & -2.944 & 5.6 & 9 & 11 & 14 & 14.8 & 18.7 & 22.8 & 25.6 & 27.8 & 29.5 & 31.5 \\ \hline
1576 & -2.949 & 6.3 & 9.7 & 12.1 & 15.2 & 16 & 20.1 & 24.4 & 27.1 & 29.3 & 30.9 & 32.7 \\ \hline
1577 & -2.955 & 7.1 & 10.6 & 13.4 & 17.2 & 18 & 22.3 & 26.8 & 29.6 & 31.7 & 33.3 & 35.1 \\ \hline
1578 & -2.96 & 8 & 12.2 & 15.2 & 19.5 & 20.4 & 25.1 & 29.8 & 32.5 & 34.5 & 35.9 & 37.3 \\ \hline
1579 & -2.965 & 9.2 & 14.1 & 17.5 & 22.4 & 23.6 & 28.5 & 33 & 35 & 36.5 & 37.3 & 38 \\ \hline
1580 & -2.971 & 10.6 & 16.5 & 20.3 & 25.6 & 27.2 & 31.8 & 35 & 35.9 & 36.4 & 36.6 & 36.7 \\ \hline
1581 & -2.976 & 12.7 & 19.3 & 23.7 & 29.2 & 30.8 & 33.8 & 35.1 & 35.1 & 35.2 & 35.1 & 35 \\ \hline
1582 & -2.981 & 15.1 & 22.5 & 27.3 & 32 & 32.8 & 33.9 & 34.2 & 34.1 & 34.1 & 34.2 & 34.2 \\ \hline
1583 & -2.987 & 18 & 26.1 & 30.9 & 33.5 & 33.2 & 33.6 & 33.9 & 34.1 & 34.3 & 34.5 & 34.7 \\ \hline
1584 & -2.992 & 21.2 & 29.8 & 33.3 & 33.9 & 33.5 & 33.9 & 34.8 & 35.4 & 35.8 & 35.9 & 36.4 \\ \hline
1585 & -2.997 & 24.8 & 33 & 34.6 & 34.7 & 34.7 & 35.5 & 36.9 & 37.8 & 38.4 & 38.8 & 39.4 \\ \hline
1586 & -3.003 & 28 & 35 & 35.4 & 35.9 & 36.8 & 37.8 & 39.6 & 40.7 & 41.4 & 42 & 42.6 \\ \hline
1587 & -3.008 & 29 & 35.6 & 36.1 & 36.6 & 38.4 & 39.4 & 41.5 & 42.6 & 43.4 & 44.3 & 45 \\ \hline
1588 & -3.013 & 26 & 34.3 & 35.1 & 35.1 & 37.2 & 38.2 & 40.4 & 41.5 & 42.5 & 43.4 & 44.1 \\ \hline
1589 & -3.019 & 20.7 & 31.4 & 32.6 & 32.1 & 34 & 34.8 & 36.8 & 37.8 & 38.7 & 39.6 & 40.2 \\ \hline
1590 & -3.024 & 15.5 & 27.4 & 29.5 & 29.2 & 30.8 & 31.4 & 33 & 33.9 & 34.7 & 35.3 & 35.8 \\ \hline
1591 & -3.029 & 11.6 & 22.6 & 26.8 & 27.3 & 28.9 & 29.4 & 30.7 & 31.4 & 32 & 32.5 & 33 \\ \hline
1592 & -3.035 & 9 & 18 & 23.6 & 25.6 & 27.9 & 28.5 & 29.9 & 30.5 & 30.9 & 31.2 & 31.7 \\ \hline
1593 & -3.04 & 7.2 & 14.4 & 19.7 & 22.8 & 26.6 & 28.2 & 30 & 30.8 & 31 & 31.5 & 31.8 \\ \hline
1594 & -3.045 & 6.3 & 11.7 & 16 & 19.3 & 23.7 & 27 & 30.1 & 31.4 & 31.9 & 32.5 & 32.9 \\ \hline
1595 & -3.051 & 5.7 & 10 & 13.4 & 16.4 & 20.1 & 24.5 & 29.2 & 31.4 & 32.7 & 33.6 & 34.4 \\ \hline
1596 & -3.056 & 5.3 & 9 & 11.8 & 14.1 & 17.1 & 21.3 & 26.8 & 29.7 & 31.9 & 33.5 & 34.9 \\ \hline
1597 & -3.061 & 5.2 & 8.6 & 11 & 13.1 & 15.2 & 19.2 & 24 & 27.2 & 29.9 & 31.9 & 33.8 \\ \hline
1598 & -3.067 & 5.2 & 8.5 & 10.6 & 12.8 & 14.4 & 18 & 22.4 & 25.3 & 27.9 & 29.9 & 31.9 \\ \hline
1599 & -3.072 & 5.4 & 8.8 & 10.9 & 13.2 & 14.4 & 17.9 & 22 & 24.8 & 27.1 & 28.8 & 30.8 \\ \hline
1600 & -3.077 & 5.8 & 9.3 & 11.3 & 14.1 & 15.2 & 18.8 & 22.8 & 25.5 & 27.6 & 29.2 & 31 \\ \hline
1601 & -3.083 & 6.4 & 10.1 & 12.6 & 15.7 & 16.7 & 20.7 & 24.7 & 27.5 & 29.2 & 30.9 & 32.7 \\ \hline
1602 & -3.088 & 7.2 & 11.3 & 14.1 & 17.7 & 18.8 & 23 & 27.4 & 30 & 31.9 & 33.4 & 35 \\ \hline
1603 & -3.093 & 8.4 & 12.9 & 16.1 & 20.3 & 21.6 & 26.1 & 30.5 & 32.8 & 34.4 & 35.4 & 36.6 \\ \hline
1604 & -3.099 & 9.6 & 14.9 & 18.5 & 23.1 & 24.8 & 29.2 & 33 & 34.4 & 35.2 & 35.7 & 36 \\ \hline
1605 & -3.104 & 11.2 & 17.3 & 21.3 & 26.3 & 28.1 & 31.6 & 33.6 & 34 & 34.2 & 34.2 & 34.2 \\ \hline
1606 & -3.109 & 13.2 & 19.9 & 24.3 & 29.1 & 30.4 & 32 & 32.8 & 32.7 & 32.7 & 32.7 & 32.8 \\ \hline
1607 & -3.115 & 15.6 & 23.3 & 27.7 & 30.8 & 31.2 & 31.7 & 32.3 & 32.4 & 32.6 & 32.7 & 32.9 \\ \hline
1608 & -3.12 & 18.7 & 26.9 & 30.6 & 31.6 & 31.6 & 32 & 33 & 33.4 & 33.8 & 34 & 34.5 \\ \hline
1609 & -3.125 & 22.2 & 30.6 & 32.4 & 32.5 & 32.8 & 33.5 & 35 & 35.7 & 36.3 & 36.8 & 37.3 \\ \hline
1610 & -3.131 & 25.6 & 32.9 & 33.4 & 33.7 & 34.7 & 35.6 & 37.6 & 38.6 & 39.4 & 39.9 & 40.5 \\ \hline
1611 & -3.136 & 27.1 & 33.6 & 34.1 & 34.5 & 36.3 & 37.3 & 39.6 & 40.7 & 41.6 & 42.3 & 42.9 \\ \hline
1612 & -3.141 & 24.7 & 32.4 & 33.3 & 33.3 & 35.5 & 36.6 & 38.8 & 40 & 40.9 & 41.8 & 42.5 \\ \hline
1613 & -3.147 & 19.6 & 29.6 & 30.9 & 30.5 & 32.5 & 33.4 & 35.5 & 36.5 & 37.3 & 38.3 & 39 \\ \hline
1614 & -3.152 & 14.4 & 25.7 & 27.7 & 27.6 & 29.3 & 30 & 31.7 & 32.6 & 33.3 & 34.2 & 34.7 \\ \hline
1615 & -3.157 & 10.8 & 21.1 & 24.9 & 25.6 & 27.2 & 27.8 & 29.2 & 30 & 30.5 & 31.3 & 31.8 \\ \hline
1616 & -3.163 & 8 & 16.4 & 21.4 & 23.6 & 26 & 26.8 & 28.2 & 28.8 & 29.3 & 30 & 30.4 \\ \hline
1617 & -3.168 & 6.6 & 13 & 17.8 & 20.8 & 24.4 & 26.4 & 28.2 & 29 & 29.6 & 30.1 & 30.6 \\ \hline
1618 & -3.173 & 5.6 & 10.6 & 14.5 & 17.6 & 21.3 & 25.1 & 28.3 & 29.6 & 30.4 & 31.1 & 31.6 \\ \hline
1619 & -3.179 & 5.2 & 9.3 & 12.3 & 15.1 & 18.1 & 22.7 & 27.2 & 29.6 & 31.2 & 32.2 & 33 \\ \hline
1620 & -3.184 & 4.9 & 8.5 & 11 & 13.5 & 15.7 & 20 & 24.9 & 28 & 30.3 & 31.9 & 33.5 \\ \hline
1621 & -3.189 & 5.1 & 8.2 & 10.5 & 12.7 & 14.4 & 18.3 & 22.9 & 26 & 28.4 & 30.4 & 32.4 \\ \hline
1622 & -3.195 & 5.2 & 8.2 & 10.5 & 12.8 & 14 & 17.6 & 21.7 & 24.7 & 26.9 & 28.8 & 30.7 \\ \hline
1623 & -3.2 & 5.6 & 8.9 & 10.9 & 13.6 & 14.4 & 18 & 21.9 & 24.7 & 26.5 & 28.3 & 30.3 \\ \hline
1624 & -3.205 & 6 & 9.7 & 11.8 & 14.8 & 15.5 & 19.2 & 23.2 & 25.7 & 27.6 & 29.1 & 31 \\ \hline
1625 & -3.211 & 6.8 & 10.6 & 13.3 & 16.4 & 17.4 & 21.3 & 25.3 & 27.9 & 29.7 & 31.1 & 32.7 \\ \hline
1626 & -3.216 & 7.7 & 12.2 & 15 & 18.8 & 19.7 & 23.9 & 28.1 & 30.5 & 32.1 & 33.3 & 34.7 \\ \hline
1627 & -3.221 & 9.1 & 14.1 & 17.3 & 21.6 & 22.8 & 27.1 & 30.9 & 32.7 & 33.8 & 34.5 & 35.1 \\ \hline
1628 & -3.227 & 10.4 & 16.3 & 19.9 & 24.4 & 26.1 & 29.8 & 32.4 & 33.2 & 33.6 & 33.8 & 33.9 \\ \hline
1629 & -3.232 & 12.4 & 19 & 23 & 27.6 & 29.1 & 31.3 & 32.4 & 32.6 & 32.7 & 32.7 & 32.6 \\ \hline
1630 & -3.237 & 14.8 & 22.1 & 26.3 & 30 & 30.5 & 31.3 & 31.7 & 32 & 32.2 & 32.3 & 32.3 \\ \hline
1631 & -3.243 & 17.6 & 25.7 & 29.3 & 31.1 & 30.9 & 31.4 & 32.1 & 32.6 & 33 & 33.1 & 33.5 \\ \hline
1632 & -3.248 & 20.5 & 28.9 & 31 & 31.5 & 31.6 & 32.2 & 33.4 & 34.2 & 34.7 & 35.1 & 35.5 \\ \hline
1633 & -3.253 & 23.5 & 31 & 31.8 & 32.3 & 33 & 33.9 & 35.4 & 36.6 & 37.2 & 37.7 & 38.2 \\ \hline
1634 & -3.259 & 25.1 & 31.6 & 32.2 & 32.8 & 34.3 & 35.2 & 37.2 & 38.4 & 39.1 & 39.7 & 40.4 \\ \hline
1635 & -3.264 & 23.4 & 30.7 & 31.4 & 31.9 & 33.9 & 34.9 & 36.8 & 38 & 38.9 & 39.6 & 40.4 \\ \hline
1636 & -3.269 & 18.9 & 27.9 & 29.1 & 29.2 & 31.2 & 32.1 & 34 & 35 & 35.8 & 36.7 & 37.2 \\ \hline
1637 & -3.275 & 14.1 & 24.3 & 26.2 & 26.4 & 28.2 & 28.9 & 30.7 & 31.5 & 32.2 & 33 & 33.5 \\ \hline
1638 & -3.28 & 10.5 & 20.1 & 23.5 & 24.4 & 26.1 & 26.8 & 28.4 & 29 & 29.7 & 30.3 & 30.7 \\ \hline
1639 & -3.285 & 8 & 15.8 & 20.5 & 22.7 & 24.9 & 25.7 & 27.3 & 28.2 & 28.7 & 29.1 & 29.6 \\ \hline
1640 & -3.291 & 6.7 & 12.6 & 17 & 20 & 23.3 & 25.2 & 27.3 & 28.2 & 28.8 & 29.2 & 29.7 \\ \hline
1641 & -3.296 & 5.9 & 10.6 & 14.1 & 17.2 & 20.7 & 24.1 & 27.3 & 28.7 & 29.6 & 30 & 30.8 \\ \hline
1642 & -3.301 & 5.5 & 9.3 & 12.1 & 14.8 & 17.6 & 21.7 & 26.1 & 28.4 & 30 & 30.8 & 31.9 \\ \hline
1643 & -3.306 & 5.5 & 8.6 & 11.1 & 13.3 & 15.6 & 19.5 & 24.1 & 26.9 & 29.1 & 30.5 & 32.1 \\ \hline
1644 & -3.312 & 5.5 & 8.5 & 10.7 & 12.9 & 14.4 & 18 & 22.2 & 25.1 & 27.3 & 29 & 31 \\ \hline
1645 & -3.317 & 5.6 & 8.9 & 10.9 & 13.2 & 14.1 & 17.6 & 21.5 & 24 & 26 & 27.7 & 29.6 \\ \hline
1646 & -3.322 & 6 & 9.3 & 11.3 & 14 & 14.5 & 18.1 & 21.7 & 24.1 & 26 & 27.5 & 29.2 \\ \hline
1647 & -3.328 & 6.4 & 10.2 & 12.4 & 15.2 & 15.8 & 19.5 & 23.1 & 25.3 & 27.1 & 28.4 & 30 \\ \hline
1648 & -3.333 & 7.2 & 11.3 & 13.8 & 17.1 & 17.7 & 21.5 & 25.1 & 27.3 & 29.1 & 30.1 & 31.5 \\ \hline
1649 & -3.338 & 8.3 & 12.9 & 15.8 & 19.3 & 20.2 & 24.1 & 27.7 & 29.6 & 31.1 & 31.9 & 32.7 \\ \hline
1650 & -3.344 & 9.5 & 14.9 & 18 & 22 & 23.1 & 26.8 & 29.8 & 31.1 & 31.9 & 32.3 & 32.6 \\ \hline
1651 & -3.349 & 11.2 & 17.2 & 20.6 & 24.8 & 26 & 28.9 & 30.5 & 31.1 & 31.4 & 31.5 & 31.5 \\ \hline
1652 & -3.354 & 13.2 & 19.7 & 23.5 & 27.2 & 28 & 29.6 & 30.2 & 30.5 & 30.7 & 30.7 & 30.7 \\ \hline
1653 & -3.36 & 15.6 & 22.9 & 26.6 & 28.8 & 28.9 & 29.6 & 30.2 & 30.7 & 30.8 & 31.1 & 31.3 \\ \hline
1654 & -3.365 & 18 & 25.8 & 28.6 & 29.3 & 29.5 & 30 & 31 & 31.6 & 32 & 32.4 & 32.8 \\ \hline
1655 & -3.37 & 20.8 & 28.1 & 29.5 & 29.8 & 30.4 & 31.2 & 32.7 & 33.6 & 34.1 & 34.6 & 35.1 \\ \hline
1656 & -3.376 & 22.8 & 29 & 29.8 & 30.4 & 31.6 & 32.5 & 34.4 & 35.5 & 36.1 & 36.7 & 37.3 \\ \hline
1657 & -3.381 & 22.1 & 28.5 & 29.4 & 30 & 32 & 32.7 & 34.8 & 35.9 & 36.6 & 37.3 & 38.1 \\ \hline
1658 & -3.386 & 18.4 & 26.3 & 27.4 & 27.6 & 29.8 & 30.6 & 32.5 & 33.6 & 34.4 & 35.2 & 35.9 \\ \hline
1659 & -3.392 & 13.9 & 23 & 24.6 & 24.8 & 26.8 & 27.7 & 29.3 & 30.4 & 31.1 & 31.8 & 32.3 \\ \hline
1660 & -3.397 & 10.2 & 19 & 21.9 & 22.8 & 24.4 & 25.3 & 26.9 & 27.9 & 28.4 & 29 & 29.5 \\ \hline
1661 & -3.402 & 7.8 & 15 & 19.2 & 21.2 & 23.2 & 24.3 & 25.7 & 26.7 & 27.2 & 27.8 & 28.3 \\ \hline
1662 & -3.408 & 6.2 & 11.8 & 16 & 18.8 & 21.9 & 23.8 & 25.7 & 26.7 & 27.2 & 27.8 & 28.3 \\ \hline
1663 & -3.413 & 5.4 & 9.8 & 13.3 & 16.3 & 19.6 & 22.8 & 25.8 & 27.1 & 28 & 28.6 & 29.2 \\ \hline
1664 & -3.418 & 5 & 8.6 & 11.3 & 14 & 16.8 & 20.5 & 24.8 & 26.9 & 28.4 & 29.4 & 30.1 \\ \hline
1665 & -3.424 & 5 & 8.2 & 10.5 & 12.8 & 14.8 & 18.4 & 22.8 & 25.6 & 27.6 & 29 & 30.4 \\ \hline
1666 & -3.429 & 5.2 & 8.2 & 10.2 & 12.4 & 13.9 & 17.2 & 21.2 & 24 & 26 & 27.6 & 29.3 \\ \hline
1667 & -3.434 & 5.6 & 8.6 & 10.6 & 12.8 & 13.9 & 17.2 & 20.8 & 23.2 & 25.2 & 26.7 & 28.4 \\ \hline
1668 & -3.44 & 6 & 9.3 & 11.4 & 14 & 14.6 & 17.9 & 21.5 & 23.7 & 25.6 & 26.9 & 28.4 \\ \hline
1669 & -3.445 & 6.8 & 10.5 & 12.7 & 15.6 & 16.2 & 19.5 & 23.1 & 25.2 & 26.8 & 28.2 & 29.6 \\ \hline
1670 & -3.45 & 7.6 & 11.7 & 14.3 & 17.6 & 18.1 & 21.6 & 25.2 & 27.3 & 28.8 & 29.9 & 31.2 \\ \hline
1671 & -3.456 & 8.8 & 13.4 & 16.2 & 19.9 & 20.8 & 24.3 & 27.6 & 29.4 & 30.7 & 31.4 & 32.2 \\ \hline
1672 & -3.461 & 10 & 15.3 & 18.5 & 22.3 & 23.6 & 26.8 & 29.3 & 30.4 & 30.9 & 31.2 & 31.5 \\ \hline
1673 & -3.466 & 11.6 & 17.6 & 21.1 & 24.9 & 26.4 & 28.5 & 29.7 & 30.2 & 30.4 & 30.4 & 30.6 \\ \hline
1674 & -3.472 & 13.6 & 20.1 & 23.7 & 27 & 27.9 & 28.8 & 29.4 & 29.7 & 29.9 & 30 & 30.2 \\ \hline
1675 & -3.477 & 15.7 & 22.8 & 26.1 & 27.8 & 28.3 & 28.8 & 29.4 & 29.9 & 30.3 & 30.4 & 30.6 \\ \hline
1676 & -3.482 & 17.9 & 25.2 & 27.4 & 28 & 28.7 & 29.2 & 30.2 & 30.9 & 31.4 & 31.7 & 32.1 \\ \hline
1677 & -3.488 & 20.3 & 26.7 & 27.8 & 28.4 & 29.5 & 30.1 & 31.7 & 32.5 & 33.1 & 33.6 & 34.1 \\ \hline
1678 & -3.493 & 21.2 & 26.7 & 27.7 & 28.4 & 30.3 & 30.9 & 32.8 & 33.7 & 34.3 & 34.9 & 35.5 \\ \hline
1679 & -3.498 & 19.1 & 25.5 & 26.6 & 27.2 & 29.4 & 30.1 & 32 & 32.8 & 33.6 & 34.3 & 34.9 \\ \hline
1680 & -3.504 & 15 & 22.7 & 24.2 & 24.6 & 26.7 & 27.4 & 29.2 & 30 & 30.8 & 31.5 & 32.1 \\ \hline
1681 & -3.509 & 11 & 19.4 & 21.7 & 22.2 & 24.3 & 24.9 & 26.4 & 27.2 & 28 & 28.7 & 29.1 \\ \hline
1682 & -3.514 & 8.2 & 15.4 & 19 & 20.2 & 22.6 & 23.3 & 24.8 & 25.6 & 26.4 & 26.8 & 27.2 \\ \hline
1683 & -3.52 & 6.5 & 12.2 & 16.1 & 18.4 & 21.3 & 22.5 & 24.4 & 25.2 & 25.9 & 26.3 & 26.7 \\ \hline
1684 & -3.525 & 5.4 & 9.8 & 13.3 & 16 & 19.3 & 21.7 & 24.2 & 25.2 & 26 & 26.7 & 27.1 \\ \hline
1685 & -3.53 & 4.9 & 8.5 & 11.3 & 14 & 16.8 & 20 & 23.7 & 25.3 & 26.4 & 27.3 & 27.9 \\ \hline
1686 & -3.536 & 4.8 & 7.9 & 10.2 & 12.7 & 14.8 & 18 & 22.1 & 24.4 & 26 & 27.3 & 28.3 \\ \hline
1687 & -3.541 & 5.2 & 7.9 & 10 & 12.3 & 13.8 & 16.9 & 20.8 & 23.2 & 25 & 26.5 & 27.9 \\ \hline
1688 & -3.546 & 5.3 & 8.3 & 10.4 & 12.8 & 13.6 & 16.8 & 20.3 & 22.4 & 24.2 & 25.5 & 27.1 \\ \hline
1689 & -3.552 & 6 & 9.3 & 11.3 & 13.8 & 14.4 & 17.5 & 20.8 & 22.8 & 24.4 & 25.6 & 27.1 \\ \hline
1690 & -3.557 & 6.8 & 10.5 & 12.5 & 15.3 & 15.7 & 19 & 22 & 24 & 25.6 & 26.7 & 27.9 \\ \hline
1691 & -3.562 & 7.6 & 11.8 & 14.1 & 17.2 & 17.7 & 21 & 24 & 26 & 27.3 & 28.3 & 29.5 \\ \hline
1692 & -3.568 & 8.8 & 13.4 & 16 & 19.2 & 20 & 23.2 & 26.3 & 27.9 & 29 & 29.5 & 30.4 \\ \hline
1693 & -3.573 & 10 & 15.3 & 18.1 & 21.6 & 22.7 & 25.6 & 28.1 & 29.1 & 29.6 & 29.9 & 30.2 \\ \hline
1694 & -3.578 & 11.5 & 17.3 & 20.2 & 23.6 & 25.1 & 26.9 & 28.3 & 28.7 & 28.8 & 29.1 & 29.1 \\ \hline
1695 & -3.584 & 13.1 & 19.3 & 22.2 & 25.2 & 26.3 & 27.1 & 27.9 & 28 & 28.1 & 28.3 & 28.3 \\ \hline
1696 & -3.589 & 14.5 & 21.1 & 23.9 & 25.6 & 26.3 & 26.6 & 27.4 & 27.6 & 28 & 28.3 & 28.3 \\ \hline
1697 & -3.594 & 16.3 & 22.8 & 24.8 & 25.6 & 26.3 & 26.8 & 27.8 & 28.4 & 28.8 & 29.1 & 29.5 \\ \hline
1698 & -3.6 & 18 & 23.8 & 25 & 25.7 & 26.8 & 27.6 & 29 & 29.6 & 30.3 & 30.7 & 31.1 \\ \hline
1699 & -3.605 & 18.8 & 23.9 & 25 & 25.7 & 27.6 & 28.3 & 30 & 30.8 & 31.5 & 31.9 & 32.4 \\ \hline
1700 & -3.61 & 16.8 & 22.7 & 24 & 24.5 & 26.8 & 27.4 & 29.2 & 30 & 30.8 & 31.5 & 31.9 \\ \hline
1701 & -3.616 & 13.1 & 20.3 & 21.9 & 22.3 & 24.5 & 25.1 & 26.8 & 27.7 & 28.4 & 29.1 & 29.6 \\ \hline
1702 & -3.621 & 9.6 & 17 & 19.5 & 20.3 & 22.2 & 22.9 & 24.4 & 25.3 & 26 & 26.7 & 27.1 \\ \hline
1703 & -3.626 & 7.2 & 13.7 & 17 & 18.7 & 20.9 & 21.7 & 23.2 & 24.1 & 24.8 & 25.5 & 25.9 \\ \hline
1704 & -3.632 & 5.6 & 10.8 & 14.2 & 16.7 & 19.6 & 20.9 & 22.9 & 23.9 & 24.7 & 25.2 & 25.7 \\ \hline
1705 & -3.637 & 4.8 & 8.9 & 11.8 & 14.4 & 17.6 & 20.1 & 23 & 24.3 & 25.2 & 25.8 & 26.5 \\ \hline
1706 & -3.642 & 4.4 & 7.8 & 10.2 & 12.8 & 15.3 & 18.5 & 22.2 & 24.2 & 25.5 & 26.3 & 27.3 \\ \hline
1707 & -3.647 & 4.4 & 7.4 & 9.6 & 12 & 13.7 & 17 & 20.8 & 23.1 & 24.9 & 26 & 27.3 \\ \hline
1708 & -3.653 & 4.8 & 7.7 & 9.6 & 12 & 13.1 & 16.2 & 19.6 & 21.9 & 23.6 & 24.9 & 26.4 \\ \hline
1709 & -3.658 & 5.4 & 8.5 & 10.4 & 12.8 & 13.6 & 16.5 & 19.6 & 21.6 & 23.2 & 24.3 & 25.8 \\ \hline
1710 & -3.663 & 6.1 & 9.5 & 11.6 & 14 & 14.6 & 17.4 & 20.4 & 22.2 & 23.6 & 24.7 & 25.9 \\ \hline
1711 & -3.669 & 7.2 & 10.9 & 13 & 15.6 & 16.3 & 19.1 & 22 & 23.8 & 25 & 26 & 27.1 \\ \hline
1712 & -3.674 & 8.3 & 12.5 & 14.6 & 17.6 & 18.3 & 21.1 & 24 & 25.7 & 26.8 & 27.5 & 28.4 \\ \hline
1713 & -3.679 & 9.5 & 14.2 & 16.6 & 19.6 & 20.7 & 23.5 & 26 & 27.3 & 28 & 28.5 & 29 \\ \hline
1714 & -3.685 & 10.7 & 16.1 & 18.8 & 22 & 23.1 & 25.4 & 27.1 & 27.6 & 27.9 & 28.1 & 28.3 \\ \hline
1715 & -3.69 & 12.3 & 18.2 & 21.2 & 24 & 25.2 & 26.3 & 27.1 & 27.4 & 27.6 & 27.7 & 27.9 \\ \hline
1716 & -3.695 & 13.8 & 20.2 & 23.2 & 25.2 & 25.9 & 26.3 & 26.9 & 27.2 & 27.5 & 27.6 & 27.9 \\ \hline
1717 & -3.701 & 15.5 & 22.1 & 24.4 & 25.4 & 26 & 26.4 & 27.2 & 27.7 & 28.2 & 28.4 & 28.7 \\ \hline
1718 & -3.706 & 16.9 & 23.1 & 24.5 & 25.3 & 26.4 & 26.8 & 28 & 28.8 & 29.3 & 29.5 & 29.9 \\ \hline
1719 & -3.711 & 17.7 & 23.1 & 24.2 & 24.9 & 26.8 & 27.3 & 28.9 & 29.8 & 30.2 & 30.7 & 31.1 \\ \hline
1720 & -3.717 & 16.2 & 21.8 & 23.1 & 23.7 & 26 & 26.5 & 28.2 & 29.1 & 29.7 & 30.3 & 30.8 \\ \hline
1721 & -3.722 & 12.9 & 19.4 & 21 & 21.6 & 23.7 & 24.3 & 26 & 26.8 & 27.6 & 28.2 & 28.7 \\ \hline
1722 & -3.727 & 9.6 & 16.2 & 18.6 & 19.2 & 21.3 & 21.8 & 23.5 & 24.3 & 25.1 & 25.6 & 26.2 \\ \hline
1723 & -3.733 & 7.2 & 13 & 16.1 & 17.4 & 19.6 & 20.2 & 21.9 & 22.7 & 23.5 & 24 & 24.6 \\ \hline
1724 & -3.738 & 5.6 & 10.2 & 13.3 & 15.4 & 18.2 & 19.3 & 21.3 & 22.2 & 23.1 & 23.6 & 24.2 \\ \hline
1725 & -3.743 & 4.8 & 8.3 & 11 & 13.4 & 16.4 & 18.6 & 21.2 & 22.4 & 23.4 & 24 & 24.7 \\ \hline
1726 & -3.749 & 4.3 & 7.2 & 9.4 & 11.8 & 14.2 & 17.2 & 20.4 & 22.4 & 23.6 & 24.4 & 25.4 \\ \hline
1727 & -3.754 & 4.3 & 6.9 & 8.9 & 11.2 & 13 & 15.9 & 19.2 & 21.6 & 23.2 & 24.2 & 25.4 \\ \hline
1728 & -3.759 & 4.4 & 7 & 8.9 & 11.2 & 12.4 & 15.3 & 18.4 & 20.6 & 22 & 23.3 & 24.6 \\ \hline
1729 & -3.765 & 4.8 & 7.8 & 9.7 & 12.2 & 13.1 & 15.7 & 18.4 & 20.4 & 21.7 & 22.9 & 24.1 \\ \hline
1730 & -3.77 & 5.7 & 9 & 11.1 & 13.6 & 14.3 & 16.9 & 19.5 & 21.2 & 22.4 & 23.4 & 24.4 \\ \hline
1731 & -3.775 & 6.9 & 10.8 & 12.9 & 15.4 & 16.2 & 18.7 & 21.4 & 22.8 & 24 & 24.8 & 25.8 \\ \hline
1732 & -3.781 & 8.4 & 12.5 & 14.8 & 17.5 & 18.3 & 20.8 & 23.4 & 24.8 & 25.7 & 26.4 & 27.2 \\ \hline
1733 & -3.786 & 9.8 & 14.5 & 16.9 & 19.6 & 20.6 & 23.1 & 25.3 & 26.2 & 26.8 & 27.2 & 27.6 \\ \hline
1734 & -3.791 & 11 & 16.1 & 18.8 & 21.6 & 22.6 & 24.5 & 25.8 & 26.3 & 26.6 & 26.8 & 26.9 \\ \hline
1735 & -3.797 & 12.2 & 18 & 20.7 & 23.2 & 24.1 & 24.8 & 25.6 & 25.9 & 26.1 & 26.3 & 26.3 \\ \hline
1736 & -3.802 & 13.4 & 19.7 & 22.2 & 23.6 & 24.4 & 24.7 & 25.4 & 25.6 & 25.9 & 26.2 & 26.3 \\ \hline
1737 & -3.807 & 14.8 & 20.9 & 23 & 23.7 & 24.4 & 24.8 & 25.8 & 26.2 & 26.5 & 26.8 & 27.1 \\ \hline
1738 & -3.813 & 16 & 21.3 & 22.8 & 23.6 & 24.8 & 25.2 & 26.5 & 27 & 27.4 & 27.9 & 28.2 \\ \hline
1739 & -3.818 & 16 & 20.7 & 22.1 & 22.8 & 24.8 & 25.4 & 26.9 & 27.5 & 28 & 28.4 & 28.9 \\ \hline
1740 & -3.823 & 13.6 & 19 & 20.5 & 21.1 & 23.4 & 23.9 & 25.6 & 26.1 & 26.8 & 27.2 & 27.9 \\ \hline
1741 & -3.829 & 10.4 & 16.5 & 18.2 & 18.8 & 21 & 21.5 & 23.2 & 23.8 & 24.6 & 25.1 & 25.7 \\ \hline
1742 & -3.834 & 7.7 & 13.4 & 15.7 & 16.8 & 18.9 & 19.4 & 20.9 & 21.6 & 22.3 & 23 & 23.5 \\ \hline
1743 & -3.839 & 5.8 & 10.6 & 13.3 & 15 & 17.3 & 18.1 & 19.6 & 20.5 & 21.1 & 21.8 & 22.3 \\ \hline
1744 & -3.844 & 4.6 & 8.2 & 10.9 & 12.9 & 15.7 & 17.2 & 18.9 & 20.1 & 20.9 & 21.5 & 22.2 \\ \hline
1745 & -3.85 & 4 & 7 & 9.3 & 11.3 & 13.7 & 16.1 & 18.7 & 20.3 & 21.3 & 22.1 & 23 \\ \hline
1746 & -3.855 & 3.6 & 6.2 & 8.2 & 10.3 & 12.4 & 14.9 & 18.1 & 20 & 21.4 & 22.4 & 23.5 \\ \hline
1747 & -3.86 & 3.7 & 6.2 & 8.1 & 10.2 & 12 & 14.4 & 17.6 & 19.6 & 21.1 & 22.1 & 23.4 \\ \hline
1748 & -3.866 & 4.1 & 6.6 & 8.5 & 10.8 & 12.1 & 14.5 & 17.4 & 19 & 20.4 & 21.5 & 22.6 \\ \hline
1749 & -3.871 & 4.8 & 7.7 & 9.7 & 12 & 13.2 & 15.4 & 17.8 & 19.4 & 20.7 & 21.5 & 22.6 \\ \hline
1750 & -3.876 & 6 & 9.3 & 11.3 & 13.6 & 14.8 & 16.8 & 19 & 20.4 & 21.6 & 22.3 & 23.3 \\ \hline
1751 & -3.882 & 7.3 & 11.2 & 13.2 & 15.6 & 16.4 & 18.6 & 20.8 & 22 & 23.2 & 23.8 & 24.5 \\ \hline
1752 & -3.887 & 8.8 & 13.1 & 15.1 & 17.6 & 18.6 & 20.6 & 22.8 & 23.6 & 24.4 & 24.7 & 25.3 \\ \hline
1753 & -3.892 & 10.4 & 15.1 & 17.3 & 19.6 & 21 & 22.5 & 23.9 & 24.4 & 24.8 & 25.1 & 25.3 \\ \hline
1754 & -3.898 & 11.8 & 17.1 & 19.3 & 21.2 & 22.7 & 23.3 & 24.1 & 24.4 & 24.8 & 24.7 & 24.9 \\ \hline
1755 & -3.903 & 12.9 & 18.6 & 20.9 & 22.1 & 23.3 & 23.3 & 24 & 24.3 & 24.6 & 24.7 & 24.7 \\ \hline
1756 & -3.908 & 13.7 & 19.6 & 21.4 & 22 & 23.2 & 23.1 & 24 & 24.4 & 24.6 & 24.9 & 25.1 \\ \hline
1757 & -3.914 & 14.4 & 19.7 & 21.1 & 21.6 & 23.2 & 23.3 & 24.4 & 24.6 & 25 & 25.3 & 25.5 \\ \hline
1758 & -3.919 & 14.1 & 18.9 & 20.2 & 20.8 & 22.8 & 23 & 24.4 & 24.7 & 25.2 & 25.5 & 25.9 \\ \hline
1759 & -3.924 & 12.2 & 17.2 & 18.6 & 19.2 & 21.6 & 21.8 & 23.3 & 23.8 & 24.4 & 24.7 & 25.1 \\ \hline
1760 & -3.93 & 9.5 & 14.8 & 16.5 & 17.1 & 19.4 & 19.8 & 21.3 & 21.8 & 22.5 & 22.9 & 23.4 \\ \hline
1761 & -3.935 & 7.1 & 12.1 & 14.2 & 15.2 & 17.4 & 17.8 & 19.3 & 20 & 20.8 & 21.3 & 21.8 \\ \hline
1762 & -3.94 & 5.2 & 9.4 & 11.9 & 13.5 & 15.8 & 16.5 & 18.1 & 18.8 & 19.6 & 20.1 & 20.7 \\ \hline
1763 & -3.946 & 4.4 & 7.4 & 9.8 & 11.6 & 14.3 & 15.5 & 17.5 & 18.4 & 19.1 & 19.8 & 20.3 \\ \hline
1764 & -3.951 & 3.6 & 6.2 & 8.2 & 10 & 12.5 & 14.3 & 16.8 & 18.2 & 19.1 & 19.8 & 20.7 \\ \hline
1765 & -3.956 & 3.2 & 5.8 & 7.3 & 9.2 & 11.2 & 13.1 & 16.1 & 17.8 & 19.2 & 20 & 21.1 \\ \hline
1766 & -3.962 & 3.2 & 5.4 & 6.9 & 8.9 & 10.8 & 12.6 & 15.6 & 17.4 & 18.8 & 19.7 & 21 \\ \hline
1767 & -3.967 & 3.6 & 5.8 & 7.3 & 9.6 & 11.2 & 13 & 15.7 & 17.4 & 18.4 & 19.5 & 20.6 \\ \hline
1768 & -3.972 & 4.4 & 6.6 & 8.5 & 10.8 & 12.1 & 13.8 & 16.2 & 17.7 & 18.6 & 19.5 & 20.3 \\ \hline
1769 & -3.978 & 5.3 & 8.1 & 10.1 & 12.3 & 13.6 & 15.1 & 17.2 & 18.5 & 19.3 & 20 & 20.7 \\ \hline
1770 & -3.983 & 6.5 & 9.7 & 11.9 & 14 & 15.2 & 16.7 & 18.6 & 19.7 & 20.4 & 21 & 21.5 \\ \hline
1771 & -3.988 & 8.2 & 12.1 & 13.9 & 16 & 17.5 & 18.7 & 20.3 & 21.1 & 21.6 & 21.9 & 22.3 \\ \hline
1772 & -3.994 & 9.9 & 14.1 & 15.9 & 18 & 19.6 & 20.3 & 21.5 & 21.9 & 22 & 22.3 & 22.6 \\ \hline
1773 & -3.999 & 11.5 & 16.1 & 18 & 19.6 & 21.2 & 21.2 & 22 & 22.3 & 22.4 & 22.7 & 22.9 \\ \hline
1774 & -4.004 & 12.4 & 17.7 & 19.3 & 20.2 & 21.6 & 21.3 & 22 & 22.4 & 22.5 & 22.8 & 23.1 \\ \hline
1775 & -4.009 & 13.1 & 18.4 & 19.7 & 20 & 21.6 & 21.3 & 22 & 22.4 & 22.8 & 23.1 & 23.5 \\ \hline
1776 & -4.015 & 13.2 & 18.1 & 19.3 & 19.6 & 21.3 & 21.3 & 22.1 & 22.5 & 22.8 & 23.1 & 23.5 \\ \hline
1777 & -4.02 & 12.2 & 16.8 & 18.1 & 18.4 & 20.7 & 20.6 & 21.6 & 22.1 & 22.4 & 22.7 & 23.1 \\ \hline
1778 & -4.025 & 9.9 & 14.7 & 16.1 & 16.6 & 19.1 & 19 & 20.1 & 20.7 & 21.1 & 21.5 & 21.9 \\ \hline
1779 & -4.031 & 7.6 & 12.2 & 13.8 & 14.5 & 16.9 & 17.1 & 18.4 & 19 & 19.5 & 19.8 & 20 \\ \hline
1780 & -4.036 & 5.6 & 9.6 & 11.7 & 12.7 & 15 & 15.4 & 16.8 & 17.6 & 18.2 & 18.6 & 18.9 \\ \hline
1781 & -4.041 & 4.4 & 7.5 & 9.7 & 11.1 & 13.4 & 14.2 & 16 & 16.8 & 17.5 & 17.9 & 18.5 \\ \hline
1782 & -4.047 & 3.6 & 6.1 & 7.9 & 9.5 & 11.8 & 13 & 15.2 & 16.3 & 17.1 & 17.6 & 18.3 \\ \hline
1783 & -4.052 & 3.2 & 5.3 & 6.7 & 8.3 & 10.4 & 11.9 & 14.4 & 15.9 & 17 & 17.6 & 18.4 \\ \hline
1784 & -4.057 & 2.8 & 4.8 & 6.1 & 7.6 & 9.6 & 11.1 & 13.7 & 15.4 & 16.6 & 17.5 & 18.4 \\ \hline
1785 & -4.063 & 2.9 & 5 & 6.2 & 8 & 9.6 & 11.2 & 13.6 & 15.2 & 16.4 & 17.5 & 18.3 \\ \hline
1786 & -4.068 & 3.3 & 5.3 & 6.9 & 8.8 & 10.4 & 11.9 & 14 & 15.6 & 16.7 & 17.5 & 18.3 \\ \hline
1787 & -4.073 & 4.1 & 6.4 & 8.1 & 10 & 11.6 & 13.1 & 15.1 & 16.4 & 17.1 & 17.9 & 18.5 \\ \hline
1788 & -4.079 & 5.2 & 7.7 & 9.7 & 11.6 & 13.2 & 14.4 & 16.3 & 17.2 & 17.6 & 18.3 & 18.9 \\ \hline
1789 & -4.084 & 6.6 & 9.7 & 11.7 & 13.6 & 15.2 & 16 & 17.5 & 18.3 & 18.6 & 19.1 & 19.3 \\ \hline
1790 & -4.089 & 8.1 & 11.7 & 13.5 & 15.2 & 16.9 & 17.3 & 18.4 & 18.8 & 19 & 19.4 & 19.5 \\ \hline
1791 & -4.095 & 9.6 & 13.6 & 15.2 & 16.5 & 18.4 & 18.1 & 18.8 & 19.2 & 19.3 & 19.5 & 19.6 \\ \hline
1792 & -4.1 & 10.8 & 14.9 & 16.4 & 17.1 & 18.8 & 18.2 & 18.9 & 19.2 & 19.5 & 19.6 & 19.9 \\ \hline
1793 & -4.105 & 11.2 & 15.8 & 17 & 17.2 & 18.8 & 18.2 & 19.1 & 19.5 & 19.6 & 19.9 & 20.3 \\ \hline
1794 & -4.111 & 11.5 & 15.8 & 16.9 & 16.8 & 18.5 & 18.2 & 19.1 & 19.5 & 19.6 & 20 & 20.4 \\ \hline
1795 & -4.116 & 10.8 & 15 & 16.3 & 16.4 & 18.4 & 17.9 & 18.8 & 19.2 & 19.6 & 19.9 & 20.1 \\ \hline
1796 & -4.121 & 8.8 & 13.4 & 14.7 & 15 & 17.2 & 16.9 & 17.8 & 18.1 & 18.4 & 18.7 & 19.1 \\ \hline
1797 & -4.127 & 6.8 & 11 & 12.7 & 13.3 & 15.6 & 15.5 & 16.4 & 16.8 & 17.2 & 17.5 & 17.7 \\ \hline
1798 & -4.132 & 5.1 & 8.6 & 10.6 & 11.6 & 13.6 & 14 & 15.2 & 15.6 & 16 & 16.4 & 16.8 \\ \hline
1799 & -4.137 & 4.2 & 7 & 8.6 & 10 & 12.1 & 12.8 & 14.4 & 14.9 & 15.6 & 15.9 & 16.4 \\ \hline
1800 & -4.143 & 3.1 & 5.5 & 7 & 8.4 & 10.5 & 11.6 & 13.4 & 14.3 & 15.2 & 15.5 & 16.3 \\ \hline
1801 & -4.148 & 2.7 & 4.7 & 6.1 & 7.2 & 9.2 & 10.5 & 12.6 & 13.8 & 14.8 & 15.5 & 16.3 \\ \hline
1802 & -4.153 & 2.4 & 4.3 & 5.5 & 6.7 & 8.4 & 9.7 & 11.8 & 13.2 & 14.4 & 15.1 & 16.2 \\ \hline
1803 & -4.158 & 2.4 & 4.3 & 5.4 & 6.8 & 8.4 & 9.6 & 11.8 & 13.2 & 14.4 & 15.1 & 16.2 \\ \hline
1804 & -4.164 & 2.8 & 4.6 & 5.7 & 7.2 & 8.8 & 10.1 & 12.3 & 13.6 & 14.5 & 15.2 & 16.2 \\ \hline
1805 & -4.169 & 3.4 & 5.3 & 6.5 & 8.4 & 10.1 & 11.2 & 13.2 & 14.4 & 15.1 & 15.6 & 16.5 \\ \hline
1806 & -4.174 & 4.2 & 6.4 & 7.8 & 9.7 & 11.6 & 12.5 & 14.2 & 15 & 15.6 & 15.9 & 16.5 \\ \hline
1807 & -4.18 & 5.3 & 7.9 & 9.7 & 11.5 & 13.2 & 13.7 & 15 & 15.6 & 16 & 16.1 & 16.5 \\ \hline
1808 & -4.185 & 6.8 & 9.7 & 11.4 & 13 & 14.8 & 14.7 & 15.6 & 16 & 16 & 16.3 & 16.5 \\ \hline
1809 & -4.19 & 8.4 & 11.7 & 13.3 & 14.3 & 16 & 15.6 & 16.4 & 16.5 & 16.7 & 16.8 & 17 \\ \hline
1810 & -4.196 & 10 & 13.4 & 14.6 & 15.1 & 16.8 & 16.2 & 16.9 & 17.2 & 17.5 & 17.5 & 17.8 \\ \hline
1811 & -4.201 & 10.8 & 14.5 & 15.3 & 15.5 & 17.2 & 16.7 & 17.5 & 17.6 & 17.9 & 18.3 & 18.6 \\ \hline
1812 & -4.206 & 10.5 & 14.5 & 15 & 15.1 & 16.9 & 16.4 & 17.2 & 17.5 & 17.7 & 18.2 & 18.3 \\ \hline
1813 & -4.212 & 9.2 & 13.2 & 14.1 & 14.2 & 16.1 & 15.5 & 16.3 & 16.6 & 16.8 & 17.1 & 17.4 \\ \hline
1814 & -4.217 & 7.2 & 11.3 & 12.6 & 12.8 & 14.8 & 14.3 & 15 & 15.2 & 15.5 & 15.8 & 15.9 \\ \hline
1815 & -4.222 & 5.4 & 9 & 10.8 & 11.2 & 13.2 & 13.1 & 14 & 14.3 & 14.4 & 14.7 & 14.8 \\ \hline
1816 & -4.228 & 4.2 & 7 & 8.8 & 9.6 & 11.7 & 12.1 & 13.1 & 13.6 & 13.8 & 13.9 & 14.3 \\ \hline
1817 & -4.233 & 3.2 & 5.4 & 7 & 8 & 10.2 & 10.9 & 12.3 & 12.8 & 13.5 & 13.7 & 14.2 \\ \hline
1818 & -4.238 & 2.4 & 4.5 & 5.7 & 6.8 & 8.6 & 9.7 & 11.2 & 12 & 12.7 & 13.3 & 13.9 \\ \hline
1819 & -4.244 & 2 & 3.8 & 4.9 & 6 & 7.5 & 8.5 & 10.3 & 11.2 & 12.1 & 12.7 & 13.5 \\ \hline
1820 & -4.249 & 2 & 3.4 & 4.5 & 5.6 & 6.8 & 7.7 & 9.5 & 10.7 & 11.6 & 12.3 & 13.1 \\ \hline
1821 & -4.254 & 2 & 3.4 & 4.5 & 5.6 & 6.8 & 7.8 & 9.6 & 10.8 & 11.6 & 12.3 & 13.1 \\ \hline
1822 & -4.26 & 2.4 & 3.8 & 4.9 & 6 & 7.6 & 8.5 & 10.4 & 11.2 & 12.2 & 12.7 & 13.5 \\ \hline
1823 & -4.265 & 2.8 & 4.5 & 5.7 & 7.2 & 8.8 & 9.7 & 11.2 & 12 & 12.8 & 13.2 & 13.8 \\ \hline
1824 & -4.27 & 3.6 & 5.3 & 6.9 & 8.4 & 10 & 10.8 & 12 & 12.4 & 12.8 & 13.2 & 13.5 \\ \hline
1825 & -4.275 & 4.4 & 6.5 & 8.3 & 9.6 & 11.5 & 11.4 & 12.3 & 12.6 & 12.8 & 13.1 & 13.1 \\ \hline
1826 & -4.281 & 5.6 & 8.1 & 9.9 & 10.8 & 12.4 & 11.9 & 12.7 & 12.8 & 12.9 & 13.1 & 13.1 \\ \hline
1827 & -4.286 & 7.2 & 9.8 & 11.3 & 11.7 & 13.2 & 12.7 & 13.4 & 13.6 & 13.7 & 13.9 & 13.9 \\ \hline
1828 & -4.291 & 8.8 & 11.3 & 12.1 & 12.4 & 14 & 13.5 & 14.2 & 14.4 & 14.6 & 14.9 & 15.1 \\ \hline
1829 & -4.297 & 9.2 & 12.1 & 12.5 & 12.6 & 14.4 & 13.9 & 14.6 & 14.8 & 15.2 & 15.5 & 15.7 \\ \hline
1830 & -4.302 & 8.4 & 11.6 & 12.1 & 12 & 13.9 & 13.2 & 13.9 & 14 & 14.4 & 14.8 & 15.1 \\ \hline
1831 & -4.307 & 6.8 & 10.2 & 11 & 11 & 12.8 & 12.2 & 12.8 & 12.8 & 13.2 & 13.2 & 13.5 \\ \hline
1832 & -4.313 & 5.2 & 8.2 & 9.4 & 9.8 & 11.6 & 11.3 & 11.6 & 11.7 & 12 & 12 & 12.3 \\ \hline
1833 & -4.318 & 3.7 & 6.5 & 7.8 & 8.5 & 10.4 & 10.5 & 11.2 & 11.2 & 11.6 & 11.5 & 11.8 \\ \hline
1834 & -4.323 & 2.9 & 5 & 6.2 & 6.9 & 9.2 & 9.5 & 10.6 & 10.9 & 11.2 & 11.5 & 11.8 \\ \hline
1835 & -4.329 & 2.3 & 4.1 & 5 & 5.8 & 7.7 & 8.4 & 9.8 & 10.5 & 11 & 11.5 & 11.8 \\ \hline
1836 & -4.334 & 1.9 & 3.3 & 4.2 & 5 & 6.5 & 7.3 & 8.7 & 9.6 & 10.3 & 10.9 & 11.4 \\ \hline
1837 & -4.339 & 1.7 & 3 & 3.8 & 4.6 & 5.9 & 6.4 & 7.9 & 8.8 & 9.5 & 10.2 & 10.8 \\ \hline
1838 & -4.345 & 1.6 & 2.8 & 3.4 & 4.4 & 5.6 & 6.1 & 7.5 & 8.4 & 9.1 & 9.8 & 10.4 \\ \hline
1839 & -4.35 & 1.6 & 2.8 & 3.5 & 4.4 & 5.6 & 6.2 & 7.6 & 8.4 & 9.1 & 9.8 & 10.3 \\ \hline
1840 & -4.355 & 1.7 & 3 & 3.8 & 4.8 & 6 & 6.6 & 8 & 8.7 & 9.2 & 9.8 & 10.1 \\ \hline
1841 & -4.361 & 2.1 & 3.4 & 4.5 & 5.6 & 7 & 7.4 & 8.4 & 9.1 & 9.2 & 9.7 & 9.9 \\ \hline
1842 & -4.366 & 2.8 & 4.2 & 5.6 & 6.8 & 8.2 & 8.1 & 8.8 & 9.1 & 9.2 & 9.4 & 9.5 \\ \hline
1843 & -4.371 & 4 & 5.5 & 6.9 & 7.9 & 9.3 & 8.9 & 9.2 & 9.5 & 9.6 & 9.6 & 9.8 \\ \hline
1844 & -4.377 & 5.3 & 7.1 & 8.1 & 8.8 & 10.1 & 9.5 & 10 & 10.1 & 10.3 & 10.3 & 10.6 \\ \hline
1845 & -4.382 & 6.8 & 8.3 & 9 & 9.2 & 10.8 & 10.3 & 10.8 & 10.9 & 11.1 & 11.2 & 11.6 \\ \hline
1846 & -4.387 & 7.4 & 8.9 & 9.4 & 9.6 & 11.2 & 10.7 & 11.2 & 11.4 & 11.6 & 11.9 & 12.3 \\ \hline
1847 & -4.392 & 6.8 & 8.9 & 9.3 & 9.2 & 10.9 & 10.4 & 10.8 & 11.2 & 11.3 & 11.7 & 11.9 \\ \hline
1848 & -4.398 & 5.3 & 7.8 & 8.5 & 8.4 & 10 & 9.5 & 10 & 10 & 10.3 & 10.5 & 10.7 \\ \hline
1849 & -4.403 & 4.3 & 6.6 & 7.6 & 7.6 & 9.2 & 8.8 & 9.2 & 9.2 & 9.4 & 9.5 & 9.6 \\ \hline
1850 & -4.408 & 3.1 & 5 & 6.3 & 6.5 & 8.3 & 8.1 & 8.6 & 8.8 & 8.8 & 8.8 & 8.9 \\ \hline
1851 & -4.414 & 2.3 & 3.9 & 5 & 5.6 & 7.2 & 7.3 & 8.2 & 8.4 & 8.7 & 8.7 & 8.9 \\ \hline
1852 & -4.419 & 1.7 & 3 & 4.1 & 4.4 & 6 & 6.5 & 7.6 & 8 & 8.4 & 8.7 & 9 \\ \hline
1853 & -4.424 & 1.4 & 2.6 & 3.4 & 4 & 5.2 & 5.7 & 6.8 & 7.6 & 8 & 8.4 & 9 \\ \hline
1854 & -4.43 & 1.2 & 2.2 & 3 & 3.6 & 4.7 & 5.2 & 6 & 6.8 & 7.5 & 7.9 & 8.6 \\ \hline
1855 & -4.435 & 1.2 & 2.2 & 2.9 & 3.6 & 4.5 & 5 & 5.9 & 6.8 & 7.2 & 7.6 & 8.3 \\ \hline
1856 & -4.44 & 1.2 & 2.2 & 2.9 & 3.6 & 4.6 & 5.2 & 6.2 & 6.8 & 7.4 & 7.9 & 8.3 \\ \hline
1857 & -4.446 & 1.6 & 2.5 & 3.3 & 4 & 5.1 & 5.7 & 6.8 & 7.4 & 7.8 & 8.2 & 8.5 \\ \hline
1858 & -4.451 & 1.9 & 2.9 & 3.7 & 4.8 & 6 & 6.5 & 7.2 & 7.6 & 7.9 & 8.2 & 8.3 \\ \hline
1859 & -4.456 & 2.4 & 3.6 & 4.5 & 5.6 & 6.8 & 6.9 & 7.3 & 7.6 & 7.6 & 7.8 & 7.9 \\ \hline
1860 & -4.462 & 3.2 & 4.4 & 5.3 & 6 & 7.3 & 6.9 & 7.2 & 7.2 & 7.4 & 7.5 & 7.7 \\ \hline
1861 & -4.467 & 4.3 & 5.3 & 6.1 & 6.4 & 7.7 & 7.2 & 7.6 & 7.6 & 7.8 & 7.9 & 8.1 \\ \hline
1862 & -4.472 & 5.2 & 6.1 & 6.5 & 6.8 & 8 & 7.6 & 8 & 8.1 & 8.3 & 8.3 & 8.7 \\ \hline
1863 & -4.478 & 5.4 & 6.5 & 6.6 & 6.8 & 8.4 & 7.9 & 8.4 & 8.5 & 8.7 & 8.7 & 9.1 \\ \hline
1864 & -4.483 & 4.8 & 6.2 & 6.5 & 6.4 & 8 & 7.5 & 8 & 8.1 & 8.3 & 8.3 & 8.7 \\ \hline
1865 & -4.488 & 3.6 & 5.5 & 5.8 & 6 & 7.4 & 6.9 & 7.2 & 7.3 & 7.6 & 7.5 & 7.9 \\ \hline
1866 & -4.493 & 2.7 & 4.5 & 5.1 & 5.6 & 6.8 & 6.3 & 6.8 & 6.8 & 6.8 & 6.8 & 7.1 \\ \hline
1867 & -4.499 & 2 & 3.5 & 4.3 & 4.9 & 6.1 & 5.9 & 6.5 & 6.6 & 6.7 & 6.7 & 6.7 \\ \hline
1868 & -4.504 & 1.3 & 2.7 & 3.4 & 4.1 & 5.3 & 5.6 & 6.2 & 6.4 & 6.7 & 6.7 & 6.7 \\ \hline
1869 & -4.509 & 1.1 & 2.2 & 2.9 & 3.3 & 4.5 & 4.9 & 5.8 & 6.1 & 6.4 & 6.7 & 6.7 \\ \hline
1870 & -4.515 & 0.8 & 1.8 & 2.5 & 2.9 & 3.9 & 4.1 & 5.1 & 5.6 & 6 & 6.2 & 6.4 \\ \hline
1871 & -4.52 & 0.8 & 1.8 & 2.1 & 2.6 & 3.6 & 3.7 & 4.7 & 5.2 & 5.6 & 5.8 & 6.1 \\ \hline
1872 & -4.525 & 0.8 & 1.7 & 2.1 & 2.6 & 3.5 & 3.7 & 4.6 & 5.1 & 5.6 & 5.5 & 6 \\ \hline
1873 & -4.531 & 0.8 & 1.8 & 2.1 & 2.8 & 3.6 & 4 & 4.9 & 5.2 & 5.6 & 5.8 & 6.1 \\ \hline
1874 & -4.536 & 0.9 & 1.8 & 2.3 & 3.2 & 4 & 4.4 & 5.2 & 5.2 & 5.6 & 5.9 & 6.1 \\ \hline
1875 & -4.541 & 1.3 & 2.1 & 2.7 & 3.6 & 4.8 & 4.8 & 5.2 & 5.2 & 5.6 & 5.6 & 5.8 \\ \hline
1876 & -4.547 & 1.6 & 2.5 & 3.3 & 4 & 5.2 & 4.9 & 5.2 & 5.2 & 5.5 & 5.5 & 5.5 \\ \hline
1877 & -4.552 & 2.4 & 3.1 & 3.7 & 4.4 & 5.2 & 4.9 & 5.2 & 5.2 & 5.5 & 5.5 & 5.5 \\ \hline
1878 & -4.557 & 2.9 & 3.6 & 4.1 & 4.4 & 5.3 & 5.1 & 5.2 & 5.3 & 5.6 & 5.8 & 5.9 \\ \hline
1879 & -4.563 & 3.5 & 4 & 4.3 & 4.4 & 5.7 & 5.3 & 5.6 & 5.7 & 6 & 6.1 & 6.3 \\ \hline
1880 & -4.568 & 3.5 & 4 & 4.3 & 4.4 & 5.6 & 5.3 & 5.6 & 5.6 & 5.9 & 6.1 & 6.3 \\ \hline
1881 & -4.573 & 2.8 & 3.8 & 4.2 & 4.4 & 5.6 & 5 & 5.2 & 5.4 & 5.6 & 5.6 & 5.9 \\ \hline
1882 & -4.578 & 2 & 3.3 & 3.7 & 4 & 5.2 & 4.6 & 4.8 & 4.9 & 5.2 & 5.1 & 5.2 \\ \hline
1883 & -4.584 & 1.6 & 2.6 & 3.3 & 3.6 & 4.8 & 4.5 & 4.7 & 4.8 & 4.8 & 4.7 & 4.8 \\ \hline
1884 & -4.589 & 1.2 & 2.2 & 2.6 & 3.2 & 4.2 & 4.1 & 4.4 & 4.7 & 4.8 & 4.7 & 4.8 \\ \hline
1885 & -4.594 & 0.8 & 1.8 & 2.2 & 2.8 & 3.6 & 3.7 & 4.4 & 4.5 & 4.8 & 4.7 & 4.8 \\ \hline
1886 & -4.6 & 0.8 & 1.4 & 1.8 & 2.4 & 3.1 & 3.3 & 4 & 4.1 & 4.4 & 4.5 & 4.7 \\ \hline
1887 & -4.605 & 0.8 & 1.4 & 1.7 & 2 & 2.7 & 2.9 & 3.6 & 3.8 & 4.1 & 4.3 & 4.6 \\ \hline
1888 & -4.61 & 0.7 & 1.3 & 1.7 & 2 & 2.4 & 2.8 & 3.3 & 3.6 & 4 & 4.1 & 4.3 \\ \hline
1889 & -4.616 & 0.7 & 1.3 & 1.7 & 2 & 2.4 & 2.8 & 3.3 & 3.6 & 4 & 4.1 & 4.3 \\ \hline
1890 & -4.621 & 0.7 & 1.3 & 1.7 & 2 & 2.8 & 2.9 & 3.4 & 3.6 & 4 & 4.1 & 4.3 \\ \hline
1891 & -4.626 & 0.8 & 1.4 & 1.7 & 2.3 & 3.2 & 3.2 & 3.6 & 3.7 & 4 & 4.1 & 4.3 \\ \hline
1892 & -4.632 & 0.8 & 1.4 & 1.9 & 2.4 & 3.3 & 3.2 & 3.6 & 3.6 & 3.9 & 3.9 & 4 \\ \hline
1893 & -4.637 & 1.2 & 1.8 & 2.2 & 2.7 & 3.6 & 3.3 & 3.6 & 3.6 & 3.9 & 3.9 & 3.9 \\ \hline
1894 & -4.642 & 1.6 & 2.1 & 2.4 & 2.8 & 3.6 & 3.3 & 3.6 & 3.6 & 3.9 & 3.9 & 3.9 \\ \hline
1895 & -4.648 & 2 & 2.2 & 2.5 & 2.8 & 3.7 & 3.3 & 3.7 & 3.9 & 4 & 3.9 & 4 \\ \hline
1896 & -4.653 & 2 & 2.2 & 2.5 & 2.8 & 3.6 & 3.3 & 3.6 & 3.9 & 4 & 3.9 & 4 \\ \hline
1897 & -4.658 & 1.8 & 2.2 & 2.5 & 2.8 & 3.6 & 3.3 & 3.6 & 3.7 & 3.7 & 3.9 & 3.9 \\ \hline
1898 & -4.663 & 1.3 & 1.8 & 2.2 & 2.4 & 3.2 & 3 & 3.3 & 3.4 & 3.5 & 3.5 & 3.5 \\ \hline
1899 & -4.669 & 0.9 & 1.7 & 2 & 2.3 & 3 & 2.9 & 3.2 & 3.2 & 3.4 & 3.4 & 3.5 \\ \hline
1900 & -4.674 & 0.7 & 1.4 & 1.7 & 2 & 2.7 & 2.6 & 2.9 & 3.1 & 3.2 & 3.2 & 3.4 \\ \hline
1901 & -4.679 & 0.6 & 1.3 & 1.6 & 2 & 2.4 & 2.5 & 2.8 & 3 & 3.2 & 3.2 & 3.4 \\ \hline
1902 & -4.685 & 0.4 & 1 & 1.3 & 1.6 & 2 & 2.2 & 2.6 & 2.8 & 2.9 & 3 & 3.1 \\ \hline
1903 & -4.69 & 0.4 & 0.9 & 1.3 & 1.6 & 2 & 2.1 & 2.4 & 2.8 & 2.8 & 3 & 3.1 \\ \hline
1904 & -4.695 & 0.4 & 0.8 & 1.1 & 1.5 & 1.9 & 2 & 2.4 & 2.5 & 2.8 & 2.7 & 3 \\ \hline
1905 & -4.701 & 0.4 & 0.8 & 1.1 & 1.5 & 1.9 & 2 & 2.4 & 2.5 & 2.8 & 2.7 & 3 \\ \hline
1906 & -4.706 & 0.4 & 0.8 & 1.1 & 1.5 & 1.9 & 2 & 2.4 & 2.4 & 2.7 & 2.7 & 2.8 \\ \hline
1907 & -4.711 & 0.4 & 0.8 & 1.1 & 1.5 & 1.9 & 2 & 2.4 & 2.4 & 2.7 & 2.7 & 2.7 \\ \hline
1908 & -4.717 & 0.4 & 0.8 & 1.1 & 1.5 & 1.9 & 2 & 2.4 & 2.4 & 2.4 & 2.6 & 2.7 \\ \hline
1909 & -4.722 & 0.4 & 1 & 1.3 & 1.6 & 2 & 2 & 2.4 & 2.4 & 2.4 & 2.6 & 2.7 \\ \hline
1910 & -4.727 & 0.4 & 1 & 1.3 & 1.6 & 2 & 2 & 2.4 & 2.4 & 2.4 & 2.3 & 2.6 \\ \hline
1911 & -4.733 & 0.4 & 1 & 1.3 & 1.6 & 2 & 2 & 2.4 & 2.4 & 2.4 & 2.3 & 2.5 \\ \hline
1912 & -4.738 & 0.4 & 1 & 1.3 & 1.6 & 2 & 2 & 2 & 2.2 & 2.4 & 2.3 & 2.3 \\ \hline
1913 & -4.743 & 0.4 & 1 & 1.3 & 1.6 & 2 & 2 & 2 & 2.1 & 2.4 & 2.3 & 2.3 \\ \hline
1914 & -4.748 & 0.4 & 0.9 & 1.3 & 1.5 & 1.9 & 1.9 & 2 & 2 & 2.1 & 2.2 & 2.3 \\ \hline
1915 & -4.754 & 0.4 & 0.9 & 1.2 & 1.4 & 1.8 & 1.9 & 2 & 2 & 2 & 2.1 & 2.2 \\ \hline
1916 & -4.759 & 0.4 & 0.7 & 1 & 1.2 & 1.6 & 1.6 & 2 & 2 & 2 & 1.9 & 2 \\ \hline
1917 & -4.764 & 0.4 & 0.6 & 0.9 & 1.2 & 1.6 & 1.6 & 1.9 & 2 & 2 & 1.9 & 1.9 \\ \hline
1918 & -4.77 & 0.4 & 0.6 & 0.7 & 0.9 & 1.4 & 1.6 & 1.7 & 1.9 & 2 & 1.9 & 1.9 \\ \hline
1919 & -4.775 & 0.4 & 0.6 & 0.7 & 0.9 & 1.3 & 1.6 & 1.7 & 1.8 & 2 & 1.9 & 1.9 \\ \hline
1920 & -4.78 & 0.3 & 0.6 & 0.6 & 0.8 & 1.2 & 1.4 & 1.6 & 1.6 & 1.9 & 1.9 & 1.9 \\ \hline
1921 & -4.786 & 0.2 & 0.6 & 0.6 & 0.8 & 1.2 & 1.3 & 1.6 & 1.6 & 1.8 & 1.9 & 1.9 \\ \hline
1922 & -4.791 & 0.2 & 0.6 & 0.6 & 0.8 & 1.1 & 1.2 & 1.6 & 1.6 & 1.6 & 1.7 & 1.8 \\ \hline
1923 & -4.796 & 0.2 & 0.6 & 0.5 & 0.8 & 1.1 & 1.2 & 1.6 & 1.6 & 1.6 & 1.6 & 1.8 \\ \hline
1924 & -4.802 & 0.2 & 0.6 & 0.5 & 0.8 & 1 & 1.2 & 1.4 & 1.6 & 1.6 & 1.5 & 1.6 \\ \hline
1925 & -4.807 & 0.2 & 0.6 & 0.5 & 0.8 & 0.9 & 1.2 & 1.4 & 1.6 & 1.6 & 1.5 & 1.6 \\ \hline
1926 & -4.812 & 0.1 & 0.6 & 0.5 & 0.8 & 0.8 & 1 & 1.2 & 1.4 & 1.6 & 1.5 & 1.5 \\ \hline
1927 & -4.818 & 0.1 & 0.6 & 0.5 & 0.8 & 0.8 & 1 & 1.2 & 1.3 & 1.6 & 1.5 & 1.5 \\ \hline
1928 & -4.823 & 0.1 & 0.5 & 0.5 & 0.8 & 0.8 & 0.8 & 1.2 & 1.2 & 1.4 & 1.4 & 1.5 \\ \hline
1929 & -4.828 & 0 & 0.5 & 0.5 & 0.8 & 0.8 & 0.8 & 1.2 & 1.2 & 1.3 & 1.4 & 1.5 \\ \hline
1930 & -4.833 & 0 & 0.5 & 0.5 & 0.8 & 0.8 & 0.8 & 1.1 & 1.2 & 1.2 & 1.4 & 1.4 \\ \hline
1931 & -4.839 & 0 & 0.5 & 0.5 & 0.8 & 0.8 & 0.8 & 1.1 & 1.2 & 1.2 & 1.4 & 1.4 \\ \hline
1932 & -4.844 & 0 & 0.5 & 0.5 & 0.8 & 0.8 & 0.8 & 0.9 & 1.1 & 1.2 & 1.2 & 1.3 \\ \hline
1933 & -4.849 & 0 & 0.5 & 0.5 & 0.8 & 0.8 & 0.8 & 0.9 & 1.1 & 1.2 & 1.2 & 1.3 \\ \hline
1934 & -4.855 & 0 & 0.5 & 0.5 & 0.8 & 0.8 & 0.8 & 0.9 & 1.1 & 1.2 & 1.1 & 1.2 \\ \hline
1935 & -4.86 & 0 & 0.5 & 0.5 & 0.8 & 0.8 & 0.8 & 0.8 & 1.1 & 1.2 & 1.1 & 1.1 \\ \hline
1936 & -4.865 & 0 & 0.5 & 0.5 & 0.8 & 0.8 & 0.8 & 0.8 & 1 & 1.1 & 1.1 & 1.1 \\ \hline
1937 & -4.871 & 0 & 0.5 & 0.5 & 0.8 & 0.8 & 0.8 & 0.8 & 1 & 1.1 & 1.1 & 1.1 \\ \hline
1938 & -4.876 & 0 & 0.5 & 0.5 & 0.8 & 0.8 & 0.8 & 0.8 & 0.8 & 1.1 & 1.1 & 1.1 \\ \hline
1939 & -4.881 & 0 & 0.5 & 0.5 & 0.8 & 0.8 & 0.8 & 0.8 & 0.8 & 1.1 & 1.1 & 1.1 \\ \hline
1940 & -4.887 & 0 & 0.5 & 0.5 & 0.6 & 0.8 & 0.8 & 0.8 & 0.8 & 1 & 1.1 & 1.1 \\ \hline
1941 & -4.892 & 0 & 0.4 & 0.5 & 0.6 & 0.8 & 0.8 & 0.8 & 0.8 & 1 & 1.1 & 1.1 \\ \hline
1942 & -4.897 & 0 & 0.3 & 0.5 & 0.5 & 0.8 & 0.8 & 0.8 & 0.8 & 0.9 & 1 & 1.1 \\ \hline
1943 & -4.902 & 0 & 0.3 & 0.5 & 0.5 & 0.8 & 0.8 & 0.8 & 0.8 & 0.8 & 1 & 1 \\ \hline
1944 & -4.908 & 0 & 0.2 & 0.5 & 0.5 & 0.8 & 0.8 & 0.8 & 0.8 & 0.8 & 0.9 & 1 \\ \hline
1945 & -4.913 & 0 & 0.2 & 0.5 & 0.5 & 0.8 & 0.8 & 0.8 & 0.8 & 0.8 & 0.9 & 1 \\ \hline
1946 & -4.918 & 0 & 0.2 & 0.4 & 0.4 & 0.8 & 0.8 & 0.8 & 0.8 & 0.8 & 0.8 & 1 \\ \hline
1947 & -4.924 & 0 & 0.2 & 0.4 & 0.4 & 0.8 & 0.8 & 0.8 & 0.8 & 0.8 & 0.8 & 1 \\ \hline
1948 & -4.929 & 0 & 0.2 & 0.4 & 0.4 & 0.8 & 0.8 & 0.8 & 0.8 & 0.8 & 0.8 & 1 \\ \hline
1949 & -4.934 & 0 & 0.2 & 0.4 & 0.4 & 0.8 & 0.8 & 0.8 & 0.8 & 0.8 & 0.8 & 1 \\ \hline
1950 & -4.94 & 0 & 0.2 & 0.4 & 0.4 & 0.8 & 0.8 & 0.8 & 0.8 & 0.8 & 0.7 & 1 \\ \hline
1951 & -4.945 & 0 & 0.2 & 0.4 & 0.4 & 0.8 & 0.8 & 0.8 & 0.8 & 0.8 & 0.7 & 1 \\ \hline
1952 & -4.95 & 0 & 0.2 & 0.4 & 0.4 & 0.8 & 0.8 & 0.8 & 0.8 & 0.8 & 0.7 & 0.9 \\ \hline
1953 & -4.956 & 0 & 0.2 & 0.4 & 0.4 & 0.8 & 0.8 & 0.8 & 0.8 & 0.8 & 0.7 & 0.9 \\ \hline
1954 & -4.961 & 0 & 0.2 & 0.4 & 0.4 & 0.7 & 0.8 & 0.8 & 0.8 & 0.8 & 0.7 & 0.8 \\ \hline
1955 & -4.966 & 0 & 0.2 & 0.4 & 0.4 & 0.7 & 0.8 & 0.8 & 0.8 & 0.8 & 0.7 & 0.9 \\ \hline
1956 & -4.971 & 0 & 0.2 & 0.4 & 0.4 & 0.7 & 0.8 & 0.8 & 0.8 & 0.8 & 0.7 & 0.8 \\ \hline
1957 & -4.977 & 0 & 0.2 & 0.4 & 0.4 & 0.7 & 0.8 & 0.8 & 0.8 & 0.8 & 0.7 & 0.8 \\ \hline
1958 & -4.982 & 0 & 0.2 & 0.3 & 0.4 & 0.7 & 0.8 & 0.8 & 0.8 & 0.8 & 0.7 & 0.7 \\ \hline
1959 & -4.987 & 0 & 0.2 & 0.3 & 0.4 & 0.7 & 0.8 & 0.8 & 0.8 & 0.8 & 0.7 & 0.7 \\ \hline
1960 & -4.993 & 0 & 0.2 & 0.3 & 0.4 & 0.6 & 0.8 & 0.8 & 0.8 & 0.8 & 0.7 & 0.7 \\ \hline
1961 & -4.998 & 0 & 0.2 & 0.3 & 0.4 & 0.6 & 0.8 & 0.8 & 0.8 & 0.8 & 0.7 & 0.7 \\ \hline
1962 & -5.003 & 0 & 0.2 & 0.3 & 0.4 & 0.6 & 0.8 & 0.8 & 0.8 & 0.8 & 0.7 & 0.7 \\ \hline
1963 & -5.009 & 0 & 0.2 & 0.3 & 0.4 & 0.6 & 0.8 & 0.8 & 0.8 & 0.8 & 0.7 & 0.7 \\ \hline
1964 & -5.014 & 0 & 0.2 & 0.2 & 0.4 & 0.5 & 0.8 & 0.8 & 0.8 & 0.8 & 0.7 & 0.7 \\ \hline
1965 & -5.019 & 0 & 0.2 & 0.2 & 0.4 & 0.5 & 0.8 & 0.8 & 0.8 & 0.8 & 0.7 & 0.7 \\ \hline
1966 & -5.025 & 0 & 0.2 & 0.2 & 0.4 & 0.5 & 0.8 & 0.8 & 0.8 & 0.8 & 0.7 & 0.7 \\ \hline
1967 & -5.03 & 0 & 0.2 & 0.2 & 0.4 & 0.5 & 0.8 & 0.8 & 0.8 & 0.8 & 0.7 & 0.7 \\ \hline
1968 & -5.035 & 0 & 0.2 & 0.2 & 0.4 & 0.5 & 0.7 & 0.8 & 0.8 & 0.8 & 0.7 & 0.7 \\ \hline
1969 & -5.04 & 0 & 0.2 & 0.2 & 0.4 & 0.5 & 0.7 & 0.8 & 0.8 & 0.8 & 0.7 & 0.7 \\ \hline
1970 & -5.046 & 0 & 0.2 & 0.2 & 0.4 & 0.4 & 0.7 & 0.8 & 0.8 & 0.8 & 0.7 & 0.7 \\ \hline
1971 & -5.051 & 0 & 0.2 & 0.2 & 0.4 & 0.4 & 0.7 & 0.8 & 0.8 & 0.8 & 0.7 & 0.7 \\ \hline
1972 & -5.056 & 0 & 0.2 & 0.2 & 0.4 & 0.4 & 0.6 & 0.8 & 0.8 & 0.8 & 0.7 & 0.7 \\ \hline
1973 & -5.062 & 0 & 0.2 & 0.2 & 0.4 & 0.4 & 0.6 & 0.8 & 0.8 & 0.8 & 0.7 & 0.7 \\ \hline
1974 & -5.067 & 0 & 0.2 & 0.1 & 0.4 & 0.4 & 0.5 & 0.8 & 0.8 & 0.8 & 0.7 & 0.7 \\ \hline
1975 & -5.072 & 0 & 0.2 & 0.1 & 0.4 & 0.4 & 0.5 & 0.8 & 0.8 & 0.8 & 0.7 & 0.7 \\ \hline
1976 & -5.078 & 0 & 0.2 & 0.1 & 0.4 & 0.4 & 0.5 & 0.8 & 0.8 & 0.8 & 0.7 & 0.7 \\ \hline
1977 & -5.083 & 0 & 0.2 & 0.1 & 0.4 & 0.4 & 0.5 & 0.8 & 0.8 & 0.8 & 0.7 & 0.7 \\ \hline
1978 & -5.088 & 0 & 0.2 & 0.1 & 0.4 & 0.4 & 0.5 & 0.8 & 0.8 & 0.8 & 0.7 & 0.7 \\ \hline
1979 & -5.094 & 0 & 0.2 & 0.1 & 0.4 & 0.4 & 0.5 & 0.8 & 0.8 & 0.8 & 0.7 & 0.7 \\ \hline
1980 & -5.099 & 0 & 0.2 & 0.1 & 0.4 & 0.4 & 0.4 & 0.8 & 0.8 & 0.8 & 0.7 & 0.7 \\ \hline
1981 & -5.104 & 0 & 0.2 & 0.1 & 0.4 & 0.4 & 0.4 & 0.8 & 0.8 & 0.8 & 0.7 & 0.7 \\ \hline
1982 & -5.109 & 0 & 0.2 & 0.1 & 0.4 & 0.4 & 0.4 & 0.8 & 0.8 & 0.8 & 0.7 & 0.7 \\ \hline
1983 & -5.115 & 0 & 0.2 & 0.1 & 0.4 & 0.4 & 0.4 & 0.7 & 0.8 & 0.8 & 0.7 & 0.7 \\ \hline
1984 & -5.12 & 0 & 0.2 & 0.1 & 0.4 & 0.4 & 0.4 & 0.7 & 0.8 & 0.8 & 0.7 & 0.7 \\ \hline
1985 & -5.125 & 0 & 0.2 & 0.1 & 0.4 & 0.4 & 0.4 & 0.7 & 0.8 & 0.8 & 0.7 & 0.7 \\ \hline
1986 & -5.131 & 0 & 0.2 & 0.1 & 0.4 & 0.4 & 0.4 & 0.7 & 0.8 & 0.8 & 0.7 & 0.7 \\ \hline
1987 & -5.136 & 0 & 0.2 & 0.1 & 0.4 & 0.4 & 0.4 & 0.7 & 0.8 & 0.8 & 0.7 & 0.7 \\ \hline
1988 & -5.141 & 0 & 0.2 & 0.1 & 0.4 & 0.4 & 0.4 & 0.7 & 0.7 & 0.8 & 0.7 & 0.7 \\ \hline
1989 & -5.147 & 0 & 0.2 & 0.1 & 0.4 & 0.4 & 0.4 & 0.7 & 0.7 & 0.8 & 0.7 & 0.7 \\ \hline
1990 & -5.152 & 0 & 0.2 & 0.1 & 0.4 & 0.4 & 0.4 & 0.6 & 0.7 & 0.8 & 0.7 & 0.7 \\ \hline
1991 & -5.157 & 0 & 0.2 & 0.1 & 0.4 & 0.4 & 0.4 & 0.6 & 0.7 & 0.8 & 0.7 & 0.7 \\ \hline
1992 & -5.162 & 0 & 0.2 & 0.1 & 0.4 & 0.4 & 0.4 & 0.5 & 0.7 & 0.8 & 0.7 & 0.7 \\ \hline
1993 & -5.168 & 0 & 0.2 & 0.1 & 0.4 & 0.4 & 0.4 & 0.5 & 0.7 & 0.8 & 0.7 & 0.7 \\ \hline
1994 & -5.173 & 0 & 0.2 & 0.1 & 0.4 & 0.4 & 0.4 & 0.5 & 0.7 & 0.7 & 0.7 & 0.7 \\ \hline
1995 & -5.178 & 0 & 0.2 & 0.1 & 0.4 & 0.4 & 0.4 & 0.5 & 0.7 & 0.7 & 0.7 & 0.7 \\ \hline
1996 & -5.184 & 0 & 0.2 & 0.1 & 0.4 & 0.4 & 0.4 & 0.5 & 0.6 & 0.7 & 0.7 & 0.7 \\ \hline
1997 & -5.189 & 0 & 0.2 & 0.1 & 0.4 & 0.4 & 0.4 & 0.5 & 0.6 & 0.7 & 0.7 & 0.7 \\ \hline
1998 & -5.194 & 0 & 0.2 & 0.1 & 0.4 & 0.4 & 0.4 & 0.4 & 0.6 & 0.7 & 0.7 & 0.7 \\ \hline
1999 & -5.2 & 0 & 0.2 & 0.1 & 0.4 & 0.4 & 0.4 & 0.4 & 0.6 & 0.7 & 0.7 & 0.7 \\ \hline
2000 & -5.205 & 0 & 0.2 & 0.1 & 0.4 & 0.4 & 0.4 & 0.4 & 0.5 & 0.7 & 0.7 & 0.7 \\ \hline
2001 & -5.21 & 0 & 0.2 & 0.1 & 0.4 & 0.4 & 0.4 & 0.4 & 0.5 & 0.7 & 0.7 & 0.7 \\ \hline
2002 & -5.216 & 0 & 0.2 & 0.1 & 0.4 & 0.4 & 0.4 & 0.4 & 0.5 & 0.6 & 0.7 & 0.7 \\ \hline
2003 & -5.221 & 0 & 0.2 & 0.1 & 0.4 & 0.4 & 0.4 & 0.4 & 0.5 & 0.6 & 0.7 & 0.7 \\ \hline
2004 & -5.226 & 0 & 0.2 & 0.1 & 0.3 & 0.4 & 0.4 & 0.4 & 0.4 & 0.6 & 0.7 & 0.7 \\ \hline
2005 & -5.231 & 0 & 0.2 & 0.1 & 0.3 & 0.4 & 0.4 & 0.4 & 0.4 & 0.6 & 0.7 & 0.7 \\ \hline
2006 & -5.237 & 0 & 0.2 & 0.1 & 0.3 & 0.4 & 0.4 & 0.4 & 0.4 & 0.5 & 0.6 & 0.6 \\ \hline
2007 & -5.242 & 0 & 0.2 & 0.1 & 0.3 & 0.4 & 0.4 & 0.4 & 0.4 & 0.5 & 0.6 & 0.6 \\ \hline
2008 & -5.247 & 0 & 0.2 & 0.1 & 0.3 & 0.4 & 0.4 & 0.4 & 0.4 & 0.5 & 0.6 & 0.6 \\ \hline
2009 & -5.253 & 0 & 0.2 & 0.1 & 0.3 & 0.4 & 0.4 & 0.4 & 0.4 & 0.5 & 0.6 & 0.6 \\ \hline
2010 & -5.258 & 0 & 0.2 & 0.1 & 0.3 & 0.4 & 0.4 & 0.4 & 0.4 & 0.4 & 0.6 & 0.6 \\ \hline
2011 & -5.263 & 0 & 0.2 & 0.1 & 0.3 & 0.4 & 0.4 & 0.4 & 0.4 & 0.4 & 0.6 & 0.6 \\ \hline
2012 & -5.269 & 0 & 0.2 & 0.1 & 0.3 & 0.4 & 0.4 & 0.4 & 0.4 & 0.4 & 0.5 & 0.6 \\ \hline
2013 & -5.274 & 0 & 0.2 & 0.1 & 0.3 & 0.4 & 0.4 & 0.4 & 0.4 & 0.4 & 0.5 & 0.6 \\ \hline
2014 & -5.279 & 0 & 0.2 & 0.1 & 0.3 & 0.4 & 0.4 & 0.4 & 0.4 & 0.4 & 0.5 & 0.6 \\ \hline
2015 & -5.284 & 0 & 0.2 & 0.1 & 0.3 & 0.4 & 0.4 & 0.4 & 0.4 & 0.4 & 0.5 & 0.6 \\ \hline
2016 & -5.29 & 0 & 0.2 & 0.1 & 0.3 & 0.4 & 0.4 & 0.4 & 0.4 & 0.4 & 0.4 & 0.5 \\ \hline
2017 & -5.295 & 0 & 0.2 & 0.1 & 0.3 & 0.4 & 0.4 & 0.4 & 0.4 & 0.4 & 0.4 & 0.5 \\ \hline
2018 & -5.3 & 0 & 0.2 & 0.1 & 0.2 & 0.4 & 0.4 & 0.4 & 0.4 & 0.4 & 0.4 & 0.5 \\ \hline
2019 & -5.306 & 0 & 0.2 & 0.1 & 0.2 & 0.4 & 0.4 & 0.4 & 0.4 & 0.4 & 0.4 & 0.5 \\ \hline
2020 & -5.311 & 0 & 0.2 & 0.1 & 0.2 & 0.4 & 0.4 & 0.4 & 0.4 & 0.4 & 0.4 & 0.4 \\ \hline
2021 & -5.316 & 0 & 0.2 & 0.1 & 0.2 & 0.4 & 0.4 & 0.4 & 0.4 & 0.4 & 0.3 & 0.4 \\ \hline
2022 & -5.322 & 0 & 0.2 & 0.1 & 0.2 & 0.3 & 0.4 & 0.4 & 0.4 & 0.4 & 0.3 & 0.4 \\ \hline
2023 & -5.327 & 0 & 0.2 & 0.1 & 0.2 & 0.3 & 0.4 & 0.4 & 0.4 & 0.4 & 0.3 & 0.4 \\ \hline
2024 & -5.332 & 0 & 0.2 & 0.1 & 0.2 & 0.3 & 0.4 & 0.4 & 0.4 & 0.4 & 0.3 & 0.3 \\ \hline
2025 & -5.337 & 0 & 0.2 & 0.1 & 0.2 & 0.3 & 0.4 & 0.4 & 0.4 & 0.4 & 0.3 & 0.3 \\ \hline
2026 & -5.343 & 0 & 0.2 & 0.1 & 0.1 & 0.3 & 0.4 & 0.4 & 0.4 & 0.4 & 0.3 & 0.3 \\ \hline
2027 & -5.348 & 0 & 0.2 & 0.1 & 0.1 & 0.3 & 0.4 & 0.4 & 0.4 & 0.4 & 0.3 & 0.3 \\ \hline
2028 & -5.353 & 0 & 0.2 & 0.1 & 0.1 & 0.3 & 0.4 & 0.4 & 0.4 & 0.4 & 0.3 & 0.3 \\ \hline
2029 & -5.359 & 0 & 0.2 & 0.1 & 0.1 & 0.3 & 0.4 & 0.4 & 0.4 & 0.4 & 0.3 & 0.3 \\ \hline
2030 & -5.364 & 0 & 0.2 & 0.1 & 0.1 & 0.3 & 0.4 & 0.4 & 0.4 & 0.4 & 0.3 & 0.3 \\ \hline
2031 & -5.369 & 0 & 0.2 & 0.1 & 0.1 & 0.3 & 0.4 & 0.4 & 0.4 & 0.4 & 0.3 & 0.3 \\ \hline
2032 & -5.375 & 0 & 0.2 & 0.1 & 0.1 & 0.3 & 0.4 & 0.4 & 0.4 & 0.4 & 0.3 & 0.3 \\ \hline
2033 & -5.38 & 0 & 0.2 & 0.1 & 0.1 & 0.3 & 0.4 & 0.4 & 0.4 & 0.4 & 0.3 & 0.3 \\ \hline
2034 & -5.385 & 0 & 0.2 & 0.1 & 0.1 & 0.3 & 0.4 & 0.4 & 0.4 & 0.4 & 0.3 & 0.3 \\ \hline
2035 & -5.39 & 0 & 0.2 & 0.1 & 0.1 & 0.3 & 0.4 & 0.4 & 0.4 & 0.4 & 0.3 & 0.3 \\ \hline
2036 & -5.396 & 0 & 0.2 & 0.1 & 0.1 & 0.3 & 0.4 & 0.4 & 0.4 & 0.4 & 0.3 & 0.3 \\ \hline
2037 & -5.401 & 0 & 0.2 & 0.1 & 0.1 & 0.3 & 0.4 & 0.4 & 0.4 & 0.4 & 0.3 & 0.3 \\ \hline
2038 & -5.406 & 0 & 0.2 & 0.1 & 0.1 & 0.3 & 0.4 & 0.4 & 0.4 & 0.4 & 0.3 & 0.3 \\ \hline
2039 & -5.412 & 0 & 0.2 & 0.1 & 0.1 & 0.3 & 0.4 & 0.4 & 0.4 & 0.4 & 0.3 & 0.3 \\ \hline
2040 & -5.417 & 0 & 0.2 & 0.1 & 0 & 0.3 & 0.4 & 0.4 & 0.4 & 0.4 & 0.3 & 0.3 \\ \hline
2041 & -5.422 & 0 & 0.2 & 0.1 & 0 & 0.3 & 0.4 & 0.4 & 0.4 & 0.4 & 0.3 & 0.3 \\ \hline
2042 & -5.428 & 0 & 0.2 & 0.1 & 0 & 0.2 & 0.4 & 0.4 & 0.4 & 0.4 & 0.3 & 0.3 \\ \hline
2043 & -5.433 & 0 & 0.2 & 0.1 & 0 & 0.2 & 0.4 & 0.4 & 0.4 & 0.4 & 0.3 & 0.3 \\ \hline
2044 & -5.438 & 0 & 0.2 & 0.1 & 0 & 0.2 & 0.4 & 0.4 & 0.4 & 0.4 & 0.3 & 0.3 \\ \hline
2045 & -5.443 & 0 & 0.2 & 0.1 & 0 & 0.2 & 0.4 & 0.4 & 0.4 & 0.4 & 0.3 & 0.3 \\ \hline
2046 & -5.449 & 0 & 0.2 & 0.1 & 0 & 0.2 & 0.4 & 0.4 & 0.4 & 0.4 & 0.3 & 0.3 \\ \hline
2047 & -5.454 & 0 & 0.2 & 0.1 & 0 & 0.2 & 0.4 & 0.4 & 0.4 & 0.4 & 0.3 & 0.3 \\ \hline

          \caption{Aufnahmen der CCD - Kamera.}
          \label{tab:CCD}
        \end{longtable}
      \end{scriptsize}
      
    \end{section}
    %%%%%%%%%%%%%%%%%%%%%%%%%%%%%%%%%%%%%%%%
    
  \end{chapter}
  %%%%%%%%%%%%%%%%%%%%%%%%%%%%%%
  
  
  
  %%%%%%%%%%%%%%%%%%%%%%%%%%%%%%
  %%%%%%%%%%%%%%%%%%%%%%%%%%%%%%
  %%%%%%%%%%%%%%%%%%%%%%%%%%%%%%
  \begin{chapter}{ZWEITER TEIL - Franck-Hertz-Versuch}
    \label{Anhang:chp:FH}
    
    
    
    %%%%%%%%%%%%%%%%%%%%%%%%%%%%%%%%%%%%%%%%
    %%%%%%%%%%%%%%%%%%%%%%%%%%%%%%%%%%%%%%%%
    %%%%%%%%%%%%%%%%%%%%%%%%%%%%%%%%%%%%%%%%
    \begin{section}{Bremsspannung}
      \label{Anhang:chp:FHbremsspannung}
      \begin{figure}[htbp!]
        \centering
        \begin{minipage}{0.48\textwidth}
          \centering
          \includegraphics[width=\textwidth]
              {Figures/Versuch401-Franck-Hertz-1_5VBremsspannung_Beschl_Spannung_Anodenspannung.png}
          \caption{Angepasste Gauss-Funktion für eine Bremsspannung von 
              $\SI{1.5}{\volt}$.}
          \label{fig:AnhangFHB15V}
        \end{minipage} \quad
        \begin{minipage}{0.48\textwidth}
          \centering
          \includegraphics[width=\textwidth]
              {Figures/Versuch401-Franck-Hertz-2_0VBremsspannung_Beschl_Spannung_Anodenspannung.png}
          \caption{Angepasste Gauss-Funktion für eine Bremsspannung von 
              $\SI{2.0}{\volt}$.}
          \label{fig:AnhangFHB20V}
        \end{minipage} \\
        \begin{minipage}{0.48\textwidth}
          \centering
          \includegraphics[width=\textwidth]
              {Figures/Versuch401-Franck-Hertz-2_5VBremsspannung_Beschl_Spannung_Anodenspannung.png}
          \caption{Angepasste Gauss-Funktion für eine Bremsspannung von 
              $\SI{2.5}{\volt}$.}
          \label{fig:AnhangFHB25V}
        \end{minipage} \quad
        \begin{minipage}{0.48\textwidth}
          \centering
          \includegraphics[width=\textwidth]
              {Figures/Versuch401-Franck-Hertz-3_0VBremsspannung_Beschl_Spannung_Anodenspannung.png}
          \caption{Angepasste Gauss-Funktion für eine Bremsspannung von 
              $\SI{3.0}{\volt}$.}
          \label{fig:AnhangFHB30V}
        \end{minipage} \\
        \begin{minipage}{0.48\textwidth}
          \centering
          \includegraphics[width=\textwidth]
              {Figures/Versuch401-Franck-Hertz-3_5VBremsspannung_Beschl_Spannung_Anodenspannung.png}
          \caption{Angepasste Gauss-Funktion für eine Bremsspannung von 
              $\SI{3.5}{\volt}$.}
          \label{fig:AnhangFHB35V}
        \end{minipage}
      \end{figure}
      
      \begin{scriptsize}
        \begin{longtable}[htbp]{|c|c|c|c|c|c|c|c|c|c|}
          \hline
          \multicolumn{10}{|c|}{Bremsspannungen/V} \\ \hline 
          \multicolumn{2}{|c|}{1.5} & \multicolumn{2}{|c|}{2.0} &
          \multicolumn{2}{|c|}{2.5} & \multicolumn{2}{|c|}{3.0} & 
          \multicolumn{2}{|c|}{3.5} \\ \hline
          $U_{B}/V$ & $U_{A}/V$ & $U_{B}/V$ & $U_{A}/V$ &
          $U_{B}/V$ & $U_{A}/V$ & $U_{B}/V$ & $U_{A}/V$ &
          $U_{B}/V$ & $U_{A}/V$ \\ \hline\hline \endhead
          \input{./Tables/FH_Bremsspannung.tex}
          \caption{Messreihen zur Abhängigkeit des Anodenspannungs von der
              Bremsspannung.}
          \label{tab:FHbremsspannung}
        \end{longtable}
      \end{scriptsize}
    \end{section}
    %%%%%%%%%%%%%%%%%%%%%%%%%%%%%%%%%%%%%%%%
    
    
    
    \newpage
    %%%%%%%%%%%%%%%%%%%%%%%%%%%%%%%%%%%%%%%%
    %%%%%%%%%%%%%%%%%%%%%%%%%%%%%%%%%%%%%%%%
    %%%%%%%%%%%%%%%%%%%%%%%%%%%%%%%%%%%%%%%%
    \begin{section}{Temperatur}
      \label{Anhang:chp:FHtemperatur}
      \begin{figure}[htbp!]
        \centering
        \begin{minipage}{0.48\textwidth}
          \centering
          \includegraphics[width=\textwidth]
              {Figures/Versuch401-Franck-Hertz-130CTemperatur_Beschl_Spannung_Anodenspannung.png}
          \caption{Angepasste Gauss-Funktion für eine Temperatur von 
              $\SI{130}{\celsius}$.}
          \label{fig:AnhangFHT130C}
        \end{minipage} \quad
        \begin{minipage}{0.48\textwidth}
          \centering
          \includegraphics[width=\textwidth]
              {Figures/Versuch401-Franck-Hertz-140CTemperatur_Beschl_Spannung_Anodenspannung.png}
          \caption{Angepasste Gauss-Funktion für eine Temperatur von 
              $\SI{140}{\celsius}$.}
          \label{fig:AnhangFHT140C}
        \end{minipage} \\
        \begin{minipage}{0.48\textwidth}
          \centering
          \includegraphics[width=\textwidth]
              {Figures/Versuch401-Franck-Hertz-150CTemperatur_Beschl_Spannung_Anodenspannung.png}
          \caption{Angepasste Gauss-Funktion für eine Temperatur von 
              $\SI{150}{\celsius}$.}
          \label{fig:AnhangFHT150C}
        \end{minipage} \quad
        \begin{minipage}{0.48\textwidth}
          \centering
          \includegraphics[width=\textwidth]
              {Figures/Versuch401-Franck-Hertz-165CTemperatur_Beschl_Spannung_Anodenspannung.png}
          \caption{Angepasste Gauss-Funktion für eine Temperatur von 
              $\SI{165}{\celsius}$.}
          \label{fig:AnhangFHT165C}
        \end{minipage} \\
        \begin{minipage}{0.48\textwidth}
          \centering
          \includegraphics[width=\textwidth]
              {Figures/Versuch401-Franck-Hertz-175CTemperatur_Beschl_Spannung_Anodenspannung.png}
          \caption{Angepasste Gauss-Funktion für eine Temperatur von 
              $\SI{175}{\celsius}$.}
          \label{fig:AnhangFHT175C}
        \end{minipage}
      \end{figure}
      
      \begin{scriptsize}
        \begin{longtable}[htbp]{|c|c|c|c|c|c|c|c|c|c|}
          \hline
          \multicolumn{10}{|c|}{Temperaturen/\textdegree $C$} \\ \hline 
          \multicolumn{2}{|c|}{130} & \multicolumn{2}{|c|}{140} &
          \multicolumn{2}{|c|}{150} & \multicolumn{2}{|c|}{165} & 
          \multicolumn{2}{|c|}{175} \\ \hline
          $U_{B}/V$ & $U_{A}/V$ & $U_{B}/V$ & $U_{A}/V$ &
          $U_{B}/V$ & $U_{A}/V$ & $U_{B}/V$ & $U_{A}/V$ &
          $U_{B}/V$ & $U_{A}/V$ \\ \hline\hline \endhead
          % & Temperatur & 130 & Temperatur & 140 & Temperatur & 150 & Temperatur & 165 & Temperatur & 175 \\ \hline
% & "U_B & / & V" & "U_A & / & V" & "U_B & / & V" & "U_A & / & V" & "U_B & / & V" & "U_A & / & V" & "U_B & / & V" & "U_A & / & V" & "U_B & / & V" & "U_A & / & V" \\ \hline
0.1 & 0 & 2.8 & 0 & 0.1 & 0 & 0.1 & 0 & 0.1 & 0 \\ \hline
0.2 & 0 & 3 & 0 & 0.1 & 0 & 0.1 & 0 & 0.1 & 0 \\ \hline
0.3 & 0 & 3.1 & 0 & 0.1 & 0 & 0.1 & 0 & 0.1 & 0 \\ \hline
0.5 & 0 & 3.3 & 0 & 0.1 & 0 & 0.3 & 0 & 0.1 & 0 \\ \hline
0.6 & 0 & 3.5 & 0 & 0.1 & 0 & 0.6 & 0 & 0.1 & 0 \\ \hline
0.8 & 0 & 3.6 & 0 & 0.1 & 0 & 0.8 & 0 & 0.1 & 0 \\ \hline
0.9 & 0 & 3.7 & 0 & 0.1 & 0 & 1 & 0 & 0 & 0 \\ \hline
1 & 0 & 3.9 & 0 & 0.1 & 0 & 1.2 & 0 & 0 & 0 \\ \hline
1.2 & 0 & 4.1 & 0 & 0.2 & 0 & 1.4 & 0 & 0.1 & 0 \\ \hline
1.3 & 0 & 4.3 & 0 & 0.4 & 0 & 1.5 & 0 & 0.1 & 0 \\ \hline
1.4 & 0 & 4.4 & 0 & 0.5 & 0 & 1.8 & 0 & 0.2 & 0 \\ \hline
1.6 & 0 & 4.6 & 0 & 0.8 & 0 & 2 & 0 & 0.4 & 0 \\ \hline
1.7 & 0 & 4.7 & 0 & 0.9 & 0 & 2.1 & 0 & 0.6 & 0 \\ \hline
1.8 & 0 & 4.9 & 0 & 1.1 & 0 & 2.4 & 0 & 0.8 & 0 \\ \hline
2 & 0 & 5 & 0 & 1.2 & 0 & 2.5 & 0 & 1 & 0 \\ \hline
2.2 & 0 & 5.2 & 0 & 1.4 & 0 & 2.8 & 0 & 1.2 & 0 \\ \hline
2.2 & 0 & 5.4 & 0 & 1.6 & 0 & 3 & 0 & 1.5 & 0 \\ \hline
2.4 & 0 & 5.5 & 0 & 1.8 & 0 & 3.1 & 0 & 1.7 & 0 \\ \hline
2.6 & 0 & 5.6 & 0 & 1.9 & 0 & 3.3 & 0 & 1.8 & 0 \\ \hline
2.7 & 0 & 5.8 & 0 & 2.1 & 0 & 3.5 & 0 & 2.1 & 0 \\ \hline
2.8 & 0 & 6 & 0 & 2.2 & 0 & 3.7 & 0 & 2.2 & 0 \\ \hline
2.9 & 0 & 6.2 & 0 & 2.4 & 0 & 3.9 & 0 & 2.5 & 0 \\ \hline
3 & 0 & 6.3 & 0 & 2.6 & 0 & 4.2 & 0 & 2.7 & 0 \\ \hline
3.2 & 0 & 6.4 & 0 & 2.8 & 0 & 4.3 & 0 & 2.9 & 0 \\ \hline
3.3 & 0 & 6.6 & 0 & 2.9 & 0 & 4.6 & 0 & 3.1 & 0 \\ \hline
3.5 & 0 & 6.8 & 0 & 3.1 & 0 & 4.7 & 0 & 3.3 & 0 \\ \hline
3.7 & 0 & 6.9 & 0 & 3.3 & 0 & 5 & 0 & 3.5 & 0 \\ \hline
3.7 & 0 & 7.1 & 0 & 3.5 & 0 & 5.1 & 0 & 3.7 & 0 \\ \hline
3.9 & 0 & 7.2 & 0 & 3.6 & 0 & 5.3 & 0 & 3.9 & 0 \\ \hline
4 & 0 & 7.4 & 0 & 3.8 & 0 & 5.5 & 0 & 4.1 & 0 \\ \hline
4.2 & 0 & 7.6 & 0 & 3.9 & 0 & 5.7 & 0 & 4.3 & 0 \\ \hline
4.3 & 0 & 7.7 & 0 & 4.1 & 0 & 5.9 & 0.1 & 4.5 & 0 \\ \hline
4.4 & 0 & 7.9 & 0 & 4.3 & 0 & 6.1 & 0.1 & 4.7 & 0 \\ \hline
4.6 & 0 & 8 & 0 & 4.5 & 0 & 6.3 & 0.1 & 4.9 & 0 \\ \hline
4.7 & 0 & 8.2 & 0 & 4.6 & 0 & 6.5 & 0.1 & 5.1 & 0 \\ \hline
4.9 & 0 & 8.4 & 0 & 4.8 & 0 & 6.7 & 0.1 & 5.3 & 0 \\ \hline
5 & 0 & 8.5 & 0 & 5 & 0 & 6.9 & 0.1 & 5.5 & 0 \\ \hline
5.1 & 0 & 8.7 & 0 & 5.1 & 0 & 7.1 & 0.1 & 5.7 & 0 \\ \hline
5.3 & 0 & 8.8 & 0 & 5.3 & 0 & 7.4 & 0.1 & 6 & 0 \\ \hline
5.4 & 0 & 9 & 0 & 5.5 & 0 & 7.5 & 0.1 & 6.1 & 0 \\ \hline
5.5 & 0 & 9.2 & 0 & 5.7 & 0 & 7.7 & 0.1 & 6.3 & 0 \\ \hline
5.7 & 0 & 9.3 & 0 & 5.9 & 0 & 7.9 & 0.1 & 6.6 & 0 \\ \hline
5.9 & 0 & 9.4 & 0.1 & 6 & 0 & 8.1 & 0.1 & 6.7 & 0 \\ \hline
5.9 & 0 & 9.6 & 0.1 & 6.2 & 0 & 8.3 & 0.1 & 7 & 0 \\ \hline
6.1 & 0 & 9.8 & 0.2 & 6.4 & 0 & 8.5 & 0.1 & 7.1 & 0 \\ \hline
6.3 & 0 & 9.9 & 0.2 & 6.5 & 0 & 8.7 & 0.1 & 7.4 & 0 \\ \hline
6.4 & 0 & 10.1 & 0.3 & 6.7 & 0 & 8.9 & 0.1 & 7.6 & 0 \\ \hline
6.5 & 0.1 & 10.2 & 0.3 & 6.9 & 0 & 9.1 & 0.1 & 7.8 & 0 \\ \hline
6.6 & 0.1 & 10.4 & 0.4 & 7 & 0 & 9.3 & 0.1 & 8 & 0 \\ \hline
6.8 & 0.1 & 10.5 & 0.5 & 7.2 & 0 & 9.5 & 0.1 & 8.2 & 0 \\ \hline
6.9 & 0.1 & 10.8 & 0.7 & 7.4 & 0 & 9.7 & 0.1 & 8.4 & 0 \\ \hline
7 & 0.1 & 10.9 & 0.8 & 7.5 & 0 & 9.9 & 0.1 & 8.6 & 0 \\ \hline
7.2 & 0.1 & 11 & 0.9 & 7.7 & 0 & 10.1 & 0.1 & 8.8 & 0 \\ \hline
7.3 & 0.1 & 11.2 & 1.1 & 7.9 & 0 & 10.3 & 0.1 & 9 & 0 \\ \hline
7.5 & 0.1 & 11.4 & 1.3 & 8 & 0 & 10.5 & 0.2 & 9.2 & 0 \\ \hline
7.6 & 0.1 & 11.5 & 1.5 & 8.2 & 0 & 10.6 & 0.2 & 9.4 & 0 \\ \hline
7.7 & 0.1 & 11.7 & 1.5 & 8.4 & 0 & 10.9 & 0.2 & 9.6 & 0 \\ \hline
7.9 & 0.1 & 11.9 & 1.5 & 8.5 & 0 & 11 & 0.3 & 9.8 & 0 \\ \hline
8 & 0.1 & 11.9 & 1.4 & 8.8 & 0 & 11.2 & 0.3 & 10 & 0 \\ \hline
8.1 & 0.1 & 12.2 & 1.2 & 8.9 & 0 & 11.4 & 0.4 & 10.2 & 0.1 \\ \hline
8.3 & 0.1 & 12.4 & 1.1 & 9.1 & 0 & 11.7 & 0.4 & 10.5 & 0.1 \\ \hline
8.4 & 0.1 & 12.5 & 0.9 & 9.2 & 0 & 11.9 & 0.4 & 10.6 & 0.1 \\ \hline
8.6 & 0.1 & 12.6 & 0.8 & 9.4 & 0 & 12 & 0.4 & 10.8 & 0.2 \\ \hline
8.7 & 0.1 & 12.8 & 0.6 & 9.6 & 0 & 12.2 & 0.4 & 11.1 & 0.2 \\ \hline
8.8 & 0.1 & 12.9 & 0.6 & 9.8 & 0 & 12.4 & 0.4 & 11.2 & 0.3 \\ \hline
9 & 0.1 & 13.1 & 0.5 & 9.9 & 0 & 12.6 & 0.4 & 11.5 & 0.3 \\ \hline
9.1 & 0.2 & 13.2 & 0.4 & 10.1 & 0.1 & 12.9 & 0.3 & 11.6 & 0.4 \\ \hline
9.2 & 0.2 & 13.4 & 0.4 & 10.3 & 0.1 & 13 & 0.3 & 11.9 & 0.4 \\ \hline
9.4 & 0.3 & 13.6 & 0.3 & 10.5 & 0.1 & 13.2 & 0.3 & 12.1 & 0.4 \\ \hline
9.5 & 0.4 & 13.7 & 0.3 & 10.6 & 0.1 & 13.4 & 0.2 & 12.2 & 0.4 \\ \hline
9.7 & 0.5 & 13.9 & 0.4 & 10.8 & 0.2 & 13.6 & 0.2 & 12.5 & 0.3 \\ \hline
9.8 & 0.6 & 14 & 0.4 & 10.9 & 0.2 & 13.8 & 0.2 & 12.6 & 0.3 \\ \hline
9.9 & 0.7 & 14.2 & 0.5 & 11.2 & 0.3 & 14.1 & 0.2 & 12.9 & 0.3 \\ \hline
10 & 0.9 & 14.4 & 0.7 & 11.3 & 0.3 & 14.2 & 0.2 & 13.1 & 0.2 \\ \hline
10.2 & 1 & 14.6 & 0.8 & 11.5 & 0.4 & 14.4 & 0.2 & 13.3 & 0.2 \\ \hline
10.3 & 1.3 & 14.7 & 1 & 11.6 & 0.5 & 14.6 & 0.3 & 13.5 & 0.2 \\ \hline
10.5 & 1.5 & 14.8 & 1.2 & 11.8 & 0.5 & 14.8 & 0.3 & 13.7 & 0.2 \\ \hline
10.6 & 1.7 & 15 & 1.5 & 11.9 & 0.4 & 15 & 0.3 & 13.9 & 0.2 \\ \hline
10.7 & 2.1 & 15.1 & 1.8 & 12.2 & 0.4 & 15.2 & 0.4 & 14.1 & 0.2 \\ \hline
10.9 & 2.3 & 15.4 & 2.1 & 12.3 & 0.3 & 15.4 & 0.5 & 14.3 & 0.2 \\ \hline
11 & 2.7 & 15.5 & 2.5 & 12.5 & 0.3 & 15.6 & 0.5 & 14.5 & 0.2 \\ \hline
11.2 & 3.2 & 15.6 & 2.9 & 12.7 & 0.2 & 15.8 & 0.7 & 14.7 & 0.3 \\ \hline
11.3 & 3.5 & 15.8 & 3.3 & 12.9 & 0.2 & 16 & 0.8 & 14.9 & 0.3 \\ \hline
11.4 & 3.9 & 15.9 & 3.8 & 13 & 0.2 & 16.2 & 0.9 & 15.2 & 0.4 \\ \hline
11.6 & 4.2 & 16.1 & 4.3 & 13.2 & 0.2 & 16.4 & 1 & 15.3 & 0.5 \\ \hline
11.7 & 4.4 & 16.2 & 4.9 & 13.3 & 0.1 & 16.6 & 1.1 & 15.5 & 0.6 \\ \hline
11.9 & 4.4 & 16.4 & 5.3 & 13.5 & 0.1 & 16.8 & 1.1 & 15.7 & 0.7 \\ \hline
12 & 4.1 & 16.6 & 5.6 & 13.7 & 0.1 & 17 & 1 & 15.9 & 0.9 \\ \hline
12.1 & 3.8 & 16.7 & 5.6 & 13.9 & 0.1 & 17.2 & 0.9 & 16.1 & 1.1 \\ \hline
12.2 & 3.4 & 16.9 & 5.3 & 14 & 0.2 & 17.4 & 0.8 & 16.4 & 1.2 \\ \hline
12.4 & 3.1 & 17.1 & 4.7 & 14.2 & 0.2 & 17.6 & 0.7 & 16.6 & 1.3 \\ \hline
12.6 & 2.7 & 17.2 & 4.3 & 14.4 & 0.3 & 17.8 & 0.6 & 16.7 & 1.3 \\ \hline
12.7 & 2.3 & 17.4 & 3.7 & 14.5 & 0.4 & 18 & 0.5 & 17 & 1.2 \\ \hline
12.8 & 1.8 & 17.5 & 3.2 & 14.7 & 0.4 & 18.2 & 0.4 & 17.2 & 1.1 \\ \hline
12.9 & 1.6 & 17.6 & 2.5 & 14.9 & 0.6 & 18.4 & 0.3 & 17.4 & 0.9 \\ \hline
13.1 & 1.3 & 17.9 & 2.1 & 15.1 & 0.7 & 18.6 & 0.3 & 17.6 & 0.8 \\ \hline
13.2 & 1.1 & 18 & 1.7 & 15.2 & 0.9 & 18.9 & 0.3 & 17.7 & 0.6 \\ \hline
13.4 & 1 & 18.2 & 1.5 & 15.4 & 1 & 19 & 0.3 & 18 & 0.5 \\ \hline
13.5 & 0.9 & 18.3 & 1.2 & 15.6 & 1.3 & 19.2 & 0.3 & 18.2 & 0.4 \\ \hline
13.6 & 0.9 & 18.4 & 1.1 & 15.8 & 1.5 & 19.4 & 0.4 & 18.4 & 0.3 \\ \hline
13.8 & 0.9 & 18.6 & 1.1 & 15.9 & 1.8 & 19.6 & 0.5 & 18.6 & 0.3 \\ \hline
13.9 & 1 & 18.8 & 1.1 & 16.1 & 2.1 & 19.7 & 0.6 & 18.8 & 0.3 \\ \hline
14.1 & 1.1 & 18.9 & 1.2 & 16.2 & 2.4 & 19.9 & 0.7 & 19 & 0.3 \\ \hline
14.2 & 1.4 & 19.1 & 1.4 & 16.4 & 2.7 & 20.2 & 0.9 & 19.2 & 0.3 \\ \hline
14.3 & 1.6 & 19.2 & 1.6 & 16.6 & 2.8 & 20.3 & 1 & 19.4 & 0.4 \\ \hline
14.5 & 1.9 & 19.4 & 1.9 & 16.8 & 2.7 & 20.5 & 1.2 & 19.6 & 0.5 \\ \hline
14.6 & 2.3 & 19.5 & 2.2 & 16.9 & 2.5 & 20.8 & 1.5 & 19.8 & 0.7 \\ \hline
14.7 & 2.6 & 19.7 & 2.7 & 17.1 & 2.2 & 20.9 & 1.7 & 20.1 & 0.8 \\ \hline
14.9 & 3.2 & 19.9 & 3.2 & 17.3 & 1.9 & 21.1 & 2 & 20.2 & 1 \\ \hline
15 & 3.7 & 20.1 & 3.8 & 17.4 & 1.7 & 21.3 & 2.2 & 20.5 & 1.3 \\ \hline
15.1 & 4.2 & 20.2 & 4.4 & 17.6 & 1.3 & 21.6 & 2.2 & 20.6 & 1.5 \\ \hline
15.3 & 5 & 20.4 & 5.1 & 17.8 & 1.1 & 21.8 & 2.1 & 20.9 & 1.8 \\ \hline
15.4 & 5.7 & 20.5 & 6.1 & 17.9 & 0.9 & 22 & 1.9 & 21.1 & 2.1 \\ \hline
15.5 & 6.5 & 20.7 & 6.9 & 18.1 & 0.7 & 22.2 & 1.7 & 21.3 & 2.4 \\ \hline
15.7 & 7.3 & 20.8 & 8 & 18.3 & 0.6 & 22.3 & 1.5 & 21.5 & 2.6 \\ \hline
15.8 & 8.3 & 21 & 9 & 18.5 & 0.5 & 22.5 & 1.2 & 21.7 & 2.5 \\ \hline
16 & 9.3 & 21.2 & 10.1 & 18.6 & 0.5 & 22.7 & 1 & 21.9 & 2.3 \\ \hline
16.1 & 10.3 & 21.3 & 10.9 & 18.8 & 0.6 & 22.9 & 0.8 & 22 & 2 \\ \hline
16.2 & 11 & 21.5 & 11.1 & 19 & 0.6 & 23.1 & 0.6 & 22.3 & 1.7 \\ \hline
16.4 & 11.1 & 21.6 & 11.1 & 19.1 & 0.8 & 23.3 & 0.5 & 22.5 & 1.4 \\ \hline
16.6 & 11.2 & 21.8 & 11.1 & 19.4 & 0.9 & 23.5 & 0.5 & 22.7 & 1.1 \\ \hline
16.7 & 11.2 & 22 & 11 & 19.5 & 1.1 & 23.7 & 0.5 & 22.9 & 0.8 \\ \hline
16.8 & 11.2 & 22.1 & 10.7 & 19.7 & 1.3 & 23.9 & 0.5 & 23.1 & 0.7 \\ \hline
16.9 & 11.2 & 22.3 & 9.8 & 19.8 & 1.7 & 24.1 & 0.6 & 23.3 & 0.5 \\ \hline
17.1 & 11.2 & 22.4 & 8.8 & 20 & 2 & 24.4 & 0.7 & 23.6 & 0.5 \\ \hline
17.2 & 11.1 & 22.6 & 7.4 & 20.2 & 2.4 & 24.5 & 0.8 & 23.7 & 0.5 \\ \hline
17.4 & 11.1 & 22.8 & 6.5 & 20.4 & 2.8 & 24.7 & 1.1 & 23.9 & 0.5 \\ \hline
17.4 & 11 & 22.9 & 5.6 & 20.5 & 3.3 & 24.9 & 1.3 & 24.1 & 0.6 \\ \hline
17.6 & 9.6 & 23 & 4.9 & 20.7 & 3.8 & 25.1 & 1.5 & 24.3 & 0.7 \\ \hline
17.7 & 8 & 23.2 & 4.1 & 20.8 & 4.4 & 25.3 & 1.9 & 24.5 & 0.9 \\ \hline
17.9 & 7.2 & 23.4 & 3.7 & 21.1 & 5 & 25.6 & 2.2 & 24.7 & 1.1 \\ \hline
18 & 6.2 & 23.5 & 3.4 & 21.2 & 5.7 & 25.7 & 2.6 & 25 & 1.4 \\ \hline
18.1 & 5.6 & 23.7 & 3.3 & 21.4 & 6.1 & 25.9 & 3 & 25.1 & 1.7 \\ \hline
18.3 & 4.8 & 23.8 & 3.4 & 21.6 & 6.2 & 26.1 & 3.4 & 25.3 & 2.1 \\ \hline
18.4 & 4.4 & 24 & 3.6 & 21.8 & 6 & 26.3 & 3.7 & 25.6 & 2.5 \\ \hline
18.6 & 4.3 & 24.2 & 4 & 21.9 & 5.6 & 26.5 & 3.7 & 25.8 & 3.1 \\ \hline
18.7 & 4.2 & 24.4 & 4.5 & 22.1 & 5.1 & 26.7 & 3.5 & 26 & 3.5 \\ \hline
18.8 & 4.2 & 24.5 & 5.2 & 22.2 & 4.5 & 26.9 & 3.2 & 26.2 & 3.9 \\ \hline
19 & 4.4 & 24.7 & 6.2 & 22.4 & 3.9 & 27.1 & 2.8 & 26.4 & 4.2 \\ \hline
19.2 & 5 & 24.8 & 7.1 & 22.6 & 3.2 & 27.3 & 2.4 & 26.6 & 4.1 \\ \hline
19.3 & 5.6 & 25 & 8.4 & 22.7 & 2.6 & 27.5 & 2 & 26.8 & 3.8 \\ \hline
19.4 & 6.3 & 25.1 & 9.8 & 22.9 & 2.1 & 27.7 & 1.6 & 27 & 3.5 \\ \hline
19.5 & 7.2 & 25.3 & 11.1 & 23.1 & 1.7 & 27.9 & 1.3 & 27.2 & 3 \\ \hline
19.7 & 8.4 & 25.5 & 11.2 & 23.3 & 1.4 & 28.1 & 1 & 27.4 & 2.5 \\ \hline
19.8 & 9.8 & 25.6 & 11.3 & 23.4 & 1.3 & 28.3 & 0.9 & 27.6 & 2 \\ \hline
19.9 & 11 & 25.8 & 11.3 & 23.6 & 1.2 & 28.5 & 0.8 & 27.8 & 1.6 \\ \hline
20.1 & 11.2 & 25.9 & 11.3 & 23.8 & 1.2 & 28.7 & 0.8 & 28 & 1.3 \\ \hline
20.2 & 11.2 & 26.1 & 11.3 & 24 & 1.3 & 28.9 & 0.8 & 28.2 & 1 \\ \hline
20.3 & 11.3 & 26.3 & 11.3 & 24.1 & 1.5 & 29.1 & 1 & 28.4 & 0.9 \\ \hline
20.5 & 11.3 & 26.4 & 11.3 & 24.3 & 1.8 & 29.2 & 1.1 & 28.6 & 0.8 \\ \hline
20.6 & 11.3 & 26.6 & 11.3 & 24.5 & 2.1 & 29.5 & 1.4 & 28.8 & 0.9 \\ \hline
20.8 & 11.3 & 26.7 & 11.3 & 24.6 & 2.5 & 29.7 & 1.7 & 29.1 & 1 \\ \hline
20.9 & 11.3 & 27 & 11.3 & 24.8 & 3 & 29.9 & 2 & 29.3 & 1.2 \\ \hline
21.1 & 11.3 & 27.1 & 11.3 & 25 & 3.6 & 30.1 & 2.4 & 29.5 & 1.4 \\ \hline
21.2 & 11.3 & 27.2 & 11.3 & 25.1 & 4.2 & 30.3 & 2.9 & 29.6 & 1.7 \\ \hline
21.4 & 11.3 & 27.4 & 11.3 & 25.3 & 5 & 30.4 & 3.5 & 29.8 & 2.1 \\ \hline
21.5 & 11.3 & 27.5 & 11.3 & 25.5 & 5.8 & 30.6 & 4 & 30 & 2.6 \\ \hline
21.6 & 11.3 & 27.6 & 11.3 & 25.6 & 6.8 & 30.8 & 4.6 & 30.2 & 3.2 \\ \hline
21.7 & 11.3 & 27.9 & 11.3 & 25.9 & 7.7 & 31 & 5.1 & 30.5 & 3.7 \\ \hline
21.9 & 11.3 & 28 & 11.3 & 26 & 8.9 & 31.2 & 5.5 & 30.6 & 4.4 \\ \hline
22 & 11.3 & 28.2 & 11.3 & 26.2 & 9.7 & 31.5 & 5.6 & 30.9 & 5.1 \\ \hline
22.2 & 11.3 & 28.3 & 11.3 & 26.3 & 10.6 & 31.8 & 5.4 & 31 & 5.5 \\ \hline
22.3 & 11.3 & 28.5 & 11.3 & 26.5 & 10.9 & 31.8 & 5.1 & 31.3 & 6 \\ \hline
22.5 & 11.3 & 28.6 & 11.3 & 26.7 & 10.9 & 32 & 4.7 & 31.5 & 6 \\ \hline
22.6 & 11.3 & 28.8 & 11.4 & 26.8 & 10.7 & 32.2 & 4.1 & 31.7 & 5.9 \\ \hline
22.8 & 11.3 & 29 & 11.3 & 27 & 10.2 & 32.4 & 3.5 & 31.9 & 5.6 \\ \hline
22.8 & 11.3 & 29.1 & 11.3 & 27.2 & 9.5 & 32.7 & 2.9 & 32.1 & 5.1 \\ \hline
22.9 & 11.3 & 29.2 & 11.3 & 27.3 & 8.5 & 32.8 & 2.4 & 32.3 & 4.3 \\ \hline
23.1 & 11.3 & 29.4 & 11.3 & 27.5 & 7.5 & 33.1 & 1.9 & 32.5 & 3.7 \\ \hline
23.2 & 11.3 & 29.6 & 11.3 & 27.7 & 6.4 & 33.2 & 1.7 & 32.7 & 3 \\ \hline
23.4 & 11.3 & 29.7 & 11.3 & 27.9 & 5.6 & 33.4 & 1.5 & 32.9 & 2.4 \\ \hline
23.5 & 11.3 & 29.9 & 11.3 & 28 & 4.7 & 33.6 & 1.4 & 33.2 & 2 \\ \hline
23.7 & 11.3 & 30 & 11.3 & 28.2 & 4 & 33.8 & 1.4 & 33.4 & 1.7 \\ \hline
23.7 & 11.3 & 30.2 & 11.3 & 28.4 & 3.6 & 34 & 1.6 & 33.5 & 1.5 \\ \hline
23.9 & 11.3 & 30.3 & 11.3 & 28.5 & 3.3 & 34.2 & 1.8 & 33.7 & 1.5 \\ \hline
24 & 11.3 & 30.5 & 11.3 & 28.7 & 3.2 & 34.4 & 2.1 & 33.9 & 1.6 \\ \hline
24.2 & 11.3 & 30.6 & 11.3 & 28.9 & 3.3 & 34.6 & 2.6 & 34.1 & 1.7 \\ \hline
24.3 & 11.3 & 30.8 & 11.3 & 29.1 & 3.5 & 34.8 & 3 & 34.3 & 2.1 \\ \hline
24.4 & 11.3 & 31 & 11.3 & 29.3 & 3.9 & 35 & 3.7 & 34.6 & 2.5 \\ \hline
24.6 & 11.3 & 31.1 & 11.3 & 29.4 & 4.4 & 35.2 & 4.4 & 34.8 & 3.1 \\ \hline
24.7 & 11.3 & 31.3 & 11.3 & 29.6 & 5.2 & 35.4 & 5.2 & 35 & 3.7 \\ \hline
24.9 & 11.3 & 31.5 & 11.3 & 29.7 & 5.9 & 35.6 & 6.1 & 35.2 & 4.4 \\ \hline
25 & 11.3 & 31.6 & 11.3 & 29.9 & 7.1 & 35.8 & 6.9 & 35.4 & 5.4 \\ \hline
25.2 & 11.3 & 31.8 & 11.3 & 30.1 & 8.3 & 36 & 7.7 & 35.6 & 6.2 \\ \hline
25.3 & 11.3 & 32 & 11.3 & 30.2 & 9.9 & 36.2 & 8.2 & 35.8 & 7.3 \\ \hline
25.4 & 11.3 & 32.1 & 11.3 & 30.4 & 11 & 36.4 & 8.6 & 35.9 & 8.2 \\ \hline
25.6 & 11.3 & 32.2 & 11.3 & 30.6 & 11.2 & 36.6 & 8.6 & 36.2 & 9 \\ \hline
25.8 & 11.3 & 32.4 & 11.3 & 30.8 & 11.2 & 36.8 & 8.4 & 36.4 & 9.4 \\ \hline
25.9 & 11.3 & 32.6 & 11.3 & 31 & 11.3 & 37 & 7.8 & 36.6 & 9.6 \\ \hline
26 & 11.3 & 32.7 & 11.3 & 31.1 & 11.3 & 37.2 & 7.1 & 36.8 & 9.4 \\ \hline
26.1 & 11.3 & 32.9 & 11.3 & 31.3 & 11.3 & 37.4 & 6.4 & 37 & 8.9 \\ \hline
26.1 & 11.3 & 33 & 11.3 & 31.4 & 11.3 & 37.6 & 5.6 & 37.2 & 8.1 \\ \hline
26.2 & 11.3 & 32.9 & 11.3 & 31.6 & 11.3 & 37.8 & 4.8 & 37.4 & 7.3 \\ \hline
26.4 & 11.3 & 32.9 & 11.3 & 31.8 & 11.3 & 38 & 4.1 & 37.6 & 6.5 \\ \hline
26.5 & 11.3 & 33 & 11.3 & 31.9 & 11.3 & 38.2 & 3.6 & 37.9 & 5.4 \\ \hline
27.1 & 11.3 & 32.9 & 11.3 & 32.1 & 11.3 & 38.4 & 3.2 & 38.1 & 4.7 \\ \hline
27 & 11.3 & 33 & 11.3 & 32.3 & 11.3 & 38.6 & 2.9 & 38.3 & 4.1 \\ \hline
27.1 & 11.3 & 33 & 11.3 & 32.5 & 11.3 & 38.8 & 2.9 & 38.4 & 3.6 \\ \hline
27.2 & 11.3 & 33 & 11.3 & 32.7 & 11.3 & 39 & 3 & 38.6 & 3.4 \\ \hline
27.4 & 11.3 & 33 & 11.3 & 32.8 & 11.3 & 39.2 & 3.3 & 38.9 & 3.3 \\ \hline
27.5 & 11.3 & 33 & 11.3 & 33 & 11.3 & 39.4 & 3.7 & 39.1 & 3.5 \\ \hline
27.6 & 11.3 & 32.9 & 11.3 & 33.2 & 11.3 & 39.6 & 4.4 & 39.3 & 3.9 \\ \hline
27.8 & 11.3 & 32.9 & 11.3 & 33.3 & 11.3 & 39.8 & 5.1 & 39.5 & 4.4 \\ \hline
28 & 11.3 & 33 & 11.3 & 33.5 & 11.3 & 40 & 6.1 & 39.7 & 5.2 \\ \hline
28 & 11.3 & 32.9 & 11.3 & 33.7 & 11.3 & 40.2 & 7.2 & 39.9 & 6.1 \\ \hline
28.2 & 11.3 & 32.9 & 11.3 & 33.8 & 11.3 & 40.4 & 8.3 & 40.1 & 7.2 \\ \hline
28.3 & 11.3 & 32.9 & 11.3 & 34 & 11.3 & 40.6 & 9.8 & 40.3 & 8.6 \\ \hline
28.5 & 11.3 & 32.9 & 11.3 & 34.2 & 11.3 & 40.8 & 11.1 & 40.5 & 10.1 \\ \hline
28.5 & 11.3 & 32.9 & 11.3 & 34.3 & 11.3 & 41 & 11.2 & 40.7 & 11.1 \\ \hline
28.6 & 11.3 & 33 & 11.3 & 34.5 & 11.3 & 41.1 & 11.3 & 40.9 & 11.2 \\ \hline
28.6 & 11.3 & 32.9 & 11.3 & 34.7 & 11.3 & 41.3 & 11.2 & 41.1 & 11.3 \\ \hline
28.5 & 11.3 & 32.9 & 11.3 & 34.9 & 11.3 & 41.3 & 11.2 & 41.3 & 11.3 \\ \hline
28.5 & 11.3 & 33 & 11.3 & 35.1 & 11.3 & 41.3 & 11.2 & 41.5 & 11.3 \\ \hline
28.5 & 11.3 & 33 & 11.3 & 35.2 & 11.3 & 41.3 & 11.2 & 41.7 & 11.3 \\ \hline
28.5 & 11.3 & 33 & 11.3 & 35.5 & 11.3 & 41.3 & 11.2 & 41.9 & 11.3 \\ \hline
28.5 & 11.3 & 33 & 11.3 & 35.5 & 11.3 & 41.3 & 11.2 & 42.1 & 11.3 \\ \hline
28.5 & 11.3 & 33 & 11.3 & 35.5 & 11.3 & 41.3 & 11.2 & 42.3 & 11.3 \\ \hline
28.5 & 11.3 & 33 & 11.3 & 35.5 & 11.4 & 41.3 & 11.2 & 42.5 & 11.3 \\ \hline

          \caption{Messreihen zur Abhängigkeit des Anodenspannungs von der
              Temperatur.}
          \label{tab:FHtemperatur}
        \end{longtable}
      \end{scriptsize}
    \end{section}
    %%%%%%%%%%%%%%%%%%%%%%%%%%%%%%%%%%%%%%%%
    
  \end{chapter}
  %%%%%%%%%%%%%%%%%%%%%%%%%%%%%%
  
\end{appendix}
%%%%%%%%%%%%%%%%%%%%
 