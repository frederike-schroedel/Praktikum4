%%%%%%%%%%%%%%%%%%%%
%%% Document
%%%%%%%%%%%%%%%%%%%%
\documentclass[pdftex, a4paper,11pt, twoside, ngerman]{report}
% \documentclass[11pt,xcolor=dvipsnames]{beamer}

% für deutsche zeichen äüö ohne kile auto-ersetzen
% \usepackage[utf8x]{inputenc}

% kile auto-ersetzen: einstellungen->latex:general-> hacken bei special
% characters
% \usepackage[ansinew]{inputenc}
% \usepackage[UKenglish]{babel}          %Englisch
\usepackage[ngerman]{babel}          %Deutsch


%%%%%%%%%%
%%% Geometry
%%%%%%%%%%
% \usepackage{showframe}
\usepackage[scale=0.8, hmarginratio=4:2]{geometry}
  \geometry{textheight=1.05\textheight, textwidth=.95\textwidth,
            marginparwidth=25 pt}
  \parskip=7pt


%%%%%%%%%%
%%% Packages (aus header datei)
%%%%%%%%%%
\IfFileExists{header_TobiasBrauell-DOCUMENT.tex}{
    % Copyright © 2014 Tobias Brauell <tobiasbrauell@gmail.com>

% This is my general purpose LaTeX header file for writing German documents.
% Ideally, you include this using a simple ``\input{header.tex}`` in your main
% document and start with ``\title`` and ``\begin{document}`` afterwards.

% If you need to add additional packages, I recommend not doing this in this
% file, but in your main document. That way, you can just drop in a new
% ``header.tex`` and get all the new commands without having to merge manually.

%%%%%%%%%%%%%%%%%%%%%%%%%%%%%
%%% Locale, date
%%%%%%%%%%%%%%%%%%%%%%%%%%%%%
\usepackage[german]{isodate}



%%%%%%%%%%%%%%%%%%%%%%%%%%%%%
%%% Margins and other spacing
%%%%%%%%%%%%%%%%%%%%%%%%%%%%%
\usepackage[activate]{pdfcprot}
% \usepackage[parfill]{parskip}
\usepackage{setspace}
  \setlength{\columnsep}{2 cm}
  \setlength{\parindent}{0 pt}


%%%%%%%%%%%%%%%%%%%%%%%%%%%%%
%%% Input encoding
%%%%%%%%%%%%%%%%%%%%%%%%%%%%%
\usepackage[T1]{fontenc}
\usepackage[utf8x]{inputenc}



%%%%%%%%%%%%%%%%%%%%%%%%%%%%%
%%% Indexing
%%%%%%%%%%%%%%%%%%%%%%%%%%%%%
\usepackage{makeidx}
  \makeindex



%%%%%%%%%%%%%%%%%%%%%%%%%%%%%
%%% Blindtext
%%%%%%%%%%%%%%%%%%%%%%%%%%%%%
\usepackage{blindtext}


%%%%%%%%%%%%%%%%%%%%%%%%%%%%%
%%% Global Counter
%%%%%%%%%%%%%%%%%%%%%%%%%%%%%



%%%%%%%%%%%%%%%%%%%%%%%%%%%%%
%%% Geometry
%%%%%%%%%%%%%%%%%%%%%%%%%%%%%
\usepackage{layout}
% \usepackage[scale=0.8]{geometry}
%   \geometry{textheight=1.05\textheight, marginparwidth=50 pt}

% \usepackage{multirow}
% \usepackage{dcolumn}



%%%%%%%%%%%%%%%%%%%%%%%%%%%%%
%%% Pagestyle
%%%%%%%%%%%%%%%%%%%%%%%%%%%%%
% \usepackage{fancyhdr}
% \usepackage{microtype} 

% \pagestyle{fancy}



%%%%%%%%%%%%%%%%%%%%%%%%%%%%%
%%% Fonts/Colors
%%%%%%%%%%%%%%%%%%%%%%%%%%%%%
\usepackage{lmodern}
\usepackage{xcolor}
% This replaces all fonts with Bitstream Charter, Bitstream Vera Sans and
% Bitstream Vera Mono. Math will be rendered in Charter.
% \usepackage[charter, greekuppercase=italicized]{mathdesign}
% \usepackage{beramono}
% \usepackage{berasans}

% Bold, sans-serif tensors. This fragment is taken from “egreg” from
% http://tex.stackexchange.com/a/82747/8945 and licensed under `CC-BY-SA
% <https://creativecommons.org/licenses/by-sa/3.0/>`_.
% \usepackage{bm}
%   \DeclareMathAlphabet{\mathsfit}{\encodingdefault}{\sfdefault}{m}{sl}
%   \SetMathAlphabet{\mathsfit}{bold}{\encodingdefault}{\sfdefault}{bx}{sl}
%   \newcommand{\tens}[1]{\bm{\mathsfit{#1}}}

% Bold vectors.
% \renewcommand{\vec}[1]{\boldsymbol{#1}}



%%%%%%%%%%%%%%%%%%%%%%%%%%%%%
%%% Code/Listings
%%%%%%%%%%%%%%%%%%%%%%%%%%%%%
\usepackage{listings}



%%%%%%%%%%%%%%%%%%%%%%%%%%%%%
%%% Enumerations
%%%%%%%%%%%%%%%%%%%%%%%%%%%%%
\usepackage{enumitem}
% \usepackage{paralist}


%%%%%%%%%%%%%%%%%%%%%%%%%%%%%
%%% Figures
%%%%%%%%%%%%%%%%%%%%%%%%%%%%%
% \usepackage[pdftex]{graphicx}
\usepackage{graphicx}
\usepackage{epsfig}
\usepackage{epstopdf}
\usepackage{subfigure}
\usepackage{wrapfig}
\makeatletter \newcommand\hyper@makecurrent[1]{} \makeatother
\usepackage{caption}
% \usepackage{subcaption}

\addto\captionsUKenglish{\renewcommand{\figurename}{Fig.}}
\addto\captionsngerman{\renewcommand{\figurename}{Abb.}}



%%%%%%%%%%%%%%%%%%%%%%%%%%%%%
%%% PDF Pages
%%%%%%%%%%%%%%%%%%%%%%%%%%%%%
\usepackage{pdfpages}



%%%%%%%%%%%%%%%%%%%%%%%%%%%%%
%%% Personal Graphics
%%%%%%%%%%%%%%%%%%%%%%%%%%%%%
\usepackage{tikz}
% \usepackage{tikz-3dplot}
  \usetikzlibrary{calc}
  \usetikzlibrary{decorations.markings}



%%%%%%%%%%%%%%%%%%%%%%%%%%%%%
%%% Math
%%%%%%%%%%%%%%%%%%%%%%%%%%%%%
\usepackage{amsmath}
\usepackage{amssymb}
\usepackage{mathtools}
\usepackage{dcolumn}
\usepackage[
    separate-uncertainty  = true,
    uncertainty-separator =  {\,}, 
%   mode = text, 
    output-decimal-marker ={,}, 
    multi-part-units      = brackets, 
    range-units           = brackets, 
    range-phrase          = {\,--\,}
  ]{siunitx}
% \usepackage{feynmf}



%%%%%%%%%%%%%%%%%%%%%%%%%%%%%
%%% Referenzen
%%%%%%%%%%%%%%%%%%%%%%%%%%%%%
\usepackage{hyperref}
\usepackage{url}
% \usepackage{cleveref}%\label{abc}--\cref{abc} \Cref{abc[,def]}-und \crefrange{abc}{def}-bis
\usepackage[ngerman]{cleveref}%\label{abc}--\cref{abc} \Cref{abc[,def]}-und \crefrange{abc}{def}-bis



%%%%%%%%%%%%%%%%%%%%%%%%%%%%%
%%% Table's
%%%%%%%%%%%%%%%%%%%%%%%%%%%%%
\usepackage{rotating}
\usepackage{longtable}
\usepackage{multirow}
\usepackage{tabularx}
  \newcolumntype{L}[1]{>{\raggedright\arraybackslash}p{#1}} % linksbündig mit Breitenangabe
  \newcolumntype{C}[1]{>{\centering\arraybackslash}p{#1}} % zentriert mit Breitenangabe
  \newcolumntype{R}[1]{>{\raggedleft\arraybackslash}p{#1}} % rechtsbündig mit Breitenangabe



%%%%%%%%%%%%%%%%%%%%%%%%%%%%%
%%% Todo's
%%%%%%%%%%%%%%%%%%%%%%%%%%%%%
% \usepackage{xkeyval}
\usepackage{todonotes} %\todo{text} oder \todo[inline]{text}
%   \presetkeys{todonotes}{inline}{}
%   \let\todox\todo
%   \renewcommand\todo{1}{\todox[inline]{#1}}


%%%%%%%%%%%%%%%%%%%%%%%%%%%%%%%%%%%%%%%%%%%%%%%%%%%%%%%%%%
%%% Settings
%%%%%%%%%%%%%%%%%%%%%%%%%%%%%%%%%%%%%%%%%%%%%%%%%%%%%%%%%%
\usepackage{cancel}

\newcommand{\HRule}{\rule{\linewidth}{0.5mm}}



%%%%%%%%%%%%%%%%%%%%%%%%%%%%%
%%% Theme
%%%%%%%%%%%%%%%%%%%%%%%%%%%%%



%%%%%%%%%%%%%%%%%%%%%%%%%%%%%
%%% header
%%%%%%%%%%%%%%%%%%%%%%%%%%%%%
% \lhead{text}
% \chead{text}
% \rhead{text}



%%%%%%%%%%%%%%%%%%%%%%%%%%%%%
%%% footer
%%%%%%%%%%%%%%%%%%%%%%%%%%%%%
%%% Tobias Brauell       	Versuch....		Ruth Jacobs
% \renewcommand\footrulewidth{.4pt}
% \lfoot{\scriptsize Ruth Jacobs - Tobias Brauell \\ {\ \ \ \ \ \ \ \ \ \ } Gruppe $\alpha 9$} 
% \cfoot{\thepage\ / \ \pageref{LastPage}}
% \rfoot{\scriptsize Versuch 518: Höhenstrahlung \\ Tutor: Christoph Krieger {\ \ \ } } 



%%%%%%%%%%%%%%%%%%%%%%%%%%%%%
%%% Title Page
%%%%%%%%%%%%%%%%%%%%%%%%%%%%%
% \title[ITER { } International Thermonuclear Experimental Reactor]{\huge{\bf{ITER}} \\ \large{\bf{International Thermonuclear Experimental Reactor}}}
% \author[T. Brauell]{Tobias Brauell}
% \institute{Universität Bonn}
% 
% \date{09.~Dez.~2013}
% \logo{\includegraphics[width=.15\textwidth]{Figures/toplogo.png}}


}{
    % Copyright © 2014 Tobias Brauell <tobiasbrauell@gmail.com>

% This is my general purpose LaTeX header file for writing German documents.
% Ideally, you include this using a simple ``\input{header.tex}`` in your main
% document and start with ``\title`` and ``\begin{document}`` afterwards.

% If you need to add additional packages, I recommend not doing this in this
% file, but in your main document. That way, you can just drop in a new
% ``header.tex`` and get all the new commands without having to merge manually.

%%%%%%%%%%%%%%%%%%%%%%%%%%%%%
%%% Locale, date
%%%%%%%%%%%%%%%%%%%%%%%%%%%%%
\usepackage[german]{isodate}



%%%%%%%%%%%%%%%%%%%%%%%%%%%%%
%%% Margins and other spacing
%%%%%%%%%%%%%%%%%%%%%%%%%%%%%
\usepackage[activate]{pdfcprot}
% \usepackage[parfill]{parskip}
\usepackage{setspace}
  \setlength{\columnsep}{2 cm}
  \setlength{\parindent}{0 pt}


%%%%%%%%%%%%%%%%%%%%%%%%%%%%%
%%% Input encoding
%%%%%%%%%%%%%%%%%%%%%%%%%%%%%
\usepackage[T1]{fontenc}
\usepackage[utf8x]{inputenc}



%%%%%%%%%%%%%%%%%%%%%%%%%%%%%
%%% Indexing
%%%%%%%%%%%%%%%%%%%%%%%%%%%%%
\usepackage{makeidx}
  \makeindex



%%%%%%%%%%%%%%%%%%%%%%%%%%%%%
%%% Blindtext
%%%%%%%%%%%%%%%%%%%%%%%%%%%%%
\usepackage{blindtext}


%%%%%%%%%%%%%%%%%%%%%%%%%%%%%
%%% Global Counter
%%%%%%%%%%%%%%%%%%%%%%%%%%%%%



%%%%%%%%%%%%%%%%%%%%%%%%%%%%%
%%% Geometry
%%%%%%%%%%%%%%%%%%%%%%%%%%%%%
\usepackage{layout}
% \usepackage[scale=0.8]{geometry}
%   \geometry{textheight=1.05\textheight, marginparwidth=50 pt}

% \usepackage{multirow}
% \usepackage{dcolumn}



%%%%%%%%%%%%%%%%%%%%%%%%%%%%%
%%% Pagestyle
%%%%%%%%%%%%%%%%%%%%%%%%%%%%%
% \usepackage{fancyhdr}
% \usepackage{microtype} 

% \pagestyle{fancy}



%%%%%%%%%%%%%%%%%%%%%%%%%%%%%
%%% Fonts/Colors
%%%%%%%%%%%%%%%%%%%%%%%%%%%%%
\usepackage{lmodern}
\usepackage{xcolor}
% This replaces all fonts with Bitstream Charter, Bitstream Vera Sans and
% Bitstream Vera Mono. Math will be rendered in Charter.
% \usepackage[charter, greekuppercase=italicized]{mathdesign}
% \usepackage{beramono}
% \usepackage{berasans}

% Bold, sans-serif tensors. This fragment is taken from “egreg” from
% http://tex.stackexchange.com/a/82747/8945 and licensed under `CC-BY-SA
% <https://creativecommons.org/licenses/by-sa/3.0/>`_.
% \usepackage{bm}
%   \DeclareMathAlphabet{\mathsfit}{\encodingdefault}{\sfdefault}{m}{sl}
%   \SetMathAlphabet{\mathsfit}{bold}{\encodingdefault}{\sfdefault}{bx}{sl}
%   \newcommand{\tens}[1]{\bm{\mathsfit{#1}}}

% Bold vectors.
% \renewcommand{\vec}[1]{\boldsymbol{#1}}



%%%%%%%%%%%%%%%%%%%%%%%%%%%%%
%%% Code/Listings
%%%%%%%%%%%%%%%%%%%%%%%%%%%%%
\usepackage{listings}



%%%%%%%%%%%%%%%%%%%%%%%%%%%%%
%%% Enumerations
%%%%%%%%%%%%%%%%%%%%%%%%%%%%%
\usepackage{enumitem}
% \usepackage{paralist}


%%%%%%%%%%%%%%%%%%%%%%%%%%%%%
%%% Figures
%%%%%%%%%%%%%%%%%%%%%%%%%%%%%
% \usepackage[pdftex]{graphicx}
\usepackage{graphicx}
\usepackage{epsfig}
\usepackage{epstopdf}
\usepackage{subfigure}
\usepackage{wrapfig}
\makeatletter \newcommand\hyper@makecurrent[1]{} \makeatother
\usepackage{caption}
% \usepackage{subcaption}

\addto\captionsUKenglish{\renewcommand{\figurename}{Fig.}}
\addto\captionsngerman{\renewcommand{\figurename}{Abb.}}



%%%%%%%%%%%%%%%%%%%%%%%%%%%%%
%%% PDF Pages
%%%%%%%%%%%%%%%%%%%%%%%%%%%%%
\usepackage{pdfpages}



%%%%%%%%%%%%%%%%%%%%%%%%%%%%%
%%% Personal Graphics
%%%%%%%%%%%%%%%%%%%%%%%%%%%%%
\usepackage{tikz}
% \usepackage{tikz-3dplot}
  \usetikzlibrary{calc}
  \usetikzlibrary{decorations.markings}



%%%%%%%%%%%%%%%%%%%%%%%%%%%%%
%%% Math
%%%%%%%%%%%%%%%%%%%%%%%%%%%%%
\usepackage{amsmath}
\usepackage{amssymb}
\usepackage{mathtools}
\usepackage{dcolumn}
\usepackage[
    separate-uncertainty  = true,
    uncertainty-separator =  {\,}, 
%   mode = text, 
    output-decimal-marker ={,}, 
    multi-part-units      = brackets, 
    range-units           = brackets, 
    range-phrase          = {\,--\,}
  ]{siunitx}
% \usepackage{feynmf}



%%%%%%%%%%%%%%%%%%%%%%%%%%%%%
%%% Referenzen
%%%%%%%%%%%%%%%%%%%%%%%%%%%%%
\usepackage{hyperref}
\usepackage{url}
% \usepackage{cleveref}%\label{abc}--\cref{abc} \Cref{abc[,def]}-und \crefrange{abc}{def}-bis
\usepackage[ngerman]{cleveref}%\label{abc}--\cref{abc} \Cref{abc[,def]}-und \crefrange{abc}{def}-bis



%%%%%%%%%%%%%%%%%%%%%%%%%%%%%
%%% Table's
%%%%%%%%%%%%%%%%%%%%%%%%%%%%%
\usepackage{rotating}
\usepackage{longtable}
\usepackage{multirow}
\usepackage{tabularx}
  \newcolumntype{L}[1]{>{\raggedright\arraybackslash}p{#1}} % linksbündig mit Breitenangabe
  \newcolumntype{C}[1]{>{\centering\arraybackslash}p{#1}} % zentriert mit Breitenangabe
  \newcolumntype{R}[1]{>{\raggedleft\arraybackslash}p{#1}} % rechtsbündig mit Breitenangabe



%%%%%%%%%%%%%%%%%%%%%%%%%%%%%
%%% Todo's
%%%%%%%%%%%%%%%%%%%%%%%%%%%%%
% \usepackage{xkeyval}
\usepackage{todonotes} %\todo{text} oder \todo[inline]{text}
%   \presetkeys{todonotes}{inline}{}
%   \let\todox\todo
%   \renewcommand\todo{1}{\todox[inline]{#1}}


%%%%%%%%%%%%%%%%%%%%%%%%%%%%%%%%%%%%%%%%%%%%%%%%%%%%%%%%%%
%%% Settings
%%%%%%%%%%%%%%%%%%%%%%%%%%%%%%%%%%%%%%%%%%%%%%%%%%%%%%%%%%
\usepackage{cancel}

\newcommand{\HRule}{\rule{\linewidth}{0.5mm}}



%%%%%%%%%%%%%%%%%%%%%%%%%%%%%
%%% Theme
%%%%%%%%%%%%%%%%%%%%%%%%%%%%%



%%%%%%%%%%%%%%%%%%%%%%%%%%%%%
%%% header
%%%%%%%%%%%%%%%%%%%%%%%%%%%%%
% \lhead{text}
% \chead{text}
% \rhead{text}



%%%%%%%%%%%%%%%%%%%%%%%%%%%%%
%%% footer
%%%%%%%%%%%%%%%%%%%%%%%%%%%%%
%%% Tobias Brauell       	Versuch....		Ruth Jacobs
% \renewcommand\footrulewidth{.4pt}
% \lfoot{\scriptsize Ruth Jacobs - Tobias Brauell \\ {\ \ \ \ \ \ \ \ \ \ } Gruppe $\alpha 9$} 
% \cfoot{\thepage\ / \ \pageref{LastPage}}
% \rfoot{\scriptsize Versuch 518: Höhenstrahlung \\ Tutor: Christoph Krieger {\ \ \ } } 



%%%%%%%%%%%%%%%%%%%%%%%%%%%%%
%%% Title Page
%%%%%%%%%%%%%%%%%%%%%%%%%%%%%
% \title[ITER { } International Thermonuclear Experimental Reactor]{\huge{\bf{ITER}} \\ \large{\bf{International Thermonuclear Experimental Reactor}}}
% \author[T. Brauell]{Tobias Brauell}
% \institute{Universität Bonn}
% 
% \date{09.~Dez.~2013}
% \logo{\includegraphics[width=.15\textwidth]{Figures/toplogo.png}}


}



%%%%%%%%%%
%%%%%%%%%%
%%%%%%%%%%
\begin{document}
%   \layout
  
  
  
  %%%%%%%%%%%%%%%%%%%%
  %%%%%%%%%%%%%%%%%%%%
  %%%%%%%%%%%%%%%%%%%%
  \input{./Titlepage-Versuch424.tex}
  %%%%%%%%%%%%%%%%%%%%
  
  
  
%   \setcounter{page}{2}
  
  \begin{chapter}*{Abstract}
    Ziel des Versuchs ist es...
    
    \todo{TO-DO}
    
  \end{chapter}
  
  \tableofcontents
  
  
  
  %%%%%%%%%%%%%%%%%%%%
  %%%%%%%%%%%%%%%%%%%%
  %%%%%%%%%%%%%%%%%%%%
  \begin{chapter}{Theorie des Versuchs}
    \label{chp:Theorie}
    
    
    
  \end{chapter}
  %%%%%%%%%%%%%%%%%%%%
         
         
         
  %%%%%%%%%%%%%%%%%%%%
  %%%%%%%%%%%%%%%%%%%%
  %%%%%%%%%%%%%%%%%%%%
  \begin{chapter}{Aufbau des Versuches}
    \label{chp:Aufbau}
    Der Aufbau dieses Versuches ist in \cref{fig:Aufbau} zu sehen.
    Die Probe wird im Rezipienten montiert und angeschlossen.
    An der Probe kann von außen ein Magnetfeld mittels eines permanent Magneten
    angelegt werden. An der Stromquelle wird ein Strom nach den Vorgaben der
    ausliegenden Anleitungen und der Art der Probe eingestellt.
    Dieser Strom bleibt für alle Messungen dieser Probe unbedingt konstant.
    An einem Voltmeter keine die anliegende Spannung gemessen werden.
    Stromquelle und Voltmeter sind über einen Schaltkasten mit der Probe im
    Rezipienten verbunden.
    Über den Schaltkasten lässt sich die Beschaltung der Probe mit zwei
    Drehreglern einstellen.
    Welche Art der Beschaltung zur Verfügung stehen kann ebenfalls in den
    ausliegenden Tabellen nachgeschlagen werden.
    Am benutzte permanent Magnet wurden die Pole markiert und kann um die Probe
    gedreht werden um die Ausrichtung des Magnetfeldes umzukehren.
    
    Weiter ist für den zweiten Teil dieses Versuches ein Helium Kühlaggregat am
    Rezipienten und eine Heizung am Probenträger selbst angebracht.
    Über das Kühlaggregat kann die Probe auf sehr niedrige Temperaturen
    abgekühlt werden. Zum erwärmen, kann ein Strom von einer zweiten Stromquelle
    durch den Heizdraht geschickt werden. Die Temperatur der Probe kann über
    die Spannung an einem Temperatursensor und einer ausliegenden
    Tabelle umgerechnet werden.
    Für die Messung der Temperaturabhängigkeit wird der Rezipient auf einen
    Druck von etwa \SI{1e-8}{\bar} evakuiert.
    
    
  \end{chapter}
  %%%%%%%%%%%%%%%%%%%%
  
  
  
  %%%%%%%%%%%%%%%%%%%%
  %%%%%%%%%%%%%%%%%%%%
  %%%%%%%%%%%%%%%%%%%%
  \begin{chapter}{Durchführung der Messungen}
    \label{chp:Durchführung}
    
    
    
    %%%%%%%%%%%%%%%%%%%%%%%%%%%%%%
    %%%%%%%%%%%%%%%%%%%%%%%%%%%%%%
    %%%%%%%%%%%%%%%%%%%%%%%%%%%%%%
    \begin{section}{Messung des Widerstandes bei Raumtemperatur}
      \label{chp:MessungWiderstandRaumtemperatur}
      Die Messung des Widerstandes führen wir für die Proben
      \textit{GaAs (alt)} bei \SI{100,000(5)}{\micro\ampere} und
      \textit{InAs - HF-301-040} bei \SI{15,003(5)}{\milli\ampere} durch.
      Die Raumtemperatur bei den Messungen betrug \SI{26}{\celsius} und die
      stärke des permanent Magneten ist mit \SI{0,138(1)}{\tesla} angegeben.
      
      Für beide Proben messen wir die Spannungen aller acht Schaltungen zur
      Bestimmung des Widerstandes. Um die Messwerte statistisch mitteln zu
      können, nehmen wir für jede Schaltung fünf verschiedene Werte.
      Die gewonnenen Messwerte können in
      \cref{tab:WiderstandGaAs,tab:WiderstandInAs} eingesehen werden.
      \begin{table}[htbp]
        \centering
        \footnotesize
        \begin{tabular}{|c|c|c|c|c|c|c|c|}
          \hline
          Schaltung & $U / mV$  & $U / mV$  & $U / mV$  & $U / mV$  & $U / mV$
               & $<U> / mV$  & $\sigma(<U>) / mV$ \\ \hline
          % & mess & nr. & durchgang & 1 & durchgang & 2 & durchgang & 3 & durchgang & 4 & durchgang & 5 & Average & std & dev \\ 
1 & -0,244 & -0,244 & -0,244 & -0,244 & -0,243 & -0,2438 & 0,0004 \\ 
2 & 0,241 & 0,241 & 0,242 & 0,242 & 0,241 & 0,2414 & 0,000489898 \\ 
3 & -0,193 & -0,193 & -0,193 & -0,193 & -0,193 & -0,193 & 0 \\ 
4 & 0,19 & 0,19 & 0,19 & 0,19 & 0,19 & 0,19 & 0 \\ 
5 & -0,243 & -0,243 & -0,243 & -0,243 & -0,242 & -0,2428 & 0,0004 \\ 
6 & 0,242 & 0,242 & 0,242 & 0,242 & 0,242 & 0,242 & 0 \\ 
7 & -0,193 & -0,192 & -0,193 & -0,192 & -0,192 & -0,1924 & 0,000489898 \\ 
8 & 0,19 & 0,191 & 0,19 & 0,19 & 0,191 & 0,1904 & 0,000489898 \\ 

        \end{tabular}
        \caption{Messdaten der Widerstands Messung für die \textit{GaAs (alt)}
            Probe.}
        \label{tab:WiderstandGaAs}
      \end{table}
      \begin{table}[htbp]
        \centering
        \footnotesize
        \begin{tabular}{|c|c|c|c|c|c|c|c|}
          \hline
          Schaltung & $U / mV$  & $U / mV$  & $U / mV$  & $U / mV$  & $U / mV$
               & $<U> / mV$  & $\sigma(<U>) / mV$ \\ \hline
          \input{Tables/InAsWiderstand.tex}
        \end{tabular}
        \caption{Messdaten der Widerstands Messung für die
            \textit{InAs - HF-301-040} Probe.}
        \label{tab:WiderstandInAs}
      \end{table}
      
      
      
    \end{section}
    %%%%%%%%%%%%%%%%%%%%%%%%%%%%%%
    
    
    
    %%%%%%%%%%%%%%%%%%%%%%%%%%%%%%
    %%%%%%%%%%%%%%%%%%%%%%%%%%%%%%
    %%%%%%%%%%%%%%%%%%%%%%%%%%%%%%
    \begin{section}{Messung der Hallkonstanten bei Raumtemperatur}
      \label{chp:MessungHallkonstanteRaumtemperatur}
      Bei der Messung der Hallkonstanten bei Raumtemperatur wird dem Aufbau der
      permanent Magnet hinzugefügt und die Schaltungen zur Messung der
      Hallkonstanten verwendet.
      Auch hier nehmen wir wieder fünf Werte und
      bilden den Mittelwert um statistisch aussagekräftige Messwerte und
      statistische Fehler zu erhalten.
      Die gewonnenen Messwerte können in
      \cref{tab:HallGaAs,tab:HallInAs} eingesehen werden.
      \begin{table}[htbp]
        \centering
        \footnotesize
        \begin{tabular}{|c|c|c|c|c|c|c|c|c|}
          \hline
          Schaltung & Magn. & $U / mV$ & $U / mV$  & $U / mV$  & $U / mV$  & $U / mV$
               & $<U> / mV$  & $\sigma(<U>) / mV$ \\ \hline
          % & mess & nr \\ \hline
1 & $B+$ & -0,008 & -0,008 & -0,008 & -0,007 & -0,007 & -0,0076 & 0,000489898 \\ \hline
2 & $B+$ & 0,005 & 0,006 & 0,006 & 0,006 & 0,006 & 0,0058 & 0,0004 \\ \hline
3 & $B+$ & 0,094 & 0,094 & 0,094 & 0,095 & 0,095 & 0,0944 & 0,000489898 \\ \hline
4 & $B+$ & -0,097 & -0,096 & -0,0965 & -0,096 & -0,096 & -0,0963 & 0,0004 \\ \hline
5 & $B-$ & -0,096 & -0,097 & -0,096 & -0,096 & -0,095 & -0,096 & 0,000632456 \\ \hline
6 & $B-$ & 0,056 & 0,095 & 0,095 & 0,096 & 0,095 & 0,0874 & 0,015704776 \\ \hline
7 & $B-$ & 0,005 & 0,005 & 0,005 & 0,005 & 0,006 & 0,0052 & 0,0004 \\ \hline
8 & $B-$ & -0,008 & -0,008 & -0,008 & -0,008 & -0,007 & -0,0078 & 0,0004 \\ \hline
9 & 0 & -0,051 & -0,051 & -0,051 & -0,051 & -0,05 & -0,0508 & 0,0004 \\ \hline
10 & 0 & 0,05 & 0,05 & 0,05 & 0,05 & 0,051 & 0,0502 & 0,0004 \\ \hline

        \end{tabular}
        \caption{Messdaten zur Bestimmung der Hallkonstanten für die
            \textit{GaAs (alt)} Probe.}
        \label{tab:HallGaAs}
      \end{table}
      \begin{table}[htbp]
        \centering
        \footnotesize
        \begin{tabular}{|c|c|c|c|c|c|c|c|c|}
          \hline
          Schaltung & Magn. & $U / mV$ & $U / mV$  & $U / mV$  & $U / mV$  & $U / mV$
               & $<U> / mV$  & $\sigma(<U>) / mV$ \\ \hline
          \input{Tables/InAsHall.tex}
        \end{tabular}
        \caption{Messdaten zur Bestimmung der Hallkonstanten für die
            \textit{InAs - HF-301-040} Probe.}
        \label{tab:HallInAs}
      \end{table}
      
      Bei den Schaltungen $1-4$ wird das externe Magnetfeld in \textit{$B+$}
      angelegt, d.h. der markierte Nordpol des permanent Magneten liegt auf
      der Seite des Probenträger, auf dem die Probe montiert ist.
      Schaltungen $5-8$ wird in \textit{$B-$} gemessen. Dabei wird der Magnet
      kurzerhand um \SI{180}{\degree} gedreht, sodass nun der Südpol zur Probe
      zeigt.
      Die Schaltungen $9-10$ werden ohne den externen Magneten gemessen.
      
      
    \end{section}
    %%%%%%%%%%%%%%%%%%%%%%%%%%%%%%
    
    
    
    %%%%%%%%%%%%%%%%%%%%%%%%%%%%%%
    %%%%%%%%%%%%%%%%%%%%%%%%%%%%%%
    %%%%%%%%%%%%%%%%%%%%%%%%%%%%%%
    \begin{section}{Messungen bei unterschiedlichen Temperaturen}
      \label{chp:MessungUnterschiedlicheTemperaturen}
      Nachdem alle Messungen am ersten Tag abgeschlossen sind schalten wir die
      Vakuum Turbopumpe und die Helium Kühlung des Rezipienten ein.
      Dabei haben wir die Probe \textit{InAs - HF-301-040} für die kommenden
      Messungen im Rezipienten.
      An dem angeschlossenen Voltmeter lässt sich die Thermospannung ablesen
      und mit der ausliegenden Tabelle in \SI{}{\celsius} umrechnen.
      Über Nacht wurde die Probe auf etwa \SI{-205}{\kelvin} abgekühlt.
      Nun messen wir die Temperaturabhängigkeiten des Widerstandes und der
      Hallkonstante. Dazu messen wir alle $18$ Schaltungen bei einer Temperatur
      jeweils einmal, sobald wir mithilfe der Heizung die Temperatur
      stabilisiert haben.
      
      \todo{noch was zur Durchführung?}
      
      In \cref{tab:TemperaturInAs} sind alle Messwerte
      festgehalten. Die Temperatur wurde während der Messung direkt von
      uns gegen die Raumtemperatur korrigiert.
      \begin{sidewaystable}[htbp]
        \centering
        \footnotesize
        \rowcolors{3}{}{lightgray!40}
        \begin{tabular}{c c c c c c c c c c c c c c c}
          % \cline{2-15}
\\ \hline
\multicolumn{15}{|c|}{Temperaturen der Messungen $/K$} \\ \hline
$t_1$ & -206 & -196 & -183 & -172 & -163 & -153 & -140 & -125 & -110 & -93 & -79 & -64 & -27 & 0 \\
$t_2$ & -205 & -195 & -183 & -173 & -160 & -151 & -140 & -125 & -109 & -93 & -79 & -62 & -27 & 0 \\
$<t>$ & -205,5 & -195,5 & -183 & -172,5 & -161,5 & -152 & -140 & -125 & -109,5 & -93 & -79 & -63 & -27 & 0 \\ \hline
\hiderowcolors 
\multicolumn{15}{|c|}{Messungen zur Bestimmung des Widerstandes} \\ \hline
\showrowcolors
Schaltung & $U/mV$ & $U/mV$ & $U/mV$ & $U/mV$ & $U/mV$ & $U/mV$ & $U/mV$ & $U/mV$ & $U/mV$ & $U/mV$ & $U/mV$ & $U/mV$ & $U/mV$ & $U/mV$ \\
1 & -82,382 & -82,54 & -82,9 & -83,3 & -83,596 & -83,793 & -84,02 & -84,54 & -85,195 & -85,71 & -86,432 & -87,258 & -88,869 & -90,2 \\
2 & 82,505 & 82,665 & 83,027 & 83,412 & 83,712 & 83,9 & 84,12 & 84,615 & 85,274 & 85,75 & 86,476 & 87,288 & 88,869 & 90,166 \\
3 & -82,077 & -82,24 & -82,608 & -82,987 & -83,305 & -83,495 & -83,714 & -84,2 & -84,888 & -85,344 & -86,102 & -86,907 & -88,523 & -89,837 \\
4 & 82,111 & 82,277 & 82,65 & 83,025 & 83,321 & 83,55 & 83,755 & 84,228 & 84,928 & 85,372 & 86,129 & 86,924 & 88,527 & 89,802 \\
5 & -82,429 & -82,598 & -82,97 & -83,335 & -83,627 & -83,884 & -84,058 & -84,52 & -85,243 & -85,688 & -86,455 & -87,253 & -88,875 & -90,157 \\
6 & 82,458 & 82,622 & 82,99 & 83,349 & 83,64 & 83,902 & 84,066 & 84,513 & 85,251 & 85,693 & 86,463 & 87,265 & 88,91 & 90,182 \\
7 & -82,193 & -82,364 & -82,73 & -83,081 & -83,378 & -83,63 & -83,786 & -84,223 & -84,934 & -85,393 & -86,151 & -86,94 & -88,563 & -89,8 \\
8 & 81,992 & 82,172 & 82,48 & 82,905 & 83,218 & 83,467 & 83,628 & 84,076 & 84,805 & 85,28 & 86,057 & 86,865 & 88,547 & 89,823 \\ \hline
\hiderowcolors
\multicolumn{15}{|c|}{Messungen zur Bestimmung der Hallkonstanten} \\ \hline
\showrowcolors
Schaltung & $U/mV$ & $U/mV$ & $U/mV$ & $U/mV$ & $U/mV$ & $U/mV$ & $U/mV$ & $U/mV$ & $U/mV$ & $U/mV$ & $U/mV$ & $U/mV$ & $U/mV$ & $U/mV$ \\
1 & 26,978 & 27,03 & 27,041 & 27,048 & 27,044 & 27,047 & 27,028 & 26,99 & 26,922 & 26,879 & 26,831 & 26,779 & 26,621 & 26,483 \\
2 & -27,15 & -27,205 & -27,214 & -27,219 & -27,2 & -27,197 & -27,173 & -27,127 & -27,04 & -26,983 & -26,916 & -26,85 & -26,63 & -26,443 \\
3 & 27,808 & 27,849 & 27,848 & 27,85 & 27,832 & 27,828 & 27,806 & 27,761 & 27,684 & 27,631 & 27,573 & 27,518 & 27,317 & 27,154 \\
4 & -27,742 & -27,787 & -27,786 & -27,789 & -27,775 & -27,771 & -27,746 & -27,707 & -27,63 & -27,584 & -27,533 & -27,481 & -27,303 & -27,16 \\
5 & -27,889 & -27,923 & -27,941 & -27,938 & -27,914 & -27,914 & -27,88 & -27,832 & -27,754 & -27,7 & -27,639 & -27,563 & -27,352 & -27,177 \\
6 & 27,722 & 27,752 & 27,773 & 27,774 & 27,761 & 27,766 & 27,74 & 27,699 & 27,642 & 27,6 & 27,558 & 27,5 & 27,345 & 27,219 \\
7 & -27,085 & -27,12 & -27,14 & -27,143 & -27,122 & -27,125 & -27,095 & -27,052 & -26,98 & -26,928 & -26,88 & -26,811 & -26,628 & -26,48 \\
8 & 27,15 & 27,18 & 27,198 & 27,199 & 27,175 & 27,177 & 27,15 & 27,101 & 27,028 & 26,973 & 26,918 & 26,842 & 26,642 & 26,471 \\
9 & -0,44 & -0,439 & -0,435 & -0,431 & -0,426 & -0,421 & -0,416 & -0,414 & -0,404 & -0,399 & -0,391 & -0,384 & -0,36 & -0,339 \\
10 & 0,268 & 0,263 & 0,263 & 0,263 & 0,266 & 0,269 & 0,272 & 0,275 & 0,287 & 0,296 & 0,304 & 0,318 & 0,47 & 0,376 \\

        \end{tabular}
        \caption{Messdaten zur Bestimmung des Widerstandes und der 
            Hallkonstanten für \textit{InAs - HF-301-040} bei
            unterschiedlichen Temperaturen.}
        \label{tab:TemperaturInAs}
      \end{sidewaystable}
      
      
    \end{section}
    %%%%%%%%%%%%%%%%%%%%%%%%%%%%%%
   
    
    
  \end{chapter}
  %%%%%%%%%%%%%%%%%%%%
  
  
  
  %%%%%%%%%%%%%%%%%%%%
  %%%%%%%%%%%%%%%%%%%%
  %%%%%%%%%%%%%%%%%%%%
  \begin{chapter}{Auswertung der Messdaten}
    \label{chp:Auswertung}
    
    
    
    %%%%%%%%%%%%%%%%%%%%%%%%%%%%%%
    %%%%%%%%%%%%%%%%%%%%%%%%%%%%%%
    %%%%%%%%%%%%%%%%%%%%%%%%%%%%%%
    \begin{section}{Spezifischer Widerstand bei Raumtemperatur}
      \label{chp:AuswertungSpezifischerWiderstandRaumtemperatur}
      
      
      
    \end{section}
    %%%%%%%%%%%%%%%%%%%%%%%%%%%%%%
   
   
   
    %%%%%%%%%%%%%%%%%%%%%%%%%%%%%%
    %%%%%%%%%%%%%%%%%%%%%%%%%%%%%%
    %%%%%%%%%%%%%%%%%%%%%%%%%%%%%%
    \begin{section}{Hallkonstante bei Raumtemperatur}
      \label{chp:AuswertungHallkonstanteRaumtemperatur}
      
      
      
    \end{section}
    %%%%%%%%%%%%%%%%%%%%%%%%%%%%%%
    
    
    
    %%%%%%%%%%%%%%%%%%%%%%%%%%%%%%
    %%%%%%%%%%%%%%%%%%%%%%%%%%%%%%
    %%%%%%%%%%%%%%%%%%%%%%%%%%%%%%
    \begin{section}{Beweglichkeit der Ladungen}
      \label{chp:AuswertungBeweglichkeitLadungen}
      
      
     
    \end{section}
    %%%%%%%%%%%%%%%%%%%%%%%%%%%%%%
   
   
   
    %%%%%%%%%%%%%%%%%%%%%%%%%%%%%%
    %%%%%%%%%%%%%%%%%%%%%%%%%%%%%%
    %%%%%%%%%%%%%%%%%%%%%%%%%%%%%%
    \begin{section}{Temperaturabhängigkeit}
      \label{chp:AuswertungTemperaturen}
      
      
      
    \end{section}
    %%%%%%%%%%%%%%%%%%%%%%%%%%%%%%
    
    
    
    %%%%%%%%%%%%%%%%%%%%%%%%%%%%%%
    %%%%%%%%%%%%%%%%%%%%%%%%%%%%%%
    %%%%%%%%%%%%%%%%%%%%%%%%%%%%%%
    \begin{section}{Fazit}
      \label{chp:Fazit}
      
      
      
    \end{section}
    %%%%%%%%%%%%%%%%%%%%%%%%%%%%%%
    
  \end{chapter}
  %%%%%%%%%%%%%%%%%%%%
  
  
  
  %%%%%%%%%%%%%%%%%%%%
  %%%%%%%%%%%%%%%%%%%%
  %%%%%%%%%%%%%%%%%%%%
  \input{./Versuch424_TobiasBrauell_FrederikeSchroedel_Appendix.tex}
  %%%%%%%%%%%%%%%%%%%%
  
  
  
  %%%%%%%%%%%%%%%%%%%%
  %%%%%%%%%%%%%%%%%%%%
  %%%%%%%%%%%%%%%%%%%%
  \begin{thebibliography}{99}
    \scriptsize
    \bibitem{bib:Anleitung}\url{http://www.praktika.physik.uni-bonn.de/module/physik412/downloads/p441d}
\bibitem{bib:Theorieteil}\textit{Weißlichtspektroskopie an metallischen
Nanostrukturen - Erstellung eines Versuchs für ein Praktikum im Rahmen des
Physikstudiums an der Universität Bonn} von Roberto Röll 2013
\bibitem{bib:Wiki}\url{http://en.wikipedia.org/wiki/Plasmon}



  \end{thebibliography}
  %%%%%%%%%%%%%%%%%%%%
 
\end{document}
%%%%%%%%%%
